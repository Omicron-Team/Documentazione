\documentclass[12pt,a4paper]{letter}

\usepackage[a4paper, margin=1in, includefoot]{geometry}
\usepackage[utf8]{inputenc}
\usepackage{graphicx}
\usepackage{float}
\usepackage{fancyhdr}
\usepackage[italian]{babel}
\usepackage{ucs}
\usepackage[table]{xcolor} 
\usepackage{longtable}
\usepackage[colorlinks=true]{hyperref}
\usepackage{lastpage}
\usepackage{lipsum}
\usepackage{array}
\usepackage{graphicx}
\usepackage{amsmath}
\usepackage{amssymb}
\usepackage{../../Utilita/latexSetup/styleLettera}
%%Comandi particolari per i nomi delle figure. Chiamare come una funzione LaTeX
\newcommand{\Omicron}{\emph{Omicron }}
\newcommand{\respProg}{\emph{Responsabile di Progetto }}
\newcommand{\ammProg}{\emph{Amministratore di Progetto }}
\newcommand{\analProg}{\emph{Analista }}
\newcommand{\verifProg}{\emph{Verificatore }}
\newcommand{\programProg}{\emph{Programmatore }}
\newcommand{\progetProg}{\emph{Progettista }}
\newcommand{\proponProg}{\emph{Proponente }}
\newcommand{\commitProg}{\emph{Committente }}
\newcommand{\nameproject}{\emph{Dopo scelta del progetto}}

%committenti
\newcommand{\Committente}{\VT \newline \CR}
\newcommand{\VT}{Prof. Vardanega Tullio}
\newcommand{\CR}{Prof. Cardin Riccardo}


% proponenti
\newcommand{\Proponente}{Dopo scelta del progetto}
\newcommand{\ZD}{Dopo scelta del progetto}
\newcommand{\CT}{Dopo scelta del progetto}

% Omicron team
\newcommand{\MB}{Matthew Balzan}
\newcommand{\DF}{Vasile Tusa}
\newcommand{\FD}{Francesco Dallan}
\newcommand{\NM}{Niccolò Mantovani}
\newcommand{\SB}{Silvia Bazzeato}
\newcommand{\GB}{Gabriel Bizzo}
\newcommand{\MDI}{Marco Dello Iacovo}

% documenti
\newcommand{\SdF}{Studio di Fattibilità}
\newcommand{\SdFv}[1]{\textit{Studio di Fattibilità {#1}}}
\newcommand{\PdQ}{Piano di Qualifica}
\newcommand{\PdQv}[1]{\textit{Piano di Qualifica {#1}}}
\newcommand{\PdP}{Piano di Progetto}
\newcommand{\PdPv}[1]{\textit{Piano di Progetto {#1}}}
\newcommand{\NdP}{Norme di Progetto}
\newcommand{\NdPv}[1]{\textit{Norme di Progetto {#1}}}
\newcommand{\AdR}{Analisi dei Requisiti}
\newcommand{\AdRv}[1]{\textit{Analisi dei Requisiti {#1}}}
\newcommand{\Glossario}{Glossario}
\newcommand{\Glossariov}[1]{\textit{Glossario {#1}}}
\newcommand{\MM}{Manuale Manutentore}
\newcommand{\MMv}[1]{\textit{Manuale Manutentore {#1}}}
\newcommand{\MU}{Manuale Utente}
\newcommand{\MUv}[1]{\textit{Manuale Utente {#1}}}


%command HRule
\newcommand{\HRule}{\rule{\linewidth}{0.5mm}}
\usepackage[official]{eurosym}



\date{} %rimuove la data automatica
\address{\VT{} \\ \CR{} \\ Università degli Studi di Padova \\ Dipartimento di Matematica \\ Via Trieste, 63\\ 35121 Padova }


\begin{document}
    \begin{letter}
        { Egregio \VT{},\\Egregio \CR{},}
        \begin{minipage}{.5\textwidth}
            \begin{flushleft}
                \includegraphics[width=.77\linewidth]{../../Utilita/img/OmicronLogo.png}
            \end{flushleft}
        \end{minipage}
        \begin{minipage}{.5\textwidth}
            \begin{flushright}
               \includegraphics[width=.67\linewidth]{../../Utilita/img/LogoUnipd.png}
            \end{flushright}
        \end{minipage}
        {    
        \begin{flushleft}
			\vspace{1cm}
            Corso di Ingegneria del Software\\ Gruppo \Omicron{}\\ Email:  \textit{omicronswe@gmail.com} \\ 28 Maggio 2021 
        \end{flushleft}
        }
        \opening{ Con la presente il gruppo \Omicron{} intende comunicarVi la partecipazione alla Revisione di Accettazione fissata in data 1 Giugno 2021, con lo scopo di esporVi il prodotto completo da Voi commissionato, denominato:}
        \begin{center}
           \textbf{\nameproject{} : piattaforma di e-commerce in stile Serverless} 
        \end{center}
        proposto dall'azienda \textit{\Proponente{}}.\\
        In allegato sono inclusi i seguenti documenti:
            \begin{itemize}
                \item \Glossariov{4.0.0};
                \item \PdPv{4.0.0};
                \item \PdQv{4.0.0};
                \item \NdPv{4.0.0};
                \item \VIv{2021-04-24};
                \item \VIv{2021-05-09};
                \item \VIv{2021-05-17};
                \item \VEv{2021-05-19};
                \item \MMv{1.0.0};
                \item \MUv{1.0.0}.
            \end{itemize}
            
            
La documentazione ufficiale relativa al prodotto è disponibile all'interno del file \textit{Documentazione.zip} al seguente indirizzo:\\ \url{https://github.com/Omicron-Team/Documentazione/releases/tag/RA}.

Il codice del prodotto, invece, è disponibile in questi due indirizzi, suddivisi per modulo di sviluppo:
\begin{itemize}
	\item \textbf{Modulo Front-end:} \url{https://github.com/OmicronSwe/EmporioLambda-FE};
	\item \textbf{Modulo Back-end:} \url{https://github.com/OmicronSwe/EmporioLambda-BE}.
\end{itemize}
        
        Come riportato nel documento \PdPv{4.0.0}, il preventivo di costo finale è pari a \textbf{13.148,00\euro{}}.
        
        Il gruppo è formato dai seguenti componenti:
        
        {

\rowcolors{2}{azzurro2}{azzurro3}

\centering
\renewcommand{\arraystretch}{1.8}
\begin{longtable}{C{4cm} C{3cm}}

\rowcolor{azzurro1}
\textbf{Nominativo} &
\textbf{Matricola}\\
\endhead

\MB & 1193093 \\
\VAS & 1121847 \\
\FD & 1193405 \\
\NM & 1187325 \\
\SB & 1096809 \\
\GB & 1170734 \\
\MDI & 1193421 \\

\end{longtable}
}
Rimaniamo a Vostra completa disposizione per eventuali chiarimenti. \\
        Cordiali saluti,\\
        \closing{\centering{\MDI{}, \\ \textit{\respProg{}} \\ \includegraphics[width=.7\linewidth]{Firma/Marco.png}}}
        

    \end{letter}
\end{document}