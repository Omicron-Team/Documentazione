\section{Resoconto}
\subsection{Organizzazione per la revisione di progettazione}
La video chiamata è iniziata con una discussione mirata ad individuare le azioni necessarie da parte del gruppo per presentarsi alla Revisione di Progettazione. \\
I punti principalmente affrontati riguardano:
\begin{itemize}
\item le modifiche da effettuare ai documenti già presentati alla Revisione dei Requisiti;
\item l'adeguatezza di ciò che è stato implementato per il Proof of Concept\ped{G}.
\end{itemize}
\subsection{Discussione sui documenti che necessitano di modifiche}
Dopo una attenta analisi da parte del gruppo, sono stati individuati i seguenti documenti da modificare:
\begin{itemize}
\item \AdRv{v1.0.0};
\item \PdPv{v1.0.0};
\item \PdQv{v1.0.0};
\item \NdPv{v1.0.0}.
\end{itemize}
Ogni membro del gruppo, in base al suo ruolo, provvederà ad effettuare le opportune modifiche sulla base dell'esito della Revisione dei Requisiti.
\subsection{Discussione sul Proof of Concept}
Sono state presentate le funzionalità implementate dai vari membri del gruppo per il Proof of Concept\ped{G}, verificando che funzionasse tutto correttamente.
\subsection{Organizzazione per la presentazione del Proof of Concept}
Infine è stata effettuata una discussione su come presentare il Proof of Concept\ped{G} al \CR , riscontrando come metodo più efficace l'utilizzo di alcune slide per la spiegazione delle scelte tecnologiche, in modo da supportare al meglio l'esecuzione del dimostratore eseguibile.