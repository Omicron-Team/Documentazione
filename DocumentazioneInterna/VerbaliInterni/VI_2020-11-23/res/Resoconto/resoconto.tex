\section{Resoconto}
\subsection{Scelta del capitolato}
La maggior parte dell'incontro è stato utilizzato per scegliere definitivamente il capitolato su cui lavorare durante i prossimi mesi.\\
Dall'ultimo meeting svolto, erano rimaste queste preferenze (in ordine):
\begin{itemize}
\item C1 - \textit{BlockCOVID}
\item C2 - \textit{EmporioLambda}
\item C7 - \textit{SSD}
\end{itemize}
Successivamente ad una discussione approfondita sui capitolati in questione, sono state cambiate le preferenze, in base a qualità, rischi e problemi di disponibilità:
\begin{itemize}
\item C2 - \textit{EmporioLambda}
\item C1 - \textit{BlockCOVID}
\item C3 - \textit{GDP}
\end{itemize}
In particolare, il capitolato C1, nonostante fosse molto popolare all'interno del gruppo, risultava abbastanza complicato in pratica e, inoltre, molto richiesto dagli altri team. Il capitolato definitivo quindi ricade sulla nostra originale seconda scelta, \textit{EmporioLambda} di \textit{Red Babel}. La terza preferenza è stata rimpiazzata dal capitolato C3, dato che la proposta di \textit{Zextras} non era più piaciuta ai membri del gruppo.\\
In caso non ci fossero conflitti di interesse con altri gruppi, il capitolato C2 - \textit{EmporioLambda} di \textit{Red Babel} sarà la nostra scelta definitiva.
\subsection{Scelta degli strumenti di documentazione}
Dopo un'attenta analisi dei vari strumenti disponibili per scrivere documentazione, sono stati scelti codesti:
\begin{itemize}
\item LaTeX\ped{G}.
\item Github\ped{G}.
\end{itemize}
È stato inoltre consigliato l'utilizzo di TexMaker\ped{G} come editor e MiKTeX\ped{G} come distribuzione di LaTeX\ped{G}.\\
Github\ped{G} verrà utilizzato per la scrittura collaborativa, il versionamento\ped{G} dei documenti e la tracciabilità delle modifiche e dei problemi che risalgono.\\
\SB{} si occuperà, insieme a \NM{}, della creazione del template\ped{G} LaTeX\ped{G}, che sarà finalizzato in settimana.
\NM{} si occuperà dell'intero setup di Github\ped{G}, tra cui: organizzazione cartelle, inserimento template\ped{G} LaTeX\ped{G} e inserimento di una Github Action\ped{G} per la compilazione remota dei documenti.
\subsection{Affidamento primo studio di fattibilità}
È stato affidato il compito a tutti membri del gruppo di lavorare in un semplice \SdF{}\ped{G} (un capitolato per persona) in modo da discuterne durante il prossimo incontro e aiutare i futuri redattori del documento. Sono stati assegnati in base a preferenze personali e viene chiesto solo di riportare le tecnologie interessate, alcuni aspetti positivi e alcune criticità.