\section{Resoconto}
\subsection{Organizzazione per la revisione di qualifica}
La video chiamata è iniziata con una discussione mirata ad individuare le azioni necessarie da parte del gruppo per presentarsi alla Revisione di Qualifica. \\
I punti principalmente affrontati riguardano:
\begin{itemize}
\item le modifiche da effettuare ai documenti già presentati alla Revisione di Progettazione, che ancora non erano stati aggiornati;
\item il livello di avanzamento della parte di codifica.
\end{itemize}
\subsection{Discussione sui documenti che necessitano di modifiche}
Dopo un'attenta analisi da parte del gruppo, sono stati individuati i seguenti documenti da modificare:
\begin{itemize}
\item \PdPv{v2.0.0};
\item \PdQv{v2.0.0};
\item \NdPv{v2.0.0}.
\end{itemize}
Ogni membro del gruppo, in base al suo ruolo, provvederà ad incrementare i documenti.
\subsection{Discussione sull'allegato tecnico}
Sono stati presentati i vari diagrammi e i design pattern utilizzati per l'implementazione del modulo di back-end e di front-end, verificando che fosse tutto completo.
\subsection{Organizzazione per la presentazione dell'allegato tecnico}
Infine è stata effettuata una discussione su come presentare l'allegato tecnico al \CR , riscontrando come metodo più efficace l'utilizzo di alcune slide per la spiegazione dell'architettura.
\subsection{Valutazione per il miglioramento}
Il gruppo ha effettuato un'autovalutazione riguardo al lavoro svolto e alle problematiche riscontrate al fine di compiere un miglioramento nella prossima fase.
\setcounter{table}{-1}
{

\rowcolors{2}{azzurro2}{azzurro3}

\centering
\renewcommand{\arraystretch}{1.5}
\begin{longtable}{>{\centering}p{0.20\textwidth} >{}p{0.35\textwidth} >{}p{0.35\textwidth}}
\rowcolor{azzurro1}
\textbf{Problema} &
\textbf{Descrizione}&
\textbf{Soluzione}\\
\endhead

Covid-19 & Vista la situazione di emergenza sanitaria, gli incontri dovranno svolgersi ancora in modalità telematica. & Il gruppo continuerà ad usare Discord\ped{G} per comunicare. \\
Impegni personali & Alcuni componenti del gruppo hanno impegni personali e lavorativi. & Il gruppo gestirà il carico di lavoro tenendo in considerazione gli impegni di ciascuno. \\
Tecnologie & Alcune tecnologie si sono rivelate più complicate da apprendere. & I membri più esperti aiutano quelli meno nell'apprendimento. \\

		
\end{longtable}
}