\documentclass[12pt]{article}

\usepackage[a4paper, margin=1in, includefoot]{geometry}
\usepackage[utf8]{inputenc}
\usepackage{graphicx}
\usepackage{float}
\usepackage{fancyhdr}
\usepackage[italian]{babel}
\usepackage{ucs}
\usepackage[table]{xcolor} 
\usepackage{longtable}
\usepackage[colorlinks=true]{hyperref}
\usepackage{lastpage}
\usepackage{lipsum}
\usepackage{array}
\usepackage{graphicx}
\usepackage{amsmath}
\usepackage{amssymb}
\usepackage{../../../Utilita/latexSetup/styleTemplate}
%% INSERIRE QUI IL NOME DEL DOCUMENTO (INSERITE SEMPRE UNO SPAZIO ALLA FINE DEL NOME)
\newcommand{\doctitle}{Verbale interno \datared{} }

%% INSERIRE QUI LA VERSIONE ATTUALE DEL DOCUMENTO (INSERITE SEMPRE UNO SPAZIO ALLA FINE DELLA VERSIONE)
\newcommand{\versiondoc}{0.0.1 }

%%INSERITE QUI LA DATA DI COMPILAZIONE FINALE DEL DOCUMENTO
\newcommand{\datared}{2020-12-02}

%%INSERIRE QUI IL/I REDATTORI
\newcommand{\redattore}{\SB}

%%INSERIRE IL/I NOME DEI VERIFICATORI CHE HANNO VERIFICATO IL DOCUMENTO
\newcommand{\verificatori}{xxxxxx}

%%INSERIRE IL NOME DI CHI HA APPROVATO IL DOCUMENTO
\newcommand{\approvazione}{xxxxxx}

%%INSERIRE LA TIPOLOGIA DI USO DEL DOCUMENTO [Interno/Esterno]
\newcommand{\usodoc}{Interno}

%%INSERIRE LA LISTA DI DISTRIBUZIONE DEL DOCUMENTO
\newcommand{\listadistr}{
    \Omicron\\
    \emph{\VT}\\
    \emph{\CR}
}

%%INSERIRE IL SOMMARIO DEL DOCUMENTO
\newcommand{\testosommario}{Resoconto dell'incontro online effettuato dal gruppo \Omicron il giorno \datared}

%INSERIRE IL PATH RELATIVO ALL'IMMAGINE IN BASE ALLA CARTELLA DI DOVE CI SI TROVA
\newcommand{\relativePathImg}{../../../Utilita/img/}
%%Comandi particolari per i nomi delle figure. Chiamare come una funzione LaTeX
\newcommand{\Omicron}{\emph{Omicron }}
\newcommand{\respProg}{\emph{Responsabile di Progetto }}
\newcommand{\ammProg}{\emph{Amministratore di Progetto }}
\newcommand{\analProg}{\emph{Analista }}
\newcommand{\verifProg}{\emph{Verificatore }}
\newcommand{\programProg}{\emph{Programmatore }}
\newcommand{\progetProg}{\emph{Progettista }}
\newcommand{\proponProg}{\emph{Proponente }}
\newcommand{\commitProg}{\emph{Committente }}
\newcommand{\nameproject}{\emph{Dopo scelta del progetto}}

%committenti
\newcommand{\Committente}{\VT \newline \CR}
\newcommand{\VT}{Prof. Vardanega Tullio}
\newcommand{\CR}{Prof. Cardin Riccardo}


% proponenti
\newcommand{\Proponente}{Dopo scelta del progetto}
\newcommand{\ZD}{Dopo scelta del progetto}
\newcommand{\CT}{Dopo scelta del progetto}

% Omicron team
\newcommand{\MB}{Matthew Balzan}
\newcommand{\DF}{Vasile Tusa}
\newcommand{\FD}{Francesco Dallan}
\newcommand{\NM}{Niccolò Mantovani}
\newcommand{\SB}{Silvia Bazzeato}
\newcommand{\GB}{Gabriel Bizzo}
\newcommand{\MDI}{Marco Dello Iacovo}

% documenti
\newcommand{\SdF}{Studio di Fattibilità}
\newcommand{\SdFv}[1]{\textit{Studio di Fattibilità {#1}}}
\newcommand{\PdQ}{Piano di Qualifica}
\newcommand{\PdQv}[1]{\textit{Piano di Qualifica {#1}}}
\newcommand{\PdP}{Piano di Progetto}
\newcommand{\PdPv}[1]{\textit{Piano di Progetto {#1}}}
\newcommand{\NdP}{Norme di Progetto}
\newcommand{\NdPv}[1]{\textit{Norme di Progetto {#1}}}
\newcommand{\AdR}{Analisi dei Requisiti}
\newcommand{\AdRv}[1]{\textit{Analisi dei Requisiti {#1}}}
\newcommand{\Glossario}{Glossario}
\newcommand{\Glossariov}[1]{\textit{Glossario {#1}}}
\newcommand{\MM}{Manuale Manutentore}
\newcommand{\MMv}[1]{\textit{Manuale Manutentore {#1}}}
\newcommand{\MU}{Manuale Utente}
\newcommand{\MUv}[1]{\textit{Manuale Utente {#1}}}


%command HRule
\newcommand{\HRule}{\rule{\linewidth}{0.5mm}}



\begin{document}
	
	\begin{titlepage} 

\begin{center}
\includegraphics[scale=0.25]{\relativePathImg OmicronLogo} \\
\textit{omicronswe@gmail.com}\\[0.7cm]
\HRule \\[0.7cm] \textbf{\huge \doctitle}\\[0.4cm] \HRule \\[1.5cm] 
\par\end{center}{\huge \par}

\begin{center}
	\textbf{\Large Informazioni sul documento} \\[0.5cm]
	\begin{longtable}{ r | p{5cm} }
		\textbf{Progetto} & \nameproject \\
		\textbf{Versione documento} & \versiondoc \\
		\textbf{Data documento} & \datared \\
		\textbf{Redattori} & \parbox[t]{\textwidth}{\redattore} \\[0.5cm]
		\textbf{Verificatori} & \parbox[t]{\textwidth}{\verificatori} \\[0.5cm]
		\textbf{Approvazione} & \approvazione \\
		\textbf{Uso documento} & \usodoc \\
		\textbf{Lista distribuzione} & \parbox[t]{\textwidth}{\listadistr} \\
	\end{longtable}
\vspace{0.7cm}
\textbf{\Large Sommario} \\[0.4cm]
\testosommario
\end{center}%

\end{titlepage} 


	
	\section*{Registro delle modifiche}
\setcounter{table}{-1}
{

\rowcolors{2}{azzurro2}{azzurro3}

\centering
\renewcommand{\arraystretch}{1.5}
\begin{longtable}{c C{2.5cm} C{2.5cm} C{3.5cm} C{3.5cm}}
\rowcolor{azzurro1}
\textbf{Versione} &
\textbf{Data}&
\textbf{Autore}&
\textbf{Ruolo}&
\textbf{Descrizione}\\
\endhead

3.0.0 & 2021-04-17 & \GB & \respProg & Approvazione del documento. \\
2.1.0 & 2021-04-16 & \SB & \verifProg & Verifica complessiva del documento. \\
2.0.3 & 2021-03-16 & \MDI{}, \VAS & \analProg{    } \verifProg & Aggiornato tracciamento Use Case e riferimenti nei requisiti funzionali, verificato documento. \\
2.0.2 & 2021-03-14 & \MDI{}, \VAS & \analProg{    } \verifProg & Rimossi schemi generali, rinumerati Use Case, verificato documento. \\
2.0.1 & 2021-03-14 & \MB{}, \VAS & \analProg{    } \verifProg & Modificati UC8, rimossi UC6, UC9, UC12 e UC13, verificato documento. \\
2.0.0 & 2021-03-01 & \VAS & \respProg & Approvazione documento per RP. \\
1.1.0 & 2021-02-28 & \MDI & \verifProg & Verifica documento. \\
1.0.6 & 2021-02-25 & \FD & \analProg & Modificato UC3. \\
1.0.5 & 2021-02-25 & \SB & \analProg & Rimosso admin dagli attori primari. \\
1.0.4 & 2021-02-13 & \FD & \analProg & Separato UC12 in UC14, UC15, UC16, UC17 e UC18. Separato UC6 in UC6 e UC7. \\
1.0.3 & 2021-02-12 & \SB & \analProg & Rimossi UC10 e UC11.5. \\
1.0.2 & 2021-02-11 & \SB & \analProg & Rimossi UC13, UC14 e relativi requisiti. Modificato UC11. \\
1.0.1 & 2021-02-10 & \SB & \analProg & Aggiornato UC7. \\
1.0.0 & 2021-01-11 & \FD & \respProg & Approvazione documento per RR. \\
0.3.0 & 2021-01-08 & \SB & \verifProg & Verifica documento. \\
0.2.2 & 2021-01-07 & \MDI & \analProg & Correzione errori dopo verifica \\
0.2.1 & 2021-01-07 & \MB & \analProg & Correzione errori dopo verifica \\
0.2.0 & 2021-01-06 & \SB & \verifProg & Verifica intero documento. \\
0.1.4 & 2021-01-06 & \MB & \analProg & Stesura tracciamento (\S{5}) \\
0.1.3 & 2021-01-04 & \MDI & \analProg & Stesura UC3, UC4, UC5, UC7, UC12 \\
0.1.2 & 2021-01-04 & \MB & \analProg & Stesura UC1, UC2, UC6, UC8, UC13, UC14 \\
0.1.1 & 2021-01-02 & \GB & \analProg & Stesura UC9, UC10, UC11 \\
0.1.0 & 2020-12-28 & \NM & \verifProg & Verifica requisiti. \\
0.0.4 & 2020-12-26 & \MB & \analProg & Stesura requisiti (\S{4}) \\
0.0.3 & 2020-12-24 & \MB & \analProg & Aggiunta introduzione e struttura a \S{4} \\
0.0.2 & 2020-12-24 & \GB & \analProg & Stesura introduzione e descrizione generale. \\
0.0.1 & 2020-12-23 & \GB & \analProg & Creazione della bozza del documento. \\

		
\end{longtable}
}
	\newpage
	\tableofcontents
	\newpage
	
	\section{Informazioni generali}
	\begin{itemize}
		\item \textbf{Luogo:} Video chiamata tramite Zoom.
		\item \textbf{Data:} 2020-11-16.
		\item \textbf{Ora Inizio:} 16:30.
		\item \textbf{Ora fine:} 18:00.
		\item \textbf{Segretario} \MB.
		\item \textbf{Partecipanti:}
		\begin{itemize}
			\item \MB
			\item \DF
			\item \FD
			\item \NM
			\item \SB
			\item \GB
			\item \MDI
		\end{itemize}
		
	\end{itemize}
	
	\section{Ordine del giorno}	
	\section{Ordine del giorno}
\begin{itemize}
\item item
\end{itemize}	
	
	\newpage
	
	\section{Resoconto}
	\subsection{Nome del gruppo}
	È stato fatto un veloce brainstorming per la ricerca del nome del gruppo. Dopo alcune idee, siamo arrivati a due scelte:
	\begin{itemize}
	\item \textit{Omicron}
	\item \textit{Sputnik-14}
	\end{itemize}
	Successivamente, la maggioranza ha optato per il nome \textit{Omicron}.
	
	\subsection{Discussione capitolati}
	La maggior parte dell'incontro è stato occupato dalla discussione di ogni capitolato, per avere già sotto mano delle opinioni e delle preferenze. Sono risultati positivi 3 capitolati (in ordine di piacimento):
	\begin{itemize}
		\item Capitolato C1 - \textit{BlockCOVID} di \textit{Imola Informatica}
		\item Capitolato C2 - \textit{EmporioLambda} di \textit{Red Babel}
		\item Capitolato C7 - \textit{SSD} di \textit{Zextras}
	\end{itemize}
	È stato comunque dato il compito a tutti i membri del gruppo di analizzare attentamente tutti i capitolati, in modo da prendere una decisione più ufficiosa per il prossimo incontro.
	
	\subsection{Logo del gruppo}
	È stato assegnato a \NM{} il compito di provare alcuni design per il logo del gruppo, contenente la scritta \textit{Omicron}. In settimana verrà scelto definitivamente tramite i principali canali di comunicazione.
	
	\subsection{Scelta email del gruppo}
	Si è scelto l'indirizzo e-mail: \textit{omicronswe@gmail.com}.\\
	\MDI{} è stato incaricato di creare l'indirizzo e fornire le credenziali al resto dei componenti. È stato inoltre scelto \textit{Google} come provider, in modo da sfruttare il loro servizio di cartelle condivise \textit{Google Drive}, per pubblicare al suo interno tutti i documenti utili, visibili dall'intero gruppo.
	
	\subsection{Scelta canale di comunicazione}
	Oltre al canale \textit{Telegram}, creato dopo la formazione dei gruppi e utilizzato per le prime direttive (tra cui l'organizzazione di questo incontro), non è risultato abbastanza ricco di funzionalità per le comunicazione tra membri del gruppo durante lo svolgimento del progetto.\\
	Per questo motivo abbiamo cercato una soluzione più vicina alle nostre esigenze.\\
	Inizialmente, per la sua notorietà, è stato proposto \textit{Slack}; ma dopo una ricerca più dettagliata, si è notato che non supportava chiamate di gruppo (nella sua versione base), cosa che sarebbe molto utile da avere integrata ai canali di testo.\\
	Si ha quindi optato per \textit{Discord}, un software \textit{VOIP}, orientato alla collaborazione tra utenti, molto simile a \textit{Slack}. In aggiunta alle sue estensive funzionalità, diversi membri del gruppo risultavano avere già molta esperienza d'uso a riguardo, quindi la scelta è stata approvata facilmente.
	
	\section{Riepilogo delle decisioni}
	

{

\rowcolors{2}{azzurro2}{azzurro3}
\centering
\renewcommand{\arraystretch}{1.5}
\begin{longtable}{ >{\centering}p{0.20\textwidth} >{}p{0.70\textwidth}}
\rowcolor{azzurro1}
\textbf{Codice} &
\textbf{Decisione}\\
\endhead

VI\_2020-11-16.1 & Scelto il nome del gruppo: \textit{Omicron}\\
VI\_2020-11-16.2 & Le preferenze dei capitolati sono (in ordine):
\begin{itemize}
\item C1 - \textit{BlockCOVID}
\item C2 - \textit{EmporioLambda}
\item C7 - \textit{SSD}
\end{itemize}
La scelta definitiva verrà eseguita nel prossimo incontro.\\
VI\_2020-11-16.3 & Il logo del gruppo verrà ipotizzato da \NM{} in settimana. \\
VI\_2020-11-16.4 & Scelta la mail del gruppo:\textit{omicronswe@gmail.com}, creata da \MDI{}, impostando nello stesso momento la cartella \textit{Google Drive}.\\
VI\_2020-11-16.5 & Scelto il principale canale di comunicazione: \textit{Discord}\\
		
\end{longtable}
}
\end{document}