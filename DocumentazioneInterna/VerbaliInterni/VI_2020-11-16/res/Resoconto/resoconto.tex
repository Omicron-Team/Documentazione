\section{Resoconto}
	\subsection{Nome del gruppo}
	È stato fatto un veloce brainstorming\ped{G} per la ricerca del nome del gruppo. Dopo alcune idee, siamo arrivati a due scelte:
	\begin{itemize}
	\item \textit{Omicron}
	\item \textit{Sputnik-14}
	\end{itemize}
	Successivamente, la maggioranza ha optato per il nome \textit{Omicron}.
	
	\subsection{Discussione capitolati}
	La maggior parte dell'incontro è stato occupato dalla discussione di ogni capitolato, per avere già sotto mano delle opinioni e delle preferenze. Sono risultati positivi 3 capitolati (in ordine di piacimento):
	\begin{itemize}
		\item Capitolato C1 - \textit{BlockCOVID} di \textit{Imola Informatica}
		\item Capitolato C2 - \textit{EmporioLambda} di \textit{Red Babel}
		\item Capitolato C7 - \textit{SSD} di \textit{Zextras}
	\end{itemize}
	È stato comunque dato il compito a tutti i membri del gruppo di analizzare attentamente tutti i capitolati, in modo da prendere una decisione più ufficiosa per il prossimo incontro.
	
	\subsection{Logo del gruppo}
	È stato assegnato a \NM{} il compito di provare alcuni design per il logo del gruppo, contenente la scritta \textit{Omicron}. In settimana verrà scelto definitivamente tramite i principali canali di comunicazione.
	
	\subsection{Scelta email del gruppo}
	Si è scelto l'indirizzo e-mail: \textit{omicronswe@gmail.com}.\\
	\MDI{} è stato incaricato di creare l'indirizzo e fornire le credenziali al resto dei componenti. È stato inoltre scelto Google\ped{G} come provider\ped{G}, in modo da sfruttare il loro servizio di cartelle condivise Google Drive\ped{G}, per pubblicare al suo interno tutti i documenti utili, visibili dall'intero gruppo.
	
	\subsection{Scelta canale di comunicazione}
	Oltre al canale Telegram\ped{G}, creato dopo la formazione dei gruppi e utilizzato per le prime direttive (tra cui l'organizzazione di questo incontro), non è risultato abbastanza ricco di funzionalità per le comunicazione tra membri del gruppo durante lo svolgimento del progetto.\\
	Per questo motivo abbiamo cercato una soluzione più vicina alle nostre esigenze.\\
	Inizialmente, per la sua notorietà, è stato proposto Slack\ped{G}; ma dopo una ricerca più dettagliata, si è notato che non supportava chiamate di gruppo (nella sua versione base), cosa che sarebbe molto utile da avere integrata ai canali di testo.\\
	Si ha quindi optato per Discord\ped{G}, un software VOIP\ped{G}, orientato alla collaborazione tra utenti, molto simile a Slack\ped{G}. In aggiunta alle sue estensive funzionalità, diversi membri del gruppo risultavano avere già molta esperienza d'uso a riguardo, quindi la scelta è stata approvata facilmente.