\section{Resoconto}
\subsection{Discussione e suddivisione per \AdR}
Abbiamo discusso delle sigle da utilizzare per i casi d'uso all'interno dell'\AdR{}. Sono state riportate le regole nel documento di \NdP{} nella sezione \S{2.2.4.2}. \\
Si è inoltre decisa la piattaforma da usare per la rappresentazione dei diagrammi UML\ped{G}, cioè Draw.io\ped{G}.

\subsection{Finalizzazione dei ruoli}
Dopo un attento confronto si sono decisi i ruoli per la trascrizione dei documenti mancanti e, nel caso dei responsabili, dei documenti da approvare. Le assegnazioni sono state riportate nel documento interno per la gestione dei ruoli, presente nella piattaforma Google Drive.

\subsection{Domande generali riguardo i problemi riscontrati}
È stata riscontrata un'ambiguità riguardo al modo in cui doveva avvenire il cambiamento di versione di un documento a seguito di una verifica.\\ 
Infine si è deciso che data una versione X.Y.Z:
\begin{itemize}
\item se il documento è stato modificato, viene incrementato il valore di Z;
\item se il documento (o la parte di documento scritta al momento) viene verificato completamente, allora si incrementa il valore di Y e successivamente Z viene messo a 0;
\item se il documento viene approvato, viene incrementato il valore di X, mentre Y e Z vengono messi a 0.

\end{itemize}

