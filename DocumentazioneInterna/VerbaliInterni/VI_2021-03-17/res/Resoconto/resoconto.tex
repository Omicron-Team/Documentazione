\section{Resoconto}
\subsection{Discussione sulla scelta del design pattern da utilizzare per il modulo di front-end}
La video chiamata è iniziata con una discussione mirata ad analizzare l'esito dell'incontro effettuato con il proponente \Proponente{} il 2021-03-16. Successivamente a quella riunione ci siamo trovati a dover decidere tra il design pattern\ped{G} proposto dal gruppo, Model-View-ViewModel (MVVM), e i design pattern\ped{G} di React\ped{G} suggeriti dall'azienda.\\
Dopo un'attenta comparazione tra le due opzioni abbiamo deciso di seguire i consigli del proponente in quanto le motivazioni che ci sono state fornite sono risultate convincenti; infatti, mentre l'utilizzo del MVVM rischiava di risultare un tentativo di overfitting del design pattern nel contesto di sviluppo con il framework\ped{G} web Next.js\ped{G}, l'utilizzo delle best practices come la suddivisione dei componenti React\ped{G} in contenitori e presentazionali e l'isolamento verticale dei componenti ci permettono di sfruttare a pieno la semplicità e l'efficienza offerta dal framework\ped{G}.
\subsection{Discussione per la riorganizzazione del modulo di back-end}
Un'ulteriore osservazione sollevata durante l'incontro del 2021-03-16 da parte di \Proponente{} è la poca organizzazione del codice corrispondente al modulo di back-end\ped{G}, il quale risulta avere un'architettura ibrida tra layered e a microservizi. A questo proposito il gruppo ha deciso di realizzare un'architettura a strati in quanto le modifiche da effettuare per l'implementazione e l'integrazione dei microservizi nel sistema risultano molto complesse ed onerose. Inoltre, sempre sotto consiglio del proponente, è stato deciso di adottare il Domain-driven design (DDD) come approccio per l'organizzazione delle funzioni lambda in modo da porre il focus principale del progetto sulla logica del dominio.