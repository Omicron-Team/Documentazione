\section{Capitolato C7}

\subsection{Informazioni generali}
\begin{itemize}
\item \textbf{\emph{Nome:}} SSD - Soluzioni di sincronizzazione desktop.
\item \textbf{\commitProg:} Zextras.
\item \textbf{\proponProg:} \VT{} e \CR.
\end{itemize}

\subsection{Descrizione}
Questo capitolato pone come obbiettivo la creazione di un algoritmo solido ed efficiente in grado di garantire il salvataggio in cloud del lavoro e contemporaneamente la sincronizzazione su disco dei cambiamenti presenti nel cloud. Questo algoritmo dovrà poi interfacciarsi con il servizio Zextras Drive. É anche richiesto lo sviluppo di un interfaccia multipiattaforma per utilizzare l'algoritmo in tutti i principali sistemi operativi.

\subsection{Tecnologie coinvolte}
\begin{itemize}
\item \textbf{Qt:} framework\ped{G} consigliato da Zextras per lo sviluppo di programmi con interfaccia grafica.
\item \textbf{Python\ped{G}:} linguaggio di programmazione, consigliato da Zextras per lo sviluppo della business logic.
\item \textbf{Zextras Drive:} prodotto per la condivisione e l'archiviazione di file nel cloud.
\item \textbf{GraphQL:} linguaggio utilizzato per effettuare query\ped{G} alle API\ped{G} di Zextras Drive.
\end{itemize}

\subsection{Aspetti positivi}
\begin{itemize}
\item L'ambito trattato è di facile comprensione, dato che tutti i componenti del gruppo hanno già utilizzato servizi simili, come ad esempio Google Dive o Dropbox.
\item Il pattern MVC\ped{G} e il framework\ped{G} Qt sono stati già trattati e utilizzati nel corso di programmazione ad oggetti.
\end{itemize}

\subsection{Vincoli}
\begin{itemize}
\item L'algoritmo e l'interfaccia utente devono essere utilizzabili nei tre sistemi operativi principali (Windows, Mac, Linux), senza richiedere all'utente l'installazione di ulteriori prodotti.
\item É richiesta una forte distinzione tra interfaccia utente e business logic.
\item É richiesto l'uso del pattern MVC\ped{G} che permetta un cambiamento rapido della business logic o dell'interfaccia utente senza un eccessivo sviluppo aggiuntivo.
\end{itemize}

\subsection{Criticità e fattori di rischio}
\begin{itemize}
\item É necessario imparare come gestire la sincronizzazione locale dei dati su 3 sistemi operativi diversi.
\item Il linguaggio di programmazione Python\ped{G} consigliato da Zextras non è ben conosciuto all'interno del gruppo.
\item Sarà necessario familiarizzarsi con il software Zimbra e il linguaggio GraphQL richiesto per le chiamate alle API\ped{G}.
\item É necessario molto lavoro di ricerca per la creazione dell'algoritmo di sincronizzazione.
\item Le idee e le richieste di questo capitolato non sono risultate interessanti alla maggior parte dei componenti del gruppo.
\item L'apprendimento del funzionamento della suite Zimbra probabilmente non risulterà spendibile in un futuro contesto lavorativo.
\end{itemize}

\subsection{Conclusione}
La grande quantità di ricerca necessaria ad imparare a gestire la sincronizzazione locale in 3 sistemi operativi diversi, la poca spendibilità delle conoscenze acquisite in un eventuale contesto lavorativo futuro e lo scarso interesse da parte del gruppo hanno portato alla decisione di scartare questo progetto.
