\section{Capitolato C4}

\subsection{Informazioni generali}
\begin{itemize}
\item \textbf{Nome:} HD Viz - Visualizzazione di dati multidimensionali.
\item \textbf{\commitProg:} Zucchetti.
\item \textbf{\proponProg:} \VT{} e \CR.
\end{itemize}

\subsection{Descrizione}
Il capitolato C4 ha per oggetto l’affidamento della fornitura per la realizzazione di un'applicazione di visualizzazione di dati con molte dimensioni a supporto della fase esplorativa dell'analisi di dati, in modo da riusire poi a identificare possibili pattern o cluster.

\subsection{Tecnologie coinvolte}
\begin{itemize}
\item \textbf{HTML, CSS, Javascript} 
\item \textbf{D3.js:} Libreria javascript che permette di visualizzare dati in diverse modalità.
\item \textbf{SQL, NoSQL:} Database, necessari per memorizzare i dati da rappresentare.
\item \textbf{Matrici multidimensionali}
\item \textbf{Tomcat} o \textbf{Node.js} per la parte server di supporto alla presentazione nel browser e alle query al database.
\end{itemize}

\subsection{Aspetti positivi}
\begin{itemize}
\item La presentazione è stata molto chiara e l’azienda è risultata disponibile al chiarimento.
\item La presenza di vare immagini di esempio fornite dall'azienda ci permette di capire meglio il prodotto da sviluppare.
\item Molte delle tecnologie richieste sono già state trattate durante il nostro percorso di studi.
\item L’azienda Zucchetti è la prima software house in italia, con più di 40 anni di esperienza nel settore.
\end{itemize}

\subsection{Vincoli}
\begin{itemize}
\item Capacità di visualizzare dati di almeno 15 dimensioni.
\item I dati devono poter essere forniti al sistema di visualizzazione sia con query ad un database che da file in formato csv.
\item Uso della libreria D3.js.
\item Capacità di rappresentare le seguenti visualizzazioni:
\begin{itemize}
	\item Scatter plot matrix fino a 5 dimensioni.
	\item Force field.
	\item Heat map, con ordinazione obbligatoria dei punti per evidenziare i 'Cluster' nei dati.
	\item Proiezione lineare multi asse.
\end{itemize}
\end{itemize}

\subsection{Criticità e fattori di rischio}
\begin{itemize}
\item La presenza di una forte componente matematica non ha convinto alcuni membri del gruppo.
\item Rispetto ad altri capitolati il gruppo percepisce le tecnologie trattate come poco interessanti.
\item La libreria D3.js è usata solo in alcuni ambiti lavorativi di nicchia, quindi saperla usare non tornerà utile ai membri del gruppo in fututo.
\item Sono presenti dei vincoli molto stretti, sopratutto nella parte che riguarda il front-end.
\end{itemize}

\subsection{Conclusione}
L'obbiettivo e le tecnologie trattate non sono risultate stimolanti e non hanno suscitato l'interesse di tutti i membri del gruppo. La libreria D3.js, pur essendo interessante non risulta spendibile nella maggior parte degli ambiti lavorativi. Si è quindi deciso di scartare il capitolato.