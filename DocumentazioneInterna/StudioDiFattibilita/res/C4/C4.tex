\section{Capitolato C4}

\subsection{Informazioni generali}
\begin{itemize}
\item \textbf{\emph{Nome:}} HD Viz - Visualizzazione di dati multidimensionali;
\item \textbf{\commitProg:} Zucchetti;
\item \textbf{\proponProg:} \VT{} e \CR.
\end{itemize}

\subsection{Descrizione}
Il capitolato C4 ha per oggetto l’affidamento della fornitura per la realizzazione di un'applicazione di visualizzazione di dati con molte dimensioni a supporto della fase esplorativa dell'analisi di dati, in modo da riuscire poi a identificare possibili pattern o cluster\ped{G}.

\subsection{Tecnologie coinvolte}
\begin{itemize}
\item \textbf{HTML\ped{G}, CSS\ped{G}, Javascript\ped{G}};
\item \textbf{D3.js\ped{G}:} Libreria javascript usata per la visualizzazione dei grafici;
\item \textbf{SQL\ped{G}, NoSQL\ped{G}:} Database, necessari per memorizzare i dati da rappresentare;
\item \textbf{Matrici multidimensionali};
\item \textbf{Tomcat\ped{G}} o \textbf{Node.js\ped{G}} per la parte server di supporto alla presentazione nel browser e alle query\ped{G} al database.
\end{itemize}

\subsection{Vincoli}
\begin{itemize}
\item Capacità di visualizzare dati di almeno 15 dimensioni;
\item I dati devono poter essere forniti al sistema di visualizzazione sia con query\ped{G} ad un database che da file in formato csv;
\item Uso della libreria D3.js\ped{G};
\item Capacità di rappresentare le seguenti visualizzazioni:
\begin{itemize}
	\item Scatter plot matrix fino a 5 dimensioni;
	\item Force field;
	\item Heat map\ped{G}, con ordinazione obbligatoria dei punti per evidenziare i cluster\ped{G} nei dati;
	\item Proiezione lineare multi asse.
\end{itemize}
\end{itemize}

\subsection{Aspetti positivi}
\begin{itemize}
\item La presentazione è stata molto chiara e l’azienda è risultata disponibile al chiarimento;
\item La presenza di vare immagini di esempio fornite dall'azienda ci permette di capire meglio il prodotto da sviluppare;
\item Molte delle tecnologie richieste sono già state trattate durante il nostro percorso di studi;
\item L’azienda Zucchetti è la prima software house in italia, con più di 40 anni di esperienza nel settore.
\end{itemize}

\subsection{Aspetti critici}
\begin{itemize}
\item La presenza di una forte componente matematica non ha convinto alcuni membri del gruppo;
\item Rispetto ad altri capitolati il gruppo percepisce le tecnologie trattate come poco interessanti;
\item La libreria D3.js\ped{G} è usata solo in alcuni ambiti lavorativi di nicchia, quindi saperla usare non tornerà utile ai membri del gruppo in futuro;
\item Sono presenti dei vincoli molto stretti, sopratutto nella parte che riguarda il front-end\ped{G}.
\end{itemize}

\subsection{Conclusione}
L'obbiettivo e le tecnologie trattate non sono risultate stimolanti e non hanno suscitato l'interesse di tutti i membri del gruppo. La libreria D3.js\ped{G}, pur essendo interessante non risulta spendibile nella maggior parte degli ambiti lavorativi. Si è quindi deciso di scartare il capitolato.
