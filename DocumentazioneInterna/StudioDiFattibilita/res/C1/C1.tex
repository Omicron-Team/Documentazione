\section{Capitolato C1}

\subsection{Informazioni generali}{

% sezione testuale o a lista puntata?
Il capitolato si chiama "\textit{BlockCOVID}", è stato proposto dall'azienda \textit{Imola Informatica} e i committenti sono il Prof. Vardanega Tullio e il Prof. Cardin Riccardo.
}

\subsection{Descrizione}{
L'obiettivo del capitolato in questione è quello di creare una piattaforma che consenta il tracciamento immutabile e certificato tramite tag RFID delle presenze in tempo reale alle postazioni di lavoro di un laboratorio informatico. Attraverso un'applicazione dedicata deve essere possibile, in base alla tipologia di utente (dipendente/studente o addetto alla pulizia), visualizzare le postazioni libere, prenotare e/o visualizzare le informazioni relative allo stato di sanificazione di una postazione, segnalare la pulizia di una postazione o dell'intera stanza. Il server, comunicando con l'applicazione, deve offrire ad un amministratore del sistema la gestione e il monitoraggio delle singole stanze e degli utenti che usufruiscono delle postazioni di lavoro attraverso un'interfaccia con procedura di autenticazione.

}

\subsection{Tecnologie coinvolte}{
\begin{itemize}
\item \textit{Java} (versione 8+), \textit{Python} o \textit{Node.js} per lo sviluppo del server back-end;
\item Protocolli asincroni per le comunicazioni app mobile-server;
\item \textit{Blockchain}: sistema in cui i dati vengono raggruppati in blocchi concatenati tra loro tramite funzioni matematiche per salvare in modo immutabile i dati relativi alle sanificazioni;
\item \textit{Kubernetes}, \textit{Openshift} o \textit{Rancher} per il rilascio delle componenti
del server nonché per la gestione della scalabilità orizzontale;
\item API \textit{Rest} attraverso le quali sia possibile utilizzare l'applicativo;

% tenere questo punto?
\item \textit{Java} (attraverso il kit di sviluppo software Android) o \textit{Swift} per lo sviluppo dell'applicazione mobile rispettivamente Android o iOS.
\end{itemize}
}

\subsection{Vincoli}{
\begin{itemize}
\item Scalabilità orizzontale del server in base alla modifica del numero di utilizzatori;
\item Creazione di un'applicazione Android/iOS con relativa interfaccia grafica;
\item Utilizzo dei \textit{tag RFID} per il tracciamento in tempo reale;

% tenere questi punti?
\item Implementazione di un'architettura Client-Server;
\item Report di test effettuati relativamente all'ottimizzazione dell'applicazione rispetto al consumo della
batteria dei cellulari.
\end{itemize}
}

\subsection{Aspetti positivi}{
\begin{itemize}
\item L'azienda ha 35 anni di esperienza nel settore informatico;
\item Possibilità di sviluppare un applicativo che possa effettivamente venire utilizzato in ambito aziendale;
\item Libertà nella scelta del linguaggio di programmazione da utilizzare;
\item I linguaggi di programmazione consigliati sono molto quotati anche al di fuori dell'ambito universitario;
\item Spiegazione chiara, da parte dell’azienda, sui compiti che l'applicativo deve svolgere e gli obiettivi che devono essere raggiunti;
\item Vengono messi a disposizione dall'azienda dei server dedicati per effettuare test sui nostri componenti applicativi sviluppati, con relativa spiegazione sull'uso di \textit{Docker} (attraverso un seminario tecnologico).
\end{itemize}
}

\subsection{Aspetti critici}{
\begin{itemize}
\item Applicativo dipendente dall'utente, infatti vi è la necessità che gli impiegati abbiano sempre a disposizione il cellulare;
\item Dubbi sulla gestione dello stato di occupazione della postazione di lavoro nel caso l'impiegato si sposti da quest'ultima;
\item Possibile difficoltà nell’apprendimento di alcune complesse tecnologie coinvolte per i membri del gruppo che non le hanno mai utilizzate precedentemente.
\end{itemize}
}

\subsection{Conclusione}{
Il capitolato proposto da \textit{Imola Informatica} ha suscitato un discreto interesse da parte del gruppo, sia per la disponibilità dimostrata da parte dell'azienda, sia per la libertà nell'approccio alle tecnologie coinvolte e la loro effettiva utilità a livello curricolare. Tuttavia, a causa del molto interesse da parte di un buon numero di altri gruppi, si è deciso di optare per un capitolato differente.
}

