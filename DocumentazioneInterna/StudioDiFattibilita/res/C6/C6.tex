\section{Capitolato C6}

\subsection{Informazioni generali}
\begin{itemize}
\item \textbf{\emph{Nome:}} RG - Realtime Gaming Platform;
\item \textbf{\commitProg:} Zero12;
\item \textbf{\proponProg:} \VT{} e \CR.
\end{itemize}

\subsection{Descrizione}
Questo capitolato pone come obbiettivo la realizzazione di un videogioco a scorrimento verticale fruibile da device mobili, con la possibilità di giocare in multiplayer\ped{G} in real-time\ped{G}. Il focus principale è proprio la componente di gioco tra più giocatori, che dovrà eseguire su server cloud-based\ped{G} utilizzando tecnologie Amazon Web Services\ped{G}.

\subsection{Tecnologie coinvolte}
\begin{itemize}
\item \textbf{AWS Gamelift:} servizio AWS\ped{G} specifico per supportare giochi multiplayer\ped{G};
\item \textbf{AWS Appsync:} servizio AWS\ped{G} che permette lo sviluppo rapito di API GraphQL;
\item \textbf{DynamoDB:} database SQL\ped{G} dalle alte performance, ideale per la conservazione di tag o altre informazioni di supporto;
\item \textbf{Node.js\ped{G}:} software open-source\ped{G} orientato agli eventi che permette l'esecuzione di codice JavaScript\ped{G}, necessario se verrà scelto un servizio che richiede l'esecuzione di codice;
\item \textbf{Swift\ped{G}:} linguaggio di programmazione usato per sviluppare app per dispositivi iOS\ped{G}/MacOS;
\item \textbf{Kotlin:} linguaggio di programmazione usato per sviluppare app per dispositivi Android\ped{G}.
\end{itemize}

\subsection{Vincoli}
\begin{itemize}
\item Produrre un videogioco a scorrimento verticale per dispositivi mobile;
\item Permettere sessioni multiplayer\ped{G} con 2-6 giocatori;
\item Poter vedere, durante le partite multiplayer\ped{G}, i movimenti dei rivali e, allo stesso tempo, sincronizzare eventuali nemici e powerup\ped{G} in modo che tutti i giocatori abbiano la medesima sfida;
\item Avere una modalità singleplayer\ped{G} con livelli infiniti a difficoltà crescente, che terminerà quando il giocatore finirà le vite o non avrà raccolto in tempo i powerup\ped{G};
\item Utilizzare tecnologie AWS\ped{G} per il supporto al multiplayer\ped{G};
\item Produrre un analisi preliminare sulle tecnologie AWS\ped{G} per capire quali servizi si adattano meglio ad un gioco multiplayer\ped{G} real-time\ped{G};
\item Una struttura server scalabile\ped{G}.
\end{itemize}

\subsection{Aspetti positivi}
\begin{itemize}
\item L'azienda risulta disponibile a fornire chiarificazioni e consigli per la scelta dei servizi cloud e per l'impostazione della struttura server;
\item Richiesto l'apprendimento di tecnologie cloud AWS\ped{G}, molto popolari in ambito lavorativo.
\end{itemize}

\subsection{Aspetti critici}
\begin{itemize}
\item La creazione di un videogioco non interessa ai membri del gruppo, che in futuro non hanno intenzione di lavorare nell'ambito dello sviluppo di videogiochi;
\item É necessario imparare diverse nuove tecnologie e linguaggi di programmazione, cosa che ci porterebbe via una notevole quantità di tempo.
\end{itemize}

\subsection{Conclusione}
Pur essendo un progetto interessante e che ci permetterebbe di approcciarci all'ambito molto richiesto delle tecnologie cloud AWS\ped{G}, imparare a programmare un videogioco per dispositivi mobile non è qualcosa che interessa al gruppo. Abbiamo quindi deciso di scartare questo capitolato.