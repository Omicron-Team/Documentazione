\section{Capitolato C3}

\subsection{Informazioni generali}{
\begin{itemize}
\item \textbf{\emph{Nome :}} GDP - Gathering Detection Platform.
\item \textbf{\commitProg:} Sync Lab.
\item \textbf{\proponProg:} \VT{} e \CR.
\end{itemize}
}

\subsection{Descrizione}{
Il capitolato presentato si pone come obiettivo finale la realizzazione di una piattaforma che, attraverso l'elaborazione real-time di una varietà di sorgenti di dati anche molto eterogenee tra loro (quali ad esempio videocamere, flussi di prenotazioni Uber\ped{G}, tabelle degli orari di autobus/treni, etc.), permetta di ottenere indicazioni riguardo a potenziali assembramenti e di poter quindi, in tale ottica, fornire un supporto alle decisioni per l'ottimizzazione del traffico.
Gli utenti devono poter interagire con la piattaforma tramite una applicazione Web\ped{G} che riesca a rappresentare la situazione globale dei flussi.
}

\subsection{Tecnologie coinvolte}{
\begin{itemize}
\item \textbf{Java}\ped{G} e \textbf{Angular}\ped{G} per lo sviluppo delle parti di back-end\ped{G} e di front-end\ped{G} della componente relativa all'applicazione Web\ped{G} del sistema; 
\item Framework\ped{G} \textbf{Leaflet}\ped{G} per la gestione delle mappe (heat-map\ped{G});
\item Protocolli asincroni\ped{G} per la comunicazione tra diversi componenti del software;
\item Design pattern \textbf{Publisher/Subscriber}\ped{G} e adozione del protocollo \textbf{MQTT}\ped{G} per la comunicazione asincrona\ped{G} tra i vari componenti del sistema;
\item \textbf{Apache Kafka}\ped{G} per lo stream\ped{G} processing dei dati in tempo reale;
\item Database per la collezione dei dati risultanti dalla fase di elaborazione;
\item \textbf{Scikit-learn}\ped{G}, \textbf{Numpy}\ped{G} o \textbf{TensorFlow}\ped{G} per la fase di esplorazione e analisi dei dati utilizzando tecniche di machine learning\ped{G}.
\end{itemize}
}

\subsection{Vincoli}{
\begin{itemize}
\item Implementazione di una soluzione di tipo '\textbf{Predictive Analytics}\ped{G}' per prevedere e limitare un eccessivo afflusso di persone in un dato luogo; 
\item Creazione di una Web\ped{G} dashboard\ped{G} che rappresenti tramite heat-map\ped{G} la situazione totale dei flussi di persone. Inoltre tale afflusso deve poter essere valutato in tempo reale, previsto in intervalli temporali futuri e raccolto/storicizzato nel tempo.
\end{itemize}
}

\subsection{Aspetti positivi}{
\begin{itemize}
\item Utilizzo di tecnologie molto attuali e interessanti, che sono in continua crescita;
\item L'azienda sembra molto professionale e disponibile, mettendo inoltre a disposizione personale qualificato attraverso canali di comunicazione diretta e/o asincrona (ad esempio attraverso chat, Discord\ped{G}, etc.) e server dedicati per installare i nostri componenti applicativi;
\item Molta libertà nella scelta dei linguaggi di programmazione e delle tecnologie da utilizzare, avendo a disposizione molti consigli basati sull'esperienza dell'azienda.
\end{itemize}
}

\subsection{Aspetti critici}{
\begin{itemize}
\item La piattaforma richiesta in questo capitolato risulta molto complessa da implementare, specialmente per le tecnologie coinvolte, col rischio di impiegare molto tempo nella fase di auto-apprendimento e di realizzazione;
\item Per una buona realizzazione del prodotto risultano necessarie conoscenze importanti riguardanti l'esplorazione e la manipolazione dei dati, i quali sono presenti in forme anche molto eterogenee tra loro.
\end{itemize}
}

\subsection{Conclusione}{
Questo particolare capitolato, proposto da \textit{Sync Lab}, ha suscitato un bel dibattito nel gruppo: i molti aspetti positivi rilevati analizzando il capitolato, come l'interesse per le tecnologie coinvolte e la disponibilità dell'azienda, hanno dovuto scontrarsi tuttavia con l'elevata complessità richiesta nell'implementazione della piattaforma e infine, aggiungendo l'interesse riscontrato da molti gruppi per questo capitolato, è stato deciso di scartarlo e optare per un'altra soluzione.
}

