
\section{Introduzione}
\subsection{Scopo del documento}
Questo documento ha lo scopo di indicare le norme da seguire al fine del corretto svolgimento del progetto per tutto il suo ciclo di vita. 
Ogni membro del gruppo \Omicron{} deve obbligatoriamente seguire le norme descritte in questo documento per garantire la massima coerenza del materiale prodotto.

\subsection{Scopo generale del prodotto}
Lo scopo del capitolato C2 è la realizzazione di una generica piattaforma di e-commerce\ped{G}, chiamata \nameproject{}, basata su tecnologia Serverless\ped{G} da vendere a mercanti. \nameproject{} deve essere distribuibile usando l'account AWS\ped{G} del mercante con una configurazione manuale minima. Deve inoltre essere prodotta una piattaforma dimostrativa per l'utilizzo di \nameproject{}.   

\subsection{Glossario}
Viene fornito un glossario il quale scopo è quello di evitare ambiguità nel linguaggio utilizzato fornendo una definizione ai vari termini usati nella documentazione. Il glossario può essere trovato nell'apposito documento \Glossario{}.pdf.

\subsection{Riferimenti}
\subsubsection{Normativi}
\begin{itemize}
\item Standard ISO/IEC 12207:1995:\\ \url{https://www.math.unipd.it/~tullio/IS-1/2009/Approfondimenti/ISO_12207-1995.pdf}
\item Capitolato d'appalto C2 - \nameproject{}: piattaforma di e-commerce in stile Serverless:\\ \url{https://www.math.unipd.it/~tullio/IS-1/2020/Progetto/C2.pdf}
\end{itemize}

\subsubsection{Informativi}
da aggiungere
\begin{itemize}
	\item Standard \textbf{ISO 9001: } \url{https://it.wikipedia.org/wiki/Norme_della_serie_ISO_9000#ISO_9001};
	\item Standard \textbf{ISO/IEC 12207:2008: } \url{https://www.edatlas.it/scarica/TPSIT_4/Capitolo3/ContenutiDigitaliIntegrativi/1NormaISOIEC12207.pdf}.
\end{itemize}