
\section{Introduzione}
\subsection{Scopo del documento}
Questo documento ha lo scopo di indicare le norme da seguire al fine del corretto svolgimento del progetto per tutto il suo ciclo di vita. 
Ogni membro del gruppo \Omicron{} deve obbligatoriamente seguire le norme descritte in questo documento per garantire la massima coerenza del materiale prodotto.

\subsection{Scopo generale del prodotto}
Lo scopo del capitolato C2 è la realizzazione di una generica piattaforma di e-commerce\ped{G}, chiamata \nameproject{}, basata su tecnologia Serverless\ped{G} da vendere a venditori. \nameproject{} deve essere distribuibile usando l'account AWS\ped{G} del venditore con una configurazione manuale minima. Deve inoltre essere prodotta una piattaforma dimostrativa per l'utilizzo di \nameproject{}.   

\subsection{Glossario}
Viene fornito un glossario il cui scopo è quello di evitare ambiguità nel linguaggio utilizzato fornendo una definizione ai vari termini usati nella documentazione. Il glossario può essere trovato nell'apposito documento \Glossario{}.pdf.

\subsection{Riferimenti}
\subsubsection{Normativi}
\begin{itemize}
\item Capitolato d'appalto C2 - \nameproject{}: piattaforma di e-commerce in stile Serverless:\\ \url{https://www.math.unipd.it/~tullio/IS-1/2020/Progetto/C2.pdf}.
\end{itemize}

\subsubsection{Informativi}
\begin{itemize}
\item Standard \textbf{ISO/IEC 12207:1995}:\\ \url{https://www.math.unipd.it/~tullio/IS-1/2009/Approfondimenti/ISO_12207-1995.pdf};\\
\url{https://en.wikipedia.org/wiki/ISO/IEC_12207};
\item Standard \textbf{ISO/IEC 9126: }\\ \url{https://it.wikipedia.org/wiki/ISO/IEC_9126};
\item Standard \textbf{ISO/IEC 15504: }\\ \url{https://en.wikipedia.org/wiki/ISO/IEC_15504};
\item Standard \textbf{ISO 8601: }\\ \url{https://it.wikipedia.org/wiki/ISO_8601};
\item \textbf{Approccio incrementale}:\\ \url{https://it.wikipedia.org/wiki/Modello_incrementale};
\item \textbf{Camel Case} e \textbf{Pascal Case}:\\ \url{https://it.wikipedia.org/wiki/Notazione_a_cammello};
\item Spunti per stile di codifica:\\ \url{https://github.com/airbnb/javascript};
\item Informazioni su \textbf{Design Pattern}:\\ \url{https://it.wikipedia.org/wiki/Design_pattern};
\item Documentazione \textbf{GitHub}:\\ \url{https://docs.github.com/en};
\item Documentazione \textbf{Git}:\\ \url{https://www.atlassian.com/git};
\item Documentazione \textbf{\LaTeX}:\\ \url{https://www.latex-project.org/help/documentation/};
\item \textbf{Diagrammi dei casi d'uso} - Materiale didattico del corso di Ingegneria del Software:\\ \url{https://www.math.unipd.it/~rcardin/swea/2021/Diagrammi%20Use%20Case_4x4.pdf};
\item \textbf{Diagrammi delle classi} - Materiale didattico del corso di Ingegneria del Software:\\ \url{https://www.math.unipd.it/~rcardin/swea/2021/Diagrammi%20delle%20Classi_4x4.pdf};
\item \textbf{Diagrammi dei package} - Materiale didattico del corso di Ingegneria del Software:\\ \url{https://www.math.unipd.it/~rcardin/swea/2021/Diagrammi%20dei%20Package_4x4.pdf};
\item \textbf{Diagrammi di seguenza} - Materiale didattico del corso di Ingegneria del Software:\\ \url{https://www.math.unipd.it/~rcardin/swea/2021/Diagrammi%20di%20Sequenza_4x4.pdf};
\end{itemize}
