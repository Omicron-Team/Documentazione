\subsubsection{Elementi grafici}
\myparagraph{Tabelle}
Ogni tabella deve essere centrata orizzontalmente nella pagina e deve presentare sotto di essa la propria didascalia, ad eccezione del \textit{Registro delle modifiche} di ciascun documento e il \textit{Riepilogo delle decisioni} dei verbali. \\
La didascalia deve contenere il numero della sezione a cui si riferisce e, in modo incrementale, dal numero progressivo delle tabelle di quella sezione.
\begin{itemize}
	\item \textbf{X.Y}: rappresenta la sezione;
	\item \textbf{Z}: rappresenta il numero progressivo della tabella all'interno della sezione.
\end{itemize}

\myparagraph{Immagini}
Le immagini devo essere centrate orizzontalmente ed essere nettamente separate dai paragrafi. Devono, inoltre, essere accompagnate da una didascalia analoga a quella descritta per le tabelle.

\myparagraph{Diagrammi UML}
Tutti i diagrammi UML\ped{G} sono inseriti nei documenti come immagini.