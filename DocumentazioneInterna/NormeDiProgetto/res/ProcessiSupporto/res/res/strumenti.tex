\subsubsection{Strumenti}
\myparagraph{\LaTeX}
Per la stesura di tutti i documenti si deve usare \LaTeX{} per le varie possibilità che offre:
\begin{itemize}
\item creazione di documenti formali e facilmente divisibili in sezioni;
\item separazione della formattazione dal contenuto che permette di avere un unico file per lo stile applicabile a tutti i documenti;
\item personalizzazione dei documenti grazie alle numerose librerie.
\end{itemize}

\myparagraph{\TeX maker}
Per la stesura del codice \LaTeX{} è stato utilizzato l'editor \TeX maker, utile in quanto open source. Fornisce, inoltre, suggerimenti per completare i comandi \LaTeX{} e dizionari per il controllo dell'ortografia in varie lingue, tra cui l'italiano.
\begin{center}
\url{https://www.xm1math.net/texmaker/}
\end{center}

\myparagraph{Draw.io}
Per la produzione di diagrammi UML\ped{G} viene utilizzato Draw.io in quanto offre numerose agevolazioni per la produzione veloce dei diagrammi e risulta semplice da utilizzare.
\begin{center}
\url{https://www.draw.io/}
\end{center}