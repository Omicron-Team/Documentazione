\subsubsection{Metriche}
Per la valutazione del prodotto del processo di Codifica, il gruppo adotta le seguenti metriche:
\begin{itemize}
	\item \textbf{Code coverage}: per capire la percentuale di righe di codice che vengono percorse dai test durante la loro esecuzione si userà la seguente formula:
\[CC=\frac{linee \ di \ codice \ percorse \ dai \ test}{linee \ di \ codice \ totali}*100;\]
	\item \textbf{Densità degli errori}: la capacità di resistenza agli errori viene espressa tramite l'utilizzo di una percentuale. Per calcolarla si userà la seguente formula:
\[DE=\frac{numero \ errori}{numero \ test \ effettuati}*100;\]



	\item \textbf{Copertura dei requisiti da parte dei test}: si esprime in percentuale e si intende la capacità da parte dei test di controllare i requisiti implementati, e deve assumere un valore uguale a 100. Si utilizza la seguente formula:
	\[CRT=\frac{numero \ requisiti \ testati}{numero \ requisiti \ implementati}*100;\]
	\item \textbf{Percentuale test passati}: espresso in percentuale, indica la quantità dei test superati rispetto al numero di test totali, si auspica a una percentuale prossima al 100. Si calcola con la seguente formula:
	\[PTP=\frac{numero \ test \ passati}{numero \ test \ totali}*100.\]
\end{itemize}
 