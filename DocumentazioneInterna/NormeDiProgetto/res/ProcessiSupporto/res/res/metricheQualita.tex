\subsection{Metriche di qualità}
\subsubsection{Metriche per la documentazione}
\myparagraph{Indice di Gulpease}
Per riuscire a comprendere il livello di leggibilità del testo verrà usato l'indice di Gulpease la cui formula è la seguente:
\begin{center}
\[IG=89+\frac{300*(numero \ frasi)-10*(numero \ lettere)}{(numero \ parole)}\]\\
\end{center}

\myparagraph{Correttezza ortografica}
Al fine di assicurarsi che non vi siano errori ortografici o grammaticali verrà usato un indice "CO" il quale indica quante incorrettezze(grammaticali o ortografiche) sono state trovate nel testo.

\subsubsection{Metriche per la verifica}
\myparagraph{Code coverage}
Per capire la percentuale di righe di codice che vengono percorse dai test durante la loro esecuzione si userà la seguente formula:
\[CC=\frac{linee \ di \ codice \ percorse \ dai \ test}{linee \ di \ codice \ totali}*100\]\\

 