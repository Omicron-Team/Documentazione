\subsubsection{Norme tipografiche}
Ogni documento deve rispettare le norme tipografiche descritte in questa sezione, in modo da rendere uniforme la stesura.

\myparagraph{Convenzioni sui nomi dei file}
I nomi delle cartelle e dei file (estensione esclusa) utilizzano la convenzione "CamelCase"\ped{G} con le seguenti regole:
\begin{itemize}
	\item il nome del file è composto da tutte le parole che compongono il nome del documento;
	\item le preposizioni non si omettono;
	\item i nomi dei file composti da più parole non avranno caratteri speciali a separarle ma saranno tutte unite con le iniziali di tutte le parole maiuscole.
\end{itemize}
Esempi \textbf{corretti} sono:
\begin{itemize}
	\item NormeDiProgetto;
	\item AnalisiDeiRequisiti;
\end{itemize}
Esempi \textbf{non corretti} sono:
\begin{itemize}
	\item Norme\_Prog (uso di un carattere speciale come separatore e troncamento di una parte del nome del documento);
	\item AnalisiRequisiti(omissione della preposizione "Dei");
		\item pianodiprogetto (omissione delle prime lettere maiuscole di ogni parola).
\end{itemize}

\myparagraph{Glossario}
Ogni parola contenuta nel \Glossario{} deve essere marcata con una \textbf{G} maiuscola a pedice in ogni sua occorrenza.

\myparagraph{Collegamenti a pagine internet}
Per indicare la presenza di un link esterno, questo viene contrassegnato con l'uso del colore blu.

\myparagraph{Stile del testo}
\begin{itemize}
	\item \textbf{Grassetto}: viene applicato agli elementi di un elenco puntato che riassumono il contenuto di tale voce e ai titoli;
	\item \textbf{Corsivo}: vengono scritte in corsivo le seguenti occorrenze:
	\begin{itemize}
		\item riferimenti a sezioni di altri documenti;
		\item il nome del gruppo \Omicron ;
		\item il nome del progetto \nameproject ;
		\item il nome del documento, in caso si voglia fare riferimento alla versione;
		\item i ruoli del progetto.
	\end{itemize}
	\item \textbf{Maiuscolo}: riservato agli acronimi e alle iniziali di nomi e titoli.
\end{itemize}

\myparagraph{Formati comuni}
Seguendo lo standard ISO 8601, i formati per la data e l'ora sono i seguenti:
\begin{itemize}
	\item \textbf{Data:}
		\begin{center}
			\textbf{YYYY-MM-DD}
		\end{center}
		\begin{itemize}
			\item \textbf{YYY}: rappresenta l'anno, utilizzando quattro cifre;
			\item \textbf{MM}: rappresenta il mese, utilizzando due cifre;
			\item \textbf{GG}: rappresenta il giorno, utilizzando due cifre.
		\end{itemize}
	\item \textbf{Ora:}
		\begin{center}
			\textbf{HH:MM}
		\end{center}
		\begin{itemize}
			\item \textbf{HH}: rappresenta l'ora e può assumere valori da 0 a 23;
			\item \textbf{MM}: rappresenta i minuti e può assumere valori da 0 a 59.
		\end{itemize}
\end{itemize}

\myparagraph{Sigle}
\textcolor{red}{DA FARE?????}