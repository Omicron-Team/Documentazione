\subsection{Gestione della qualità}\label{3.3}

\subsubsection{Scopo}\label{3.3.1}
L'obbiettivo è quello di garantire qualità sul prodotto e sui servizi finali e quindi soddisfare bisogni e richieste dei proponenti.

\subsubsection{Aspettative}
Garantire:
\begin{itemize}
	\item soddisfazione finale da parte del proponente;
	\item qualità oggettiva nel prodotto finale;
	\item qualità nella gestione di processi.
\end{itemize}

\subsubsection{Descrizione}
Questa sezione mira a descrivere come ottenere una qualità elevata sul software e sulla documentazione finale.
La gestione della qualità è approfondita interamente nel \PdQv{4.0.0}. Si rimanda a tale documento per una visione più specifica.

\subsubsection{Attività}
Le attività principali per garantire una soddisfacente qualità sono:
\begin{itemize}
	\item \textbf{Pianificazione:} definire degli obbiettivi di qualità da raggiungere e il modo in cui devono essere raggiunti;
	\item \textbf{Valutazione:} tramite standard\ped{G} prefissati, valutare l'andamento dell'obbiettivo di qualità, sia in corso d'opera che a conclusione del processo;
	\item \textbf{Miglioramento:} adattare le strategie per il raggiungimento degli obbiettivi di qualità in base ai risultati ottenuti tramite la valutazione.
\end{itemize}

\subsubsection{Strumenti}
Gli strumenti utilizzati sono:
\begin{itemize}
	\item standard \textbf{ISO/IEC 9126} per la gestione della qualità del software;
	\item standard \textbf{ISO/IEC 15504} per la gestione della qualità dei processi;
	\item metriche decise durante la fase di \textit{Pianificazione} e \textit{Miglioramento} delle attività.
\end{itemize}

\subsubsection{Denominazione delle metriche}
Il formato scelto dal gruppo per denominare le metriche\ped{G} è
\begin{center}
	\textbf{M[Categoria][Numero]}
\end{center}
dove:
\begin{itemize}
\item \textbf{M} indica che ci si sta riferendo ad una metrica;
\item \textbf{Categoria} specifica a quale categoria la metrica appartiene tra:
	\begin{itemize}
	\item \textbf{PC} per i processi;
	\item \textbf{PD} per i prodotti;
	\item \textbf{TS} per i test.
	\end{itemize}
\item \textbf{Numero} indica l'identificativo numerico ed inizia da 1.
\end{itemize}