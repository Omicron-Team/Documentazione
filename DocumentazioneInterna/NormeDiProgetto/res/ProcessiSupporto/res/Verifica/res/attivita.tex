\subsubsection{Attività}
Le attività coinvolte riguardano l'analisi, la quale si divide in \textit{statica} e \textit{dinamica}.

\myparagraph{Analisi statica}
L'analisi statica effettua controlli su documenti e codice per garantire correttezza, conformità con regole preventivamente decise e coesione delle componenti nel prodotto finale.
L'analisi statica si divide in:
\begin{itemize}
	\item \textbf{metodi manuali di lettura:} attuati da persone all'interno del team:
	\begin{itemize}
		\item \textbf{Walkthrough:} i \textit{Verificatori} analizzano i prodotti interamente per cercare difetti o elementi non conformi;
		\item \textbf{Inspection:} i \textit{Verificatori}, attraverso liste di controllo, analizzano parti specifiche del prodotto.
	\end{itemize}
	\item \textbf{metodi formali:} attuati da macchine.
\end{itemize}
\mbox{}\\
Di seguito le liste di controllo attuate per la verifica dei documenti:
\begin{itemize}
	\item \textbf{Sintassi:} la frase o il periodo risulta troppo complesso;
	\item \textbf{Grammatica:} presenza di errori grammaticali;
	\item \textbf{Formato data:} la data deve essere nel formato YYYY-MM-DD;
	\item \textbf{Elenchi:} ogni voce deve terminare con ``;", tranne l'ultima che termina con ``.";
	\item \textbf{Tabelle e figure:} se nel documento sono presenti tabelle o figure, queste devono essere riportate in una \textit{lista di tabelle} o \textit{lista di figure}. Fanno eccezione tabelle o figure facenti parte di una sezione a se stante relativa alla tabella o alla figura.
\end{itemize}



\myparagraph{Analisi dinamica}\label{3.4.4.2}
L'analisi dinamica consiste nell'eseguire il software per rilevarne problemi o malfunzionamenti.

\mbox{}\\
\textbf{Test} \mbox{}\\ \mbox{}\\
L'analisi dinamica viene eseguita tramite dei \textit{test} che vanno ad analizzare singole o molteplici linee di codice. Ogni test è caratterizzato dai seguenti elementi:
\begin{itemize}
	\item \textbf{ambiente: } il sistema hardware e software su cui verrà eseguito il test;
	\item \textbf{stato iniziale:} definizione dello stato iniziale da cui verrà eseguito il test;
	\item \textbf{input: } input inserito;
	\item \textbf{output: } output inserito;
	\item \textbf{istruzioni aggiuntive: } istruzioni di supporto per specificare meglio come eseguire un test e come interpretare i risultati.
\end{itemize}
I test, per essere considerati accettabili, devono definire gli elementi sopra citati ed inoltre devono:
\begin{itemize}
	\item essere ripetibili;
	\item fornire risultati sull'esito dell'esecuzione e segnalare eventuali effetti indesiderati, possibilmente in forma di \textit{file log}\ped{G}. 
\end{itemize}
\mbox{}\\
\textbf{Codifica dei Test} \mbox{}\\ \mbox{}\\
I test si dividono in:
\begin{enumerate}
	\item \textbf{Test di unità:} il test si concentra su singole unità software. Dati determinati input viene individuato un output atteso. Dopo l'esecuzione dell'unità, si controlla se l'output ottenuto è equivalente all'output atteso.\\
Tale tipologia di test verrà indicata con:
\begin{center}
	\textbf{TU[id]}
\end{center}
dove \textit{id} rappresenta un’unità.
	\item \textbf{Test di integrazione:} dopo aver superato i test di unità, si procede con l'assemblare le varie unità fra di loro e creare agglomerati più ampi. Superare tali test significa che le relazioni fra le varie unità sono accettabili.\\
Tale tipologia di test verrà indicata con:
\begin{center}
	\textbf{TI[id]}
\end{center}
dove \textit{id} rappresenta un componente.
	\item \textbf{Test di sistema:} una volta integrati tutti i componenti, si effettuano test sull'applicazione nella sua interezza. Questi test si concentrano sulle interazioni fra le parti, sul comportamento generale del sistema e sulla copertura di tutte le funzionalità. In tale fase ci si assicura che l'applicativo rispetti tutte le specifiche definite nell \AdRv{2.0.0}.\\
Tale tipologia di test verrà indicata con:
\begin{center}
	\textbf{TS[Importanza][Tipologia][Id]}
\end{center}
dove:
\begin{itemize}
	\item \textbf{Importanza:} indica l'importanza del requisito, che si divide in:
	\begin{itemize}
		\item \textbf{1:} requisiti obbligatori;
		\item \textbf{2:} requisiti desiderabili;
		\item \textbf{3:} requisiti facoltativi.
	\end{itemize}
	\item \textbf{Tipologia:} indica il tipo di requisito:
	\begin{itemize}
		\item \textbf{F:} requisito funzionale;
		\item \textbf{V:} requisito di vincolo;
		\item \textbf{P:} requisito prestazionale;
		\item \textbf{Q:} requisito di qualità.
\end{itemize}
	\item \textbf{Id:} rappresenta l'identificativo di una funzionalità		 
\end{itemize}
	\item \textbf{Test di regressione:} si attuano di seguito a modifiche del sistema e consiste nella riesecuzione dei test esistenti descritti precedentemente. Garantiscono che le nuove modifiche apportate non intacchino le funzionalità precedenti, evitando appunto una \textit{regressione}.
	\item \textbf{Test di accettazione o test di collaudo:} simile al test di sistema, ma con la differenza che viene eseguito con la collaborazione dei committenti. In particolare verifica il soddisfacimento del cliente. Il superamento di tale test sancisce la possibilità del rilascio del prodotto.\\
	Tale tipologia di test verrà indicata con:
\begin{center}
	\textbf{TA[Importanza][Tipologia][Id]}
\end{center}
dove \textit{Importanza, Tipologia ed Id} sono descritti come nei \textbf{test di sistema}.

	
\end{enumerate}
