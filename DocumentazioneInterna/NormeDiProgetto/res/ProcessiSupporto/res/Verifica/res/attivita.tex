\subsubsection{Attività}
Le attività coinvolte riguardano l'analisi, la quale si divide in \textit{statica} e \textit{dinamica}.

\myparagraph{Analisi statica}
L'analisi statica effettua controlli su documenti e codice per garantire correttezza, conformità con regole preventivamente decise e coesione delle componenti nel prodotto finale.
L'analisi statica si divide in:
\begin{itemize}
	\item \textbf{metodi manuali di lettura:} attuati da persone all'interno del team:
	\begin{itemize}
		\item \textbf{Walkthrough:} i \textit{Verificatori} analizzano i prodotti interamente per cercare difetti o elementi non conformi;
		\item \textbf{Inspection:} i \textit{Verificatori}, attraverso liste di controllo, analizzano parti specifiche del prodotto.
	\end{itemize}
	\item \textbf{metodi formali:} attuati da macchine.
\end{itemize}

Di seguito le liste di controllo attuate per la verifica del documenti:
\begin{itemize}
	\item \textbf{Sintassi:} la frase o il periodo risulta troppo complesso;
	\item \textbf{Grammatica:} presenza di errori grammaticali;
	\item \textbf{Formato data:} la data deve essere nel formato YYYY-MM-DD;
	\item \textbf{Elenchi:} ogni voce deve terminare con ``;", tranne l'ultima che termina con ``.";
	\item \textbf{Tabelle e figure:} se nel documento sono presenti tabelle o figure, queste devono essere riportate in una \textit{lista di tabelle} o \textit{lista di figure}. Fanno eccezione tabelle o figure facenti parte di una sezione a se stante relativa alla tabella o alla figura.
\end{itemize}



\myparagraph{Analisi dinamica}