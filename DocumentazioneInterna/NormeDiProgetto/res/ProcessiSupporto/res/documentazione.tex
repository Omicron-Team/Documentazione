\subsection{Documentazione}\label{3.1}

\subsubsection{Scopo}\label{3.1.1}
Questa sezione si pone l'obbiettivo di definire gli standard per la stesura dei documenti volti a rappresentare processi svolti durante il ciclo di vita del software. Tali documenti sono reperibili nella repository\ped{G} all'indirizzo: \url{https://github.com/Omicron-Team/Documentazione}.

\subsubsection{Aspettative}
Il processo in questione si pone di:
\begin{itemize}
\item individuare una struttura comune riguardo la documentazione proposta durante il ciclo di vita del software;
\item eligere una serie di norme da attuare per una realizzazione coerente dei documenti ufficiali.
\end{itemize}

\subsubsection{Descrizione}
Questa sezione contiene le norme che devono essere attuate per una corretta stesura, verifica e approvazione della documentazione ufficiale.

\subsubsection{Ciclo di vita del documento}
Ogni documento è caratterizzato dalle seguenti fasi durante il suo ciclo di vita:
\begin{enumerate}
	\item \textbf{Creazione:} il documento viene creato rispettando le norme definite. Il documento è strutturato a partire da un template\ped{G} situato nella cartella \textit{Utilita/latexSetup} nella repository\ped{G} citata nella sezione §\ref{3.1.1};
	\item \textbf{Strutturazione:} il documento viene correlato con un registro delle modifiche. In aggiunta, nei verbali, si aggiunge un riepilogo delle decisioni;
	\item \textbf{Stesura:} il documento viene scritto in modo incrementale da più componenti del gruppo;
	\item \textbf{Revisione:} ogni voce del documento è soggetta a verifiche volte a correggere e migliorare il documento. La verifica deve essere svolta da uno o più membri del gruppo. I verificatori di una particolare sezione devono essere diversi dalla persona che ha redatto la medesima sezione;
	\item \textbf{Approvazione:} il \respProg\ped{G} stabilisce la validità del documento. Se il documento è considerato valido, ne consegue il rilascio.
\end{enumerate}