\subsection{Documentazione}\label{3.1}

\subsubsection{Scopo}\label{3.1.1}
Questa sezione si pone l'obbiettivo di definire gli standard\ped{G} per la stesura dei documenti volti a rappresentare processi svolti durante il ciclo di vita del software. Tali documenti sono reperibili nella repository\ped{G} all'indirizzo: \url{https://github.com/Omicron-Team/Documentazione}.

\subsubsection{Aspettative}
Il processo in questione si pone di:
\begin{itemize}
\item individuare una struttura comune riguardo la documentazione proposta durante il ciclo di vita del software;
\item eligere una serie di norme da attuare per una realizzazione coerente dei documenti ufficiali.
\end{itemize}

\subsubsection{Descrizione}
Questa sezione contiene le norme che devono essere attuate per una corretta stesura, verifica e approvazione della documentazione ufficiale.

\subsubsection{Ciclo di vita del documento}
Ogni documento è caratterizzato dalle seguenti fasi durante il suo ciclo di vita:
\begin{enumerate}
	\item \textbf{Creazione:} il documento viene creato rispettando le norme definite. Il documento è strutturato a partire da un template\ped{G} situato nella cartella \textit{Utilita/latexSetup} nella repository\ped{G} citata nella sezione §\ref{3.1.1};
	\item \textbf{Strutturazione:} il documento viene correlato con un registro delle modifiche. In aggiunta, nei verbali, si aggiunge un riepilogo delle decisioni;
	\item \textbf{Stesura:} il documento viene scritto in modo incrementale da più componenti del gruppo;
	\item \textbf{Revisione:} ogni voce del documento è soggetta a verifiche volte a correggere e migliorare il documento. La verifica deve essere svolta da uno o più membri del gruppo. I verificatori di una particolare sezione devono essere diversi dalla persona che ha redatto la medesima sezione;
	\item \textbf{Approvazione:} il \respProg{} stabilisce la validità del documento. Se il documento è considerato valido, ne consegue il rilascio.
\end{enumerate}

\subsubsection{Template}
Il gruppo ha creato un template\ped{G} \LaTeX{} per rendere più uniforme la formattazione dei documenti e per facilitare la stesura di essi. Lo scopo è quello di permettere, a chi redige il documento, di adottare automaticamente le direttive previste dalle \NdPv{} e, in caso di modifiche ad esse, di agevolarne la procedura di adeguamento. \\
Tale template\ped{G} estende la documentclass\ped{G} article\ped{G} definendo:
\begin{itemize}
	\item comandi per estendere e personalizzare le normali capacità di sezionamento fornite da \LaTeX ;
	\item dimensioni delle pagine, del testo e relative spaziature;
	\item comandi personalizzati per semplificare e velocizzare la stesura tramite l'inserimento di nomi di ruoli, attività di progetto, nome del gruppo, nomi dei componenti e dei professori ed altri termini ritenuti significativi. 
\end{itemize}

\subsubsection{Struttura dei documenti}
Il file principale, che verrà rinominato con il nome del documento, raccoglie nel suo corpo le sezioni di cui è composto tramite comandi di input. Il preambolo contiene invece:
\begin{itemize}
	\item il package\ped{G} \texttt{styleTemplate.sty}, creato dal team, contiene le caratteristiche personalizzate del template\ped{G} come, ad esempio, i colori delle tabelle, la posizione degli oggetti nell'intestazione e nel piè di pagina;
	\item \texttt{package.tex}, un file contenente tutti i package\ped{G} forniti da \LaTeX che il team ha ritenuto necessari;
	\item \texttt{generalCommand.tex}, un file con i comandi generali creati dal team come, ad esempio, i nomi dei componenti, i vari documenti, i ruoli e possibili nuovi comandi utili a tutti i documenti;
	\item \texttt{command.tex} contiene tutti i nuovi comandi creati per la stesura del frontespizio di ogni singolo documento.
\end{itemize}

	\myparagraph{Prima pagina}
		La struttura della prima pagina, il frontespizio, è la seguente:
		\begin{itemize}
			\item \textbf{Logo del gruppo}: logo di \Omicron visibile come primo elemento centrato orizzontalmente in alto;
			\item \textbf{Recapito}: indirizzo di posta elettronica del gruppo, posizionato centralmente sotto al logo;
			\item \textbf{Titolo}: nome del documento posizionato centralmente in grassetto;
			\item \textbf{Tabella}: sotto al titolo del documento, centrale, è presente la tabella che contiene le seguenti informazioni:
			\begin{itemize}
				\item \textbf{Progetto}: nome del progetto;
				\item \textbf{Versione}: versione del documento;
				\item \textbf{Data documento}: Ultima data di modifica/verifica/approvazione;
				\item \textbf{Redattori}: nome e cognome dei membri del gruppo incaricati della redazione del documento;
				\item \textbf{Verificatori}: nome e cognome dei membri del gruppo incaricati della verifica del documento;
				\item \textbf{Approvazione}: nome e cognome dei membri del gruppo incaricati dell'approvazione del documento;
				\item \textbf{Uso}: tipo d'uso che può essere "interno" o "esterno";
				\item \textbf{Lista distribuzione}: i destinatari del documento.
			\end{itemize}
			\item \textbf{Sommario}: descrizione sintetica relativa al documento, centrale, posta sotto la tabella descrittiva.
		\end{itemize}
		
	\myparagraph{Registro delle modifiche}
	A seguito della prima pagina è sempre presente una tabella riassuntiva della cronologia delle versioni del documento. \\
	Nella tabella ogni riga corrisponde ad una modifica apportata, mentre le colonne indicano:
	\begin{itemize}
		\item \textbf{Versione}: indica il numero di versione secondo le norme descritte nella sezione §3.2.4.1;
		\item \textbf{Data}: data della modifica nel formato descritto nella sezione §3.1.7.4;
		\item \textbf{Autore}: nominativo di chi ha apportato modifiche, verificato o approvato il documento;
		\item \textbf{Ruolo}: ruolo dell'autore al momento della modifica;
		\item \textbf{Descrizione}: breve descrizione delle modifiche apportate.
	\end{itemize}
	
	\myparagraph{Indice}
	Tutti i documenti devono contenere l'indice che ha lo scopo di dare una visione macroscopica del contenuto del documento. La struttura è gerarchica e rispetta la numerazione delle sezioni e sottosezioni del documento. \\
	Possono essere presenti fino ad un massimo di tre indici differenti: per le sezioni del documento, per le immagini e per le tabelle se presenti, ad esclusione della tabella raffigurante il \textit{Registro delle modifiche}.
	
	\myparagraph{Contenuto principale}
	La struttura e presentazione del contenuto è comune a tutti i documenti. In particolare, sono presenti un'intestazione ed un piè di pagina così composti:
	\begin{itemize}
		\item \textbf{Intestazione}:
		\begin{itemize}
			\item logo del gruppo a sinistra;
			\item nome del documento e nome del gruppo a destra.
		\end{itemize}
		\item \textbf{Piè di pagina}:
		\begin{itemize}
			\item pagina corrente e numero totale delle pagine centrali.
		\end{itemize}
	\end{itemize}
	
	
	\myparagraph{Verbali}
	I verbali vengono redatti dal/i soggetto/i incaricato/i alla loro stesura in occasione di incontri tra i componenti del gruppo con o senza la presenza di esterni.  Questi documenti prevedono un'unica stesura in quanto una possibile modifica significherebbe un cambio di decisioni prese. \\
	I verbali seguono la struttura principale degli altri documenti ma il loro contenuto principale è così suddiviso:
	\begin{enumerate}
		\item \textbf{Informazioni generali} contenenti:
		\begin{itemize}
			\item \textbf{Luogo}: luogo dove si è svolto l'incontro;
			\item \textbf{Data}: data dell'incontro, in formato come descritto nella sezione §3.1.7.5;
			\item \textbf{Ora inizio}: l'ora di inizio dell'incontro;
			\item \textbf{Ora fine}: l'ora di fine dell'incontro;
			\item \textbf{Segretario}: nome e cognome del componente incaricato di prendere appunti durante l'incontro e di redigere il verbale alla fine di esso;
			\item \textbf{Partecipanti}: l'elenco dei membri del gruppo presente all'incontro e, se presenti, i nominativi delle persone esterne al team che hanno partecipato.
		\end{itemize}
		\item \textbf{Ordine del giorno}: un elenco puntato sintetico con la descrizione degli argomenti principali trattati durante l'incontro;
		\item \textbf{Resoconto}: ogni voce dell'elenco dell'\textit{Ordine del giorno} viene qui descritto in modo più approfondito;
		\item \textbf{Riepilogo delle decisioni}: un riepilogo dei tracciamenti in forma tabellare che elenca le decisioni prese durante l'incontro. Ognuna di esse è caratterizzata da un codice e da una descrizione.
	\end{enumerate}
	Ogni verbale dovrà essere denominato nel seguente formato:
	\begin{center}
	\textbf{VTipologia\_YYYY-MM-DD}
	\end{center}
	dove per "Tipologia" si intende il tipo di verbale:
	\begin{itemize}
		\item \textbf{I}: in caso di verbale interno, quindi un documento concentrato sul riassunto dell'incontro avvenuto tra i soli membri del gruppo;
		\item \textbf{E}: se si tratta di verbale esterno, quindi un documento concentrato sulla trattazione di argomenti con partecipanti esterni al team, in particolare domande e risposte riguardanti il progetto.
	\end{itemize}		
	Il formalismo utilizzato per il \textit{Riepilogo delle decisioni} è invece il seguente: 
	\begin{center}
	\textbf{VTipologia\_YYYY-MM-DD.X}
	\end{center}
	dove:
	\begin{itemize}
		\item per "Tipologia" si intende il tipo di verbale, come sopra;
		\item "X", invece, si intende il numero della decisione presa dal gruppo, partendo da 1 in modo incrementale.
	\end{itemize}
	\myparagraph{Note a piè di pagina}
	In caso di presenza di note da rendere esplicite, esse vanno indicate nella pagina corrente, in basso a sinistra ed ognuna deve riportare un numero e una descrizione.

\subsubsection{Norme tipografiche}
Ogni documento deve rispettare le norme tipografiche descritte in questa sezione, in modo da rendere uniforme la stesura.

\myparagraph{Convenzioni sui nomi dei file}
I nomi delle cartelle e dei file (estensione esclusa) utilizzano la convenzione "CamelCase"\ped{G} con le seguenti regole:
\begin{itemize}
	\item le preposizioni non si omettono;
	\item i nomi dei file composti da più parole non avranno caratteri speciali a separarle ma saranno tutte unite con le iniziali di tutte le parole maiuscole.
\end{itemize}

\myparagraph{Glossario}

\subsubsection{Elementi grafici}
\myparagraph{Tabelle}
Ogni tabella deve essere centrata orizzontalmente nella pagina e deve presentare sotto di essa la propria didascalia, ad eccezione del \textit{Registro delle modifiche} di ciascun documento e il \textit{Riepilogo delle decisioni} dei verbali. \\
La didascalia deve contenere il numero della sezione a cui si riferisce e, in modo incrementale, dal numero progressivo delle tabelle di quella sezione.
\begin{itemize}
	\item \textbf{X.Y}: rappresenta la sezione;
	\item \textbf{Z}: rappresenta il numero progressivo della tabella all'interno della sezione.
\end{itemize}

\myparagraph{Immagini}
Le immagini devo essere centrate orizzontalmente ed essere nettamente separate dai paragrafi. Devono, inoltre, essere accompagnate da una didascalia analoga a quella descritta per le tabelle.

\myparagraph{Diagrammi UML}
Tutti i diagrammi UML\ped{G} sono inseriti nei documenti come immagini.

\subsubsection{Strumenti}

\myparagraph{Verifica ortografica}
La verifica ortografica avviene tramite lo strumento integrato in \TeX maker, il quale sottolinea in rosso parole errate secondo la lingua italiana.

\myparagraph{Analisi documentazione}
L'analisi dinamica(§3.4.4.2.) della documentazione avviene tramite l'utilizzo delle \textit{GitHub Actions}\ped{G}. Il codice \LaTeX{} prodotto viene compilato ad ogni \textit{push} o \textit{pull request}\ped{G} verso qualsiasi branch\ped{G} restituendo errori di compilazione se presenti. Inoltre, se la compilazione ha successo, durante una \textit{push} o \textit{pull request}\ped{G} verso il branch\ped{G} \textbf{main} o \textbf{develop} verrà creato un artifacts\ped{G} della durata di 30 giorni contenete i documenti in formato \textit{PDF} creati fino ad ora.\\
Tale procedura serve per garantire dei documenti indipendenti dall'ambiente da cui è stato prodotto il codice \LaTeX, utilizzando un metodo di compilazione univoco. L'artifacts\ped{G} prodotto invece, semplifica l'analisi statica(§3.4.4.1.) dei documenti.

\subsubsection{Metriche}
Per la valutazione del prodotto del processo di documentazione, il gruppo ha deciso di adottare le seguenti metriche:
\begin{itemize}
	\item \textbf{Indice di Gulpease}: per riuscire a comprendere il livello di leggibilità del testo verrà usato l'indice di Gulpease la cui formula è la seguente:
\begin{center}
\[IG=89+\frac{300*(numero \ frasi)-10*(numero \ lettere)}{(numero \ parole)}\]
\end{center}
	\item \textbf{Correttezza ortografica}: al fine di assicurarsi che non vi siano errori ortografici o grammaticali verrà usato un indice "CO" il quale indica quante incorrettezze (grammaticali o ortografiche) sono state trovate nel testo.
\end{itemize}


 
