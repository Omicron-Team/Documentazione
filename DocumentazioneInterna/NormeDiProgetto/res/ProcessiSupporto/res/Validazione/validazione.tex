\subsection{Validazione}\label{3.5}

\subsubsection{Scopo}\label{3.5.1}
Validare un prodotto significa controllare se soddisfa il compito per cui è stato creato. Dopo la validazione, il prodotto soddisfa i requisiti e i bisogni del committente.

\subsubsection{Aspettative}
Le aspettative su questo processo sono:
\begin{itemize}
	\item accettare e approvare il documento;
	\item rigettare il documento, esplicitando i motivi per garantire un miglioramento del prodotto.
\end{itemize}

\subsubsection{Descrizione}
La validazione comporta la presa in esame di un prodotto dopo la sua fase di \textit{verifica}. Il prodotto, se validato, sancisce il soddisfacimento dei bisogni del committente. Sarà compito del \respProg{} validare il prodotto.

\subsubsection{Attività}
Per validare un prodotto si deve:
\begin{enumerate}
	\item identificare il prodotto da validare;
	\item identificare le procedure di validazione, che devono essere riutilizzabili;
	\item applicare le procedure decise;
	\item valutare se il riscontro rispetta le aspettative.
\end{enumerate}

\subsubsection{Strumenti}

\myparagraph{Verifica ortografica}
La verifica ortografica avviene tramite lo strumento integrato in \TeX maker, il quale sottolinea in rosso parole errate secondo la lingua italiana.

\myparagraph{Analisi documentazione}
L'analisi dinamica(§3.4.4.2.) della documentazione avviene tramite l'utilizzo delle \textit{GitHub Actions}\ped{G}. Il codice \LaTeX{} prodotto viene compilato ad ogni \textit{push} o \textit{pull request}\ped{G} verso qualsiasi branch\ped{G} restituendo errori di compilazione se presenti. Inoltre, se la compilazione ha successo, durante una \textit{push} o \textit{pull request}\ped{G} verso il branch\ped{G} \textbf{main} o \textbf{develop} verrà creato un artifacts\ped{G} della durata di 30 giorni contenete i documenti in formato \textit{PDF} creati fino ad ora.\\
Tale procedura serve per garantire dei documenti indipendenti dall'ambiente da cui è stato prodotto il codice \LaTeX, utilizzando un metodo di compilazione univoco. L'artifacts\ped{G} prodotto invece, semplifica l'analisi statica(§3.4.4.1.) dei documenti.
