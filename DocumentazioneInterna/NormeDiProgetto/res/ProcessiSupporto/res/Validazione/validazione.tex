\subsection{Validazione}\label{3.5}

\subsubsection{Scopo}\label{3.5.1}
Validare un prodotto significa controllare se soddisfa il compito per cui è stato creato. Dopo la validazione, il prodotto soddisfa i requisiti e i bisogni del committente.

\subsubsection{Aspettative}
Le aspettative su questo processo sono:
\begin{itemize}
	\item accettare e approvare il documento;
	\item rigettare il documento, esplicitando i motivi per garantire un miglioramento del prodotto.
\end{itemize}

\subsubsection{Descrizione}
La validazione comporta la presa in esame di un prodotto dopo la sua fase di \textit{verifica}. Il prodotto, se validato, sancisce il soddisfacimento dei bisogni del committente. Sarà compito del \respProg\ped{G} validare il prodotto.

\subsubsection{Attività}
Per validare un prodotto si deve:
\begin{enumerate}
	\item identificare il prodotto da validare;
	\item identificare le procedure di validazione, che devono essere riutilizzabili;
	\item applicare le procedure decise;
	\item valutare se il riscontro rispetta le aspettative.
\end{enumerate}

\subsubsection{Strumenti}
\myparagraph{\LaTeX}
Per la stesura di tutti i documenti si deve usare \LaTeX{} per le varie possibilità che offre:
\begin{itemize}
\item creazione di documenti formali e facilmente divisibili in sezioni;
\item separazione della formattazione dal contenuto che permette di avere un unico file per lo stile applicabile a tutti i documenti;
\item personalizzazione dei documenti grazie alle numerose librerie.
\end{itemize}

\myparagraph{\TeX maker}
Per la stesura del codice \LaTeX{} deve essere utilizzato l'editor \TeX maker, utile in quanto open source. Fornisce, inoltre, suggerimenti per completare i comandi \LaTeX{} e dizionari per il controllo dell'ortografia in varie lingue, tra cui l'italiano.
\begin{center}
\url{https://www.xm1math.net/texmaker/}
\end{center}

\myparagraph{Draw.io}
Per la produzione di diagrammi UML\ped{G} deve essere utilizzato Draw.io in quanto offre numerose agevolazioni per la produzione veloce dei diagrammi e risulta semplice da utilizzare.
\begin{center}
\url{https://www.draw.io/}
\end{center}
