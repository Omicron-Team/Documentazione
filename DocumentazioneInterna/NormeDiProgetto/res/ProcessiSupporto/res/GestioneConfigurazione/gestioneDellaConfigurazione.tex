\subsection{Gestione della configurazione}\label{3.2}

\subsubsection{Scopo}\label{3.2.1}
Lo scopo del processo di configurazione è quello di dirigere in modo ordinato i documenti creati e il software. Ogni elemento configurato è sottoposto a caratteristiche ben definite e normate:
\begin{itemize}
 \item collocazione;
 \item stato;
 \item denominazione.
\end{itemize}
Ogni elemento configurato è modificato sempre secondo procedure normate e $versionato_G$ in una $repository_G$

\subsubsection{Aspettative}
Il processo in questione si pone di:
\begin{itemize}
\item uniformare la produzione di codice software e documentazione;
\item identificare gli strumenti di supporto per la gestione della configurazione del ciclo di vita del software e della documentazione;
\item gestire il $versionamento_G$ di codice e documentazione.
\end{itemize}


\subsubsection{Descrizione}
Questa sezione descrive il tipo di configurazione necessaria per la creazione di documenti, codice software e $versionamento_G$.

\subsubsection{Versionamento}
\paragraph{Codice di versione del documento}