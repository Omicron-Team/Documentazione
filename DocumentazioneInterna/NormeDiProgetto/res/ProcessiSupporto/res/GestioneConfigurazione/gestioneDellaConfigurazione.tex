\subsection{Gestione della configurazione}\label{3.2}

\subsubsection{Scopo}\label{3.2.1}
Lo scopo del processo di configurazione è quello di dirigere in modo ordinato i documenti creati e il software. Ogni elemento configurato è sottoposto a caratteristiche ben definite e normate:
\begin{itemize}
 \item collocazione;
 \item stato;
 \item denominazione.
\end{itemize}
Ogni elemento configurato è modificato sempre secondo procedure normate e versionato\ped{G} in una repository\ped{G}.

\subsubsection{Aspettative}
Il processo in questione si pone di:
\begin{itemize}
\item uniformare la produzione di codice software e documentazione;
\item identificare gli strumenti di supporto per la gestione della configurazione del ciclo di vita del software e della documentazione;
\item gestire il versionamento\ped{G} di codice e documentazione.
\end{itemize}


\subsubsection{Descrizione}
Questa sezione descrive il tipo di configurazione necessaria per la creazione di documenti, codice software e versionamento\ped{G}.

\subsubsection{Versionamento}
\myparagraph{Codice di versione del documento}
Ogni documento ha una storia la quale è ricostruibile attraverso le sue versioni. Il numero di versione è così formato:
\begin{center}
\textbf{X.Y.Z}
\end{center}
dove
\begin{itemize}
 \item \textbf{X:} indica una versione stabile del documento
 	\begin{itemize}
 		\item parte da 0;
 		\item viene incrementata da parte del \respProg{} dopo il completamento della fase di approvazione del documento;
 	\end{itemize}
 \item \textbf{Y:}  indica una versione parzialmente stabile del documento
 	\begin{itemize}
 		\item parte da 0;
 		\item viene incrementata da parte del \verifProg{} dopo il completamento della fase di verifica del documento;
 		\item viene azzerato ogni qual volta avviene un incremento di X.
 	\end{itemize}
 \item \textbf{Z:} indica una versione instabile del documento
 \begin{itemize}
 		\item parte da 0;
 		\item viene incrementata da parte del \emph{Redattore} dopo il completamento di una modifica al documento;
 		\item viene azzerato ogni qual volta avviene un incremento di Y.
 	\end{itemize}
\end{itemize}

\myparagraph{Tecnologie adottate}
Per il codice software e la creazione dei documenti che sono soggetti a versionamento\ped{G} si è deciso di utilizzare il sistema di versionamento\ped{G} distribuito Git\ped{G}, utilizzando il servizio GitHub\ped{G} per ospitare le repository\ped{G} remote.


\myparagraph{Repository}
I membri del team \Omicron{} possono interagire con il VCS\ped{G} sia attraverso la linea di comando oppure tramite l'utilizzo di GitHub\ped{G}. Si è deciso di creare tre repository\ped{G} separate, una distinta per la documentazione alla quale se ne aggiungono altre 2 per l'implementazione, in particolare viene tenuto separato FrontEnd\ped{G} da BackEnd\ped{G}.
\begin{itemize}
 \item \textbf{Documentazione:} repository\ped{G} per il versionamento\ped{G} della documentazione, reperibile all'indirizzo \url{https://github.com/Omicron-Team/Documentazione};
  \item \textbf{FrontEnd\ped{G}:} repository\ped{G} per il versionamento\ped{G} del Front-End\ped{G}, reperibile all'indirizzo \url{https://github.com/OmicronSwe/FrontEnd-EmporioLambda};
   \item \textbf{BackEnd\ped{G}:} repository\ped{G} per il versionamento\ped{G} del Back-End\ped{G}, reperibile all'indirizzo \url{https://github.com/OmicronSwe/BackEnd-EmporioLambda};
\end{itemize}

\myparagraph{Struttura della repository}
I repository\ped{G} sono formati da due versioni:
\begin{itemize}
	\item \textbf{remota:} repository\ped{G} presente su GitHub\ped{G} che contiene il lavoro svolto condiviso con il team;
	\item \textbf{locale:} copia della repository\ped{G} remota dove ogni componente può lavorare sui file contenuti in essa.
\end{itemize}

La repository\ped{G} contenente la documentazione è strutturata in cartelle descritte di seguito:
\begin{itemize}
 \item \textbf{DocumentazioneEsterna:} contiene i documenti da fornire ai committenti e al proponente, insieme ai verbali redatti durante gli incontri con quest'ultimi;
\item \textbf{DocumentazioneInterna:} contiene i documenti ad uso e consumo per i membri del team \Omicron, nonché i relativi verbali redatti durante gli incontri settimanali del gruppo;
\item \textbf{LetterePresentazioni:} contiene le lettere da accompagnare alla presentazione dei documenti da fornire ai committenti e al proponente;
\item \textbf{Utilita:} contiene gli elementi di supporto per la stesura dei documenti:
\begin{itemize}
	\item \textbf{img:} cartella contenente le immagini da utilizzare all'interno dei documenti;
	\item \textbf{latexSetup: } cartella contenente i file di supporto per la compilazione dei documenti \LaTeX, fra cui il file contenente i package\ped{G}, la prima pagina, il file di stile e il file con i comandi personalizzati del team.
\end{itemize}
\end{itemize}
La suddivisione dei file per scopo aiuta a migliorare la tracciabilità del lavoro svolto e favorisce una classificazione ordinata volta ad aiutare i membri del team durante la creazione dei documenti.


\myparagraph{Tipologia di file}
I file utilizzati per la realizzazione dei documenti sono:
\begin{itemize}
	\item file con estensione \textit{.tex} per i documenti \LaTeX;
	\item file con estensione \textit{.pdf} per i documenti da presentare ai committenti e al proponente (questi file non vanno versionati nella repository\ped{G});
	\item immagini di supporto ai file precedenti;
	\item file \textit{.gitignore} per gestire gli elementi che non devono essere versionati all'interno della repository\ped{G} remota;
	\item file con estensione \textit{.sty} per la caratterizzazione dello stile che definisce ogni file prodotto dal gruppo \Omicron.
\end{itemize}

\myparagraph{Utilizzo di Git}
I repository\ped{G} configurati su GitHub\ped{G} sono composti da vari branch\ped{G} creati per incrementare il lavoro collaborativo. Per lavorare sulla stesura di un un documento o su una parte di codice si consiglia di seguire i seguenti passi:
\begin{enumerate}
	\item scelta del branch\ped{G} su cui si vuole lavorare;
	\item esecuzione del comando \textit{pull} per aggiornare il branch\ped{G} locale con le ultime modifiche aggiunte, se presenti;
	\item apportare modifiche o aggiungere file secondo il proprio lavoro assegnato;
	\item eseguire il comando \textit{add} per aggiungere le modifiche apportate nell'area di staging\ped{G};
	\item eseguire il comando \textit{commit} per tracciare la modifica con un messaggio significativo per favorirne il tracciamento;
	\item eseguire il comando \textit{push} per condividere le modifiche apportate con gli altri membri del gruppo nella repository\ped{G} remota.
\end{enumerate}

\myparagraph{Gestione delle modifiche}
Ogni branch\ped{G} è modificabile da ogni componente del gruppo, ad eccezione di \textbf{main} e \textbf{develop} che possono essere modificati dopo l'approvazione di una pull request\ped{G} da parte del \respProg. I file modificati all'interno della pull request\ped{G} sono prima soggetti a verifica e, solo dopo il completamento di tale fase, possono essere approvati da parte del \respProg.
Per effettuare modifiche maggiori sulla struttura e sui contenuti dei file si prevede di:
\begin{enumerate}
	\item riferire al \respProg{} del file la modifica che si vuole effettuare;
	\item in caso di valutazione positiva da parte del \respProg, la modifica verrà approvata.
\end{enumerate}
Per modifiche minori, come correzioni grammaticali o sintattiche, è possibile modificare indipendentemente. 





\subsubsection{Script automatici}
Il gruppo ha deciso, dove possibile o necessario, di scrivere alcuni script\ped{G} che semplifichino la stesura di documenti e il calcolo di formule matematiche utilizzate:
\begin{itemize}
	\item \textbf{buildGlossario.py}: il team ha deciso di creare uno script\ped{G} per la stesura del \Glossariov che, da un file denominato \textbf{glossario.csv} contenente i termini e le definizioni, crei il documento \LaTeX{} finale. Lo script\ped{G} ed il file csv sono reperibili all'interno della cartella \textit{DocumentazioneEsterna/Glossario} nella repository\ped{G} della documentazione;
	\item \textbf{indiceGulpease.py}: il team ha deciso di creare uno script\ped{G} per il calcolo dell'indice di Gulpease di ogni documento redatto. Lo script\ped{G} è reperibile all'interno della cartella \textit{DocumentazioneEsterna/PianoDiQualifica} nella repository\ped{G} della documentazione;
\end{itemize}