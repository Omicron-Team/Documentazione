\subsection{Gestione della configurazione}\label{3.2}

\subsubsection{Scopo}\label{3.2.1}
Lo scopo del processo di configurazione è quello di dirigere in modo ordinato i documenti creati e il software. Ogni elemento configurato è sottoposto a caratteristiche ben definite e normate:
\begin{itemize}
 \item collocazione;
 \item stato;
 \item denominazione.
\end{itemize}
Ogni elemento configurato è modificato sempre secondo procedure normate e versionato\ped{G} in una repository\ped{G}.

\subsubsection{Aspettative}
Il processo in questione si pone di:
\begin{itemize}
\item uniformare la produzione di codice software e documentazione;
\item identificare gli strumenti di supporto per la gestione della configurazione del ciclo di vita del software e della documentazione;
\item gestire il versionamento\ped{G} di codice e documentazione.
\end{itemize}


\subsubsection{Descrizione}
Questa sezione descrive il tipo di configurazione necessaria per la creazione di documenti, codice software e versionamento\ped{G}.

\subsubsection{Versionamento}
\paragraph{Codice di versione del documento}

\subsubsection{Script automatici}
Il gruppo ha deciso, dove possibile o necessario, di scrivere alcuni script\ped{G} che semplifichino la stesura di documenti e il calcolo di formule matematiche utilizzate:
\begin{itemize}
	\item \textbf{buildGlossario.py}: il team ha deciso di creare uno script\ped{G} per la stesura del \Glossariov che, da un file denominato \textbf{glossario.csv} contenente i termini e le definizioni, crei il documento \LaTeX{} finale. Lo script\ped{G} ed il file csv sono reperibili all'interno della cartella \textit{DocumentazioneEsterna/Glossario} nella repository\ped{G} della documentazione;
	\item \textbf{indiceGulpease.py}: il team ha deciso di creare uno script\ped{G} per il calcolo dell'indice di Gulpease di ogni documento redatto. Lo script\ped{G} è reperibile all'interno della cartella \textit{DocumentazioneEsterna/PianoDiQualifica} nella repository\ped{G} della documentazione;
\end{itemize}