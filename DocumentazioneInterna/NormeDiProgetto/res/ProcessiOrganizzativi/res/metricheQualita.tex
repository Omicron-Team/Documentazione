\subsection{Metriche di qualità}
\subsubsection{Metriche per la pianificazione}
\myparagraph{Budget at Completion}
Per assicurarsi che i costi sostenuti per il progetto non superino esageratamente il budget preventivato si è deciso di usare la sigla "BAC" (indica i costi del progetto al suo completamento).

\myparagraph{Actual Cost}  
Per assicurarsi che i costi che si stanno sostenendo durante la realizzazione del progetto non superino esageratamente il budget preventivato si è deciso di usare le sigle "AC" (indica i costi del progetto al momento del calcolo) e "PV" (indica il costo pianificato per realizzare le attività di progetto alla data corrente).


\myparagraph{Schedule Variance}
Per indicare se si è in linea con la schedulazione delle attività di progetto pianificate si è deciso di usare le sigle "SV"(schedule variance),"EV"(indica il valore delle attività realizzate alla data corrente) e "PV"(indica il costo pianificato per realizzare le attività di progetto alla data corrente). In particolare si fa uso della seguente formula:
\begin{center}
\[SV=EV-PV\]
\end{center}

\myparagraph{Budget Variance}
Per indicare se i costi sostenuti alla data corrente sono in linea con il budget previsto (sempre alla data corrente) si è deciso di usare le sigle "BV"(budget variance) e "PV"(costo pianificato per realizzare le attività di progetto alla data corrente).  In particolare si fa uso della seguente formula:
\begin{center}
\[BV=PV-AC\]
\end{center}
dove "AC"(Actual Cost) è indicato nelle metriche precedenti.

\subsubsection{Metriche per la gestione della qualità}
\myparagraph{Percentuale di metriche soddisfatte}
Per assicurarsi che la quantità di metriche (elencate nei paragrafi "Metriche di qualità")  soddisfatte raggiunga soglie accettabili si è deciso di usare la seguente formula:
\begin{center}
\[PMS=\frac{metriche \ soddisfatte}{metriche \ totali}*100\]
\end{center}

