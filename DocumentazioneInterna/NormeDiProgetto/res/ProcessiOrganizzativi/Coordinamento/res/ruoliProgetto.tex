\subsubsection{Ruoli di progetto}
Di seguito sono elencati e descritti i ruoli che i membri del gruppo ricopriranno. L'assegnazione dei ruoli avviene a rotazione. Le attività che ogni ruolo deve svolgere vengono organizzate e pianificate nel \PdP{}. 

\myparagraph{Responsabile}
Il \respProg{} è un ruolo di fondamentale importanza per tutta la durata del progetto. È il responsabile ultimo dei risultati del progetto.\\I suoi compiti sono:
\begin{itemize}
\item gestire, controllare, coordinare:
\begin{itemize}
\item le attività del gruppo;
\item le risorse del progetto;
\item gli altri membri del gruppo (esempio: assegnazione dei ruoli).
\end{itemize}
\item analizzare i rischi;
\item stimare i costi;
\item approvare la documentazione;
\item curare le relazioni con \proponProg{} e \commitProg{};
\item redigere il \PdP{} e l'organigramma.
\end{itemize} 

\myparagraph{Amministratore}
L'\ammProg{} è responsabile dell'efficienza e dell'operatività dell'ambiente di sviluppo.\\I suoi compiti sono: 
\begin{itemize}
\item dirigere le infrastrutture e i servizi necessari ai processi di supporto; 
\item gestire:
\begin{itemize}
\item le versioni del prodotto;
\item le configurazioni del prodotto;
\item la documentazione di progetto;
\item eventuali problemi legati alla gestione dei processi.
\end{itemize} 
\item redigere le norme e le procedure di lavoro (\NdP{}).
\end{itemize}

\myparagraph{Analista}
L'\analProg{} è responsabile delle attività di analisi del progetto.\\I suoi compiti sono:
\begin{itemize}
\item analizzare il problema ricavandone:
\begin{itemize}
\item casi d'uso;
\item requisiti.
\end{itemize}
\item redigere lo \SdF{} e l'\AdR{}.
\end{itemize}  

\myparagraph{Progettista}
Il \progetProg{} è responsabile delle attività di progettazione (partendo dai requisiti e i casi d'uso ricavati dall'\analProg{}).\\Il suo compito è produrre un'architettura logica del prodotto che sia:
\begin{itemize}
\item coerente;
\item consistente;
\item efficiente (per risorse e per costi, i quali devono essere inferiori a quanto stabilito nel preventivo);
\item manutenibile;
\item efficace (deve soddisfare tutti i requisiti).
\end{itemize}

\myparagraph{Programmatore} 
Il \programProg{} è responsabile della parte di codifica del progetto.\\Nel dettaglio i suoi compiti sono:
\begin{itemize}
\item codificare le scelte del \progetProg{};
\item creare le componenti per la verifica e validazione del codice;
\item redigere il \MU{}.
\end{itemize}

\myparagraph{Verificatore}
Il \verifProg{} è responsabile delle attività di verifica.\\Nel dettaglio i suoi compiti sono:
\begin{itemize}
\item verificare la correttezza dei prodotti in fase di revisione e evidenziandone eventuali difetti (seguendo le \NdP{}); 
\item redigere la parte retrospettiva del \PdQ{} che descrive l'esito e la completezza delle verifiche e delle prove effettuate.
\end{itemize}