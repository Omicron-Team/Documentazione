\subsubsection{Procedure}
\myparagraph{Gestione delle comunicazioni}
Le comunicazioni possono essere interne, cioè coinvolgono solo i partecipanti del team, oppure esterne, cioè includono anche soggetti esterni come proponente e committente.
\begin{itemize}
	\item \textbf{Comunicazioni interne}: per le comunicazioni tra i membri del gruppo è stato adottato Discord\ped{G}, un'applicazione VoIP multi piattaforma. Su Discord\ped{G} si possono integrare bot\ped{G}, creare canali tematici per rendere più efficiente lo scambio e il reperimento delle informazioni. \\
	All'interno di Discord\ped{G}, sono stati predisposti i seguenti canali tematici:
	\begin{itemize}
	\item \textbf{general}: per le comunicazioni di carattere generale;
	\item \textbf{github-updates}: canale riservato nel quale un bot\ped{G} automatico invia notifiche ogni volta che un membro del team effettua una push, apre/commenta/chiude una issue, effettua un merge su GitHub\ped{G};
	\item \textbf{resources}: per la condivisione di link utili;
	\item \textbf{date-incontri}: per decidere, in base alle disponibilità individuali, quando fissare riunioni;
	\item \textbf{template}: canale dedicato alla creazione del template\ped{G} di tutti i documenti e alla sua manutenzione nel tempo;
	\item un canale per ciascun documento in modo da avere una migliore organizzazione del contenuto di essi. 
	\end{itemize}
	\item \textbf{Comunicazioni esterne}: le comunicazioni con soggetti esterni al team sono di competenza del \respProg . I soggetti esterni con il quale intrattenere rapporti utili sono:
	\begin{itemize}
		\item i Committenti \textbf{\VT} e \textbf{\CR}, con cui si userà l'indirizzo \url{omicronswe@gmail.com};
		\item il Proponente \textbf{\Proponente}, con cui si è deciso di utilizzare un canale Slack\ped{G} per la chat testuale e il servizio Google Meet\ped{G} per le videochiamate.
	\end{itemize}
\end{itemize}

\myparagraph{Gestione degli incontri}
Gli incontri possono essere esterni o interni, sulla base della partecipazione di soggetti esterni al team o no. Per entrambe le tipologie di incontro, il \respProg{} nomina un segretario che avrà il compito di redigere il verbale dell'incontro.
\begin{itemize}
\item \textbf{Incontri interni}: il \respProg{} ha il compito di organizzare gli incontri interni utilizzando il canale Discord\ped{G} dedicato, dove ci si accorda con i membri del gruppo sul miglior timeslot;
\item \textbf{Incontri esterni}: il \respProg{} ha il compito di comunicare ed organizzare gli incontri esterni con il proponente o il committente, decidendo una data in comune accordo con le parti.
\end{itemize}

\myparagraph{Gestione degli strumenti di coordinamento}
Per suddividere il carico di lavoro in task\ped{G} che saranno poi divisi tra tutti i componenti, viene usata la funzionalità \textit{Projects}\ped{G} di GitHub\ped{G}. La procedura per l'assegnazione di un task\ped{G} segue il seguente schema:
\begin{itemize}
\item creazione di una nuova issue\ped{G} con un titolo significativo e una breve descrizione se necessaria;
\item indicare la/e persona/e a cui è stato assegnato tale compito;
\item selezionare il projects\ped{G} a cui fa riferimento il compito;
\item indicare una milestone\ped{G}.
\end{itemize}
Ogni compito passa attraverso i seguenti stati:
\begin{itemize}
\item \textbf{To do}, da fare;
\item \textbf{In progress}, in lavorazione;
\item \textbf{Done}, completato.
\end{itemize}
Una volta che i compiti sono stati eseguiti, viene aperta una pull request\ped{G} che verrà chiusa solo dopo la fase di verifica e approvazione.

\myparagraph{Gestione dei rischi}
Un altro compito del \respProg{} è quello di rilevare i rischi e renderli noti tramite il \PdP. \\
Per la gestione dei rischi, la procedura da seguire è la seguente:
\begin{itemize}
\item individuare nuovi problemi e monitorare i rischi già previsti;
\item aggiungere i nuovi rischi nel \PdP;
\item ridefinire, se necessario, le strategie di progetto.
\end{itemize}