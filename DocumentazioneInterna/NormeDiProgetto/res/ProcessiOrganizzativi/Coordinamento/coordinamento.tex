\subsection{Coordinamento}

\subsubsection{Scopo}

\subsubsection{Aspettative}

\subsubsection{Descrizione}

\subsubsection{Ruoli di Progetto}
Di seguito sono elencati e descritti i ruoli che i membri del gruppo ricopriranno. L'assegnazione dei ruoli avviene a rotazione. Le attività che ogni ruolo deve svolgere vengono organizzate e pianificate nel \PdP{}. 

\myparagraph{Responsabile}
Il \respProg{} è un ruolo di fondamentale importanza per tutta la durata del progetto. È il responsabile ultimo dei risultati del progetto.\\I suoi compiti sono:
\begin{itemize}
\item gestire, controllare, coordinare:
\begin{itemize}
\item le attività del gruppo;
\item le risorse del progetto;
\item gli altri membri del gruppo (esempio: assegnazione dei ruoli).
\end{itemize}
\item analizzare i rischi;
\item stimare i costi;
\item approvare la documentazione;
\item curare le relazioni con \proponProg{} e \commitProg{};
\item redigere il \PdP{} e l'organigramma.
\end{itemize} 

\myparagraph{Amministratore}
L'\ammProg{} è responsabile dell'efficienza e dell'operatività dell'ambiente di sviluppo.\\I suoi compiti sono: 
\begin{itemize}
\item dirigere le infrastrutture e i servizi necessari ai processi di supporto; 
\item gestire:
\begin{itemize}
\item le versioni del prodotto;
\item le configurazioni del prodotto;
\item la documentazione di progetto;
\item eventuali problemi legati alla gestione dei processi.
\end{itemize} 
\item redigere le norme e le procedure di lavoro (\NdP{}).
\end{itemize}

\myparagraph{Analista}
L'\analProg{} è responsabile delle attività di analisi del progetto.\\I suoi compiti sono:
\begin{itemize}
\item analizzare il problema ricavandone:
\begin{itemize}
\item casi d'uso;
\item requisiti.
\end{itemize}
\item redigere lo \SdF{} e l'\AdR{}.
\end{itemize}  

\myparagraph{Progettista}
Il \progetProg{} è responsabile delle attività di progettazione (partendo dai requisiti e i casi d'uso ricavati dall'\analProg{}).\\Il suo compito è produrre un'architettura logica del prodotto che sia:
\begin{itemize}
\item coerente;
\item consistente;
\item efficiente (per risorse e per costi, i quali devono essere inferiori a quanto stabilito nel preventivo);
\item manutenibile;
\item efficace (deve soddisfare tutti i requisiti).
\end{itemize}

\myparagraph{Programmatore} 
Il \programProg{} è responsabile della parte di codifica del progetto.\\Nel dettaglio i suoi compiti sono:
\begin{itemize}
\item codificare le scelte del \progetProg{};
\item creare le componenti per la verifica e validazione del codice;
\item redigere il \MU{}.
\end{itemize}

\myparagraph{Verificatore}
Il \verifProg{} è responsabile delle attività di verifica.\\Nel dettaglio i suoi compiti sono:
\begin{itemize}
\item verificare la correttezza dei prodotti in fase di revisione e evidenziandone eventuali difetti (seguendo le \NdP{}); 
\item redigere la parte retrospettiva del \PdQ{} che descrive l'esito e la completezza delle verifiche e delle prove effettuate.
\end{itemize}

\myparagraph{Gestione delle comunicazioni}
Le comunicazioni possono essere interne, cioè coinvolgono solo i partecipanti del team, oppure esterne, cioè includono anche soggetti esterni come proponente e committente.
\begin{itemize}
	\item \textbf{Comunicazioni interne}: per le comunicazioni tra i membri del gruppo è stato adottato Discord\ped{G}, un'applicazione VoIP multi piattaforma. Su Discord\ped{G} si possono integrare bot\ped{G}, creare canali tematici per rendere più efficiente lo scambio e il reperimento delle informazioni. \\
	All'interno di Discord\ped{G}, sono stati predisposti i seguenti canali tematici:
	\begin{itemize}
	\item \textbf{general}: per le comunicazioni di carattere generale;
	\item \textbf{github-updates}: canale riservato nel quale un bot\ped{G} automatico invia notifiche con i risultati delle \textit{GitHub Actions}\ped{G} (§3.4.5.2) ed ogni volta che un membro del team effettua una \textit{push}, apre/commenta/chiude una \textit{issue}, effettua un \textit{merge} su GitHub\ped{G};
	\item \textbf{resources}: per la condivisione di link utili;
	\item \textbf{date-incontri}: per decidere, in base alle disponibilità individuali, quando fissare riunioni;
	\item \textbf{template}: canale dedicato alla creazione del template\ped{G} di tutti i documenti e alla sua manutenzione nel tempo;
	\item un canale per ciascun documento in modo da avere una migliore organizzazione del contenuto di essi. 
	\end{itemize}
	\item \textbf{Comunicazioni esterne}: le comunicazioni con soggetti esterni al team sono di competenza del \respProg . I soggetti esterni con il quale intrattenere rapporti utili sono:
	\begin{itemize}
		\item i Committenti \textbf{\VT} e \textbf{\CR}, con cui si userà l'indirizzo \url{omicronswe@gmail.com};
		\item il Proponente \textbf{\Proponente}, con cui si è deciso di utilizzare un canale Slack\ped{G} per la chat testuale e il servizio Google Meet\ped{G} per le videochiamate.
	\end{itemize}
\end{itemize}

\myparagraph{Gestione degli incontri}
Gli incontri possono essere esterni o interni, sulla base della partecipazione di soggetti esterni al team o no. Per entrambe le tipologie di incontro, il \respProg{} nomina un segretario che avrà il compito di redigere il verbale dell'incontro.
\begin{itemize}
\item \textbf{Incontri interni}: il \respProg{} ha il compito di organizzare gli incontri interni utilizzando il canale Discord\ped{G} dedicato, dove ci si accorda con i membri del gruppo sul miglior timeslot;
\item \textbf{Incontri esterni}: il \respProg{} ha il compito di comunicare ed organizzare gli incontri esterni con il proponente o il committente, decidendo una data in comune accordo con le parti.
\end{itemize}

\subsubsection{Strumenti}
Durante lo svolgimento dell'intero progetto il gruppo utilizzerà questi strumenti:
\begin{itemize}
	\item \textbf{Telegram}\ped{G}: applicazione di messaggistica utilizzato per la gestione del gruppo nella fase iniziale;
	\item \textbf{Discord}\ped{G}: strumento per la gestione del gruppo e per le relative comunicazioni e videochiamate interne del team;
	\item \textbf{Slack}\ped{G}: strumento utilizzato per le comunicazioni con il proponente;
	\item \textbf{Google Meet}\ped{G}: strumento utilizzato per le videochiamate con il proponente;
	\item \textbf{Google Docs}\ped{G}: strumento utilizzato per la gestione dei ruoli e per la stesura delle domande da porre al proponente;
	\item \textbf{Git}\ped{G}: sistema per il controllo del versionamento;
	\item \textbf{GitHub}\ped{G}: strumento web per la condivisione in remoto dei file del progetto.
	\item \textbf{GitFlow}\ped{G}: strumento per facilitare l'utilizzo di Git\ped{G} in locale;
	\item \textbf{Sistemi operativi}: possibilità di utilizzare qualsiasi sistema operativo (Windows, Linux e Mac OS).
\end{itemize}