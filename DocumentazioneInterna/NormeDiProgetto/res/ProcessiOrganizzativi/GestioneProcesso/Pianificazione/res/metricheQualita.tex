\myparagraph{Metriche} %\vspace{-1cm}
Per la valutazione del prodotto della pianificazione, il gruppo ha deciso di adottare le seguenti metriche:
\begin{itemize}
\item \textbf{Budget at completion}: per assicurarsi che i costi sostenuti per il progetto non superino esageratamente il budget preventivato si è deciso di usare la sigla "BAC" (indica i costi del progetto al suo completamento);
\item \textbf{Actual cost}: per assicurarsi che i costi sostenuti durante la realizzazione del progetto non superino esageratamente il budget preventivato si è deciso di usare le sigle "AC" (indica i costi del progetto al momento del calcolo) e "PV" (indica il costo pianificato per realizzare le attività di progetto alla data corrente);
\item \textbf{Schedule variance}: per indicare se si è in linea con la schedulazione delle attività di progetto pianificate si è deciso di usare le sigle "SV"(schedule variance), "EV"(indica il valore delle attività realizzate alla data corrente) e "PV" (indica il costo pianificato per realizzare le attività di progetto alla data corrente). In particolare si fa uso della seguente formula: \vspace{-0.5cm}
\begin{center}
\[SV=EV-PV\]
\end{center}
\item \textbf{Budget variance}: per indicare se i costi sostenuti alla data corrente sono in linea con il budget previsto (sempre alla data corrente) si è deciso di usare le sigle "BV"(budget variance) e "PV"(costo pianificato per realizzare le attività di progetto alla data corrente).  In particolare si fa uso della seguente formula: \vspace{-0.5cm}
\begin{center}
\[BV=PV-AC\]
\end{center}
dove "AC"(Actual cost) è definito come nelle metriche precedenti;
\item \textbf{Percentuale di metriche soddisfatte}: per assicurarsi che la quantità di metriche (elencate nei paragrafi "Metriche") soddisfatte raggiunga soglie accettabili si è deciso di usare la seguente formula:
\begin{center}
\[PMS=\frac{metriche \ soddisfatte}{metriche \ totali}*100\]
\end{center}
\end{itemize}


