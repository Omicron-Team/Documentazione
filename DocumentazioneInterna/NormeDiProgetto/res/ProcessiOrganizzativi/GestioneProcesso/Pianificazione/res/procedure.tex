\myparagraph{Gestione degli strumenti di coordinamento}
Per suddividere il carico di lavoro in task\ped{G} che saranno poi divisi tra tutti i componenti, viene usata la funzionalità \textit{Projects}\ped{G} di GitHub\ped{G}. La procedura per l'assegnazione di un task\ped{G} segue il seguente schema:
\begin{itemize}
\item creazione di una nuova issue\ped{G} con un titolo significativo e una breve descrizione se necessaria;
\item indicare la/e persona/e a cui è stato assegnato tale compito;
\item selezionare il projects\ped{G} a cui fa riferimento il compito;
\item indicare una milestone\ped{G}.
\end{itemize}
Ogni compito passa attraverso i seguenti stati:
\begin{itemize}
\item \textbf{To do}: da fare;
\item \textbf{In progress}: in lavorazione;
\item \textbf{Done}: completato.
\end{itemize}
Una volta che i compiti sono stati eseguiti, viene aperta una pull request\ped{G} che verrà chiusa solo dopo la fase di verifica e approvazione.

\myparagraph{Gestione dei rischi}
Un altro compito del \respProg{} è quello di rilevare i rischi e renderli noti tramite il \PdP. \\
Per la gestione dei rischi, la procedura da seguire è la seguente:
\begin{itemize}
\item individuare nuovi problemi e monitorare i rischi già previsti;
\item aggiungere i nuovi rischi nel \PdP;
\item ridefinire, se necessario, le strategie di progetto.
\end{itemize}