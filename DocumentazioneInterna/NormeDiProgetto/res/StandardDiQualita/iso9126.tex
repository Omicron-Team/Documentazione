\section{Standard ISO/IEC 9126}
ISO/IEC 9126 è uno standard\ped{G} internazionale per la valutazione della qualità del software. Lo standard\ped{G} è stato sostituito dal ISO/IEC 25010:2011 ma essendo ancora una normativa cardine, il gruppo \Omicron{} ha deciso di adottarlo come standard\ped{G} di riferimento. \\
Lo scopo generale di queste linee guida è il miglioramento dell'organizzazione e dei processi e quindi, come conseguenza concreta, della qualità del prodotto sviluppato.

\subsection{Modello della qualità del software}
Il modello di qualità stabilito nello standard\ped{G} è suddiviso in sei caratteristiche generali, ognuna delle quali a sua volta presenta delle sottocaratteristiche misurabili attraverso delle metriche. \\
Il modello è articolato nel seguente modo:
\begin{itemize}
\item \textbf{Funzionalità}: capacità di un prodotto software di fornire funzioni che soddisfano esigenze stabilite, necessarie per operare sotto condizioni specifiche. Le sottocaratteristiche della \textit{funzionalità} sono:
	\begin{itemize}
	\item \textbf{Appropriatezza}: capacità del prodotto di fornire un insieme di funzioni per specifici compiti ed obiettivi prefissati dall'utente;
	\item \textbf{Accuratezza}: capacità del prodotto di fornire i risultati concordati con la precisione richiesta;
	\item \textbf{Interoperabilità}: capacità del prodotto di interagire ed operare con uno o più sistemi specificati;
	\item \textbf{Conformità}: capacità di aderire a standard\ped{G}, convenzioni e regolamentazioni rilevanti al settore operativo a cui vengono applicate;
	\item \textbf{Sicurezza}: capacità del prodotto di proteggere informazioni e dati negando in ogni modo che persone o sistemi non autorizzati possano accedervi o modificarli.
	\end{itemize}
\item \textbf{Affidabilità}: capacità del prodotto software di mantenere uno specificato livello di prestazioni quando usato in date condizioni per un dato periodo. Le sottocaratteristiche dell' \textit{affidabilità} sono:
	\begin{itemize}
	\item \textbf{Maturità}: capacità del prodotto di evitare errori, malfunzionamenti o risultati non corretti durante l'esecuzione;
	\item \textbf{Tolleranza agli errori}: capacità di mantenere livelli prestabiliti di prestazioni anche in presenza di malfunzionamenti o usi scorretti;
	\item \textbf{Recuperabilità}: capacità del prodotto di ripristinare il livello appropriato di prestazioni e di recupero delle informazioni rilevanti, in seguito a un malfunzionamento. L'arco di tempo in cui il software può risultare non accessibile è valutato proprio dalla recuperabilità;
	\item \textbf{Aderenza}: capacità di aderire a standard\ped{G}, convenzioni e regolamentazioni inerenti all'affidabilità.
	\end{itemize}
\item \textbf{Efficienza}: la capacità di fornire appropriate prestazioni relativamente alla quantità di risorse usate. Le sottocaratteristiche dell'\textit{efficienza} sono:
	\begin{itemize}
	\item \textbf{Comportamento rispetto al tempo}: capacità di fornire adeguati tempi di risposta, elaborazione e quantità di lavoro eseguendo le funzionalità richieste in date condizioni di lavoro;
	\item \textbf{Utilizzo delle risorse}: capacità di utilizzo di quantità e tipo di risorse in maniera adeguata;
	\item \textbf{Conformità}: capacità di aderire a standard\ped{G} e specifiche sull'efficienza.
	\end{itemize}
\item \textbf{Usabilità}: capacità del prodotto software di essere capito, appreso e usato dall'utente sotto condizioni specifiche. Le sottocaratteristiche dell'\textit{usabilità} sono:
	\begin{itemize}
	\item \textbf{Comprensibilità}: facilità di comprensione dei concetti del prodotto, per permettere all'utente di comprendere se il software è appropriato;
	\item \textbf{Apprendibilità}: capacità di ridurre l'impegno richiesto agli utenti di imparare ad usare l'applicazione;
	\item \textbf{Operabilità}: capacità di mettere in condizione gli utenti di farne uso per i propri scopi e controllarne l'uso;
	\item \textbf{Attrattiva}: capacità del prodotto di essere piacevole per l'utente che ne fa uso;
	\item \textbf{Conformità}: capacità di aderire a standard\ped{G} e convenzioni relativi all'usabilità.
	\end{itemize}
\item \textbf{Manutenibilità}: capacità del prodotto di essere modificato, includendo correzioni, miglioramenti o adattamenti. Le sottocaratteristiche della \textit{manutenibilità} sono:
	\begin{itemize}
	\item \textbf{Analizzabilità}: facilità di analizzare il codice per localizzare un errore nello stesso;
	\item \textbf{Modificabilità}: capacità di permettere l'implementazione di una specifica modifica;
	\item \textbf{Stabilità}: capacità del prodotto di evitare effetti inaspettati derivanti da modifiche errate;
	\item \textbf{Testabilità}: capacità di essere facilmente testato per validare le modifiche apportate al software.
	\end{itemize}
\item \textbf{Portabilità}: capacità del prodotto di essere trasportato da un ambiente di lavoro ad un altro. Le sottocaratteristiche della \textit{portabilità} sono:
	\begin{itemize}
	\item \textbf{Adattabilità}: capacità del software di essere adattato per differenti ambienti operativi senza dover richiedere modifiche diverse da quelle fornite per il prodotto considerato;
	\item \textbf{Installabilità}: capacità del prodotto si essere installato in uno specifico ambiente;
	\item \textbf{Coesistenza}: capacità del prodotto di coesistere con altre applicazioni indipendenti in ambienti comuni e di condividere le risorse;
	\item \textbf{Sostituibilità}: capacità di sostituire un altro software specifico indipendente, per lo stesso scopo e nello stesso ambiente;
	\item \textbf{Aderenza}: capacità di aderire a standard\ped{G} e convenzioni relativi alla portabilità.
	\end{itemize}
\end{itemize}

\subsection{Metriche per la qualità esterna}
Le metriche esterne, specificate nella norma ISO/IEC 9126-2, misurano i comportamenti del prodotto software sulla base di test, dall'operatività e dall'osservazione durante la sua esecuzione, in funzione degli obiettivi stabiliti.

\subsection{Metriche per la qualità interna}
Le metriche interne, specificate nella norma ISO/IEC 9126-3, si applicano al prodotto non eseguibile durante le fasi di progettazione e codifica. \\
Le misure effettuate permettono di prevedere il livello di qualità esterna ed in uso del prodotto finale, poiché gli attributi interni influiscono su quelli esterni e quelli in uso. \\
Le metriche interne permettono di individuare eventuali problemi che potrebbero influire sulla qualità finale del prodotto prima che sia realizzato il software eseguibile.

\subsection{Metriche per la qualità in uso}
La qualità in uso rappresenta il punto di vista dell'utente sulla qualità. Il livello di qualità in uso è raggiunto quando viene raggiunto sia il livello di qualità esterna sia quello di qualità interno. \\
Le norme ISO/IEC 9126-4 forniscono esempi di metriche da utilizzare per la misurazione della qualità rispetto ai diversi punti di vista (interno, esterno, in uso). \\
La qualità in uso, quindi, permette di abilitare specifici utenti ad ottenere specifici obiettivi con efficacia, produttività, sicurezza e soddisfazione.
\begin{itemize}
\item \textbf{Efficacia}: capacità del prodotto di permettere all'utente di raggiungere obiettivi specifici con accuratezza e completezza in uno specifico contesto di utilizzo;
\item \textbf{Produttività}: capacità di permettere all'utente di impegnare un numero definito di risorse, in relazione all'efficienza raggiunta in uno specifico contesto di utilizzo;
\item \textbf{Sicurezza}: capacità di raggiungere un livello accettabile di rischio per i dati, le persone, le apparecchiature tecniche e l'ambiente d'uso;
\item \textbf{Soddisfazione}: capacità di soddisfare gli utenti in uno specifico contesto di utilizzo.
\end{itemize}