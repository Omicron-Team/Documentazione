\section{Standard ISO/IEC 9126}
ISO/IEC 9126 è uno standard\ped{G} internazionale per la valutazione della qualità del software. Lo standard\ped{G} è stato sostituito dal ISO/IEC 25010:2011 ma essendo ancora una normativa cardine, il gruppo \Omicron ha deciso di adottarlo come standard\ped{G} di riferimento. \\
Lo scopo generale di queste linee guida è il miglioramento dell'organizzazione e dei processi e quindi, come conseguenza concreta, della qualità del prodotto sviluppato.

\subsection{Modello della qualità del software}
Il modello di qualità stabilito nello standard\ped{G} è suddiviso in sei caratteristiche generali, ognuna delle quali a sua volta presenta delle sottocaratteristiche misurabili attraverso delle metriche. \\
Il modello è articolato nel seguente modo:
\begin{itemize}
\item \textbf{Funzionalità}: capacità di un prodotto software di fornire funzioni che soddisfano esigenze stabilite, necessarie per operare sotto condizioni specifiche. Le sottocaratteristiche della \textit{funzionalità} sono:
	\begin{itemize}
	\item \textbf{Appropriatezza}: capacità del prodotto di fornire un insieme di funzioni per specifici compiti ed obiettivi prefissati dall'utente;
	\item \textbf{Accuratezza}: capacità del prodotto di fornire i risultati concordati con la precisione richiesta;
	\item \textbf{Interoperabilità}: capacità del prodotto di interagire ed operare con uno o più sistemi specificati;
	\item \textbf{Conformità}: capacità di aderire a standard, convenzioni e regolamentazioni rilevanti al settore operativo a cui vengono applicate;
	\item \textbf{Sicurezza}: capacità del prodotto di proteggere informazioni e dati negando in ogni modo che persone o sistemi non autorizzati possano accedervi o modificarli.
	\end{itemize}
\item \textbf{Affidabilità}: capacità del prodotto software di mantenere uno specificato livello di prestazioni quando usato in date condizioni per un dato periodo. Le sottocaratteristiche dell' \textit{affidabilità} sono:
	\begin{itemize}
	\item \textbf{Maturità}: capacità del prodotto di evitare errori, malfunzionamenti o risultati non corretti durante l'esecuzione;
	\item \textbf{Tolleranza agli errori}: capacità di mantenere livelli prestabiliti di prestazioni anche in presenza di malfunzionamenti o usi scorretti;
	\item \textbf{Recuperabilità}: capacità del prodotto di ripristinare il livello appropriato di prestazioni e di recupero delle informazioni rilevanti, in seguito a un malfunzionamento. L'arco di tempo in cui il software può risultare non accessibile è valutato proprio dalla recuperabilità;
	\item \textbf{Aderenza}: capacità di aderire a standard, convenzioni e regolamentazioni inerenti all'affidabilità.
	\end{itemize}
\item \textbf{Efficienza}: la capacità di fornire appropriate prestazioni relativamente alla quantità di risorse usate. Le sottocaratteristiche dell' \textit{efficienza} sono:
	\begin{itemize}
	\item \textbf{Comportamento rispetto al tempo}: capacità di fornire adeguati tempi di risposta, elaborazione e quantità di lavoro eseguendo le funzionalità richieste in date condizioni di lavoro;
	\item \textbf{Utilizzo delle risorse}: capacità di utilizzo di quantità e tipo di risorse in maniera adeguata;
	\item \textbf{Conformità}: capacità di aderire a standard e specifiche sull'efficienza.
	\end{itemize}
\item \textbf{Usabilità}:
\item \textbf{Manutenibilità}:
\item \textbf{Portabilità}:
\end{itemize}

\subsection{Metriche per la qualità esterna}

\subsection{Metriche per la qualità interna}

\subsection{Metriche per la qualità in uso}