\subsection{Sviluppo}
\subsubsection{Scopo}
Lo scopo del processo di Sviluppo è quello di descrivere i compiti e le attività di analisi, progettazione, codifica, integrazione, test, installazione ed accettazione del prodotto seguendo lo standard ISO/IEC 12207:1995.
<<<<<<< HEAD
\subsubsection{Descrizione}
Elenchiamo di seguite le attività che compongono il processo di sviluppo:
\begin{itemize}
    \item{\AdR{};}
    \item{Progettazione;}
    \item{Codifica del software;}
\end{itemize}

\subsubsection{Aspettative}
Le aspettative del gruppo sono:
\begin{itemize}
    \item{Fissare gli obiettivi di sviluppo;}
    \item{Fissare vincoli tecnologici;}
    \item{Fissare vincoli di design;}
    \item{Realizzare un prodotto finale che supera i test, che soddisfa i requisiti e le richieste del proponente;}
\end{itemize}

\subsubsection{Attività}
\paragraph{\AdR{}}
\subparagraph{Scopo} L'biettivo dell'\AdRv{} è quello di definire i requisiti preposti dall'\proponProg{}. È quindi necessario che gli \analProg
\begin{itemize}
  %  \item{Fornire ai progettisti indicazioni precise ed affidabili;}
    %\item{Fissare i requisiti concordati con il \proponProg;}
    \item{;}
    \item{;}
    \item{;}
    \item{;}
\end{itemize}




\paragraph{Progettazione}


\paragraph{Codifica del software}
=======

\paragraph{Casi d'uso}
Rappresenta una situazione nella quale il sistema viene utilizzato per soddisfare uno o più bisogni dell'utente.\\
La loro struttura è la seguente:
\begin{itemize}
\item \textbf{Identificatore}: codice che identifica il caso d'uso. Il suo formato è il seguente:
\begin{center}
\textbf{UC[CodicePadre].\{CodiceFiglio\}}\\
\end{center}
dove:
\begin{itemize}
\item \textbf{UC}: rappresenta un acronimo di "Use Case", ovvero "Caso d'uso";
\item \textbf{CodicePadre}: rappresenta il codice di un caso d'uso generico;
\item \textbf{CodiceFiglio} (opzionale): rappresenta il codice di un sotto caso di un caso d'uso;
\end{itemize}
\item \textbf{Nome}: Titolo del caso d'uso;
\item \textbf{Diagramma UML}\ped{G}: rappresentazione grafica del caso d'uso utilizzando il linguaggio UML\ped{G} 2.0; 
\item \textbf{Descrizione}: descrizione breve del caso d'uso;
\item \textbf{Scenario principale}: sequenza di azioni che porta al risultato atteso dal caso d'uso;
\item \textbf{Scenari alternativi}(opzionale): sequenza di azioni che porta ad un risultato inatteso (un'errore o un'eccezione) dal caso d'uso;
\item \textbf{Precondizioni}: condizioni necessarie per il corretto avvio del caso d'uso;
\item \textbf{Post-condizioni}: condizioni che devono essere vere dopo che il caso d'uso è andato a buon fine;
\item \textbf{Triggers}: eventi che portano all'avvio del caso d'uso;
\item \textbf{Attori primari}: utente che inizia un interazione con il sistema per raggiungere un obiettivo;
\item \textbf{Attori secondari}: entità che aiuta l'attore primario a raggiungere il suo obiettivo;
\item \textbf{Estensioni}(opzionale): aumenta le funzionalità di uno use case. Ogni istanza di un use case 'A' esegue un evento 'B' in modo condizionato. L'esecuzione di 'B' interrompe 'A';
\item \textbf{Inclusioni}(opzionale): include un'istanza di un use case 'B' in un altro use case 'A';
\item \textbf{Generalizzazioni}(opzionale): aggiunge o modifica le caratteristiche base di uno use case. Crea così delle specializzazioni dello use case di partenza.
\end{itemize}
\section{Requisiti}

\subsection{Introduzione} 

Sono stati classificati e assegnati i requisiti in base a quanto definito nel documento \NdPv{1.0.0} sezione §2.2.4.3

\subsection{Requisiti funzionali} 


{

\rowcolors{2}{azzurro2}{azzurro3}

\centering
\renewcommand{\arraystretch}{2}
\begin{longtable}{C{2.5cm} C{2.8cm} C{6.7cm} C{2.5cm}}
\caption{Tabella dei Requisiti funzionali}\\
\rowcolor{azzurro1}
\textbf{Codice} &
\textbf{Classificazione}&
\textbf{Descrizione}&
\textbf{Fonti}\\
\endhead

%REGISTRAZIONE

R1F1 & Obbligatorio & Un utente non autenticato può registrarsi & Capitolato \\
R1F1.1 & Obbligatorio & La registrazione necessita l'inserimento dei seguenti dati personali: nome, cognome e indirizzo di fatturazione & Interno \\
R1F1.2 & Obbligatorio & La registrazione necessita l'inserimento della email & VE\_2020-12-23 \\
R2F1.2.1 & Desiderabile & Il sistema deve mostrare un errore se l'email inserita nella registrazione è già stata utilizzata & Interno \\
R1F1.3 & Obbligatorio & La registrazione necessita l'inserimento di una password & Interno \\

%LOGIN

R1F2 & Obbligatorio & Un utente non autenticato può effettuare il login & Capitolato \\
R1F2.1 & Obbligatorio & Il login necessita l'inserimento della email & Interno \\
R1F2.2 & Obbligatorio & Il login necessita l'inserimento della password & Interno \\
R2F2.3 & Desiderabile & Il sistema deve mostrare un errore se le credenziali del login sono errate & Interno \\

%LOGOUT

R1F3 & Obbligatorio & Un utente autenticato può effettuare il logout & Interno \\

%CARRELLO

R1F4 & Obbligatorio & Un utente può visualizzare tutti i prodotti che ha precedentemente aggiunto al carrello & Capitolato \\
R1F4.1 & Obbligatorio & Un utente può visualizzare per ogni prodotto del carrello il nome e l’immagine & Interno \\
R1F4.2 & Obbligatorio & Un utente può rimuovere i singoli prodotti dal carrello & Interno \\
R1F3.3 & Obbligatorio & Un utente può modificare le quantità di ogni prodotto nel carrello & Interno \\
R1F4.4 & Obbligatorio & Un utente può visualizzare il costo totale dei prodotti nel carrello & Capitolato \\
R1F4.4.1 & Obbligatorio & Un utente può visualizzare il costo di ogni voce del carrello & Interno \\
R1F4.5 & Obbligatorio & Un utente può visualizzare le tasse applicate al costo totale dei prodotti nel carrello & Capitolato \\

%CHECKOUT

R1F5 & Obbligatorio & Un utente registrato può procedere al checkout dei prodotti presenti nel carrello & Capitolato \\
R1F5.1 & Obbligatorio & Per iniziare il processo di checkout l’utente deve avere almeno un prodotto nel carrello & Interno \\
R1F5.2 & Obbligatorio & Un utente che ha iniziato il processo di checkout deve inserire i dati del pagamento & Interno \\
R1F5.3 & Obbligatorio & Un utente, dopo aver inserito i dati del pagamento, può continuare al pagamento effettivo & Capitolato \\
R1F5.4 & Obbligatorio & A pagamento riuscito l’utente può visualizzare un riepilogo dell’ordine effettuato & Capitolato \\
R1F5.4.1 & Obbligatorio & A pagamento riuscito l’utente riceverà i prodotti acquistati tramite l’email del suo account & Capitolato \\
R1F5.4.2 & Obbligatorio & A pagamento fallito l’utente visualizza un messaggio di errore, e potrà riprovare il pagamento verificando i dati inseriti & Capitolato \\

%PROFILO



%MERCHANT DASHBOARD

%HOMEPAGE

%PLP

%PDP




\end{longtable}

}
\subsection{Requisiti di qualità}

{

\rowcolors{2}{azzurro2}{azzurro3}

\centering
\renewcommand{\arraystretch}{2}
\begin{longtable}{C{2.5cm} C{2.8cm} C{6.7cm} C{2.5cm}}
\caption{Tabella dei Requisiti di qualità}\\
\rowcolor{azzurro1}
\textbf{Codice} &
\textbf{Classificazione}&
\textbf{Descrizione}&
\textbf{Fonti}\\
\endhead


R1Q1 & Obbligatorio & Il prodotto deve essere sviluppato rispettando quanto scritto nel documento \NdPv{1.0.0} e \PdQv{1.0.0} & Interno \\
R1Q2 & Obbligatorio & Deve essere fornito un manuale d'utilizzo per l'utente & Capitolato \\
R1Q2.1 & Obbligatorio & Deve essere fornito un manuale d'utilizzo per l'utente in lingua TODO & Capitolato \\
R1Q3 & Obbligatorio & Deve essere fornito un manuale per lo sviluppatore, per il deploy e l'esecuzione del sistema & Capitolato \\
R1Q3.1 & Obbligatorio & Deve essere fornito un manuale per lo sviluppatore in lingua TODO & Capitolato \\
R1Q4 & Obbligatorio & Il codice sorgente del prodotto deve essere pubblicato e versionato usando GitHub o GitLab & Capitolato \\
R1Q5 & Obbligatorio & Il prodotto verrà distribuito con la licenza MIT & Capitolato \\
R1Q6 & Obbligatorio & ESLint deve essere usato per l'intero sviluppo del progetto, per controllare la qualità del codice & Capitolato \\

\end{longtable}

}
\subsection{Requisiti di vincolo}

{

\rowcolors{2}{azzurro2}{azzurro3}

\centering
\renewcommand{\arraystretch}{2}
\begin{longtable}{C{2.5cm} C{2.8cm} C{6.7cm} C{2.5cm}}
\caption{Tabella dei Requisiti di vincolo}\\
\rowcolor{azzurro1}
\textbf{Codice} &
\textbf{Classificazione}&
\textbf{Descrizione}&
\textbf{Fonti}\\
\endfirsthead
\rowcolor{white}
\caption*{Tabella dei Requisiti di vincolo (continuazione)}\\
\rowcolor{azzurro1}
\textbf{Codice} &
\textbf{Classificazione}&
\textbf{Descrizione}&
\textbf{Fonti}\\
\endhead


R1V1 & Obbligatorio & Deve essere implementato il modulo ad alto livello EML-FE & Capitolato \\
R1V1.1 & Obbligatorio & EML-FE viene implementato usando il framework Next.js\ped{G} & Capitolato \\
R1V1.2 & Obbligatorio & EML-FE viene implementato usando come linguaggio principale Typescript\ped{G} & Capitolato \\
R1V1.3 & Obbligatorio & In EML-FE devono essere pre-renderizzate le pagine HTML\ped{G}. Il pre-rendering può essere di tipo Server-side rendering\ped{G} o Static Generation\ped{G} & Capitolato \\


R1V2 & Obbligatorio & Deve essere implementato il modulo ad alto livello EML-BE & Capitolato \\
R1V2.1 & Obbligatorio & EML-BE viene implementato usando il framework Serverless\ped{G} & Capitolato \\
R1V2.2 & Obbligatorio & EML-BE viene implementato usando come linguaggio principale Typescript\ped{G} & Capitolato \\
R1V2.3 & Obbligatorio & EML-BE viene distribuito usando AWS Lambda\ped{G} come unica unità computazionale & Capitolato \\


R1V3 & Obbligatorio & Deve essere implementato il modulo ad alto livello EML-I & Capitolato \\
R1V3.1 & Obbligatorio & EML-I viene implementato usando il framework Serverless\ped{G} & Capitolato \\
R1V3.2 & Obbligatorio & EML-I viene implementato usando come linguaggio principale Typescript\ped{G} & Capitolato \\

R1V4 & Obbligatorio & Deve essere implementato il modulo ad alto livello EML-MON & Capitolato \\
R1V4.1 & Obbligatorio & EML-MON viene implementato usando Amazon CloudWatch\ped{G} o  Datadog\ped{G} & Capitolato \\

R1V5 & Obbligatorio & Deve essere utilizzato Stripe\ped{G} come provider della gestione dei pagamenti nel sito & Capitolato \\

R3V6 & Facoltativo & Si può utilizzare il Content Management System (CMS)\ped{G} Contentful\ped{G} per gestire il codice del prodotto senza fare deploy\ped{G} completi per ogni modifica di codice & Capitolato \\

R1V7 & Obbligatorio & I ruoli nel sito sono modellati e implementati usando AWS Cognito\ped{G} o Auth0\ped{G} come Identity manager & Capitolato \\

R1V8 & Obbligatorio & Il prodotto viene sviluppato con l'ultima versione di Typescript\ped{G} & Capitolato \\
R1V8.1 & Obbligatorio & Il codice Typescript\ped{G} viene sviluppato utilizzando l'approccio promise\ped{G}/async-await\ped{G} & Capitolato \\

R1V9 & Obbligatorio & I servizi implementati nel framework Serverless\ped{G} vengono costruiti come API\ped{G} & Capitolato \\

R1V10 & Obbligatorio & Red Babel deve essere creditata nel file README del prodotto & Capitolato \\

R1V11 & Obbligatorio & Devono essere usati degli enviroment per lo sviluppo del prodotto & Capitolato \\
R1V11.1 & Obbligatorio & Deve essere utilizzato l'enviroment ``Local'' & Capitolato \\
R1V11.2 & Obbligatorio & Deve essere utilizzato l'enviroment ``Test'' & Capitolato \\
R1V11.2.1 & Obbligatorio & L'enviroment ``Test'' deve essere disponibile in AWS\ped{G} & Capitolato \\
R1V11.3 & Obbligatorio & Deve essere utilizzato l'enviroment ``Staging'' & Capitolato \\
R1V11.3.1 & Obbligatorio & L'enviroment ``Staging'' deve essere disponibile in AWS\ped{G} e pubblico & Capitolato \\
R3V11.4 & Facoltativo & Può essere utilizzato l'enviroment ``Production'' & Capitolato \\

R1V11.5 & Obbligatorio & L'Identity manager deve essere presente negli enviroment Local, Test e Staging & Capitolato \\
R1V11.6 & Obbligatorio & Il Payment service deve essere presente negli enviroment Test e Staging & Capitolato \\
R3V11.7 & Facoltativo & Il CMS\ped{G} deve essere presente negli enviroment Local, Test e Staging & Capitolato \\

R1V12 & Obbligatorio & L'applicazione viene presentata in lingua inglese & VE\_2021-01-05 \\
R3V13 & Facoltativo & L'applicazione viene presentata in lingua italiana & VE\_2021-01-05 \\

\end{longtable}

}
\subsection{Requisiti prestazionali}
Non sono stati richiesti e individuati alcuni requisiti prestazionali per lo sviluppo dell'applicazione. L'unico requisito che può influenzare le prestazioni è il \textit{R1V8.1}, cioè l'uso dell'approccio promise\ped{G}/async-await\ped{G}, che ci permette di creare funzioni asincrone e concorrenti in Typescript\ped{G}, facendo uso della classe promise\ped{G}. Questo è un requisito di vincolo obbligatorio, ma misurare l'aumento di prestazioni e creare un requisito prestazionale da soddisfare ci risulta troppo complicato e fuori dalla nostra portata.
>>>>>>> NormeDiProgetto
