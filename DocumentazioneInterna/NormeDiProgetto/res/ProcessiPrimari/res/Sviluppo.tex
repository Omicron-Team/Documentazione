\subsection{Sviluppo}
\subsubsection{Scopo}
Lo scopo del processo di Sviluppo è quello di descrivere i compiti e le attività di analisi, progettazione, codifica, integrazione, test, installazione ed accettazione del prodotto seguendo lo standard ISO/IEC 12207:1995.

\subsubsection{Descrizione}
Elenchiamo di seguite le attività che compongono il processo di sviluppo:
\begin{itemize}
    \item{\AdR{};}
    \item{Progettazione;}
    \item{Codifica del software;}
\end{itemize}

\subsubsection{Aspettative}
Le aspettative del gruppo sono:
\begin{itemize}
    \item{Fissare gli obiettivi di sviluppo;}
    \item{Fissare vincoli tecnologici;}
    \item{Fissare vincoli di design;}
    \item{Realizzare un prodotto finale che supera i test, che soddisfa i requisiti e le richieste del proponente;}
\end{itemize}

\subsubsection{Attività}
\paragraph{\AdR{}}
\subparagraph{Scopo} L'biettivo dell'\AdRv{} è quello di definire i requisiti preposti dall'\proponProg{}. È quindi necessario che gli \analProg
\begin{itemize}
  %  \item{Fornire ai progettisti indicazioni precise ed affidabili;}
    %\item{Fissare i requisiti concordati con il \proponProg;}
    \item{;}
    \item{;}
    \item{;}
    \item{;}
\end{itemize}

\myparagraph{Casi d'uso}
Rappresenta una situazione nella quale il sistema viene utilizzato per soddisfare uno o più bisogni dell'utente.\\
La loro struttura è la seguente:
\begin{itemize}
\item \textbf{Identificatore}: codice che identifica il caso d'uso. Il suo formato è il seguente:
\begin{center}
\textbf{UC[CodicePadre].\{CodiceFiglio\}}\\
\end{center}
dove:
\begin{itemize}
\item \textbf{UC}: rappresenta un acronimo di "Use Case", ovvero "Caso d'uso";
\item \textbf{CodicePadre}: rappresenta il codice di un caso d'uso generico;
\item \textbf{CodiceFiglio} (opzionale): rappresenta il codice di un sotto caso di un caso d'uso;
\end{itemize}
\item \textbf{Nome}: Titolo del caso d'uso;
\item \textbf{Diagramma UML}\ped{G} \textbf{dei casi d'uso}: rappresentazione grafica del caso d'uso utilizzando il linguaggio UML\ped{G} 2.0; 
\item \textbf{Descrizione}: descrizione breve del caso d'uso;
\item \textbf{Scenario principale}: sequenza di azioni che porta al risultato atteso dal caso d'uso;
\item \textbf{Scenari alternativi}(opzionale): sequenza di azioni che porta ad un risultato inatteso (un errore o un'eccezione) dal caso d'uso;
\item \textbf{Precondizioni}: condizioni necessarie per il corretto avvio del caso d'uso;
\item \textbf{Post-condizioni}: condizioni che devono essere vere dopo che il caso d'uso è andato a buon fine;
\item \textbf{Attori primari}: utente che inizia un'interazione con il sistema per raggiungere un obiettivo;
\item \textbf{Attori secondari}(opzionale): entità che aiuta l'attore primario a raggiungere il suo obiettivo;
\item \textbf{Estensioni}(opzionale): aumenta le funzionalità di uno use case. Ogni istanza di un caso d'uso 'A' esegue un evento 'B' in modo condizionato. L'esecuzione di 'B' interrompe 'A';
\item \textbf{Inclusioni}(opzionale): include un'istanza di un use case 'B' in un altro use case 'A';
\item \textbf{Specializzazioni}(opzionale): casi d'uso che aggiungono o modificano le caratteristiche base di un caso d'uso di partenza.
\end{itemize}
\myparagraph{Requisiti}
Comprendono le competenze necessarie per svolgere correttamente il progetto.\\
Ogni requisito è composto da:
\begin{itemize}
\item \textbf{Identificatore}: codice che identifica un requisito. Il suo formato è il seguente:
\begin{center}
\textbf{R[Importanza][Tipologia][CodicePadre].\{CodiceFiglio\}}\\
\end{center}
dove:
\begin{itemize}
\item \textbf{Importanza}: rappresenta l'importanza del requisito e può essere:
\begin{itemize}
\item \textbf{1 (Obbligatorio)}: irrinunciabile per gli stakeholder;
\item \textbf{2 (Desiderabile)}: non strettamente necessario ma a valore aggiunto riconoscibile;
\item \textbf{3 (Facoltativo)}: relativamente utile oppure contrattabile più avanti nel progetto;
\end{itemize}
\item \textbf{Tipologia}: rappresenta il tipo di requisito e può essere:
\begin{itemize}
\item \textbf{V (Vincolo)}: descrive i vincoli sui servizi offerti dal sistema;
\item \textbf{F (Funzionale)}: descrive le funzioni che il sistema deve realizzare;
\item \textbf{P (Prestazionale)}: descrive i vincoli prestazionali che il sistema;
\item \textbf{Q (Qualità)}: descrive i vincoli di qualità che il sistema deve avere;
\end{itemize}
\item \textbf{CodicePadre}: rappresenta il codice di un requisito generico;
\item \textbf{CodiceFiglio} (opzionale): rappresenta il codice di un sotto caso di requisito;
\end{itemize}
\item \textbf{Descrizione}: riporta una breve descrizione del requisito;
\item \textbf{Classificazione}: riporta l'importanza del requisito;
\item \textbf{Fonte}: riporta da dove deriva il requisito. Il requisito può derivare da:
\begin{itemize}
\item \textbf{Capitolato}: requisito individuato dalla lettura del capitolato d'appalto.
\item \textbf{Interno}: requisito individuato dagli analisti e ritenuto opportuno.
\item \textbf{Verbale}: requisito individuato a seguito di una discussione tra i membri del gruppo di progetto o con il proponente. Verrà riportato, inoltre, il codice di riferimento del verbale.
\item \textbf{Caso d'uso}: requisito estrapolato da un caso d'uso.
\end{itemize}
\end{itemize}
\subsubsection{Progettazione}
\myparagraph{Scopo}
L'attività di progettazione precede la codifica e ha il compito di individuare i requisiti software richiesti (analizzando l' \AdRv{2.0.0}) per far si che il prodotto finale soddisfi tutti gli stakeholder\ped{G}. Per far ciò lo sviluppo del prodotto deve essere:
\begin{itemize}
\item \textbf{Efficiente}: prodotto in economia, minimizzando le risorse utilizzate;
\item \textbf{Efficace}: deve garantire la qualità del prodotto perseguendo la correttezza per costruzione;
\item \textbf{Organizzato}: i compiti devono essere suddivisi tra i vari membri del gruppo in modo da ridurre la complessità del problema.
\end{itemize}

\myparagraph{Aspettative}
Il gruppo \Omicron{} intende, tramite l'attività di progettazione, fissare l'architettura del prodotto prima di passare alla sua realizzazione. 

\myparagraph{Descrizione}
Le parti fondamentali del processo di progettazione sono due:
\begin{itemize}
\item \textbf{Technology baseline}: è redatta dal progettista e contiene:
\begin{itemize}
\item specifiche della progettazione ad alto livello del prodotto e delle sue componenti;
\item elenco dei diagrammi UML\ped{G} che saranno utilizzati per la realizzazione dell'architettura e dei test di verifica.
\end{itemize}
\item \textbf{Product baseline}: è redatta dal progettista e ha il compito di:
\begin{itemize}
\item integrare ciò che è riportato nella Technology baseline;
\item definire i test di verifica.
\end{itemize}
\end{itemize} 

\myparagraph{Technology baseline}
Le \textbf{Technology baselines} includono:
\begin{itemize}
\item \textbf{Diagrammi UML\ped{G}}:
\begin{itemize}
\item \textbf{diagrammi delle classi}: descrivono il tipo degli oggetti che compongono il sistema e le relazioni statiche esistenti tra loro;
\item \textbf{diagrammi dei package}: documentano le dipendenze fra le classi;
\item \textbf{diagrammi di attività}: descrivono le azioni di un processo;
\item \textbf{diagrammi di sequenza}: descrive uno scenario, ovvero una serie di azioni in cui tutte le scelte sono già effettuate.
\end{itemize}
\item \textbf{Design Pattern\ped{G}}: soluzioni progettuali generali ad un problema ricorrente. Devono essere descritti e rappresentati con un diagramma che ne espone significato e struttura;
\item \textbf{Tecnologie utilizzate}: descrizione delle tecnologie usate. Ne devono essere descritti l'utilizzo, i vantaggi e gli svantaggi;
\item \textbf{Tracciamento delle componenti}: i requisiti vengono soddisfatti da alcune componenti, di cui vi è necessità di tener traccia;
\item \textbf{Test di integrazione}: test il cui scopo è assicurarsi il corretto funzionamento del progetto una volta che sono state unite le sue componenti.
\end{itemize}

\myparagraph{Product baseline}
Le \textbf{Product baselines} includono:
\begin{itemize}
\item \textbf{definizione delle classi}: di ogni classe è necessario che si possano comprendere in modo esaustivo lo scopo e le funzionalità, evitando ridondanze;
\item \textbf{tracciamento delle classi}: ogni requisito deve avere una classe che lo soddisfi; 
\item \textbf{test di unità}: test il quale scopo è assicurarsi il corretto funzionamento delle singole componenti del progetto.
\end{itemize}
\subsubsection{Codifica}
\myparagraph{Scopo}
L'attività di codifica ha lo scopo di normare l'effettiva realizzazione del prodotto software partendo dall'architettura realizzata durante la fase di progettazione. È compito del programmatore assicurarsi il corretto svolgimento di questa attività.

\myparagraph{Aspettative}
L'obiettivo del gruppo \Omicron{} è quello di creare, tramite l'attività di codifica, un prodotto software che sia in grado di soddisfare i requisiti prefissati con il proponente assicurandosi, usando delle norme e delle convenzioni, di:
\begin{itemize}
\item generare codice leggibile ed uniforme;
\item agevolare le fasi di manutenzione, verifica e validazione;
\item mantenere un prodotto di ottima qualità. 
\end{itemize}

\myparagraph{Descrizione}
Il codice scritto dovrà essere di qualità, rispettando e perseguendo gli obiettivi di qualità definiti nel \PdQv.

\myparagraph{Stile di codifica}
Il codice generato da ogni membro deve rispettare le seguenti norme:
\begin{itemize}
\item \textbf{indentazione}: Ogni blocco innestato deve essere indentato usando 2 spazi per ogni livello di indentazione;
\item \textbf{parentesizzazione}: le parentesi di delimitazione dei costrutti devono essere inserite in linea e non al di sotto di essi;
\item \textbf{scrittura dei metodi}: è preferibile l'uso di metodi brevi (con poche righe di codice);
\item \textbf{univocità dei nomi}: il nome di classi, metodi, variabili deve essere univoco e autoesplicativo;
\item \textbf{struttura dei nomi}: I nomi sono strutturati nel seguente modo:
\begin{itemize}
\item \textit{metodi, variabili e costanti}: sono scritti con la prima lettera minuscola. Se il nome è composto da più parole quelle successive iniziano con la lettera maiuscola (camel Case);
\item \textit{classi}: sono scritti con la prima lettera maiuscola. Se il nome è composto da più parole quelle successive iniziano con la lettera maiuscola (Pascal Case).
\end{itemize} 
\item \textbf{lingua}: il codice e i commenti saranno scritti in lingua italiana.
\end{itemize}

\myparagraph{Ricorsione}
L'uso della ricorsione va evitato se non in casi dove risulta essere strettamente necessaria.

\myparagraph{Strumenti}
Per il processo di sviluppo, durante il progetto, sono stati usati i seguenti strumenti:
\begin{itemize}
\item \textbf{Draw.io}: piattaforma online usata per creare diagrammi di vario tipo. Nel nostro caso verrà usata per produrre diagrammi UML\ped{G};
\item \textbf{ESLint\ped{G}}: strumento di analisi statica del codice. Verrà usato per assicurarsi che il codice venga scritto rispettando i requisiti sopra imposti (§ 2.2.6.4);
\item \textbf{Vercel\ped{G}}: piattaforma cloud\ped{G} che consente di ospitare gratuitamente siti web sviluppati con Next.js\ped{G};
\item \textbf{AWS Lambda\ped{G}}: servizio di calcolo serverless facente parte di Amazon Web Services\ped{G}(AWS);
\item \textbf{AWS DynamoDB\ped{G}}: servizio di database NoSQL\ped{G} facente parte dei servizi offerti da AWS\ped{G};
\item \textbf{AWS Cognito\ped{G}}: servizio che fornisce autenticazione, autorizzazione e gestione degli utenti per applicazioni Web e mobile, facente parte dei servizi offerti da AWS\ped{G};
\item \textbf{AWS API Gateway\ped{G}}: servizio di AWS\ped{G} che permette la creazione, la pubblicazione, la gestione, il monitoraggio e la protezione di API\ped{G} REST\ped{G}, HTTP\ped{G} e WebSocket;
\item \textbf{Prettier\ped{G}}: questo tool permette la formattazione del codice prodotto in maniera specifica e automatica. Sarà possibile specificare delle regole di formattazione che verranno applicate al nostro codice;
\item \textbf{Next.js\ped{G}}: si tratta di un framework JavaScript per applicazioni React, e consente il rendering automatico lato server (SSR, server side rendering\ped{G});
\item \textbf{Serverless\ped{G}}: framework per la codifica di funzioni AWS Lambda\ped{G} ed effettuare deploy nei vari servizi AWS\ped{G}. Permette di creare un intero servizio di back-end.
\end{itemize} 

