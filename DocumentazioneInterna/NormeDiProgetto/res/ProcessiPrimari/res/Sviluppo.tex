\subsection{Sviluppo}
\subsubsection{Scopo}
Lo scopo del processo di sviluppo, secondo lo standard\ped{G} ISO/IEC 12207:1995, è di descrivere le attività necessarie allo sviluppo del prodotto software.

\subsubsection{Descrizione}
Elenchiamo di seguito le attività che compongono il processo di sviluppo:
\begin{itemize}
    \item analisi dei requisiti;
    \item progettazione;
    \item codifica del software.
\end{itemize}

\subsubsection{Aspettative}
Le aspettative del gruppo riguardo al processo sono:
\begin{itemize}
    \item fissare gli obiettivi di sviluppo;
    \item fissare vincoli tecnologici;
    \item fissare vincoli di design;
    \item realizzare un prodotto finale che supera i test, che soddisfa i requisiti e le richieste del proponente.
\end{itemize}

\subsubsection{\AdR{}}
\myparagraph{Scopo} 
L'obiettivo dell'\AdRv{}è quello di definire i requisiti preposti dal \proponProg{}. È quindi necessario che gli analisti stabiliscano questi requisiti con l'obiettivo di:
\begin{itemize}
    \item definire lo scopo del lavoro;
    \item fornire ai progettisti riferimenti precisi ed affidabili;
    \item determinare funzioni e requisiti con il \proponProg;
    \item fornire una base per garantire successivi miglioramenti del prodotto;
    \item fornire ai verificatori indicazioni per l'attività di controllo dei test;
    \item calcolare la quantità di lavoro per poter avere una stima dei costi.
\end{itemize}

\paragraph{Casi d'uso}
Rappresenta una situazione nella quale il sistema viene utilizzato per soddisfare uno o più bisogni dell'utente.\\
La loro struttura è la seguente:
\begin{itemize}
\item \textbf{Identificatore}: codice che identifica il caso d'uso. Il suo formato è il seguente:
\begin{center}
\textbf{UC[CodicePadre].\{CodiceFiglio\}}\\
\end{center}
dove:
\begin{itemize}
\item \textbf{UC}: rappresenta un acronimo di "Use Case", ovvero "Caso d'uso";
\item \textbf{CodicePadre}: rappresenta il codice di un caso d'uso generico;
\item \textbf{CodiceFiglio} (opzionale): rappresenta il codice di un sotto caso di un caso d'uso;
\end{itemize}
\item \textbf{Nome}: Titolo del caso d'uso;
\item \textbf{Diagramma UML}\ped{G}: rappresentazione grafica del caso d'uso utilizzando il linguaggio UML\ped{G} 2.0; 
\item \textbf{Descrizione}: descrizione breve del caso d'uso;
\item \textbf{Scenario principale}: sequenza di azioni che porta al risultato atteso dal caso d'uso;
\item \textbf{Scenari alternativi}(opzionale): sequenza di azioni che porta ad un risultato inatteso (un'errore o un'eccezione) dal caso d'uso;
\item \textbf{Precondizioni}: condizioni necessarie per il corretto avvio del caso d'uso;
\item \textbf{Post-condizioni}: condizioni che devono essere vere dopo che il caso d'uso è andato a buon fine;
\item \textbf{Triggers}: eventi che portano all'avvio del caso d'uso;
\item \textbf{Attori primari}: utente che inizia un interazione con il sistema per raggiungere un obiettivo;
\item \textbf{Attori secondari}: entità che aiuta l'attore primario a raggiungere il suo obiettivo;
\item \textbf{Estensioni}(opzionale): aumenta le funzionalità di uno use case. Ogni istanza di un use case 'A' esegue un evento 'B' in modo condizionato. L'esecuzione di 'B' interrompe 'A';
\item \textbf{Inclusioni}(opzionale): include un'istanza di un use case 'B' in un altro use case 'A';
\item \textbf{Generalizzazioni}(opzionale): aggiunge o modifica le caratteristiche base di uno use case. Crea così delle specializzazioni dello use case di partenza.
\end{itemize}
\section{Requisiti}

\subsection{Introduzione} 

Sono stati classificati e assegnati i requisiti in base a quanto definito nel documento \NdPv{1.0.0} sezione §2.2.4.3

\subsection{Requisiti funzionali} 


{

\rowcolors{2}{azzurro2}{azzurro3}

\centering
\renewcommand{\arraystretch}{2}
\begin{longtable}{C{2.5cm} C{2.8cm} C{6.7cm} C{2.5cm}}
\caption{Tabella dei Requisiti funzionali}\\
\rowcolor{azzurro1}
\textbf{Codice} &
\textbf{Classificazione}&
\textbf{Descrizione}&
\textbf{Fonti}\\
\endhead

%REGISTRAZIONE

R1F1 & Obbligatorio & Un utente non autenticato può registrarsi & Capitolato \\
R1F1.1 & Obbligatorio & La registrazione necessita l'inserimento dei seguenti dati personali: nome, cognome e indirizzo di fatturazione & Interno \\
R1F1.2 & Obbligatorio & La registrazione necessita l'inserimento della email & VE\_2020-12-23 \\
R2F1.2.1 & Desiderabile & Il sistema deve mostrare un errore se l'email inserita nella registrazione è già stata utilizzata & Interno \\
R1F1.3 & Obbligatorio & La registrazione necessita l'inserimento di una password & Interno \\

%LOGIN

R1F2 & Obbligatorio & Un utente non autenticato può effettuare il login & Capitolato \\
R1F2.1 & Obbligatorio & Il login necessita l'inserimento della email & Interno \\
R1F2.2 & Obbligatorio & Il login necessita l'inserimento della password & Interno \\
R2F2.3 & Desiderabile & Il sistema deve mostrare un errore se le credenziali del login sono errate & Interno \\

%LOGOUT

R1F3 & Obbligatorio & Un utente autenticato può effettuare il logout & Interno \\

%CARRELLO

R1F4 & Obbligatorio & Un utente può visualizzare tutti i prodotti che ha precedentemente aggiunto al carrello & Capitolato \\
R1F4.1 & Obbligatorio & Un utente può visualizzare per ogni prodotto del carrello il nome e l’immagine & Interno \\
R1F4.2 & Obbligatorio & Un utente può rimuovere i singoli prodotti dal carrello & Interno \\
R1F3.3 & Obbligatorio & Un utente può modificare le quantità di ogni prodotto nel carrello & Interno \\
R1F4.4 & Obbligatorio & Un utente può visualizzare il costo totale dei prodotti nel carrello & Capitolato \\
R1F4.4.1 & Obbligatorio & Un utente può visualizzare il costo di ogni voce del carrello & Interno \\
R1F4.5 & Obbligatorio & Un utente può visualizzare le tasse applicate al costo totale dei prodotti nel carrello & Capitolato \\

%CHECKOUT

R1F5 & Obbligatorio & Un utente registrato può procedere al checkout dei prodotti presenti nel carrello & Capitolato \\
R1F5.1 & Obbligatorio & Per iniziare il processo di checkout l’utente deve avere almeno un prodotto nel carrello & Interno \\
R1F5.2 & Obbligatorio & Un utente che ha iniziato il processo di checkout deve inserire i dati del pagamento & Interno \\
R1F5.3 & Obbligatorio & Un utente, dopo aver inserito i dati del pagamento, può continuare al pagamento effettivo & Capitolato \\
R1F5.4 & Obbligatorio & A pagamento riuscito l’utente può visualizzare un riepilogo dell’ordine effettuato & Capitolato \\
R1F5.4.1 & Obbligatorio & A pagamento riuscito l’utente riceverà i prodotti acquistati tramite l’email del suo account & Capitolato \\
R1F5.4.2 & Obbligatorio & A pagamento fallito l’utente visualizza un messaggio di errore, e potrà riprovare il pagamento verificando i dati inseriti & Capitolato \\

%PROFILO



%MERCHANT DASHBOARD

%HOMEPAGE

%PLP

%PDP




\end{longtable}

}
\subsection{Requisiti di qualità}

{

\rowcolors{2}{azzurro2}{azzurro3}

\centering
\renewcommand{\arraystretch}{2}
\begin{longtable}{C{2.5cm} C{2.8cm} C{6.7cm} C{2.5cm}}
\caption{Tabella dei Requisiti di qualità}\\
\rowcolor{azzurro1}
\textbf{Codice} &
\textbf{Classificazione}&
\textbf{Descrizione}&
\textbf{Fonti}\\
\endhead


R1Q1 & Obbligatorio & Il prodotto deve essere sviluppato rispettando quanto scritto nel documento \NdPv{1.0.0} e \PdQv{1.0.0} & Interno \\
R1Q2 & Obbligatorio & Deve essere fornito un manuale d'utilizzo per l'utente & Capitolato \\
R1Q2.1 & Obbligatorio & Deve essere fornito un manuale d'utilizzo per l'utente in lingua TODO & Capitolato \\
R1Q3 & Obbligatorio & Deve essere fornito un manuale per lo sviluppatore, per il deploy e l'esecuzione del sistema & Capitolato \\
R1Q3.1 & Obbligatorio & Deve essere fornito un manuale per lo sviluppatore in lingua TODO & Capitolato \\
R1Q4 & Obbligatorio & Il codice sorgente del prodotto deve essere pubblicato e versionato usando GitHub o GitLab & Capitolato \\
R1Q5 & Obbligatorio & Il prodotto verrà distribuito con la licenza MIT & Capitolato \\
R1Q6 & Obbligatorio & ESLint deve essere usato per l'intero sviluppo del progetto, per controllare la qualità del codice & Capitolato \\

\end{longtable}

}
\subsection{Requisiti di vincolo}

{

\rowcolors{2}{azzurro2}{azzurro3}

\centering
\renewcommand{\arraystretch}{2}
\begin{longtable}{C{2.5cm} C{2.8cm} C{6.7cm} C{2.5cm}}
\caption{Tabella dei Requisiti di vincolo}\\
\rowcolor{azzurro1}
\textbf{Codice} &
\textbf{Classificazione}&
\textbf{Descrizione}&
\textbf{Fonti}\\
\endfirsthead
\rowcolor{white}
\caption*{Tabella dei Requisiti di vincolo (continuazione)}\\
\rowcolor{azzurro1}
\textbf{Codice} &
\textbf{Classificazione}&
\textbf{Descrizione}&
\textbf{Fonti}\\
\endhead


R1V1 & Obbligatorio & Deve essere implementato il modulo ad alto livello EML-FE & Capitolato \\
R1V1.1 & Obbligatorio & EML-FE viene implementato usando il framework Next.js\ped{G} & Capitolato \\
R1V1.2 & Obbligatorio & EML-FE viene implementato usando come linguaggio principale Typescript\ped{G} & Capitolato \\
R1V1.3 & Obbligatorio & In EML-FE devono essere pre-renderizzate le pagine HTML\ped{G}. Il pre-rendering può essere di tipo Server-side rendering\ped{G} o Static Generation\ped{G} & Capitolato \\


R1V2 & Obbligatorio & Deve essere implementato il modulo ad alto livello EML-BE & Capitolato \\
R1V2.1 & Obbligatorio & EML-BE viene implementato usando il framework Serverless\ped{G} & Capitolato \\
R1V2.2 & Obbligatorio & EML-BE viene implementato usando come linguaggio principale Typescript\ped{G} & Capitolato \\
R1V2.3 & Obbligatorio & EML-BE viene distribuito usando AWS Lambda\ped{G} come unica unità computazionale & Capitolato \\


R1V3 & Obbligatorio & Deve essere implementato il modulo ad alto livello EML-I & Capitolato \\
R1V3.1 & Obbligatorio & EML-I viene implementato usando il framework Serverless\ped{G} & Capitolato \\
R1V3.2 & Obbligatorio & EML-I viene implementato usando come linguaggio principale Typescript\ped{G} & Capitolato \\

R1V4 & Obbligatorio & Deve essere implementato il modulo ad alto livello EML-MON & Capitolato \\
R1V4.1 & Obbligatorio & EML-MON viene implementato usando Amazon CloudWatch\ped{G} o  Datadog\ped{G} & Capitolato \\

R1V5 & Obbligatorio & Deve essere utilizzato Stripe\ped{G} come provider della gestione dei pagamenti nel sito & Capitolato \\

R3V6 & Facoltativo & Si può utilizzare il Content Management System (CMS)\ped{G} Contentful\ped{G} per gestire il codice del prodotto senza fare deploy\ped{G} completi per ogni modifica di codice & Capitolato \\

R1V7 & Obbligatorio & I ruoli nel sito sono modellati e implementati usando AWS Cognito\ped{G} o Auth0\ped{G} come Identity manager & Capitolato \\

R1V8 & Obbligatorio & Il prodotto viene sviluppato con l'ultima versione di Typescript\ped{G} & Capitolato \\
R1V8.1 & Obbligatorio & Il codice Typescript\ped{G} viene sviluppato utilizzando l'approccio promise\ped{G}/async-await\ped{G} & Capitolato \\

R1V9 & Obbligatorio & I servizi implementati nel framework Serverless\ped{G} vengono costruiti come API\ped{G} & Capitolato \\

R1V10 & Obbligatorio & Red Babel deve essere creditata nel file README del prodotto & Capitolato \\

R1V11 & Obbligatorio & Devono essere usati degli enviroment per lo sviluppo del prodotto & Capitolato \\
R1V11.1 & Obbligatorio & Deve essere utilizzato l'enviroment ``Local'' & Capitolato \\
R1V11.2 & Obbligatorio & Deve essere utilizzato l'enviroment ``Test'' & Capitolato \\
R1V11.2.1 & Obbligatorio & L'enviroment ``Test'' deve essere disponibile in AWS\ped{G} & Capitolato \\
R1V11.3 & Obbligatorio & Deve essere utilizzato l'enviroment ``Staging'' & Capitolato \\
R1V11.3.1 & Obbligatorio & L'enviroment ``Staging'' deve essere disponibile in AWS\ped{G} e pubblico & Capitolato \\
R3V11.4 & Facoltativo & Può essere utilizzato l'enviroment ``Production'' & Capitolato \\

R1V11.5 & Obbligatorio & L'Identity manager deve essere presente negli enviroment Local, Test e Staging & Capitolato \\
R1V11.6 & Obbligatorio & Il Payment service deve essere presente negli enviroment Test e Staging & Capitolato \\
R3V11.7 & Facoltativo & Il CMS\ped{G} deve essere presente negli enviroment Local, Test e Staging & Capitolato \\

R1V12 & Obbligatorio & L'applicazione viene presentata in lingua inglese & VE\_2021-01-05 \\
R3V13 & Facoltativo & L'applicazione viene presentata in lingua italiana & VE\_2021-01-05 \\

\end{longtable}

}
\subsection{Requisiti prestazionali}
Non sono stati richiesti e individuati alcuni requisiti prestazionali per lo sviluppo dell'applicazione. L'unico requisito che può influenzare le prestazioni è il \textit{R1V8.1}, cioè l'uso dell'approccio promise\ped{G}/async-await\ped{G}, che ci permette di creare funzioni asincrone e concorrenti in Typescript\ped{G}, facendo uso della classe promise\ped{G}. Questo è un requisito di vincolo obbligatorio, ma misurare l'aumento di prestazioni e creare un requisito prestazionale da soddisfare ci risulta troppo complicato e fuori dalla nostra portata.
\myparagraph{Metriche}
Per la valutazione del prodotto del processo di Analisi dei Requisiti, il gruppo adotta le seguenti metriche:
\begin{itemize}
	\item \textbf{Percentuale di requisiti soddisfatti}: al fine di assicurarsi che il prodotto realizzato o che sta venendo realizzando ricopra i requisiti necessari verrà usata la seguente metrica:
\begin{center}
\[PRS = \left(\frac{Requisiti\ soddisfatti}{Requisiti\ totali}\right)*100\]
\end{center}
la quale restituisce la percentuale di requisiti che il prodotto soddisfa rispetto a quelli totali;

	\item \textbf{Percentuale di requisiti obbligatori soddisfatti}: la metrica presentata nel punto precedente può essere facilmente adattata ai requisiti obbligatori:
\begin{center}
\[PROS=\left(\frac{Requisiti\ obbligatori\ soddisfatti}{Requisiti\ obbligatori\ totali}\right)*100\]
\end{center}
\end{itemize}



\subsection{Progettazione architetturale}
\textit{\textbf{Periodo}: dal 2021-01-18 al 2021-03-08}

L'inizio di questa fase coincide con la data della Revisione dei Requisiti e si conclude con la scadenza della Revisione di Progettazione.

\subsubsection{Attività}

\begin{itemize}
\item \textbf{Incremento e verifica documenti}: vengono realizzate le aggiunte necessarie ai documenti e le eventuali correzioni provenienti dalle segnalazioni del committente o da analisi interne. I documenti in questione sono:
\begin{itemize}
\item \NdP{};
\item \AdR{};
\item \PdQ{};
\item \PdP{};
\item Glossario.
\end{itemize}
\item \textbf{Technology Baseline\ped{G}}: viene realizzato un \textit{Proof of Concept}\ped{G} per testare le tecnologie coinvolte e per provare che le funzionalità base del prodotto possano effettivamente essere implementate, soddisfacendo i requisiti collegati. Esso verrà condiviso con il proponente, per verificare che sia soddisfacente, e con il \CR{}.\\ Come precedentemente notato, nella sezione \S{3.2}, per la realizzazione del \textit{Proof of Concept}\ped{G} sono stati individuati quattro incrementi, che suddividono lo sviluppo in  aree di interesse distinte. Per ogni incremento vengono riportati i requisiti che ci impegniamo a soddisfare nel \textit{Proof of Concept}\ped{G} (non comprendendo i requisiti figlio):
\begin{itemize}
\item \textbf{Incremento 1}: R1F1;
\item \textbf{Incremento 2}: R1F7.1, R1F9.1;
\item \textbf{Incremento 3}: R1F4;
\item \textbf{Incremento 4}: R1F11.
\end{itemize}
Questa lista di requisiti non esclude il fatto che possano essere soddisfatti altri requisiti aggiuntivi.
\item \textbf{Consolidamento}: viene realizzata la presentazione da esporre in sede di Revisione di Progettazione e si approfondiscono aspetti lacunari riguardo il progetto.
\end{itemize}

\subsubsection{Periodi}

\begin{itemize}
\item \textbf{Periodo 1}: \textit{dal 2021-01-18 al 2021-01-31}. \\
Viene svolto un approfondimento personale da ogni membro del gruppo riguardo le tecnologie da utilizzare per lo sviluppo del prodotto. Inoltre, se necessario, verranno corretti i documenti realizzati nella fase di analisi.
\item \textbf{Periodo 2}: \textit{dal 2021-01-31 al 2021-03-01}. \\
Viene realizzata la Technology Baseline\ped{G}, compresa di un'analisi iniziale degli incrementi e delle tecnologie, la realizzazione degli incrementi e una verifica finale di integrazione. Di seguito vengono riportati i periodi individuati per i singoli incrementi:
\begin{itemize}
\item \textbf{Incremento 1}: \textit{dal 2021-02-04 al 2021-02-09};
\item \textbf{Incremento 2}: \textit{dal 2021-02-09 al 2021-02-14};
\item \textbf{Incremento 3}: \textit{dal 2021-02-14 al 2021-02-19};
\item \textbf{Incremento 4}: \textit{dal 2021-01-19 al 2021-02-24}.
\end{itemize}
Il periodo si conclude con la consegna del materiale per la Revisione di Progettazione;
\item \textbf{Periodo 3}: \textit{dal 2021-03-01 al 2021-03-08}. \\
Viene svolta l'attività di consolidamento. Il periodo si conclude con la Revisione di Progettazione;
\end{itemize}

\subsubsection{Diagramma di Gantt}

\begin{figure}[H]
\centering

\centerline{\includegraphics[scale=0.5]{res/Pianificazione/Gantt/progettazione}}
\caption{Diagramma di Gantt per il periodo di progettazione architetturale}
\end{figure}
\subsection{Progettazione di dettaglio e codifica}
\textit{\textbf{Periodo}: dal 2021-03-08 al 2021-04-09}

L'inizio di questa fase coincide con data della Revisione di Progettazione e conclude con la scadenza della Revisione di Qualifica.

\subsubsection{Attività}

\begin{itemize}
\item \textbf{Incremento e verifica documenti}: vengono realizzate le aggiunte necessarie ai documenti e le eventuali correzioni provenienti dalle segnalazioni dal committente o da analisi interne. I documenti in questione sono:
\begin{itemize}
\item \NdP{};
\item \AdR{};
\item \PdQ{};
\item \PdP{};
\item \textit{Glossario}.
\end{itemize}
\item \textbf{Product Baseline\ped{G}}: viene realizzata la baseline architetturale del prodotto, in base alla Technology Baseline\ped{G}. Il codice sviluppato precedentemente nel Proof of Concept può essere riutilizzato, se ritenuto corretto per il design architetturale individuato. Viene inoltre redatto l'\textit{Allegato tecnico}\ped{G} per essere inviato e presentato al \CR{}.\\ Come precedentemente notato, nella sezione \S{3.2}, per la realizzazione del prodotto sono stati individuati quattro incrementi, che suddividono lo sviluppo in aree di interesse distinte. I requisiti obbligatori di tali incrementi saranno coloro che ci impegneremo a soddisfare in questa fase.
\item \textbf{Manuali}: Durante lo sviluppo della Product Baseline\ped{G} verranno redatti il \MU e il \MM. Il primo servirà per fornire istruzioni per l'utilizzo dell'applicazione, il secondo per fornire informazioni necessarie per il mantenimento e l'ampliamento del prodotto;
\item \textbf{Consolidamento}: viene realizzata la presentazione da esporre in sede di Revisione di Qualifica e si approfondiscono aspetti lacunari riguardo il progetto.
\end{itemize}

\subsubsection{Periodi}

\begin{itemize}
\item \textbf{Periodo 1}: \textit{dal 2021-03-8 al 2021-03-11}. \\
Viene svolto un ulteriore approfondimento personale per lo sviluppo dei requisiti non visti dal Proof of Concept\ped{G}. Inoltre, se necessario, verranno corretti i documenti realizzati nella fase di progettazione.
\item \textbf{Periodo 2}: \textit{dal 2021-03-11 al 2021-04-02}. \\
Viene realizzata la Product Baseline\ped{G}, compresa di \textit{Allegato Tecnico} e di manuali. Di seguito vengono riportati i periodi individuati per i singoli incrementi:
\begin{itemize}
\item \textbf{Incremento 1}: \textit{dal 2021-03-12 al 2021-03-17};
\item \textbf{Incremento 2}: \textit{dal 2021-03-17 al 2021-03-22};
\item \textbf{Incremento 3}: \textit{dal 2021-03-22 al 2021-03-27};
\item \textbf{Incremento 4}: \textit{dal 2021-03-27 al 2021-03-31}.
\end{itemize}
Il periodo conclude con la consegna del materiale per la Revisione di Qualifica;
\item \textbf{Periodo 3}: \textit{dal 2021-04-02 al 2021-04-09}. \\
Viene svolta l'attività di consolidamento. Il periodo conclude con la Revisione di Qualifica;
\end{itemize}

\subsubsection{Diagramma di Gantt}

\begin{figure}[H]
\centering

\centerline{\includegraphics[scale=0.6]{res/Pianificazione/Gantt/codifica}}
\caption{Diagramma di Gantt per il periodo di progettazione di dettaglio e codifica}
\end{figure}

Per migliorare la visualizzazione del diagramma, la pianificazione dei singoli incrementi viene rappresentata dal successivo diagramma, che ha una durata di 5 giorni. L'ultimo incremento è pianificato per solo 4 giorni poiché prevediamo non sia così impegnativo rispetto agli altri, che richiedono molta più codifica.\\

\begin{figure}[H]
\centering

\centerline{\includegraphics[scale=1]{res/Pianificazione/Gantt/incrementoCodifica}}
\caption{Diagramma di Gantt per i singoli incrementi nel periodo di progettazione di dettaglio e codifica}
\end{figure}

\myparagraph{Strumenti}
Per il processo di sviluppo, durante il progetto, sono stati usati i seguenti strumenti:
\begin{itemize}
\item \textbf{Draw.io}: piattaforma online usata per creare diagrammi di vario tipo. Nel nostro caso verrà usata per produrre diagrammi UML\ped{G};
\item \textbf{ESLint\ped{G}}: strumento di analisi statica del codice. Verrà usato per assicurarsi che il codice venga scritto rispettando i requisiti sopra imposti (§ 2.2.6.4);
\item \textbf{Vercel\ped{G}}: piattaforma cloud\ped{G} che consente di ospitare gratuitamente siti web sviluppati con Next.js\ped{G};
\item \textbf{AWS Lambda\ped{G}}: servizio di calcolo serverless facente parte di Amazon Web Services\ped{G}(AWS);
\item \textbf{AWS DynamoDB\ped{G}}: servizio di database NoSQL\ped{G} facente parte dei servizi offerti da AWS\ped{G};
\item \textbf{AWS Cognito\ped{G}}: servizio che fornisce autenticazione, autorizzazione e gestione degli utenti per applicazioni Web e mobile, facente parte dei servizi offerti da AWS\ped{G};
\item \textbf{AWS API Gateway\ped{G}}: servizio di AWS\ped{G} che permette la creazione, la pubblicazione, la gestione, il monitoraggio e la protezione di API\ped{G} REST\ped{G}, HTTP\ped{G} e WebSocket;
\item \textbf{Prettier\ped{G}}: Questo tool permette la formattazione del codice prdotto in maniera specifica e automatica. Sara possibile specificare delle regole di formattazione che verranno applicate al nostro codice;
\item \textbf{Next.js\ped{G}}: Si tratta di un framework JavaScript back-end per applicazioni React, e consente il rendering automatico lato server (SSR, server side rendering);
\item \textbf{Serverless\ped{G}}: È stato adottata un'architettura di sviluppo serverless che ci consente di creare ed eseguire l'applicativo senza dover gestire server;
\end{itemize} 
\myparagraph{Metriche}
Per la valutazione dei prodotti della codifica, il gruppo adotta le seguenti metriche:
\begin{itemize}
	\item \textbf{Facilità di comprensione}: produce la percentuale mettendo in rapporto le linee di commenti con le linee di codice:
\begin{center}
\[FC = \left(\frac{linee\ di\ commenti}{linee\ di\ codice}\right)*100;\]
\end{center}
	\item \textbf{Facilità di utilizzo}: un prodotto, in questo caso un sito web, è facile da utilizzare quando è necessario un numero di click relativamente basso per raggiungere l'informazione cercata;
	\item \textbf{Facilità di apprendimento}: un prodotto è di facile apprendimento se sono necessari pochi minuti per raggiungere l'informazione cercata;
	\item \textbf{Dimensioni gerarchia}: con dimensione della gerarchia si intende la profondità della gerarchia delle pagine, che non deve essere troppo profonda;
	\item \textbf{Semplicità classi}: una classe viene considerata semplice quando ha pochi metodi;
	\item \textbf{Semplicità funzioni}: una funzione viene considerata semplice quando ha pochi parametri.
\end{itemize}




