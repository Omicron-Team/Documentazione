\subsection{Sviluppo}
\subsubsection{Scopo}
Lo scopo del processo di Sviluppo è quello di descrivere i compiti e le attività di analisi, progettazione, codifica, integrazione, test, installazione ed accettazione del prodotto seguendo lo standard ISO/IEC 12207:1995.
<<<<<<< HEAD
\subsubsection{Descrizione}
Elenchiamo di seguite le attività che compongono il processo di sviluppo:
\begin{itemize}
    \item{\AdR{};}
    \item{Progettazione;}
    \item{Codifica del software;}
\end{itemize}

\subsubsection{Aspettative}
Le aspettative del gruppo sono:
\begin{itemize}
    \item{Fissare gli obiettivi di sviluppo;}
    \item{Fissare vincoli tecnologici;}
    \item{Fissare vincoli di design;}
    \item{Realizzare un prodotto finale che supera i test, che soddisfa i requisiti e le richieste del proponente;}
\end{itemize}

\subsubsection{Attività}
\paragraph{\AdR{}}
\subparagraph{Scopo} L'biettivo dell'\AdRv{} è quello di definire i requisiti preposti dall'\proponProg{}. È quindi necessario che gli \analProg
\begin{itemize}
  %  \item{Fornire ai progettisti indicazioni precise ed affidabili;}
    %\item{Fissare i requisiti concordati con il \proponProg;}
    \item{;}
    \item{;}
    \item{;}
    \item{;}
\end{itemize}




\paragraph{Progettazione}


\paragraph{Codifica del software}
=======

\myparagraph{Casi d'uso}
Rappresenta una situazione nella quale il sistema viene utilizzato per soddisfare uno o più bisogni dell'utente.\\
La loro struttura è la seguente:
\begin{itemize}
\item \textbf{Identificatore}: codice che identifica il caso d'uso. Il suo formato è il seguente:
\begin{center}
\textbf{UC[CodicePadre].\{CodiceFiglio\}}\\
\end{center}
dove:
\begin{itemize}
\item \textbf{UC}: rappresenta un acronimo di "Use Case", ovvero "Caso d'uso";
\item \textbf{CodicePadre}: rappresenta il codice di un caso d'uso generico;
\item \textbf{CodiceFiglio} (opzionale): rappresenta il codice di un sotto caso di un caso d'uso;
\end{itemize}
\item \textbf{Nome}: Titolo del caso d'uso;
\item \textbf{Diagramma UML}\ped{G} \textbf{dei casi d'uso}: rappresentazione grafica del caso d'uso utilizzando il linguaggio UML\ped{G} 2.0; 
\item \textbf{Descrizione}: descrizione breve del caso d'uso;
\item \textbf{Scenario principale}: sequenza di azioni che porta al risultato atteso dal caso d'uso;
\item \textbf{Scenari alternativi}(opzionale): sequenza di azioni che porta ad un risultato inatteso (un errore o un'eccezione) dal caso d'uso;
\item \textbf{Precondizioni}: condizioni necessarie per il corretto avvio del caso d'uso;
\item \textbf{Post-condizioni}: condizioni che devono essere vere dopo che il caso d'uso è andato a buon fine;
\item \textbf{Attori primari}: utente che inizia un'interazione con il sistema per raggiungere un obiettivo;
\item \textbf{Attori secondari}(opzionale): entità che aiuta l'attore primario a raggiungere il suo obiettivo;
\item \textbf{Estensioni}(opzionale): aumenta le funzionalità di uno use case. Ogni istanza di un caso d'uso 'A' esegue un evento 'B' in modo condizionato. L'esecuzione di 'B' interrompe 'A';
\item \textbf{Inclusioni}(opzionale): include un'istanza di un use case 'B' in un altro use case 'A';
\item \textbf{Specializzazioni}(opzionale): casi d'uso che aggiungono o modificano le caratteristiche base di un caso d'uso di partenza.
\end{itemize}
\myparagraph{Requisiti}
Comprendono le competenze necessarie per svolgere correttamente il progetto.\\
Ogni requisito è composto da:
\begin{itemize}
\item \textbf{Identificatore}: codice che identifica un requisito. Il suo formato è il seguente:
\begin{center}
\textbf{R[Importanza][Tipologia][CodicePadre].\{CodiceFiglio\}}\\
\end{center}
dove:
\begin{itemize}
\item \textbf{Importanza}: rappresenta l'importanza del requisito e può essere:
\begin{itemize}
\item \textbf{1 (Obbligatorio)}: irrinunciabile per gli stakeholder;
\item \textbf{2 (Desiderabile)}: non strettamente necessario ma a valore aggiunto riconoscibile;
\item \textbf{3 (Facoltativo)}: relativamente utile oppure contrattabile più avanti nel progetto;
\end{itemize}
\item \textbf{Tipologia}: rappresenta il tipo di requisito e può essere:
\begin{itemize}
\item \textbf{V (Vincolo)}: descrive i vincoli sui servizi offerti dal sistema;
\item \textbf{F (Funzionale)}: descrive le funzioni che il sistema deve realizzare;
\item \textbf{P (Prestazionale)}: descrive i vincoli prestazionali che il sistema;
\item \textbf{Q (Qualità)}: descrive i vincoli di qualità che il sistema deve avere;
\end{itemize}
\item \textbf{CodicePadre}: rappresenta il codice di un requisito generico;
\item \textbf{CodiceFiglio} (opzionale): rappresenta il codice di un sotto caso di requisito;
\end{itemize}
\item \textbf{Descrizione}: riporta una breve descrizione del requisito;
\item \textbf{Classificazione}: riporta l'importanza del requisito;
\item \textbf{Fonte}: riporta da dove deriva il requisito. Il requisito può derivare da:
\begin{itemize}
\item \textbf{Capitolato}: requisito individuato dalla lettura del capitolato d'appalto.
\item \textbf{Interno}: requisito individuato dagli analisti e ritenuto opportuno.
\item \textbf{Verbale}: requisito individuato a seguito di una discussione tra i membri del gruppo di progetto o con il proponente. Verrà riportato, inoltre, il codice di riferimento del verbale.
\item \textbf{Caso d'uso}: requisito estrapolato da un caso d'uso.
\end{itemize}
\end{itemize}
>>>>>>> NormeDiProgetto
