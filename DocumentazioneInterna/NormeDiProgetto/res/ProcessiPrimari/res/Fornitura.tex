\section{Processi Primari}
\subsection{Fornitura}\
\subsubsection{Scopo}
L'obiettivo del processo di fornitura è quello di determinare le risorse e le procedure necessarie per lo svolgimento del progetto. 
Prima di tutto è necessario comprendere le richieste del \proponProg{}. Una volta in possesso di queste informazioni viene redatto lo \SdFv{} con le informazioni acquisite.
Successivamente è necessario sviluppare un \PdPv{} da seguire fino alla consegna del prodotto finale, per questo motivo è necessario stipulare un contratto con il \proponProg{}.\\
Elenchiamo di seguito le fasi del processo di fornitura:
\begin{itemize}
    \item{avvio;}
    \item{approntamento di risposte alle richieste;}
    \item{contrattazione;}
    \item{pianificazione;}
    \item{esecuzione e controllo;}
    \item{revisione e valutazione;}
    \item{consegna e completamento.}
\end{itemize}

\subsubsection{Aspettative}
\Omicron, come gruppo, intende instaurare un rapporto di collaborazione mediante un dialogo continuo con il \proponProg{} per poter discutere i seguenti punti:
\begin{itemize}
    \item{comprendere vincoli e requisiti sui processi;}
    \item{stimare tempistiche di progetto;}
    \item{sostenere verifiche periodiche;}
    \item{far emergere dubbi e problematiche;}
    \item{accordarsi sulla qualifica del prodotto.}
\end{itemize}

\subsubsection{Descrizione}
In questa sezione sono descritte le norme che i membri del gruppo \Omicron{} sono tenuti a seguire durante tutte le fasi del progetto.
\subsubsection{Attività}
\myparagraph{\SdF}
È necessario che i membri del gruppo, su indicazione del \respProg, si ritrovino in riunioni periodiche volte a discutere dei capitolati facendo emergere problematiche e nuove idee in merito.\\
Lo \SdFv{}viene redatto dagli analisti e comprende i seguenti punti:
\begin{itemize}
    \item \textbf{Informazioni generali:} insieme di informazioni di base, come il nome del progetto, il \proponProg{} e il \commitProg{};
    \item \textbf{Descrizione e finalità del progetto:} presentazione del progetto e descrizione delle richieste del prodotto definendo l'obiettivo da raggiungere;
    \item \textbf{Tecnologie interessate:} elenco delle tecnologie richieste per lo svolgimento;
    \item \textbf{Aspetti positivi, criticità e fattori di rischio:} considerazione fatte dal gruppo sugli aspetti positivi e sui fattori di rischio del capitolato;
    \item \textbf{Conclusioni:} motivazioni per le quali il gruppo ha deciso di accettare o rifiutare il capitolato.
\end{itemize}
\myparagraph{\PdP} 
Il Responsabile redige il \PdPv{}da seguire durante il percorso. Questo documento contiene:
\begin{itemize}
    \item \textbf{Analisi dei rischi:} analisi dettagliata dei rischi che potrebbero presentarsi, probabilità che questi si presentino e il livello di gravità in caso succeda;
    \item \textbf{Modello di sviluppo:} descrizione del modello di sviluppo scelto, indispensabile per la pianificazione;
    \item \textbf{Pianificazione:} organizzazione delle attività relative al progetto, stabilendo le scadenze;
    \item \textbf{Preventivo e consuntivo:} presentazione di un preventivo con il costo totale per la realizzazione a seguito di un consuntivo che va a confrontarsi con il preventivo iniziale.
\end{itemize}

\myparagraph{\PdQ} 
Gli amministratori dovranno redigere un documento con tutte le strategie necessarie per garantire la qualità del materiale prodotto dal gruppo. Questo documento chiamato \PdQv{}viene così suddiviso:
\begin{itemize}
    \item \textbf{Qualità di processo:} vengono identificati dei processi dagli standard e pianificati gli obiettivi. Inoltre bisogna trovare le modalità per raggiungerli individuando metriche misurabili e controllabili;    
    \item \textbf{Qualità di prodotto:} una volta definiti gli attributi del prodotto si stabiliscono gli obiettivi con metriche misurabili;    
    \item \textbf{Specifiche dei test:} Il prodotto creato deve controllare che soddisfi i requisiti preposti sottoponendolo a dei test;
	\item \textbf{Resoconto delle attività di verifica:} vengono riportati i risultati delle metriche adottate per la verifica;
    \item \textbf{Valutazioni per il miglioramento:} vengono proposti possibili miglioramenti.
\end{itemize}

\subsubsection{Strumenti}
Per il processo di fornitura il gruppo ha deciso di utilizzare i seguenti strumenti:
\begin{itemize}
    \item \textbf{Microsoft Excel}: per il calcolo del preventivo, del consuntivo e la creazione dei diagrammi;
    \item \textbf{GanttProject}: per realizzare i diagrammi di Gantt per la pianificazione delle fasi.
\end{itemize}