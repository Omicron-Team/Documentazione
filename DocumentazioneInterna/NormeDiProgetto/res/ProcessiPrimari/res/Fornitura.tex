\section{Processi primari}
\subsection{Fornitura}
\subsubsection{Scopo}
Lo scopo del processo di fornitura, secondo lo standard\ped{G} ISO/IEC 12207:1995, è di determinare le risorse e le procedure necessarie per lo svolgimento del progetto.\\
Prima di tutto è necessario comprendere le richieste del \proponProg{}. Successivamente è possibile procedere con la stesura dello \SdFv{} sulla base delle informazioni acquisite.
È necessario inoltre sviluppare un \PdPv{} da seguire fino alla consegna del prodotto finale, effettuare la stesura di un \PdQv{}, assicurando al cliente una qualità di processi e prodotti adeguata alle sue aspettative , e stipulare un contratto con il \proponProg{} nel quale vengono esposti vincoli, requisiti e tempistiche di consegna.

\subsubsection{Aspettative}
\Omicron, come gruppo, intende instaurare un rapporto di collaborazione mediante un dialogo continuo con il \proponProg{} per poter discutere i seguenti punti:
\begin{itemize}
    \item{comprendere vincoli e requisiti sui processi;}
    \item{stimare tempistiche di progetto;}
    \item{sostenere verifiche periodiche;}
    \item{chiarire dubbi e problematiche;}
    \item{accordarsi sulla qualifica del prodotto.}
\end{itemize}

\subsubsection{Descrizione}
In questa sezione sono descritte le norme che i membri del gruppo \Omicron{} sono tenuti a seguire durante tutte le fasi del progetto.

\subsubsection{Attività}
Il processo di Fornitura comprende le seguenti attività:
\begin{itemize}
	\item avvio;
	\item contrattazione;
	\item pianificazione;
	\item esecuzione e controllo;
	\item revisione e valutazione;
	\item consegna e completamento.
\end{itemize}
Nella realizzazione del nostro prodotto, associamo alle attività i seguenti compiti:
\setcounter{table}{-1}
{

\rowcolors{2}{azzurro2}{azzurro3}

\centering
\renewcommand{\arraystretch}{1.5}
\begin{longtable}{C{4cm} C{11cm}}
\rowcolor{azzurro1}
\textbf{Attività} &
\textbf{Compiti}\\
\endhead

Avvio & \vspace{-0.5cm}
	\begin{itemize}
		\item valutazione dei capitolati proposti mediante stesura del documento di \SdFv{}(§2.1.5);
		\item scelta di un capitolato.
	\end{itemize}  \\
Preparazione della proposta & \vspace{-0.5cm}
	\begin{itemize}
		\item formulazione della nostra proposta da consegnare ai committenti.
	\end{itemize}  \\
Contrattazione & Il gruppo si metterà in contatto con il proponente per: 
	\begin{itemize}
		\item chiarire dubbi emersi durante la stesura della proposta;
		\item approfondire gli aspetti principali per soddisfare i bisogni del proponente;
		\item proporre le nostre soluzioni per assicurarsi che siano in sintonia con le loro richieste.
	\end{itemize}  \\
Pianificazione & \vspace{-0.5cm}
	\begin{itemize}
		\item scelta del modello di sviluppo e conseguente specifica delle attività e compiti, definiti nel \PdPv{}(§2.1.6);
		\item identificare e valutare i rischi che si possono presentare in determinate attività;
		\item stilare regole per garantire la qualità durante tutto lo sviluppo del prodotto, più in dettaglio nel \PdQv{}(§2.1.7).
	\end{itemize}  \\
Esecuzione e controllo & \vspace{-0.5cm}
	\begin{itemize}
		\item attuazione della pianificazione;
		\item monitoraggio dei costi, rischi e tempistiche.
	\end{itemize}  \\
Revisione e valutazione & \vspace{-0.5cm}
	\begin{itemize}
		\item rispettare i dettagli sul testing del prodotto (§2.1.8), come specificato nel capitolato.
	\end{itemize}  \\
Consegna e completamento & \vspace{-0.5cm}
	\begin{itemize}
		\item consegna del prodotto (§2.1.9) ai committenti nelle modalità da loro specificate e conforme alle richieste del proponente.
	\end{itemize}  \\

		
\end{longtable}
}

\subsubsection{\SdF}
È necessario che i membri del gruppo, su indicazione del \respProg, si ritrovino in riunioni periodiche volte a discutere dei capitolati facendo emergere problematiche e nuove idee in merito.\\
Lo \SdFv{}viene redatto dagli analisti e comprende i seguenti punti:
\begin{itemize}
    \item \textbf{Informazioni generali:} insieme di informazioni di base, come il nome del progetto, il \proponProg{} e il \commitProg{};
    \item \textbf{Descrizione e finalità del progetto:} presentazione del progetto e descrizione delle richieste del prodotto definendo l'obiettivo da raggiungere;
    \item \textbf{Tecnologie coinvolte:} elenco delle tecnologie richieste per lo svolgimento;
    \item \textbf{Aspetti positivi, criticità e vincoli:} considerazioni fatte dal gruppo sugli aspetti positivi, sui fattori di rischio del capitolato e sui vincoli individuati;
    \item \textbf{Conclusioni:} motivazioni per le quali il gruppo ha deciso di accettare o rifiutare il capitolato.
\end{itemize}

\subsubsection{\PdP}
Il Responsabile redige il \PdPv{}da seguire durante il percorso. Questo documento contiene:
\begin{itemize}
    \item \textbf{Analisi dei rischi:} analisi dettagliata dei rischi che potrebbero presentarsi, probabilità che questi si presentino e il livello di gravità in caso succeda;
    \item \textbf{Modello di sviluppo:} descrizione del modello di sviluppo scelto, indispensabile per la pianificazione;
    \item \textbf{Pianificazione:} organizzazione delle attività relative al progetto, stabilendo ruoli e scadenze;
    \item \textbf{Preventivo e consuntivo:} presentazione di un preventivo con il costo totale per la realizzazione a seguito di un consuntivo che va a confrontarsi con il preventivo iniziale.
\end{itemize}

\subsubsection{\PdQ}
Gli amministratori dovranno redigere un documento con tutte le strategie necessarie per garantire la qualità del materiale prodotto dal gruppo. Questo documento chiamato \PdQv{}viene così suddiviso:
\begin{itemize}
    \item \textbf{Qualità di processo:} vengono identificati dei processi dagli standard\ped{G} e pianificati gli obiettivi. Inoltre bisogna trovare le modalità per raggiungerli individuando metriche misurabili e controllabili;    
    \item \textbf{Qualità di prodotto:} una volta definiti gli attributi del prodotto si stabiliscono gli obiettivi con metriche misurabili;    
     \item \textbf{Specifiche dei test:} Il prodotto creato viene sottoposto a una serie di test per valutarne il grado di qualità e di stabilità;
     \item \textbf{Resoconto delle attività di verifica:} vengono riportati i risultati delle metriche adottate per la verifica.
\end{itemize}

\subsubsection{Preparazione al collaudo del prodotto}
Per poter avere esito positivo in sede di collaudo, il prodotto finito deve essere stato preventivamente preparato attraverso il superamento di un'attività di test.\\ Questa si articola in:
\begin{itemize}
	\item \textbf{test d'unità};
	\item \textbf{test di integrazione};
	\item \textbf{test di sistema};
	\item \textbf{test di regressione};
	\item \textbf{test di accettazione o test di collaudo}.
\end{itemize}
I test sopracitati vengono definiti con maggior dettaglio in §3.4.4.2.

\subsubsection{Collaudo e consegna}
Il gruppo \Omicron{} consegnerà all'azienda proponente e ai committenti \textit{\VT} e \textit{\CR} quanto segue:
\begin{itemize}
	\item \textbf{codice sorgente};
	\item \textbf{documentazione del prodotto}:
	\begin{itemize}
		\item \textit{Lettera di Presentazione}: documento con cui i membri del gruppo \Omicron{} formalizzano il loro impegno nel portare a termine il capitolato prescelto rispettandone i requisiti minimi e consegnando il prodotto finito entro i termini definiti dalla lettera stessa;
		\item \MMv{1.0.0}: permette a sviluppatori esterni di contribuire al progetto facilmente;
		\item \MUv{1.0.0}: guida l'utente finale durante l'installazione e l'utilizzo delle funzionalità del prodotto;
		\item \NdPv{4.0.0}: documento che stabilisce il way of working del gruppo durante la durata di tutto il progetto;
		\item \PdQv{4.0.0}: assicura al proponente una qualità di processi e prodotti adeguata alle sue aspettative, fornendo le modalità da adottare in sede di verifica e validazione;
		\item \PdPv{4.0.0}: contiene la pianificazione preventiva dei tempi e delle attività, l'analisi dei rischi e il consuntivo di periodo, oltre alla data e ai costi previsti per la realizzazione del prodotto finale;
		\item \Glossariov{4.0.0}: facilita la comprensione dei documenti eliminando le possibili ambiguità inerenti alla terminologia.
	\end{itemize}
\end{itemize}

\subsubsection{Strumenti}
Per il processo di fornitura il gruppo ha deciso di utilizzare i seguenti strumenti:
\begin{itemize}
    \item \textbf{Microsoft Excel}: per il calcolo del preventivo, del consuntivo e la creazione dei diagrammi;
    \item \textbf{GanttProject}: per realizzare i diagrammi di Gantt per la pianificazione delle fasi.
\end{itemize}