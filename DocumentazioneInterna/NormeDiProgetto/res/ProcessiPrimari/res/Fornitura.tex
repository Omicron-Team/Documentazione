\section{Processi Primari}
\subsection{Fornitura}\
\subsubsection{Scopo}
L'obiettivo del processo di fornitura è quello di determinare quelle che sono le risorse e le procedure necessarie per lo svolgimento del progetto. 
Prima di tutto è necessario comprendere le richieste del proponente e creare uno \textit{Studio di Fattibilità} è possibile dar via al processo tenendo conto di tutte le richieste del proponente.
Successivamente è necessario sviluppare un \textit{Piano di Progetto} da seguire fino alla consegna del prodotto finale, per questo motivo bisogna stipulare un contratto con il proponente.\\
Elenchiamo di seguito le fasi del processo di fornitura:
\begin{itemize}
    \item{Avvio;}
    \item{Approntamento di risposte alle richieste;}
    \item{Contrattazione;}
    \item{Pianificazione;}
    \item{Esecuzione e controllo;}
    \item{Revisione e valutazione;}
    \item{Consegna e completamento.}
\end{itemize}

\subsubsection{Aspettative}
Omicron, come gruppo, intende instaurare un rapporto di collaborazione mantenendo un dialogo continuo con il proponente.\\
Per un corretto lavoro è necessario poter discutere dei seguenti punti per tutta la durata del progetto:

\begin{itemize}
    \item{Comprendere vincoli e requisiti sui processi;}
    \item{Stimare tempistiche di progetto;}
    \item{Sostenere verifiche periodiche;}
    \item{Far emergere dubbi e problematiche;}
    \item{Accordarsi sulla qualifica del prodotto.}
\end{itemize}

\subsubsection{Descrizione}
In questa sezione sono descritte le norme che i membri del gruppo Omicron sono tenuti a seguire durante tutte le fasi del progetto.
