\subsubsection{Attivita}
\paragraph{Studio di Fattibilità}
È necessario che i membri del gruppo, su indicazione del responsabile di progetto,  si ritrovino in riunioni periodiche volte a discutere dei capitolati facendo emergere problematiche e nuove idee in merito.\\
Lo \textit{Studio di Fattibilità} viene redatto dagli analisti e comprende i seguenti punti:
\begin{itemize}
    \item \textbf{Informazioni generali:} insieme di informazioni di base, come il nome del progetto, il proponente e il committente;
    \item \textbf{Descrizione e finalità del progetto:} presentazione del progetto e descrizione delle richieste del prodotto definendo l'obiettivo da raggiungere;
    \item \textbf{Tecnologie interessate:} elenco delle tecnologie richieste per lo svolgimento;
    \item \textbf{Aspetti positivi, criticità e fattori di rischio:} considerazione fatte dal gruppo sugli aspetti positivi e sui fattori di rischio del capitolato;
    \item \textbf{Conclusioni:} motivazioni per le quali il gruppo ha deciso di accettare o rifiutare il capitolato.
\end{itemize}
\paragraph{Piano di Progetto} Il Responsabile redige il \textit{Piano di Progetto} da seguire durante il corso. Questo documento contiene:
\begin{itemize}
    \item \textbf{Analisi dei rischi:} analisi dettagliata dei rischi che potrebbero presentarsi, compreso di probabilità che questi si presentino e il livello di gravità in caso questo succeda;
    
    \item \textbf{Modello di sviluppo:} descrizione del modello di sviluppo scelto, indispensabile per la pianificazione;
    
    \item \textbf{Pianificazione:} si tratta di organizzare le attività relative al progetto, stabilendo le scadenze;
    
    \item \textbf{Preventivo e consuntivo:} viene fatta una stima del lavoro per ogni fase del progetto presentando quindi un preventivo con il costo totale per la realizzazione. Viene tracciato anche un consuntivo che va a confrontarsi con il preventivo iniziale.
\end{itemize}

\paragraph{Piano di Qualifica} Gli amministratori dovranno redigere un documento con tutte le strategie necessarie per garantire la qualità del materiale prodotto dal gruppo. Questo documento chiamato \textit{Piano di Qualifica} è così suddiviso:
\begin{itemize}
    \item \textbf{Qualità di processo:} vengono identificati dei processi dagli standard e pianificati gli obiettivi. Inoltre bisogna trovare le modalità per raggiungerli individuando metriche mirsurabili e controllabili;
    
    \item \textbf{Qualità di prodotto:} una volta definiti gli attributi del prodotto si stabiliscono gli obiettivi con metriche misurabili;
    
    \item \textbf{Specifiche dei test:} Il prodotto creato deve controllare che soddisfi i requisiti preposti sottoponendolo a dei test;
    
    \item \textbf{Standard di qualità:} vengono esposti gli standard di qualità scelti dal gruppo;
    
    \item \textbf{Valutazioni per il miglioramento:} vengono proposti possibili miglioramenti;
    
    \item \textbf{Resoconto delle attività di verifica:} vengono riportati i risultati delle metriche adottate per la verifica.
\end{itemize}

\subsubsection{Strumenti}
\textcolor{red}{DA FARE?????}
