\subsection{Metriche di qualità}
\myparagraph{Metriche per l'analisi dei requisiti}
Al fine di assicurarsi che il prodotto realizzato o che sta venendo realizzando ricopra i requisiti necessari verrà usata la seguente metrica:
\begin{center}
\begin{equation}
\textit{PRS} = (\frac{Requisiti\ soddisfatti}{Requisiti\ totali})*100
\end{equation}
\end{center}

la quale restituisce la percentuale di requisiti che il prodotto soddisfa rispetto a quelli totali.\\ 
In particolare:
\begin{itemize}
\item se \textbf{PRS = 100}: il livello di copertura dei requisiti è ottimo;
\item se \textbf{90<PRS<100}: il livello di copertura dei requisiti è buono;
\item se \textbf{80<PRS<=90}: il livello di copertura dei requisiti è sufficiente;
\item se \textbf{PRS<=80}: il livello di copertura dei requisiti è insufficiente.
\end{itemize}

La metrica può essere facilmente adattata ai requisiti obbligatori, che però devono venir ricoperti al 100\% a prodotto finito:
\begin{center}
\begin{equation}
\textit{PROS} =  (\frac{Requisiti\ obbligatori\ soddisfatti}{Requisiti\ obbligatori\ totali})*100
\end{equation}
\end{center}

\myparagraph{Metriche per la codifica}
Per la parte di codifica useremo delle metriche che ci consentiranno di capire quanto comprensibili sono le azioni che vengono svolte nel codice.\\
Le metriche che useremo sono le seguenti:
\begin{itemize}
\item \textbf{rapporto tra linee di commento e linee di codice}:
\begin{center}
\begin{equation}
\textit{RC} =  (\frac{numero\ di\ linee\ di\ commento}{numero\ di\ linee\ di\ codice});
\end{equation}
\end{center}
\item \textbf{livello di annidamento dei metodi}: da ridurre, in quanto un alto livello di annidamento può portare a maggiore difficoltà nel trovare eventuali problemi durante le fasi di manutenzione.
\end{itemize}
