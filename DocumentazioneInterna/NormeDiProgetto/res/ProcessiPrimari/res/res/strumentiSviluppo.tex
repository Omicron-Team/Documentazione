\myparagraph{Strumenti}
Per il processo di sviluppo, durante il progetto, sono stati usati i seguenti strumenti:
\begin{itemize}
\item \textbf{Draw.io}: piattaforma online usata per creare diagrammi di vario tipo. Nel nostro caso verrà usata per produrre diagrammi UML\ped{G};
\item \textbf{ESLint\ped{G}}: strumento di analisi statica del codice. Verrà usato per assicurarsi che il codice venga scritto rispettando i requisiti sopra imposti (§ 2.2.6.4);
\item \textbf{Vercel\ped{G}}: piattaforma cloud\ped{G} che consente di ospitare gratuitamente siti web sviluppati con Next.js\ped{G};
\item \textbf{AWS Lambda\ped{G}}: servizio di calcolo serverless facente parte di Amazon Web Services\ped{G}(AWS);
\item \textbf{AWS DynamoDB\ped{G}}: servizio di database NoSQL\ped{G} facente parte dei servizi offerti da AWS\ped{G};
\item \textbf{AWS Cognito\ped{G}}: servizio che fornisce autenticazione, autorizzazione e gestione degli utenti per applicazioni Web e mobile, facente parte dei servizi offerti da AWS\ped{G};
\item \textbf{AWS API Gateway\ped{G}}: servizio di AWS\ped{G} che permette la creazione, la pubblicazione, la gestione, il monitoraggio e la protezione di API\ped{G} REST\ped{G}, HTTP\ped{G} e WebSocket;
\item \textbf{Prettier\ped{G}}: questo tool permette la formattazione del codice prodotto in maniera specifica e automatica. Sarà possibile specificare delle regole di formattazione che verranno applicate al nostro codice;
\item \textbf{Next.js\ped{G}}: si tratta di un framework JavaScript per applicazioni React, e consente il rendering automatico lato server (SSR, server side rendering\ped{G});
\item \textbf{Serverless\ped{G}}: framework per la codifica di funzioni AWS Lambda\ped{G} ed effettuare deploy nei vari servizi AWS\ped{G}. Permette di creare un intero servizio di back-end.
\end{itemize} 