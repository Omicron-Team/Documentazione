\myparagraph{Strumenti}
Per il processo di sviluppo, durante il progetto, sono stati usati i seguenti strumenti:
\begin{itemize}
\item \textbf{Draw.io}: piattaforma online usata per diagrammi di vario tipo. Nel nostro caso verrà usata per produrre diagrammi UML\ped{G};
\item \textbf{ESLint\ped{G}}: strumento di analisi statica del codice. Verrà usato per assicurarsi che il codice venga scritto rispettando i requisiti sopra imposti (§ 2.2.6.4);
\item \textbf{Vercel\ped{G}}: piattaforma cloud per siti statici e funzioni serverless che consente agli sviluppatori di ospitare siti Web e servizi Web che vengono distribuiti istantaneamente;
\item \textbf{AWS Lambda\ped{G}}: servizio di calcolo che consente di eseguire il codice senza gestire i server facente parte di Amazon Web Services\ped{G}(AWS);
\item \textbf{AWS DynamoDB\ped{G}}: servizio di database NoSQL facente parte dei servizi offerti da AWS
\item \textbf{AWS Cognito\ped{G}}: fornisce autenticazione, autorizzazione e gestione degli utenti per le applicazioni Web e mobili facente parte dei servizi offerti da AWS;
\item \textbf{AWS API Gateway\ped{G}}: servizio di AWS che permette la creazione, la pubblicazione, la gestione, il monitoraggio e la protezione di API REST, HTTP e WebSocket;
\end{itemize} 