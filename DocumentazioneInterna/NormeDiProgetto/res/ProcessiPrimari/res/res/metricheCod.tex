\myparagraph{Metriche}
Per la valutazione dei prodotti della codifica, il gruppo adotta le seguenti metriche:
\begin{itemize}
	\item \textbf{Facilità di comprensione}: produce la percentuale mettendo in rapporto le linee di commenti con le linee di codice:
\begin{center}
\[FC = \left(\frac{linee\ di\ commenti}{linee\ di\ codice}\right)*100;\]
\end{center}
	\item \textbf{Facilità di utilizzo}: un prodotto, in questo caso un sito web, è facile da utilizzare quando è necessario un numero di click relativamente basso per raggiungere l'informazione cercata;
	\item \textbf{Facilità di apprendimento}: un prodotto è di facile apprendimento se sono necessari pochi minuti per raggiungere l'informazione cercata;
	\item \textbf{Dimensioni gerarchia}: con dimensione della gerarchia si intende la profondità della gerarchia delle pagine, che non deve essere troppo profonda;
	\item \textbf{Semplicità classi}: una classe viene considerata semplice quando ha pochi metodi;
	\item \textbf{Semplicità funzioni}: una funzione viene considerata semplice quando ha pochi parametri.
\end{itemize}


