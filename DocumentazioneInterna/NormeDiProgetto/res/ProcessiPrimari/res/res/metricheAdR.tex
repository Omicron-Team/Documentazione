\myparagraph{Metriche}
Per la valutazione del prodotto dell'Analisi dei Requisiti, il gruppo adotta le seguenti metriche:
\begin{itemize}
	\item \textbf{Percentuale completezza informazioni}: determiniamo così il grado di completamento del prodotto tramite la formula:
\begin{center}
\[PCI = \left(1-\frac{funzioni \ non \ implementate}{funzioni \ implementate}\right)\ast100;\]
\end{center}


	\item \textbf{Percentuale di requisiti soddisfatti}: al fine di assicurare che il prodotto realizzato ricopra i requisiti necessari verrà usata la seguente metrica:
\begin{center}
\[PRS = \left(\frac{Requisiti\ soddisfatti}{Requisiti\ totali}\right)*100\]
\end{center}
la quale restituisce la percentuale di requisiti che il prodotto soddisfa rispetto a quelli totali;

	\item \textbf{Percentuale di requisiti obbligatori soddisfatti}: la metrica presentata nel punto precedente può essere facilmente adattata ai requisiti obbligatori:
\begin{center}
\[PROS=\left(\frac{Requisiti\ obbligatori\ soddisfatti}{Requisiti\ obbligatori\ totali}\right)*100\]
\end{center}
\end{itemize}


