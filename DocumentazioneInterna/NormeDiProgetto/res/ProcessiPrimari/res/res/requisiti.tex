\paragraph{Requisiti}
Comprendono le competenze necessarie per svolgere correttamente il progetto.\\
Ogni requisito è composto da:
\begin{itemize}
\item \textbf{Identificatore}: codice che identifica un requisito. Il suo formato è il seguente:
\begin{center}
\textbf{R[Importanza][Tipologia][CodicePadre].\{CodiceFiglio\}}\\
\end{center}
dove:
\begin{itemize}
\item \textbf{Importanza}: rappresenta l'importanza del requisito e può essere:
\begin{itemize}
\item \textbf{O (Obbligatorio)}: irrinunciabile per gli stakeholder;
\item \textbf{D (Desiderabile)}: non strettamente necessario ma a valore aggiunto riconoscibile;
\item \textbf{F (Facoltativo)}: relativamente utile oppure contrattabile più avanti nel progetto;
\end{itemize}
\item \textbf{Tipologia}: rappresenta il tipo di requisito e può essere:
\begin{itemize}
\item \textbf{V (Vincolo)}: descrive i vincoli sui servizi offerti dal sistema;
\item \textbf{F (Funzionale)}: descrive le funzioni che il sistema deve realizzare;
\item \textbf{P (Prestazionale)}: descrive i vincoli prestazionali che il sistema;
\item \textbf{Q (Qualità)}: descrive i vincoli di qualità che il sistema deve avere;
\end{itemize}
\item \textbf{CodicePadre}: rappresenta il codice di un requisito generico;
\item \textbf{CodiceFiglio} (opzionale): rappresenta il codice di un sotto caso di requisito;
\end{itemize}
\item \textbf{Descrizione}: riporta una breve descrizione del requisito;
\item \textbf{Classificazione}: riporta l'importanza del requisito;
\item \textbf{Fonte}: riporta da dove deriva il requisito. Il requisito può derivare da:
\begin{itemize}
\item \textbf{Capitolato}: requisito individuato dalla lettura del capitolato d'appalto.
\item \textbf{Verbale Interno}: requisito individuato a seguito di una discussione tra i membri del gruppo di progetto.
\item \textbf{Verbale Esterno}: requisito individuato a seguito di una discussione con il proponente del progetto.
\item \textbf{Caso d'uso}: requisito estrapolato da un caso d'uso.
\end{itemize}
\end{itemize}