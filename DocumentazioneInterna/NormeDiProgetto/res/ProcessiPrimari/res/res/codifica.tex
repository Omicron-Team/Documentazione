\subsubsection{Codifica}
\myparagraph{Scopo}
L'attività di codifica ha lo scopo di normare l'effettiva realizzazione del prodotto software partendo dall'architettura realizzata durante la fase di progettazione. È compito del programmatore assicurarsi il corretto svolgimento di questa attività.

\myparagraph{Aspettative}
L'obiettivo del gruppo \Omicron{} è quello di creare, tramite l'attività di codifica, un prodotto software che sia in grado di soddisfare i requisiti prefissati con il proponente assicurandosi, usando delle norme e delle convenzioni, di:
\begin{itemize}
\item generare codice leggibile ed uniforme;
\item agevolare le fasi di manutenzione, verifica e validazione;
\item mantenere un prodotto di ottima qualità. 
\end{itemize}

\myparagraph{Descrizione}
Il codice scritto dovrà essere di qualità, rispettando e perseguendo gli obiettivi di qualità definiti nel \PdQv{2.0.0}.

\myparagraph{Stile di codifica}
Il codice generato da ogni membro deve rispettare le seguenti norme:
\begin{itemize}
\item \textbf{indentazione}: Ogni blocco innestato deve essere indentato usando 2 spazi per ogni livello di indentazione;
\item \textbf{parentesizzazione}: le parentesi di delimitazione dei costrutti devono essere inserite in linea e non al di sotto di essi;
\item \textbf{scrittura dei metodi}: è preferibile l'uso di metodi brevi (con poche righe di codice);
\item \textbf{univocità dei nomi}: il nome di classi, metodi, variabili deve essere univoco e autoesplicativo;
\item \textbf{struttura dei nomi}: I nomi sono strutturati nel seguente modo:
\begin{itemize}
\item \textit{metodi, variabili e costanti}: sono scritti con la prima lettera minuscola. Se il nome è composto da più parole quelle successive iniziano con la lettera maiuscola (camel Case);
\item \textit{classi}: sono scritti con la prima lettera maiuscola. Se il nome è composto da più parole quelle successive iniziano con la lettera maiuscola (Pascal Case).
\end{itemize} 
\item \textbf{lingua}: il codice e i commenti saranno scritti in lingua inglese.
\end{itemize}

\myparagraph{Ricorsione}
L'uso della ricorsione va evitato se non in casi dove risulta essere strettamente necessaria.
