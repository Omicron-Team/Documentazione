\section{Introduzione}
\subsection{Scopo del documento}
Questo documento ha lo scopo di definire la modalità in cui il progetto \nameproject{} verrà svolto dal gruppo \Omicron{}. I punti trattati dal documento sono i seguenti:
\begin{itemize}
\item analisi dei rischi;
\item descrizione breve del modello di sviluppo adottato;
\item pianificazione delle attività e suddivisione dei ruoli;
\item stima dei costi e delle risorse necessarie.
\end{itemize}
\subsection{Scopo del prodotto}
Lo scopo del capitolato C2 è la realizzazione di una generica piattaforma di e-commerce\ped{G}, chiamata \nameproject{}, basata su tecnologia Serverless\ped{G} da vendere a mercanti. \nameproject{} deve essere distribuibile usando l'account AWS\ped{G} del mercante con una configurazione manuale minima. Deve inoltre essere prodotta una piattaforma dimostrativa per l'utilizzo di \nameproject{}.
\subsection{Glossario}
Viene fornito un glossario il quale scopo è quello di evitare ambiguità nel linguaggio utilizzato fornendo una definizione ai vari termini usati nella documentazione. Il glossario può essere trovato nell'apposito documento \Glossario{}.pdf.
\subsection{Riferimenti}
\subsubsection{Normativi}
\subsubsection{Informativi}
\subsection{Scadenze}
Il gruppo \Omicron{} si impegna a rispettare le seguenti scadenze per lo svolgimento del progetto \nameproject{}:
\begin{itemize}
\item \textbf{revisione dei Requisiti:} 2020-01-11;
\item \textbf{revisione di Progettazione:} 
\item \textbf{revisione di Qualifica:}
\item \textbf{revisione di Accettazione:} 
\end{itemize}