%% INSERIRE QUI IL NOME DEL DOCUMENTO (INSERITE SEMPRE UNO SPAZIO ALLA FINE DEL NOME)
\newcommand{\doctitle}{Developer Manual }

%% INSERIRE QUI LA VERSIONE ATTUALE DEL DOCUMENTO (INSERITE SEMPRE UNO SPAZIO ALLA FINE DELLA VERSIONE)
\newcommand{\versiondoc}{1.0.0 }

%%INSERITE QUI LA DATA DI COMPILAZIONE FINALE DEL DOCUMENTO
\newcommand{\datared}{2021-05-26}

%%INSERIRE QUI IL/I REDATTORI
\newcommand{\redattore}{\FD \\ \MB \vspace{0.1cm}}

%%INSERIRE IL/I NOME DEI VERIFICATORI CHE HANNO VERIFICATO IL DOCUMENTO
\newcommand{\verificatori}{\VAS \\ \MDI }

%%INSERIRE IL NOME DI CHI HA APPROVATO IL DOCUMENTO
\newcommand{\approvazione}{\VAS}

%%INSERIRE LA TIPOLOGIA DI USO DEL DOCUMENTO [Interno/Esterno]
\newcommand{\usodoc}{External}

%%INSERIRE LA LISTA DI DISTRIBUZIONE DEL DOCUMENTO
\newcommand{\listadistr}{
    \Omicron\\
    \emph{\VT}\\
    \emph{\CR}\\
    \emph{Red Babel}
}

%%INSERIRE IL SOMMARIO DEL DOCUMENTO
\newcommand{\testosommario}{This document is a developer manual for the project \textit{EmporioLambda}.}

%INSERIRE IL PATH RELATIVO ALL'IMMAGINE IN BASE ALLA CARTELLA DI DOVE CI SI TROVA
\newcommand{\relativePathImg}{../../Utilita/img/}

%INSERIRE UN NUOVO TERMINE DA INSERIRE NEL GLOSSARIO
\newcommand{\TermineGlossario}[1]{\textbf{#1}\\}

%INSERIRE LA DEFINIZIONE DEL NUOVO TERMINE INSERITO NEL GLOSSARIO
\newcommand{\DefinizioneGlossario}[1]{#1\\}