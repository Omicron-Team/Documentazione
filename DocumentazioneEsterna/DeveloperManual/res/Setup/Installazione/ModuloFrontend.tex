\subsubsection{Installing Front-end module}
To correctly download and configure the Front-end module, follow the steps below:
\begin{enumerate}
\item get the source code by using this git command:\begin{center}git clone https://github.com/OmicronSwe/EmporioLambda-FE.git;\end{center}
\item run the command:\begin{center}npm install\end{center}which will install all the necessary dependencies. These dependencies can be found in the package.json file.\newline{} Here's the list of the dependencies:
\begin{itemize}
\item cypress/code-coverage (version 3.9.2+);
\item bootstrap (version 4.6.0+);
\item cookie-cutter (version 0.2.0+);
\item cookies (version 0.8.0+);
\item jsonwebtoken (version 8.5.1+);
\item next (version 10.0.0+);
\item next-auth (version 3.4.1+);
\item next-cookies (version 2.0.3+);
\item react (version 17.0.1+);
\item react-bootstrap (version 1.5.2+);
\item react-dom (version 17.0.1+);
\item react-icons (version 4.2.0+);
\item react-stripe-checkout (version 2.6.3+);
\item react-stripe-elements (version 6.1.2+).
\end{itemize}
Apart from these, the following dependencies are needed for development and will also be installed:
\begin{itemize}
\item types/mocha (version 8.2.1+);
\item types/next-auth (version 3.7.1+);
\item types/node (version 14.14.25+);
\item types/react (version 17.0.1+);
\item types/react-dom (version 17.0.0+);
\item typescript-eslint/eslint-plugin (version 4.15.0+);
\item babel-plugin-istanbul (version 6.0.0+);
\item codecov (version 3.8.1+);
\item cross-env (version 7.0.3+);
\item cypress (version 6.6.0+);
\item eslint (version 7.19.0+);
\item eslint-config-airbnb-typescript (version 12.3.1+);
\item eslint-config-prettier (version 7.2.0+);
\item eslint-plugin-import (version 2.22.1+);
\item eslint-plugin-jsx-a11y (version 6.4.1+);
\item eslint-plugin-prettier (version 3.3.1);
\item eslint-plugin-react (version 7.22.0+);
\item eslint-plugin-react-hooks (version 4.2.0+);
\item prettier (version 2.2.1+);
\item start-server-and-test (version 1.12.0+);
\item typescript (version 4.1.3+).
\end{itemize}
\item install vercel using the command:\begin{center}npm i -g vercel\end{center}Vercel will be used to get the required environment variables.
\end{enumerate}
