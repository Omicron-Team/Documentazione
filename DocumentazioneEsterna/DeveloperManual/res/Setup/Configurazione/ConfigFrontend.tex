\subsubsection{Configuration of the Front-end module}
It is highly suggested to link a git repository to a vercel project. Information on how to do so can be found here \url{https://vercel.com/docs/git}.

By linking a git repository, vercel will automatically deploy the website for each push in specific branches.
\begin{enumerate}
\item in your CLI move to the EmporioLambda Front-End\textsubscript{G} folder and use the command:\begin{center}\texttt{vercel login}\end{center}It will ask for your credentials.
\item run the command:\begin{center}\texttt{vercel link}\end{center} If you still haven't created your project on vercel it will ask you:
\begin{itemize}
\item Which scope do you want to deploy to? Choose your scope;
\item Link to existing project? Answer N;
\item What’s your project’s name? Your project name (emporio-lambda-fe);
\item In which directory is your code located? Choose the folder where your code is located;
\item Want to override the settings? Answer N.
\end{itemize}
If you already created one by linking to a git repository it will ask you:
\begin{itemize}
\item Which scope do you want to deploy to? Choose your scope;
\item Link to existing project? Answer y;
\item What’s the name of your existing project? Choose the correct project.
\end{itemize}
Other information can be found here: \url{https://vercel.com/docs/cli\#commands/overview/project-linking};

To setup the various deployment stages, follow the instructions on this page: \url{https://vercel.com/docs/custom-domains#assigning-a-domain-to-a-git-branch}.
\end{enumerate}