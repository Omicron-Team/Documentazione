\section{Introduction}
\subsection{Aim of this document}
This document's purpose is to illustrate in detail the following information about the development of the project \textit{EmporioLambda}:
\begin{itemize}
\item technologies used;
\item software tools used;
\item software architecture used;
\item design patterns used;
\item functionalities provided.
\end{itemize}
\subsection{What is \textit{EmporioLambda}}
\textit{EmporioLambda} by \textit{Red Babel} is a concept software, built as a generic e-commerce platform, that aims to be shown and sold to merchants, for them to deploy on their AWS\textsubscript{G} accounts and sell their own products on. It is available to visit on the following link \url{https://emporio-lambda-fe.vercel.app/}.\\
As a generic e-commerce\textsubscript{G}, \textit{EmporioLambda} provides different functionalities based on the role of the user interacting with it. In particular, there exist 4 different types of users:
\begin{itemize}
\item Unauthenticated users can:
\begin{itemize}
\item sign up as a customer and sign in;
\item browse by product category and access product pages;
\item search and filter products;
\item add, remove and view all products in the cart.
\end{itemize}
\item Customers can:
\begin{itemize}
\item do everything an unauthenticated user can except signing up and signing in;
\item sign out;
\item checkout and pay for all of the products in the cart;
\item access their profile and edit any personal information;
\item access their previous orders information;
\item delete their own account.
\end{itemize}
\item Merchants can:
\begin{itemize}
\item do everything a customer can;
\item view, add, edit and delete any product;
\item view, add, edit and delete any product category;
\item see the details of all the orders made by customers.
\end{itemize}
\item Admins can:
\begin{itemize}
\item Monitor the state of the platform.
\end{itemize}
\end{itemize}
\subsection{Glossary}
In order to clarify some terms that may otherwise appear ambiguous or overly technical, at the end of this document you can find a small alphabetically-ordered glossary. In this document you will find such terms marked with a subscript G.
