\subsection{Front-end module architecture}
This section shows how the Front-end\textsubscript{G} module of EmporioLambda works. \\The Front-end\textsubscript{G} module can be summarized in 4 primary parts:
\begin{itemize}
\item \textbf{data pre-fetching:} managed with Next.js\textsubscript{G};
\item \textbf{components:} managed with React;
\item \textbf{services:} functions that communicate with the Back-end\textsubscript{G} module;
\item \textbf{types:} classes and data types used.
\end{itemize} 
The image below shows how these parts communicate with each other:
\begin{figure}[H]
\centering
\includegraphics[scale=0.58]{res/Architettura/Frontend/img/general_frontend}\\
\caption{Front-end\textsubscript{G} module general scheme}
\end{figure}
\newpage
Here's also a package diagram of the Front-end module:
\begin{figure}[H]
\centering
\includegraphics[scale=0.50]{res/Architettura/Frontend/img/package_frontend}\\
\caption{Front-end\textsubscript{G} module package\textsubscript{G} diagram}
\end{figure}
\newpage
In the following sequence diagram representing the insert of an item in the cart is possible to see how the different parts of the Front-end module work:
\begin{figure}[H]
\centering
\includegraphics[scale=0.70]{res/Architettura/Frontend/img/sequence_frontend_insertCart}\\
\caption{Front-end\textsubscript{G} module insert of an item in cart sequence diagram}
\end{figure}

