\subsection{Services}
This part of the architecture lets the \textit{EmporioLambda} Front-end module retrieve or send data to the \textit{EmporioLambda} Back-end module. Services can be found inside the \textit{pages/api/Services} of the Front-end module.\\
Each service has the following composition:
\begin{itemize}
\item a Fetcher class which contains an URL generated with the pulled enviroment variables. The constructor requires the name of the function that will be called in the Back-end;
\item a getJSONResponse function which tries to make an API call to the Back-end and return the response. The return type is a Promise, as this is an asyncronous operation;
\item a getLambdaResponse function which accepts the following parameters:\begin{itemize}
\item a string which indicates the name of the function you want to call in your Back-end; 
\item a string which indicates the type of the call you want to do (GET, POST, PUT...);
\item a string containing the access token of the current session; 
\item a string containing the parameters needed for the API call. 
\end{itemize} 
This function creates a new Fetcher object and executes the getJSONResponse method. This is the function you must call in your services.
\end{itemize}
