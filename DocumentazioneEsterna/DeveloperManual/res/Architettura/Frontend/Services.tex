\subsection{Services}
This part of the architecture let EmporioLambda Front-end module retrieve or send data to the EmporioLambda Back-end module. Services can be found inside the \textit{pages/api/Services} of the Front-end module.\\
Each service has this composition:
\begin{itemize}
\item a class Fetcher which contains an URL generated with the enviroment variables pulled. The constructor requires the name of the function needed to call from the Back-end;
\item a function getJSONResponse which tries to make an API call to the Back-end and return the response. The return type is a Promise, since this is an asyncronous operation;
\item a function getLambdaResponse which accepts the following parameters:\begin{itemize}
\item a string which indicates the function name you want to call in your Back-end; 
\item a string which indicates the type of the call you want to do (GET, POST, PUT...);
\item a string containing the token of the current session; 
\item a string containing parameters if needed for the API call. 
\end{itemize} 
This function creates a new Fetcher object and execute the getJSONResponse method. This is the function you must call in your services.
\end{itemize}
