\subsection{Developing the Back-End Module}
To create a new endpoint function (a function that can be called from the Front-end\textsubscript{G} module) on the Back-End\textsubscript{G} module you can run the following command on your CLI: 
\begin{center}
\texttt{sls create function -f \$functionName --handler src/endpoints/\$functionName/handler.\$functionName}
\end{center}
but there are more downsides than upsides in doing so.\\
Here's a list of the downsides and additional things you have to do to make it work:
\begin{itemize}
\item the command will create a folder named \$functionName inside the src/endpoints folder, which will contain a handler.js file (with the name of the function) whose extension must be changed to .ts (since we want to use Typescript\textsubscript{G} instead of Javascript\textsubscript{G}). You will also need to change the code inside the created file, since it would also be written in Javascript\textsubscript{G}.
\item a file named \$functionName.js will also be created in the test folder (change the extension to .ts here aswell), which sets up an empty unit test for the new function. This is problematic because it would create a single unit test file for each function. This is not how unit tests are written in EmporioLambda, in fact each test file (named with a specific state or actions of a component for example cartPopulatedTable or cartPopulateTable) can contain multiple function tests.
In the automatically generated empty unit test file you will find a function called describe, into which you need to insert the following code:
\begin{lstlisting}
const mochaPlugin = require('serverless-mocha-plugin');
const expect = mochaPlugin.chai.expect;
const wrapped = mochaPlugin.getWrapper(
'$functionName', 
'/functions/$functionName/handler.ts',
'$functionName'
);
\end{lstlisting}
\item the command will also add the function row inside the serverless.yml file, but not the API\textsubscript{G} Gateway one, so you will need to add this code after the handler section of the function you just created:
\begin{lstlisting}
events:
   - http:
       path: $functionName
       method: get #(or post, put...)
       authorizer:
            name: auth
\end{lstlisting}
\end{itemize}
It is better to manually create all of the needed files.\\
Here's how:
\begin{enumerate}
\item create a Typescript\textsubscript{G} file which will contain your function inside the correct folder.\\The folders are the following:
\begin{itemize}
\item \textbf{src/endpoints:} contains all the endpoints functions;
\item \textbf{src/model:} contains all the classes and interfaces used by the Back-end\textsubscript{G} module;
\item \textbf{src/services:} contains all the functions related to the external services used by the \textit{EmporioLambda} Back-end\textsubscript{G} module;
\item \textbf{src/lib:} contains all the utilities functions used by the Back-end\textsubscript{G} module.
\item \textbf{test:} contains the test files.
\item \textbf{bashScript:} contains script that support the creation of a local database.
\end{itemize}
This folder also contains other folders, each one corresponding to a specific domain of \textit{EmporioLambda} (for example the endpoints folder contains the following folders: cart, category, checkout, order...). 
\item if you want to create a test for the new function go into the test folder and add it in the file whose name indicates the component your function work on.
\item now you need to add the function to the serverless.yml file. So open the file and insert this information about your function under the appropriate component:
\begin{lstlisting}
#function name
getUser:
   #The handler of the function (locationOfYourFunction.index)
   handler: src/endpoints/user/getUser.index
   #enviroment variables needed
   environment:
     USER_POOL_ID: !Ref CognitoUserPool
   #the API gateway data if it's an endpoint function
   events:
     - http:
         path: user/{username}/
         method: get
         cors: true
         authorizer:
            name: auth
\end{lstlisting}
\end{enumerate}

Most endpoints use a lamba authorizer for security purposes. If you create a new endpoint you have to define which users have access to it by adding the endpoint name in the /src/lib/auth.ts file in the 'API Visibility' section. By default the admin user has access to all endpoints.

To deploy your back-end module, follow the serverless documentation on this website: \url{https://www.serverless.com/framework/docs/providers/aws/cli-reference/deploy/}