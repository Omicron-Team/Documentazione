\subsection{Developing the Back-End Module}
To create a new function on the Back-End module you can run the following command on your CLI: 
\begin{center}
sls create function -f \$functionName --handler functions/\$functionName/handler.\$functionName
\end{center}
this will create a folder named \$functionName inside the functions folder, which will contain a handler.js file whose extension must be changed to .ts (since we want to use Typescript instead of Javascript). 
The command will also create a file named \$functionName.js into the test folder(change the extension to .ts here aswell), which sets up an empty unit test for the new function.
In this file you will find a function called describe, into which you need to insert the following code:
\begin{lstlisting}
const mochaPlugin = require('serverless-mocha-plugin');
const expect = mochaPlugin.chai.expect;
const wrapped = mochaPlugin.getWrapper(
'$functionName', 
'/functions/$functionName/handler.ts',
'$functionName'
);
\end{lstlisting}
You now have to open the serverless.yml file (the command automatically creates the row for the AWS Lambda function, but not for the API Gateway) and add this code after the handler section of the function you just created:
\begin{lstlisting}
events:
   - http:
       path: $functionName
       method: GET #(or POST)
\end{lstlisting}.
You can also manually create all of the needed files, even without the unit test ones.