\subsection{Testing}
This section shows how to run the testing process to check whether your code is working correctly and whether it follows the correct syntax. \\ 
To test the EmporioLambda code the following tools will be used:
\begin{itemize}
\item \textbf{Cypress}: for unit testing;
\item \textbf{ESLint} and \textbf{Prettier} for the syntax check of the code.
\end{itemize}

\subsubsection{Unit testing}
To run all of the unit tests, use the following command on your CLI:
\begin{center}
npm run test
\end{center}
If you want to test a single function in the Back-End module you can run the previous command while specifying the name of the function as a parameter.\\
Example:
\begin{center}
npm run test -f \$functionName
\end{center}
You can also check the code coverage for the Back-End module by running the command:
\begin{center}
npm run coverage
\end{center}

\subsubsection{Static code analysis}
Information on the rules used to run the static code analysis can be found in the files:
\begin{itemize}
\item .eslintrc.js
\item .prettierrc.js
\end{itemize}
Static code analysis can be executed by running the following commands in your CLI:
\begin{center}
npm run lint
\end{center}
this will check the code style with lint and prettier and try to fix the errors where it is possible.\\ 
If you just want to check the code without automatic error fixing you can run this command:
\begin{center}
npm run checkWithLint
\end{center}