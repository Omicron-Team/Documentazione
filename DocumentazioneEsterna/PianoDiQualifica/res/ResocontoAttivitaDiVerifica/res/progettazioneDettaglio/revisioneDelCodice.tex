\subsubsection{Revisione del codice}

\myparagraph{Analisi statica del codice}
Il codice prodotto durante questa fase è stato sottoposto ad opportuni controlli statici forniti da:
\begin{itemize}
	\item \textbf{ESLint\ped{G} v7.19.0:} linter per il linguaggio JavaScript\ped{G} e TypeScript\ped{G};
	\item \textbf{Prettier\ped{G} v2.2.1}: formattatore di codice.
\end{itemize}

Tramite questi due plugin il codice risulta sintatticamente corretto e standardizzato secondo opportune regole.

\myparagraph{Analisi dinamica del codice}
Il codice prodotto durante questa fase è stato sottoposto ad opportuni controlli dinamici all'interno delle repository\ped{G} su GitHub\ped{G} tramite GitHub Actions\ped{G}. I controlli dinamici sono stati effettuati tramite due software di test:
\begin{itemize}
	\item \textbf{Mocha\ped{G} v1.12.0:} software per l'esecuzione dei test all'interno del modulo back-end\ped{G};
	\item \textbf{Cypress\ped{G} v6.6.0}: software per l'esecuzione dei test all'interno del modulo front-end\ped{G}.
\end{itemize}

I test vengono effettuati dopo un \textit{commit} all'interno della repository\ped{G}. La GitHub Actions\ped{G} controllerà che tutti i test precedentemente effettuati (\textit{test di regressione}) e quelli appena aggiunti diano esito positivo, in caso contrario segnalerà tale esito negativo a colui che ha effettuato il \textit{commit}. Questa modalità garantisce di avere sempre un controllo su quello che viene inserito all'interno della repository\ped{G} e la relativa correttezza.\\
Per il dettaglio dei test controllare la sezione \S{4}.