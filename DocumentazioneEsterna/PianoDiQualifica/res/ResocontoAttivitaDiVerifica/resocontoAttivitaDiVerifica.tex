\section{Resoconto attività di verifica}
In tale sezione vengono descritti gli esiti delle attività di verifica svolte sui documenti redatti.

\subsection{Analisi dinamica dei documenti}
L'analisi dinamica mediante l'utilizzo dello strumento GitHub\ped{G} Actions\ped{G} ha garantito una stesura del codice \LaTeX{}\ped{G} priva di errori e più coesa fra tutti i membri del gruppo. Infatti, tramite questo strumento, è stato possibile individuare errori che potevano essere ignorati durante la compilazione locale del documento da qualche membro del team.

\subsection{Analisi statica dei documenti}
L'analisi statica dei documenti tramite \textit{Walkthrough}\ped{G} ha portato all'individuazione di errori comuni, i quali hanno portato ad un aggiornamento della lista di controllo già stilata in precedenza e consultabile all'interno delle \NdPv{} nella sezione \textit{Analisi statica}. In tale modo è stata resa più semplice e mirata l'attività di \textit{Inspection}\ped{G}. Inoltre sono stati riscontrati errori e/o migliorie per il template\ped{G} \LaTeX\ped{G} a cui fa riferimento ogni documento.