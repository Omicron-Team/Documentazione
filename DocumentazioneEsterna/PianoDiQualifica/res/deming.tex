\section{Ciclo di Deming - Ciclo PDCA}
Il ciclo di Deming (o ciclo PDCA, acronimo di \textit{Plan-Do-Check-Act}) è un metodo di gestione iterativo in quattro fasi utilizzato per il controllo e il miglioramento continuo dei processi e dei prodotti.\\
La sequenza logica dei quattro punti ripetuti per un miglioramento continuo è la seguente:
\begin{itemize}
\item \textbf{P} - \textit{Plan}: stabilire gli obiettivi e i processi necessari per fornire risultati in accordo con i risultati attesi;
\item \textbf{D} - \textit{Do}: esecuzione del programma, inizialmente in contesti circoscritti. Attuare il piano, eseguire il processo, creare il prodotto. Raccogliere i dati per la creazione di grafici e analisi da destinare alla fase di \textit{Check} e \textit{Act};
\item \textbf{C} - \textit{Check}: test e controllo, studio e raccolta dei risultati e dei riscontri;
\item \textbf{A} - \textit{Act}: azione per rendere definitivo e/o migliore il processo. Richiede azioni correttive sulle differenze significative tra i risultati effettivi e quelli previsti. 
\end{itemize}
Quando un procedimento, attraverso questi quattro passaggi, non comporta la necessità di migliorare la portata a cui è applicato, il ciclo PDCA può essere raffinato per pianificare e migliorare con maggior dettaglio la successiva iterazione, oppure l'attenzione deve essere posta in una diversa fase del processo.