\subsection{Processi organizzativi}
\subsubsection{Pianificazione}
\myparagraph{Metrica - Estimate at Completion}
\vspace{-1cm}
\begin{itemize}
	\item \textbf{Codice}: MPC6;
	\item \textbf{Descrizione}: Preventivo totale ricalcolato alla fine di un periodo;
	\item \textbf{Processo di riferimento}: Gestione;
	\item \textbf{Sigla}: EAC;
	\item \textbf{Indicatori utili}: 
		\begin{itemize}
		\item[$\ast$] \textbf{BAC - Budget at Completion}: budget preventivato per periodo;
		\end{itemize}
	\item \textbf{Range di valori che può assumere}: 
		\begin{itemize}
			\item \textbf{Ottimale} : $EAC \leq BAC$;
			\item \textbf{Accettabile} : $BAC - 5 \% \leq EAC \leq BAC + 5 \%$.
		\end{itemize}
\end{itemize}
\myparagraph{Metrica - Variance at Completion}
\vspace{-1cm}
\begin{itemize}
	\item \textbf{Codice}: MPC7;
	\item \textbf{Descrizione}: Spesa effettivamente sostenuta in percentuale;
	\item \textbf{Processo di riferimento}: Gestione;
	\item \textbf{Sigla}: VAC;
	\item \textbf{Indicatori utili}: 
		\begin{itemize}
		\item[$\ast$] \textbf{BAC - Budget at Completion}: budget preventivato precedentemente;
		\item[$\ast$] \textbf{EAC - Estimate at Completion}: §2.3.1.1.
		\end{itemize}
	\item \textbf{Formula}: \[ VAC = \frac{BAC - EAC}{100};\]
	\item \textbf{Range di valori che può assumere}: 
		\begin{itemize}
			\item \textbf{Ottimale} : $ VAC \geq 0 $;
			\item \textbf{Accettabile} : $ VAC \geq 0 $.
		\end{itemize}
\end{itemize}
\myparagraph{Metrica - Actual Cost}
\vspace{-1cm}
\begin{itemize}
	\item \textbf{Codice}: MPC8;
	\item \textbf{Descrizione}: Denaro speso fino al momento del calcolo;
	\item \textbf{Processo di riferimento}: Gestione;
	\item \textbf{Sigla}: AC;
	\item \textbf{Indicatori utili}: 
		\begin{itemize}
			\item[$\ast$] \textbf{PV - Planned Value}: costo pianificato per realizzare le attività di progetto alla data corrente.
		\end{itemize}
	\item \textbf{Range di valori che può assumere}: 
		\begin{itemize}
			\item \textbf{Ottimale} : $0 \leq AC \leq PV $;
			\item \textbf{Accettabile} : $0 \leq AC \leq budget \ totale $.
		\end{itemize}
\end{itemize}
\myparagraph{Metrica - Schedule Variance}
\vspace{-1cm}
\begin{itemize}
	\item \textbf{Codice}: MPC9;
	\item \textbf{Descrizione}: Indica se si è in linea, in anticipo o in ritardo rispetto alla schedulazione delle attività di progetto pianificate nella baseline\ped{G};
	\item \textbf{Processo di riferimento}: Gestione;
	\item \textbf{Sigla}: SV;
	\item \textbf{Indicatori utili}: 
		\begin{itemize}
		\item[$\ast$] \textbf{EV - Earned Value}: valore delle attività realizzate alla data corrente;
		\item[$\ast$] \textbf{PV - Planned Value}: costo pianificato per realizzare le attività di progetto alla data corrente.
		\end{itemize}
	\item \textbf{Formula}: \[ SV = EV - PV; \]
	\item \textbf{Range di valori che può assumere}: 
		\begin{itemize}
			\item \textbf{Ottimale} : $SV \geq 0$;
			\item \textbf{Accettabile} : $SV = 0$.
		\end{itemize}
\end{itemize}
\myparagraph{Metrica - Budget Variance}
\vspace{-1cm}
\begin{itemize}
	\item \textbf{Codice}: MPC10;
	\item \textbf{Descrizione}: Indica se alla data corrente si è speso di più o di meno rispetto a quanto previsto a budget alla data corrente;
	\item \textbf{Processo di riferimento}: Gestione;
	\item \textbf{Sigla}: BV;
	\item \textbf{Indicatori utili}: 
		\begin{itemize}
			\item[$\ast$] \textbf{PV - Planned Value}: costo pianificato per realizzare le attività di progetto alla data corrente;
			\item[$\ast$] \textbf{AC - Actual Cost}: denaro speso fino al momento del calcolo (§ 2.3.1.2).
		\end{itemize}
	\item \textbf{Formula}: \[ BV = PV - AC;\]
	\item \textbf{Range di valori che può assumere}: 
		\begin{itemize}
			\item \textbf{Ottimale} : $BV > 0$;
			\item \textbf{Accettabile} : $BV \geq 0$.
		\end{itemize}
\end{itemize}

\subsubsection{Gestione qualità}
\myparagraph{Metrica - Percentuale di metriche soddisfatte}
\vspace{-1cm}
\begin{itemize}
	\item \textbf{Codice}: MPC11;
	\item \textbf{Descrizione}: La percentuale di metriche\ped{G} soddisfatte valuta quante metriche\ped{G} raggiungono soglie accettabili sul numero totale delle metriche\ped{G} calcolate;
	\item \textbf{Processo di riferimento}: Gestione;
	\item \textbf{Sigla}: PMS;
	\item \textbf{Formula}: \[ PMS = \frac{metriche \ soddisfatte}{metriche \ totali} \ast 100;\]
	\item \textbf{Range di valori che può assumere}: 
		\begin{itemize}
			\item \textbf{Ottimale} : $PMS \geq 80 \%$;
			\item \textbf{Accettabile} : $PMS \geq 60 \%$.
		\end{itemize}
\end{itemize}