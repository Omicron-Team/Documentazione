\subsection{Processi di supporto}
\subsubsection{Documentazione}
\myparagraph{Metrica - Indice Gulpease}
\vspace{-1cm}
\begin{itemize}
	\item \textbf{Codice}: MPC3;
	\item \textbf{Descrizione}: Indice della leggibilità del testo. Utilizza la lunghezza delle parole in lettere anziché in sillabe, semplificandone il calcolo automatico;
	\item \textbf{Processo di riferimento}: Documentazione;
	\item \textbf{Sigla}: IG;
	\item \textbf{Formula}: \[ IG = 89 + \frac{300 \ast (numero \ frasi) - 10 \ast (numero \ lettere)}{(numero \ parole)} \]
	\item \textbf{Range di valori che può assumere}:
		\begin{itemize}
			\item \textbf{Ottimale}: $80 < IG < 100$;
			\item \textbf{Accettabile}: $40 < IG < 100$.
		\end{itemize}
\end{itemize}
\myparagraph{Metrica - Correttezza ortografica}
\vspace{-1cm}
\begin{itemize}
	\item \textbf{Codice}: MPC4;
	\item \textbf{Descrizione}: Nessun documento deve contenere errori grammaticali o errori ortografici;
	\item \textbf{Processo di riferimento}: Documentazione;
	\item \textbf{Sigla}: CO;
	\item \textbf{Range di valori che può assumere}:
		\begin{itemize}
			\item \textbf{Ottimale} : $CO = 0$;
			\item \textbf{Accettabile} : $CO = 0$.
		\end{itemize}
\end{itemize}

\subsubsection{Verifica}
\myparagraph{Metrica - Code Coverage}
\vspace{-1cm}
\begin{itemize}
	\item \textbf{Codice}: MPC5;
	\item \textbf{Descrizione}: Indica il numero di righe di codice percorse dai test durante la loro
esecuzione. Per linee di codice totali si intende tutte quelle appartenenti all'unità in fase di test;
	\item \textbf{Processo di riferimento}: Verifica;
	\item \textbf{Sigla}: CC;
	\item \textbf{Formula}: \[ CC = \frac{linee \ di \ codice \ percorse \ dai \ test}{linee \ di \ codice \ totali} \ast 100\]
	\item \textbf{Range di valori che può assumere}: 
		\begin{itemize}
			\item \textbf{Ottimale} : $CC = 100 \%$;
			\item \textbf{Accettabile} : $CC = 80 \%$.
		\end{itemize}
\end{itemize}

