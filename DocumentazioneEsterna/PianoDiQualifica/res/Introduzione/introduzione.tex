\section{Introduzione}

\subsection{Scopo del documento}
Il Piano di Qualifica ha lo scopo di definire le strategie di verifica e validazione al fine di garantire la qualità di prodotto e di processo. Per garantire ciò viene applicato un sistema di verifica continuo sui processi e sulle attività in corso.\\
Il presente documento verrà più volte modificato durante l'intera durata del progetto, questo perchè molte delle metriche scelte nella fase iniziale possono essere valutate a livello pratico solo in fasi successive. A tale scopo alcune parti sono prodotte in fasi temporali successive, come le specifiche dei test (§4), il resoconto delle attività di verifica (§A) e le valutazioni per il miglioramento (\S{B}). Per le ragioni appena citate, il documento è prodotto in modo incrementale.

\subsection{Scopo del prodotto}
Il prodotto trattato ha lo scopo di realizzare una piattaforma E-commerce\ped{G} generica, che possa potenzialmente vendere qualsiasi tipo di prodotto in modo da essere appetibile sia a privati che ad aziende.

\subsection{Glossario}
Al fine di migliorare la chiarezza del documento ed evitare possibili ambiguità, viene fornito un
\Glossario{} contenente i termini più critici scelti dai membri del gruppo, e una loro spiegazione.
In questo documento, tali termini verranno indicati con la lettera `G’ a pedice della parola.

\subsection{Riferimenti}

\subsubsection{Riferimenti normativi}
\begin{itemize}
	\item \textbf{\NdP}: \NdPv{3.0.0}.
\end{itemize}

\subsubsection{Riferimenti informativi}
\sloppy
\begin{itemize}
	\item \textbf{Indice di Gulpease:}\\
	\url{https://it.wikipedia.org/wiki/Indice_Gulpease};
	\item \textbf{ISO/IEC 15504:}\\
	\url{https://en.wikipedia.org/wiki/ISO/IEC_15504};
	\item \textbf{ISO/IEC 9126:}\\
	\url{https://it.wikipedia.org/wiki/ISO/IEC_9126};
	\item \textbf{Metriche pianificazione costi:}\\
	\url{https://www.insight.com/en_US/content-and-resources/tech-tutorials/using-earned-value-to-monitor-project-performance.html}.
\end{itemize}
\fussy