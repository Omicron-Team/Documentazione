\subsection{Valutazione dell'organizzazione}

{
\rowcolors{2}{azzurro2}{azzurro3}

\centering
\renewcommand{\arraystretch}{2}
\begin{longtable}{C{2cm} C{5cm} C{2cm} C{5.2cm}}
\caption{Tabella valutazione dell'organizzazione}\\
\rowcolor{azzurro1}
\textbf{Problema} &
\textbf{Descrizione}&
\textbf{Gravità}&
\textbf{Soluzione}\\
\endhead


Incontri interni del gruppo & Difficoltà nel trovare giorni d'incontro del gruppo che rispettassero gli impegni personali di ognuno & 2 & Qualora qualche membro non fosse presente, verrà registrato l'audio della video chiamata e resa fruibile a tutto il team tramite Google Drive\ped{G}\\
Incontri esterni con il proponente & Difficoltà nel trovare giorni d'incontro con il proponente che rispettassero gli impegni personali dei membri del gruppo & 2 & Qualora qualche membro non fosse presente, il team si preoccuperà di descrivere l'incontro avvenuto con il proponente al successivo incontro interno del team\\
Assegnazioni compiti & Difficoltà nell'organizzazione, nell'assegnazione dei compiti da svolgere e la comunicazione a tutti i membri del gruppo & 1 & Si è notato che fare video chiamate fra i membri del gruppo che dovevano redigere un documento era molto oneroso in termini temporali. Si è deciso, quindi,  di incrementare l'utilizzo dell'issue di GitHub\ped{G} e di dividerle in Board Kanban\ped{G} differenti.\\


\end{longtable}
}
