\subsection{Valutazione dell'organizzazione}

{
\rowcolors{2}{azzurro2}{azzurro3}

\centering
\renewcommand{\arraystretch}{2}
\begin{longtable}{C{2cm} C{5cm} C{2cm} C{5.2cm}}
\caption{Tabella valutazione dell'organizzazione}\\
\rowcolor{azzurro1}
\textbf{Problema} &
\textbf{Descrizione}&
\textbf{Gravità}&
\textbf{Soluzione}\\
\endhead

\rowcolor{azzurro1}
\multicolumn{4}{c}{\textbf{Fase di analisi}}\\

Incontri interni del gruppo & Difficoltà nel trovare giorni d'incontro del gruppo che rispettassero gli impegni personali di ognuno & 2 & Qualora qualche membro non fosse presente, verrà registrato e reso fruibile a tutto il team tramite Google Drive\ped{G} l'audio della chiamata di gruppo\\
Incontri esterni con il proponente & Difficoltà nel trovare giorni d'incontro con il proponente che rispettassero gli impegni personali dei membri del gruppo & 2 & Qualora qualche membro non fosse presente, il team si preoccuperà di descrivere l'incontro avvenuto con il proponente al successivo incontro interno del team\\
Assegnazioni compiti & Difficoltà nell'organizzazione, nell'assegnazione dei compiti da svolgere e la comunicazione a tutti i membri del gruppo & 1 & Si è notato che fare video chiamate fra i membri del gruppo che dovevano redigere un documento era molto oneroso in termini temporali. Si è deciso, quindi,  di incrementare l'utilizzo dell'issue di GitHub\ped{G} e di dividerle in Board Kanban\ped{G} differenti\\

\rowcolor{azzurro1}
\multicolumn{4}{c}{\textbf{Fase di progettazione architetturale}}\\

Organizzazione durante la sessione d'esame & Durante il periodo della sessione invernale, la maggior parte dei membri del team sono stati occupati con gli esami universitari. Questo ha necessitato un'organizzazione del carico di lavoro del progetto mirata in base agli impegni dei singoli. & 3 & Il gruppo ha utilizzato lo strumento \textit{Google Calendar} per avere una visione generale degli impegni universitari di ognuno e per organizzare meglio i compiti da assegnare.\\
Stesura criticità dopo la fase di RR & La valutazione dei documenti durante la fase di Revisione dei Requisiti ha evidenziato alcune criticità e/o miglioramenti da apportare. & 4 & Le varie criticità dapprima sono state riassunte in un documento \textit{Google Docs}, poi inserite come \textit{issue} all'interno della repository\ped{G} relativa alla documentazione in GitHub\ped{G}\\


\end{longtable}
}
