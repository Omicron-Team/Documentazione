\section{Qualità di prodotto}
È necessario poter valutare la qualità\ped{G} del prodotto, per questo motivo il gruppo \Omicron{} ha deciso di far riferimento allo standard\ped{G} ISO/IEC 9126 che determina le caratteristiche di un prodotto di qualità\ped{G}. Si è scelto di far riferimento alle seguenti caratteristiche:
\begin {itemize}
	\item{Funzionalità;}
	\item{Affidabilità;}
	\item{Usabilità;}
	\item{Manutenibilità.}
\end{itemize}

\subsection{Funzionalità}
La \textit{funzionalità} è la capacità del prodotto di soddisfare i requisiti.
\subsubsection{Obiettivi}
Gli obiettivi preposti sono:
\begin {itemize}
	\item \textbf{Adeguatezza:} capacità del prodotto di mettere a disposizione le funzioni necessarie per soddisfare gli obiettivi stabiliti;
	\item \textbf{Accuratezza:} capacità di fornire i risultati attesi con la precisione richiesta;
	\item \textbf{Conformità:} il prodotto deve essere conforme e rispettare determinati standard\ped{G}.
\end{itemize}
\subsubsection{Metriche}
\myparagraph{Completezza informazioni}
\vspace{-1cm}
\begin{itemize}
	\item \textbf{Codice}: MPD1;
	\item \textbf{Descrizione}: il grado di completezza del prodotto viene determinato tramite una percentuale;
	\item \textbf{Processo di riferimento}: sviluppo;
	\item \textbf{Sigla}: CI;
	\item \textbf{Formula}: \[CI=\left(1-\frac{funzioni \ non \ implementate}{funzioni \ implementate}\right)\ast100;\]
	\item \textbf{Range di valori che può assumere}: 
		\begin{itemize}
			\item \textbf{Ottimale} : $CI = 100 \%$;
			\item \textbf{Accettabile} : $CI = 100 \%$.
		\end{itemize}
\end{itemize}

\subsection{Affidabilità}
L'\textit{affidabilità} è la capacità del prodotto di funzionare anche in situazioni prolungate, di gestire una grande quantità di informazioni o di utenti.
\subsubsection{Obiettivi}
Gli obiettivi preposti sono:
\begin {itemize}
	\item \textbf{Maturità:} non si devono verificare errori o malfunzionamenti durante l'esecuzione;
	\item \textbf{Tolleranza degli errori:} il prodotto deve mantenere alte prestazioni anche in caso di un uso scorretto o malfunzionamenti.
\end{itemize}
\subsubsection{Metriche}
\myparagraph{Densità degli errori}
\vspace{-1cm}
\begin{itemize}
	\item \textbf{Codice}: MPD2;
	\item \textbf{Descrizione}: la capacità di resistenza agli errori viene espressa tramite l'utilizzo di una percentuale;
	\item \textbf{Processo di riferimento}: verifica;
	\item \textbf{Sigla}: DE;
	\item \textbf{Formula}: \[DE=\frac{numero \ errori}{numero \ test \ effettuati}\ast100;\]
	\item \textbf{Range di valori che può assumere}: 
		\begin{itemize}
			\item \textbf{Ottimale} : $DE = 0 \%$;
			\item \textbf{Accettabile} : $DE \leq 15 \%$.
		\end{itemize}
\end{itemize}

\subsection{Usabilità}
Con \textit{usabilità} si intende un prodotto \textit{user friendly}\ped{G}, in particolare quindi si fa riferimento alla capacità del prodotto di essere comprensibile e di facile utilizzo da parte del cliente finale.


\subsubsection{Obiettivi}
\begin {itemize}
	\item \textbf{Comprensibilità:} l'utente deve comprendere il prodotto e poter trovare le informazioni facilmente;
	\item \textbf{Apprendibilità:} deve essere facile per l'utente imparare ad utilizzarlo;
	\item \textbf{Attrattività:} deve essere interessante e catturare l'attenzione dell'utente.
\end{itemize}
\subsubsection{Metriche}
\myparagraph{Facilità di utilizzo}
\vspace{-1cm}
\begin{itemize}
	\item \textbf{Codice}: MPD3;
	\item \textbf{Descrizione}: questa metrica\ped{G} viene espressa con il numero di click necessari per raggiungere un'informazione;
		\item \textbf{Processo di riferimento}: sviluppo;
	\item \textbf{Misurazione:} click necessari per raggiungere il checkout;
	\item \textbf{Range di valori che può assumere}: 
		\begin{itemize}
			\item \textbf{Ottimale} : $Click \leq10$;
			\item \textbf{Accettabile} : $Click \leq15$.
		\end{itemize}
\end{itemize}
\myparagraph{Facilità di apprendimento}
\vspace{-1cm}
\begin{itemize}
	\item \textbf{Codice}: MPD4;
	\item \textbf{Descrizione}: questa metrica\ped{G} viene espressa come il numero di minuti necessari per raggiungere l'informazione (in media);
	\item \textbf{Processo di riferimento}: sviluppo;
	\item \textbf{Misurazione:} numero di minuti necessari per raggiungere il checkout;
	\item \textbf{Range di valori che può assumere}: 
		\begin{itemize}
			\item \textbf{Ottimale} : $Minuti \leq6$;
			\item \textbf{Accettabile} : $Minuti \leq8$.
		\end{itemize}
\end{itemize}
\myparagraph{Dimensione gerarchia}
\vspace{-1cm}
\begin{itemize}
	\item \textbf{Codice}: MPD5;
	\item \textbf{Descrizione}: la mappa di un sito per ritenersi di facile comprensione non deve essere troppo profonda;
	\item \textbf{Processo di riferimento}: sviluppo;
	\item \textbf{Misurazione:} livello di profondità mappa del sito;
	\item \textbf{Range di valori che può assumere}: 
		\begin{itemize}
			\item \textbf{Ottimale} : $Dimensione \leq5$;
			\item \textbf{Accettabile} : $Dimensione \leq8$.
		\end{itemize}
\end{itemize}
\subsection{Manutenibilità}
Capacità del prodotto di essere modificato, mediante miglioramenti o correzioni.
\subsubsection{Obiettivi}
\begin{itemize}
	\item \textbf{Analizzabilità:} capacità del prodotto di essere analizzato per localizzare gli errori;
	\item \textbf{Modificabilità:} capacità del prodotto di permettere l'aggiunta di una o più modifiche.
\end{itemize}
\subsubsection{Metriche}
\myparagraph{Facilità di comprensione}
\vspace{-1cm}
\begin{itemize}
	\item \textbf{Codice}: MPD6;
	\item \textbf{Descrizione}: si ottiene una percentuale mettendo in rapporto le linee di commenti con le linee di codice;
	\item \textbf{Processo di riferimento}: sviluppo;
	\item \textbf{Sigla}: FC;
	\item \textbf{Formula}: \[FC=\frac{linee \ di \ commenti}{linee \ di \ codice}\ast100;\]
	\item \textbf{Range di valori che può assumere}: 
		\begin{itemize}
			\item \textbf{Ottimale} : $FC \geq 20 \%$;
			\item \textbf{Accettabile} : $FC \geq 10 \%$.
		\end{itemize}
\end{itemize}
\myparagraph{Semplicità classi}
\vspace{-1cm}
\begin{itemize}
	\item \textbf{Codice}: MPD7;
	\item \textbf{Descrizione}: una classe viene considerata semplice quando ha pochi metodi;
	\item \textbf{Processo di riferimento}: sviluppo;
	\item \textbf{Sigla}: \textit{SC};
	\item \textbf{Indicatori utili}: 
		\begin{itemize}
		\item[$\ast$] \textbf{N:} numero di classi;
		\item[$\ast$] \textbf{m:} numero di metodi.
		\end{itemize}
	\item \textbf{Misurazione:} \[SC=\frac{\sum_{i=1}^{N} m_i}{N};\]
	\item \textbf{Range di valori che può assumere}: 
		\begin{itemize}
			\item \textbf{Ottimale} : $SC \leq 10$;
			\item \textbf{Accettabile} : $SC \leq 15$.
		\end{itemize}
\end{itemize}
\myparagraph{Semplicità funzioni}
\vspace{-1cm}
\begin{itemize}
	\item \textbf{Codice}: MPD8;
	\item \textbf{Descrizione}: una funzione viene considerata semplice quando ha pochi parametri;
	\item \textbf{Processo di riferimento}: sviluppo;
	\item \textbf{Sigla}: \textit{SF};
	\item \textbf{Indicatori utili}: 
		\begin{itemize}
		\item[$\ast$] \textbf{N:} numero di funzioni;
		\item[$\ast$] \textbf{p:} numero di parametri.
		\end{itemize}
	\item \textbf{Misurazione:} \[SF=\frac{\sum_{i=1}^{N} p_i}{N};\]
	\item \textbf{Range di valori che può assumere}: 
		\begin{itemize}
			\item \textbf{Ottimale} : $SF \leq 3$;
			\item \textbf{Accettabile} : $SF \leq 6$.
		\end{itemize}
\end{itemize}

\myparagraph{Dipendenze di una classe}
\vspace{-1cm}
\begin{itemize}
	\item \textbf{Codice}: MPD9;
	\item \textbf{Descrizione}: numero di relazioni che una classe ha con altre classi;
	\item \textbf{Processo di riferimento}: sviluppo;
	\item \textbf{Sigla}: \textit{DC};
	\item \textbf{Indicatori utili}: 
		\begin{itemize}
		\item[$\ast$] \textbf{N:} numero di classi;
		\item[$\ast$] \textbf{d:} numero di dipendenze verso altre classi.
		\end{itemize}
	\item \textbf{Misurazione:} \[DC=\frac{\sum_{i=1}^{N} d_i}{N};\]
	\item \textbf{Range di valori che può assumere}:
		\begin{itemize}
			\item \textbf{Ottimale}: $DC \leq 3 $;
			\item \textbf{Accettabile}: $DC \leq 6 $.
		\end{itemize}
\end{itemize}
