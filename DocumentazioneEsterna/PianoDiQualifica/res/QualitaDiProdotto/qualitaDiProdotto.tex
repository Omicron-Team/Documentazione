\section{Qualità di prodotto}
È necessario poter valutare la qualità del prodotto, per questo motivo il gruppo \Omicron{} ha deciso di far riferimento allo standard ISO/IEC 9126 che determina le caratteristiche di un prodotto di qualità. Si è scelto di far riferimento alle seguenti caratteristiche:
\begin {itemize}
	\item{Funzionalità;}
	\item{Affidabilità;}
	\item{Usabilità;}
	\item{Manutenibilità;}
\end{itemize}

\subsection{Funzionalità}
La \textit{Funzionalità} è la capacità del prodotto di soddisfare i requisiti.
\subsubsection{Obiettivi}
Gli obiettivi preposti sono:
\begin {itemize}
	\item \textbf{Adeguatezza:} capacità del prodotto di mettere a disposizione le funzioni necessarie per soddisfare gli obiettivi stabili;
	\item \textbf{Accuratezza:} capacità di fornire i risultati attesi con la precisione richiesta;
	\item \textbf{Conformità:} il prodotto deve essere conforme e rispettare determinati standard.
\end{itemize}
\subsubsection{Metriche}
\paragraph{MPD1 - Completezza implementazione} è possibile determinare il grado di completezza del prodotto tramite una percentuale.
\begin{itemize}
	\item \textbf{Misurazione:} \begin{math}C=\left(1-\frac{N_{FNI}}{N_{FI}}\right)*100\end{math}\\
	dove:
	\begin {itemize}
		\item \begin{math}N_{FNI}\end{math}: numero di funzioni NON implementate;
		\item \begin{math}N_{FI}\end{math}: numero di funzioni implementate.
	\end{itemize}
	\item \textbf{valore ideale:} 100\%;
	\item \textbf{valore accettabile:} 100\%.
\end{itemize}

\subsection{Affidabilità}
L'\textit{affidabilità} è la capacità del prodotto di funzionare anche in situazioni prolungate, di gestire una grande quantità di informazioni o di utenti.
\subsubsection{Obiettivi}
Gli obiettivi preposti sono:
\begin {itemize}
	\item \textbf{maturità:} non si devono verificare errori o malfunzionamenti durante l'esecuzione;
	\item \textbf{tolleranza degli errori:} il prodotto deve mantenere alte prestazioni anche in caso di un uso scorretto o malfunzionamenti.
\end{itemize}
\subsubsection{Metriche}
\paragraph{MPD2 - Densita degli errori} La capacità di resistenza agli errori viene espressa tramite l'utilizzo di una percentuale.
\begin{itemize}
	\item \textbf{Misurazione:} \begin{math}M=\frac{N_{ER}}{N_{TE}}*100\end{math}\\
	dove:
	\begin {itemize}
		\item \begin{math}N_{ER}\end{math}: numero di errori rilevati durante il testing;
		\item \begin{math}N_{TE}\end{math}: numero test effetuati.
	\end{itemize}
	\item \textbf{valore ideale:} 0\%;
	\item \textbf{valore accettabile:}  \begin{math}\leq15\%\end{math}.
\end{itemize}

\subsection{Usabilità}
Con \textit{usabilità} si intende la capacità del prodotto di essere di facile utilizzo e comprensione da parte dell'utente, quello che frequentemente viene definito \textit{user-friendly}.
\subsubsection{Obiettivi}
\begin {itemize}
	\item \textbf{Comprensibile:} l'utente deve comprendere il prodotto e poter trovare le informazioni facilmente;
	\item \textbf{Apprendibile:} deve essere facile per l'utente imparare ad utilizzarlo;
	\item \textbf{Attrattivo:} deve essere interessante e catturare l'attenzione dell'utente.
\end{itemize}
\subsubsection{Metriche}
\paragraph{MPD3 - Facilità di utilizzo} Questa metrica viene espressa come il numero di click necessari per raggiungere un'informazione.
\begin{itemize}
	\item \textbf{misurazione:} click necessari per raggiungere il checkout;
	\item \textbf{valore preferibile:} \begin{math}\leq10\end{math};
	\item \textbf{valore accettabile:} \begin{math}\leq15\end{math}.
\end{itemize}
\paragraph{MPD4 - Facilità di apprendimento} Questa metrica viene espressa come il numero di minuti necessari per raggiungere l'informazione (in media).
\begin{itemize}
	\item \textbf{misurazione:} numero di minuti necessari per raggiungere il checkout;
	\item \textbf{valore preferibile:} \begin{math}\leq6\end{math};
	\item \textbf{valore accettabile:} \begin{math}\leq8\end{math}.
\end{itemize}
\paragraph{MPD5 - Dimensione gerarchia} Un sito per ritenersi di facile comprensione non deve essere troppo profondo.
\begin{itemize}
	\item \textbf{misurazione:} livello di profondità;
	\item \textbf{valore preferibile:} \begin{math}\leq5\end{math};
	\item \textbf{valore accettabile:} \begin{math}\leq8\end{math}.
\end{itemize}
\subsection{Manutenibilità}
Capacità del prodotto di essere modificato, mediante miglioramenti o correzioni.
\subsubsection{Obiettivi}
\begin{itemize}
	\item \textbf{analizzabile:} capacità del prodotto di essere analizzato per localizzare gli errori;
	\item \textbf{modificabile:} capacità del prodotto di permettere l'aggiunta di una o più modifiche.
\end{itemize}
\subsubsection{Metriche}
\paragraph{MPD6 - Facilità di comprensione}  È possibile comprendere la facilità del codice tramite una percentuale mettendo in rapporto le linee di commenti con le linee di codice, in particolare:
\begin{itemize}
	\item \textbf{Misurazione:} \begin{math}R=\frac{N_{LCOM}}{N_{LCOD}}*100\end{math}\\
	dove:
	\begin {itemize}
		\item \begin{math}N_{LCOM}\end{math}: indica il numero delle linee di commenti;
		\item \begin{math}N_{LCOD}\end{math}: indica il numero delle linee di codice.
	\end{itemize}
	\item \textbf{valore ideale:} \begin{math}\geq20\%\end{math};
	\item \textbf{valore accettabile:} \begin{math}\geq10\%\end{math}.
\end{itemize}
\paragraph{MPD7 - Semplicità classi} una classe potrebbe considerarsi semplice quando ha pochi metodi, quindi:
\begin{itemize}
	\item \textbf{Misurazione:} numero metodi per classe;
	\item \textbf{valore ideale:} \begin{math}\leq10\end{math};
	\item \textbf{valore accettabile:} \begin{math}\leq15\end{math}.
\end{itemize}
\paragraph{MPD8 - Semplicità funzioni} una funzione potrebbe considerarsi semplice quando ha pochi parametri.
\begin{itemize}
	\item \textbf{Misurazione:} numero di parametri in un metodo;
	\item \textbf{valore ideale:} \begin{math}\leq3\end{math};
	\item \textbf{valore accettabile:} \begin{math}\leq6\end{math}.
\end{itemize}
