\section{Qualità di prodotto}
È necessario poter valutare la qualità del prodotto, per questo motivo il gruppo \Omicron{} ha deciso di far riferimento allo standard ISO/IEC 9126 che determina le caratteristiche di un prodotto di qualità. Si è scelto di far riferimento alle seguenti caratteristiche:
\begin {itemize}
	\item{Funzionalità;}
	\item{Affidabilità;}
	\item{Efficienza;}
	\item{Usabilità;}
	\item{Manutenibilità;}
	\item{Portabilità;}
\end{itemize}

\subsection{Funzionalità}
La \textit{Funzionalità} è la capacità del prodotto di soddisfare i requisiti.
\subsubsection{Obiettivi}
gli obiettivi preposti sono:
\begin {itemize}
	\item \textbf{Adeguatezza:} capacità del prodotto di mettere a disposizione le funzioni necessarie per soddisfare gli obiettivi stabili;
	\item \textbf{Accuratezza:} capacità di fornire i risultati attesi con la precisione richiesta;
	\item \textbf{Conformità:} il prodotto deve essere conforme e rispettare determinati standard.
\end{itemize}
\subsubsection{Metriche}
\paragraph{MPD1 - Completezza implementazione} tramite una percentuale viene determinata quanto il prodotto è completo
\begin{itemize}
	\item \textbf{Misurazione:} \begin{math}C=\left(1-\frac{N_{FNI}}{N_{FI}}\right)*100\end{math}\\
	dove:
	\begin {itemize}
		\item \begin{math}N_{FNI}\end{math}: numero di funzioni NON implementate;
		\item \begin{math}N_{FI}\end{math}: numero di funzioni implementate.
	\end{itemize}
	\item \textbf{valore ideale:} 100\%;
	\item \textbf{valore accettabile:} 100\%.
\end{itemize}

\subsection{Affidabilità}
L'\textit{affidabilità} è la capacità del prodotto di funzionare anche in situazioni prolungate o di gestire una grande quantità di informazioni.
\subsubsection{Obiettivi}
gli obiettivi preposti sono:
\begin {itemize}
	\item \textbf{maturità:} non si devono verificare errori o malfunzionamenti durante l'esecuzione;
	\item \textbf{tolleranza degli errori:} il prodotto deve mantenere alte prestazioni anche in caso di un uso scorretto o malfunzionamenti.
\end{itemize}
\subsubsection{Metriche}
\paragraph{MPD2 - Densita degli errori} La capacità di resistenza agli errori viene espressa tramite l'utilizzo di una percentuale.
\begin{itemize}
	\item \textbf{Misurazione:} \begin{math}M=\frac{N_{ER}}{N_{TE}}*100\end{math}\\
	dove:
	\begin {itemize}
		\item \begin{math}N_{ER}\end{math}: numero di errori rilevati durante il testing;
		\item \begin{math}N_{TE}\end{math}: numero test effetuati.
	\end{itemize}
	\item \textbf{valore ideale:} 0\%;
	\item \textbf{valore accettabile:}  \begin{math}\leq15\%\end{math}.
\end{itemize}

\subsection{Efficienza} Per efficienza si intende la capacità del prodotto di soddisfare le richieste sfruttando in modo ottimale le risorse messe a disposizione impiegando il minor tempo possibile.
\subsubsection{Obiettivi}
\begin{itemize}
	\item \textbf{uso delle risorse:} capacità di sfruttare solo le risorse necessarie per risolvere una data problematica;
	\item \textbf{Tempistiche di risposta:} capacità del prodotto di fornire risposta al problema in tempi accettabili.
\end{itemize}
\subsubsection{Metriche}


\subsection{Usabilità}
\subsection{Manutenibilità}
\subsection{Portabilità}