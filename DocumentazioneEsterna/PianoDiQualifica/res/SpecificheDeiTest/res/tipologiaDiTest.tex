\subsection{Tipologia di test}

\subsubsection{Test di unità}
Il test ha lo scopo di valutare la correttezza delle singole \textit{unità} software indipendenti fra di loro. Tale tipologia di test verrà indicata con:\\
\begin{center}
	\textbf{TU[id]}
\end{center}
dove \textit{id} rappresenta un'unità.

\subsubsection{Test di integrazione}
Il test ha lo scopo di valutare la correttezza delle relazioni fra le varie unità, che vanno a formare un \textit{componente}. Tale tipologia di test verrà indicata con:\\
\begin{center}
	\textbf{TI[id]}
\end{center}
dove \textit{id} rappresenta un componente.
\subsubsection{Test di sistema}
Il test ha lo scopo di valutare la correttezza dell'intero sistema, formato dai vari componenti. Tale tipologia verrà indicata con:\\
\begin{center}
	\textbf{TS}
\end{center}

\subsubsection{Test di accettazione o test di collaudo}
Il test ha lo scopo di valutare il soddisfacimento del cliente e dimostrare che il software soddisfi i requisiti concordati. Essi vengono effettuati durante il collaudo finale.