\subsection{Tabelle riassuntive dei test}

\subsubsection{Test di unità}
La tabella verrà redatta con l'avanzare del progetto.

\subsubsection{Test di integrazione}
La tabella verrà redatta con l'avanzare del progetto.

\subsubsection{Test di sistema}
{
\rowcolors{2}{azzurro2}{azzurro3}

\centering
\renewcommand{\arraystretch}{2}
\begin{longtable}{C{3cm} C{10.2cm} C{1.5cm}}
\caption{Tabella riassuntiva test di sistema}\\
\rowcolor{azzurro1}
\textbf{Codice} &
\textbf{Descrizione}&
\textbf{Stato}\\
\endhead


TS\_O\_F\_1 & Un utente deve aver la possibilità di registrasi con:
\begin{itemize}
	\item nome;
	\item cognome;
	\item indirizzo di fatturazione;
	\item email;
	\item password.
\end{itemize} & NI\\
TS\_D\_F\_1 & Il sistema deve mostrare un errore se
l’email inserita nella registrazione è già stata utilizzata & NI\\
TS\_O\_F\_2 & Un utente registrato può effettuare il login inserendo:
\begin{itemize}
	\item email;
	\item password.
\end{itemize} &  NI\\
TS\_D\_F\_2 & Il sistema deve mostrare un errore se le credenziali del login sono errate & NI\\
TS\_O\_F\_3 & Un utente autenticato deve poter effettuare il
logout & NI\\
TS\_O\_F\_4 & Un utente deve accedere alla pagina del carrello dalla Homepage, dalla PLP e dal PDP & NI\\

\end{longtable}


}
\subsubsection{Test di accettazione o test di collaudo}
La tabella verrà redatta con l'avanzare del progetto.