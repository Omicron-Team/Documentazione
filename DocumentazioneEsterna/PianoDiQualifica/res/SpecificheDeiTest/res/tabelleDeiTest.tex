\subsection{Tabelle riassuntive dei test}

\subsubsection{Test di unità}
La tabella verrà redatta con l'avanzare del progetto.

\subsubsection{Test di integrazione}
La tabella verrà redatta con l'avanzare del progetto.

\subsubsection{Test di sistema}
{
\rowcolors{2}{azzurro2}{azzurro3}

\centering
\renewcommand{\arraystretch}{2}
\begin{longtable}{C{3cm} C{10.2cm} C{1.5cm}}
\caption{Tabella riassuntiva test di sistema}\\
\rowcolor{azzurro1}
\textbf{Codice} &
\textbf{Descrizione}&
\textbf{Stato}\\
\endhead


TS\_O\_F\_1 & Un utente deve aver la possibilità di registrasi con:
\begin{itemize}
	\item nome;
	\item cognome;
	\item indirizzo di fatturazione;
	\item email;
	\item password.
\end{itemize} & NI\\
TS\_D\_F\_1 & Il sistema deve mostrare un errore se
l’email inserita nella registrazione è già stata utilizzata & NI\\
TS\_O\_F\_2 & Un utente registrato può effettuare il login inserendo:
\begin{itemize}
	\item email;
	\item password.
\end{itemize} &  NI\\
TS\_D\_F\_2 & Il sistema deve mostrare un errore se le credenziali del login sono errate & NI\\
TS\_O\_F\_3 & Un utente autenticato deve poter effettuare il
logout & NI\\
TS\_O\_F\_4 & Un utente deve accedere alla pagina del carrello dalla Homepage, dalla PLP\ped{G} e dal PDP\ped{G} & NI\\




TS\_O\_F\_5 & Un utente dal carrello può:
\begin{itemize}
	\item visualizzare i prodotti precedentemente aggiunti;
	\item visualizzare nome e immagine del prodotto;
	\item rimuovere i singoli prodotti dal carrello;
	\item modificare la quantità dei prodotti;
	\item visualizzare il costo delle voci del carrello;
	\item visualizzare il prezzo totale del carrello;
	\item visualizzare le tasse applicate al prezzo finale del carrello.
\end{itemize}
& NI\\

TS\_O\_F\_6 & Un utente può effettuare il checkout se:
\begin{itemize}
	\item ha almeno un prodotto nel carrello;
	\item si è registrato.
\end{itemize}
& NI\\

TS\_O\_F\_7 & Durante la fase di checkout:
\begin{itemize}
	\item viene visualizzato il totale da pagare e i prodotti che verranno acquistati;
	\item l'utente deve inserire i dati di pagamento tramite Stripe\ped{G};
	\item dopo aver inserito i dati di pagamento, l'utente continua il pagamento effettivo tramite Stripe\ped{G}.
\end{itemize}
& NI\\

TS\_O\_F\_8 & A pagamento riuscito:
\begin{itemize}
	\item l'utente visualizza un riepilogo dell'ordine effettuato;
	\item l'utente riceve i prodotti acquistati tramite l'email del suo account.
\end{itemize}
& NI\\

TS\_O\_F\_9 & A pagamento fallito:
\begin{itemize}
	\item l'utente visualizza un messaggio d'errore;
	\item l'utente può riprovare il pagamento verificando i dati inseriti.
\end{itemize}
& NI\\


TS\_O\_F\_10 & Un utente registrato può visualizzare il suo profilo, contenente:
\begin{itemize}
	\item nome;
	\item cognome;
	\item password;
	\item indirizzo di fatturazione;
	\item email;
	\item lista degli ordini effettuati.
\end{itemize}
& NI\\

TS\_O\_F\_11 & Un utente registrato può aggiornare, nel suo profilo, le seguenti informazioni:
\begin{itemize}
	\item nome;
	\item cognome;
	\item password;
	\item indirizzo di fatturazione;
	\item email.
\end{itemize}
& NI\\

TS\_O\_F\_12 & Per ogni ordine effettuato vengono visualizzati:
\begin{itemize}
	\item prodotti acquistati;
	\item quantità dei prodotti acquistati;
	\item costo singolo prodotto;
	\item costo totale;
	\item tasse applicate;
	\item data di acquisto.
\end{itemize}
& NI\\

TS\_O\_F\_13 & Un utente registrato può eliminare il suo account. & NI\\

TS\_O\_F\_14 & Il venditore ha a disposizione una dashboard dove può inserire nuovi prodotti. & NI\\

TS\_D\_F\_14 & Il venditore può accedere agli strumenti esterni riservati per gli admin dalla dashboard & NI\\

TS\_O\_F\_15 & Durante l'inserimento di un nuovo prodotto, il venditore, deve assegnare i seguenti dati:
\begin{itemize}
	\item nome;
	\item descrizione;
	\item immagine.
\end{itemize} & NI\\

TS\_O\_F\_16 & Il venditore può:
\begin{itemize}
	\item visualizzare i prodotti da lui venduti;
	\item modificare la descrizione dei prodotti da lui venduti;
	\item eliminare i prodotti da lui venduti.
	\item visualizzare i dettagli di tutti gli ordini effettuati dai clienti;
\end{itemize} & NI\\

TS\_D\_F\_16 & Il venditore può modificare il nome e l'immagine dei prodotti da lui venduti. & NI\\

TS\_O\_F\_17 & Il venditore, durante la visualizzazione di un prodotto, visualizza:
\begin{itemize}
	\item nome;
	\item descrizione;
	\item immagine.
\end{itemize} & NI\\

TS\_O\_F\_18 & Il venditore, per ogni ordine effettuato, può visualizzare:
\begin{itemize}
	\item prodotti acquistati;
	\item quantità dei prodotti acquistati;
	\item costo singolo prodotto;
	\item costo totale;
	\item tasse applicate;
	\item data di acquisto.
\end{itemize}
& NI\\





TS\_O\_F\_19 & L'utente può effettuare una ricerca tra i prodotti in vendita & NI\\

TS\_O\_F\_20 & L'utente può accedere ad una PLP\ped{G} corrispondente ad una categoria di prodotti. Nella PLP\ped{G} viene visualizzata la lista di prodotti corrispondente & NI\\

TS\_O\_F\_21 & Per ogni prodotto listato nella PLP\ped{G} viene visualizzato:
\begin{itemize}
	\item nome;
	\item immagine;
	\item prezzo.
\end{itemize} & NI\\

TS\_O\_F\_22 & Nella PLP\ped{G} un utente può:
\begin{itemize}
	\item selezionare alcuni prodotti;
	\item aggiungere i prodotti al carrello.
\end{itemize} & NI\\

TS\_D\_F\_22 & Nella PLP\ped{G} un utente può:
\begin{itemize}
	\item filtrare l'insieme dei prodotti visualizzati in base al prezzo;
	\item modificare la quantità del prodotto da aggiungere al carrello.
\end{itemize} & NI\\

TS\_O\_F\_23 & Un utente può accedere alla PDP\ped{G} cliccando su un prodotto nella PLP\ped{G}. & NI\\

TS\_O\_F\_24 & Un utente, nella PDP\ped{G} può:
\begin{itemize}
	\item visualizzare nome, immagine, descrizione, prezzo, tasse applicate del prodotto.
	\item aggiungere il prodotto al carrello scegliendone la quantità.
\end{itemize} & NI\\



\end{longtable}


}
\subsubsection{Test di accettazione o test di collaudo}
La tabella verrà redatta con l'avanzare del progetto.