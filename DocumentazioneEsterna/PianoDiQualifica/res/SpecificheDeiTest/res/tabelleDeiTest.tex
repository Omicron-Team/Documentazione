\subsection{Tabelle riassuntive dei test}

\subsubsection{Test di unità modulo Back-End}
{
\rowcolors{2}{azzurro2}{azzurro3}

\centering
\renewcommand{\arraystretch}{2}
\begin{longtable}{C{2cm} C{9.6cm} C{1.5cm} C{1cm}}
\caption{Tabella riassuntiva test di sistema nel modulo back-end\ped{G}}\\
\rowcolor{azzurro1}
\textbf{Codice} &
\textbf{Descrizione}&
\textbf{Stato}&
\textbf{Esito}\\
\endhead


TUB1 & Viene verificato il corretto inserimento di un prodotto & I & S\\

TUB2 & Viene verificata la correttezza del messaggio d'errore durante l'inserimento di un prodotto con categoria non esistente & I & S\\


TUB3 & Viene verificata la correttezza del messaggio d'errore durante l'inserimento di un prodotto con un'immagine non passata correttamente & I & S\\


TUB4 & Viene verificata la correttezza del messaggio d'errore durante l'inserimento di un prodotto con un'immagine con un'estensione non supportata & I & S\\


TUB5 & Viene verificata la correttezza del messaggio d'errore durante l'inserimento di un prodotto dopo l'errore di decodifica di un'immagine & I & S\\

TUB6 & Viene verificata la correttezza del messaggio d'errore durante l'inserimento di un prodotto non conforme alle attese & I & S\\


TUB7 & Viene verificata la correttezza del messaggio d'errore durante l'inserimento di un prodotto con una richiesta senza il parametro \textit{body} & I & S\\


TUB8 & Viene verificata la correttezza del messaggio d'errore durante l'inserimento di un prodotto senza nome & I & S\\


TUB9 & Viene verificata la correttezza del messaggio d'errore durante l'inserimento di un prodotto senza descrizione & I & S\\

TUB10 & Viene verificata la correttezza del messaggio d'errore durante la modifica di un prodotto con una richiesta senza il parametro \textit{body} & I & S\\


TUB11 & Viene verificata la correttezza del messaggio d'errore durante la modifica di un prodotto con una richiesta senza il parametro \textit{PathParameters} & I & S\\

TUB12 & Viene verificata la correttezza del messaggio d'errore durante la modifica di un prodotto non esistente & I & S\\

TUB13 & Viene verificata la corretta cancellazione di un prodotto & I & S\\

TUB14 & Viene verificata la correttezza del messaggio d'errore durante la cancellazione di un prodotto con una richiesta senza il parametro \textit{PathParameters} & I & S\\

TUB15 & Viene verificata la correttezza del messaggio d'errore durante la cancellazione di un prodotto non esistente & I & S\\

TUB16 & Viene verificata la corretta visualizzazione della lista di tutti i prodotti & I & S\\

TUB17 & Viene verificata la corretta ricerca di un prodotto & I & S\\

TUB18 & Viene verificata la corretta ricerca di molteplici prodotti & I & S\\

TUB19 & Viene verificata la corretta ricerca di un prodotto tramite prezzo massimo & I & S\\

TUB20 & Viene verificata la corretta ricerca di un prodotto tramite categoria & I & S\\

TUB21 & Viene verificata la corretta ricerca di un prodotto tramite prezzo minimo & I & S\\

TUB22 & Viene verificata la corretta ricerca dei prodotti limitando il numero di risultati da visualizzare & I & S\\

TUB23 & Viene verificata la corretta visualizzazione di un prodotto richiesto tramite il suo ID & I & S\\

TUB24 & Viene verificata la correttezza del messaggio d'errore durante la richiesta di visualizzazione di tutti i prodotti con la tabella vuota & I & S\\

TUB25 & Viene verificata la correttezza del messaggio d'errore durante la ricerca di un prodotto con una richiesta senza \textit{PathParameters} & I & S\\

TUB26 & Viene verificata la correttezza del messaggio d'errore durante la ricerca di un prodotto con una richiesta con \textit{PathParameters} errato & I & S\\

TUB27 & Viene verificata la correttezza del messaggio d'errore durante la ricerca di un prodotto non presente & I & S\\

TUB28 & Viene verificata la correttezza del messaggio d'errore durante la richiesta di un prodotto non presente tramite l'ID & I & S\\

TUB29 & Viene verificata la correttezza del messaggio d'errore durante una richiesta senza \textit{PathParameters} di un prodotto tramite l'ID & I & S\\

TUB30 & Viene verificata la correttezza del messaggio d'errore durante una richiesta non conforme di un prodotto tramite l'ID & I & S\\

TUB31 & Viene verificata la correttezza del messaggio d'errore durante la creazione di un ordine con un utente non corretto & I & S\\

TUB32 & Viene verificata la correttezza del messaggio d'errore durante la creazione di un ordine non conforme & I & S\\

TUB33 & Viene verificata la correttezza del messaggio d'errore durante la creazione di un ordine con un \textit{cart} non conforme & I & S\\

TUB34 & Viene verificata la correttezza del messaggio d'errore durante la creazione di un ordine con un \textit{cart} non presente & I & S\\

TUB35 & Viene verificata la correttezza del messaggio d'errore durante la creazione di un ordine senza un'email a cui inviare l'ordine & I & S\\

TUB36 & Viene verificata la correttezza della visualizzazione dei dati degli ordini effettuati da uno specifico utente & I & S\\

TUB37 & Viene verificata la correttezza del messaggio d'errore durante la visualizzazione degli ordini di uno specifico utente con una richiesta senza \textit{PathParameters} & I & S\\

TUB38 & Viene verificata la correttezza del messaggio d'errore durante la visualizzazione degli ordini di un utente che non ne ha effettuati & I & S\\

TUB39 & Viene verificata la correttezza della visualizzazione dei dati di un ordine tramite il proprio ID & I & S\\

TUB40 & Viene verificata la correttezza del messaggio d'errore durante la visualizzazione di un ordine tramite il proprio ID con una richiesta senza senza \textit{PathParameters} & I & S\\

TUB41 & Viene verificata la correttezza del messaggio d'errore durante la visualizzazione di un ordine tramite il proprio ID con una richiesta errata & I & S\\

TUB42 & Viene verificata la correttezza del messaggio d'errore durante la visualizzazione di un ordine non presente tramite un ID & I & S\\

TUB43 & Viene verificata la correttezza del messaggio d'errore durante la creazione di una sessione Stripe\ped{G} con una richiesta senza \textit{body} & I & S\\

TUB44 & Viene verificata la correttezza del messaggio d'errore durante la creazione di una sessione Stripe\ped{G} con un \textit{cart} non presente & I & S\\


TUB45 & Viene verificata la correttezza della modifica dei prodotti all'interno del carrello durante la creazione di una sessione Stripe\ped{G} con alcuni prodotti modificati durante la fase di checkout & I & S\\

TUB46 & Viene verificata la correttezza della cancellazione dei prodotti all'interno del carrello durante la creazione di una sessione Stripe\ped{G} con alcuni prodotti eliminati durante la fase di checkout & I & S\\

TUB47 & Viene verificata la correttezza del messaggio d'errore durante la cancellazione di un utente & I & S\\


\end{longtable}

}

\subsubsection{Test di integrazione}
La tabella verrà redatta con l'avanzare del progetto.

\subsubsection{Test di sistema}
{
\rowcolors{2}{azzurro2}{azzurro3}

\centering
\renewcommand{\arraystretch}{2}
\begin{longtable}{C{2cm} C{9.6cm} C{1.5cm} C{1cm}}
\caption{Tabella riassuntiva test di sistema}\\
\rowcolor{azzurro1}
\textbf{Codice} &
\textbf{Descrizione}&
\textbf{Stato}&
\textbf{Esito}\\
\endhead


TS1F1 & Un utente deve aver la possibilità di registrasi con:
\begin{itemize}
	\item nome;
	\item cognome;
	\item indirizzo di fatturazione;
	\item email;
	\item password.
\end{itemize} & I & S\\
TS2F1 & Il sistema deve mostrare un errore se
i campi inseriti durante la fase di registrazione non sono validi & NI & NS\\
TS1F2 & Un utente può effettuare il login inserendo:
\begin{itemize}
	\item email;
	\item password.
\end{itemize} & I & S\\
TS2F2 & Il sistema deve mostrare un errore se le credenziali del login sono errate & I & S\\
TS1F3 & Un utente autenticato deve poter effettuare il
logout & I & S\\
TS1F4 & Un utente generico deve accedere alla pagina del carrello dalla Homepage, dalla PLP\ped{G} e dal PDP\ped{G} & I & S\\
TS1F5 & Un utente generico, dal carrello, può:
\begin{itemize}
	\item visualizzare i prodotti precedentemente aggiunti;
	\item visualizzare nome, immagine e quantità del prodotto;
	\item rimuovere i singoli prodotti dal carrello;
	\item modificare la quantità dei prodotti;
	\item visualizzare il costo delle voci del carrello;
	\item visualizzare il prezzo totale dei prodotti nel carrello;
	\item visualizzare le tasse applicate al prezzo totale dei prodotti nel carrello.
\end{itemize}
& I & S\\

TS1F6 & Un cliente può effettuare il checkout se:
\begin{itemize}
	\item ha almeno un prodotto nel carrello;
	\item si è registrato.
\end{itemize}
& I & S\\

TS1F7 & Durante la fase di checkout:
\begin{itemize}
	\item viene visualizzata l'email a cui verranno mandati i prodotti;
	\item il cliente deve inserire i dati di pagamento tramite Stripe\ped{G};
	\item dopo aver inserito i dati di pagamento, il cliente continua il pagamento effettivo tramite Stripe\ped{G}.
\end{itemize}
& I & S\\

TS2F7 & Durante la fase di checkout il cliente deve poter modificare l'email a cui verranno mandati i prodotti
& I & S\\

TS1F8 & A pagamento riuscito:
\begin{itemize}
	\item il cliente visualizza un riepilogo dell'ordine effettuato;
	\item il cliente riceve i prodotti acquistati tramite l'email usata per l'acquisto.
\end{itemize}
& NI & NS\\

TS1F9 & A pagamento fallito:
\begin{itemize}
	\item il cliente visualizza un messaggio d'errore;
	\item il cliente può riprovare il pagamento verificando i dati inseriti.
\end{itemize}
& NI & NS\\


TS1F10 & Un cliente può visualizzare il suo profilo, contenente:
\begin{itemize}
	\item nome;
	\item cognome;
	\item indirizzo di fatturazione;
	\item email;
	\item lista degli ordini effettuati.
\end{itemize}
& I & S\\

TS1F11 & Un cliente può aggiornare, nel suo profilo, le seguenti informazioni:
\begin{itemize}
	\item nome;
	\item cognome;
	\item password;
	\item indirizzo di fatturazione;
	\item email.
\end{itemize}
& I & S\\

TS2F11 & Se la modifica delle informazioni nel profilo non va a buon fine, viene visualizzato un errore & NI & NS\\

TS1F12 & Per ogni ordine effettuato vengono visualizzati:
\begin{itemize}
	\item id ordine;
	\item prodotti acquistati;
	\item quantità dei prodotti acquistati;
	\item costo singolo prodotto;
	\item costo totale;
	\item tasse applicate;
	\item data di acquisto.
\end{itemize}
& I & S\\

TS1F13 & Un cliente può eliminare il suo account & NI & NS\\

TS1F14 & Il venditore ha a disposizione una dashboard\ped{G} dove può inserire nuovi prodotti e gestire il proprio catalogo digitale & I & S\\

TS2F14 & Il venditore, dalla dashboard\ped{G}, può accedere agli strumenti esterni riservati per gli admin & I & S\\

TS1F15 & Durante l'inserimento di un nuovo prodotto, il venditore, deve assegnare i seguenti dati:
\begin{itemize}
	\item nome;
	\item descrizione;
	\item immagine;
	\item prezzo;
	\item categoria.
\end{itemize} & I & S\\

TS1F16 & Il venditore può:
\begin{itemize}
	\item visualizzare i prodotti da lui venduti;
	\item modificare la descrizione dei prodotti da lui venduti;
	\item eliminare i prodotti da lui venduti;
	\item visualizzare i dettagli di tutti gli ordini effettuati dai clienti.
\end{itemize} & I & S\\

TS2F16 & Il venditore può modificare il nome, il prezzo, l'immagine e la categoria dei prodotti da lui venduti & I & S\\

TS1F17 & Il venditore, durante la visualizzazione di un prodotto, visualizza:
\begin{itemize}
	\item nome;
	\item descrizione;
	\item categoria;
	\item prezzo;
	\item immagine.
\end{itemize} & I & S\\

TS1F18 & Il venditore, per ogni ordine effettuato, può visualizzare:
\begin{itemize}
	\item numero ordine;
	\item prodotti acquistati;
	\item quantità dei prodotti acquistati;
	\item costo singolo prodotto;
	\item costo totale;
	\item tasse applicate;
	\item data di acquisto.
\end{itemize}
& I & S\\

TS1F19 & Il venditore può gestire le categorie dei prodotti, ovvero:
\begin{itemize}
	\item visualizzare una lista di categorie di prodotti;
	\item inserire nuove categorie di prodotti;
	\item eliminare categorie esistenti di prodotti.
\end{itemize}
& I & S\\





TS1F20 & Un utente generico può effettuare una ricerca tra i prodotti in vendita & I & S\\

TS1F21 & L'utente generico può accedere alla PLP\ped{G} corrispondente ad una categoria di prodotti. Nella PLP\ped{G} viene visualizzata la lista di prodotti corrispondente & I & S\\

TS1F22 & Per ogni prodotto listato nella PLP\ped{G} viene visualizzato:
\begin{itemize}
	\item nome;
	\item immagine;
	\item prezzo.
\end{itemize} & I & S\\

TS1F23 & Nella PLP\ped{G} un utente generico può:
\begin{itemize}
	\item selezionare alcuni prodotti;
	\item aggiungere i prodotti al carrello.
\end{itemize} & I & S\\

TS2F23 & Nella PLP\ped{G} un utente generico può filtrare l'insieme dei prodotti visualizzati in base al prezzo & NI & NS\\

TS1F24 & Un utente generico può accedere alla PDP\ped{G} cliccando su un prodotto nella PLP\ped{G} & I & S\\

TS1F25 & Un utente generico, nella PDP\ped{G} può:
\begin{itemize}
	\item visualizzare nome, immagine, descrizione, prezzo e tasse applicate del prodotto;
	\item aggiungere il prodotto al carrello scegliendone la quantità.
\end{itemize} & I & S\\



\end{longtable}


}
\subsubsection{Test di accettazione o test di collaudo}
La tabella verrà redatta con l'avanzare del progetto.