\section{Specifiche dei test}
Il team intende attuare una strategia di testing che segue il \textbf{modello a V}\ped{G}. Tale modello accoppia vari tipi di test alle attività di analisi e progettazione e garantisce il controllo del prodotto software nel suo complesso. A correlazione dei test verranno redatte delle tabelle che daranno un'indicazione sull'output prodotto, il quale preciserà se il risultato ottenuto è quello atteso oppure no.\\
Le tabelle saranno così descritte:
\begin{enumerate}
	\item \textbf{Codice:} codice identificativo del test;
	\item \textbf{Descrizione:} una breve descrizione di ciò che verifica il test;
	\item \textbf{Stato:} identifica lo stato del test:
	\begin{itemize}
		\item \textbf{I:} identifica un test \textit{implementato};
		\item \textbf{NI:} identifica un test \textit{non implementato}. 
	\end{itemize}
	\item \textbf{Esito:} identifica l'esito del test (colonna presente solo se lo stato del test è \textit{implementato}):
	\begin{itemize}
		\item \textbf{S:} identifica un test \textit{superato};
		\item \textbf{NS:} identifica un test \textit{non superato}. 
	\end{itemize}
\end{enumerate}
La descrizione dei \textit{codici} dei test sono documentati nelle \NdPv{4.0.0} nella sezione \textit{Codifica dei test}.

\subsection{Metriche dei test}
Il gruppo ha deciso di stabilire delle metriche per l'esecuzione e il soddisfacimento dei test. Tali metriche verranno ampliate, se necessario, durante la prosecuzione del progetto.


{
\rowcolors{2}{azzurro2}{azzurro3}

\centering
\renewcommand{\arraystretch}{2}
\begin{longtable}{C{8.7cm} C{3cm} C{3cm}}
\caption{Tabella metriche dei test}\\
\rowcolor{azzurro1}
\textbf{Nome} &
\textbf{Valore accettabile}&
\textbf{Valore desiderabile}\\
\endhead


Copertura dei requisiti da parte dei test & 100\% & 100\%\\
Copertura del codice da parte dei test & 85\% & 100\%\\
Percentuale test passati & 85\% & 100\%\\


\end{longtable}
}


\subsection{Tabelle riassuntive dei test}

\subsubsection{Test di unità}
{
\rowcolors{2}{azzurro2}{azzurro3}

\centering
\renewcommand{\arraystretch}{2}
\begin{longtable}{C{2cm} C{9.6cm} C{1.5cm} C{1cm}}
\caption{Tabella riassuntiva test di unità}\\
\rowcolor{azzurro1}
\textbf{Codice} &
\textbf{Descrizione}&
\textbf{Stato}&
\textbf{Esito}\\
\endhead


TU1 & Viene verificato il corretto inserimento di un prodotto & I & S\\

TU2 & Viene verificata la correttezza del messaggio d'errore durante l'inserimento di un prodotto con categoria non esistente & I & S\\


TU3 & Viene verificata la correttezza del messaggio d'errore durante l'inserimento di un prodotto con un'immagine non passata correttamente & I & S\\


TU4 & Viene verificata la correttezza del messaggio d'errore durante l'inserimento di un prodotto con un'immagine con un'estensione non supportata & I & S\\


TU5 & Viene verificata la correttezza del messaggio d'errore durante l'inserimento di un prodotto dopo l'errore di decodifica di un'immagine & I & S\\

TU6 & Viene verificata la correttezza del messaggio d'errore durante l'inserimento di un prodotto non conforme alle attese & I & S\\


TU7 & Viene verificata la correttezza del messaggio d'errore durante l'inserimento di un prodotto con una richiesta senza il parametro \textit{body} & I & S\\


TU8 & Viene verificata la correttezza del messaggio d'errore durante l'inserimento di un prodotto senza nome & I & S\\


TU9 & Viene verificata la correttezza del messaggio d'errore durante l'inserimento di un prodotto senza descrizione & I & S\\

TU10 & Viene verificato il corretto funzionamento della modifica di un prodotto & I & S\\

TU11 & Viene verificata la correttezza del messaggio d'errore durante la modifica di un prodotto con una richiesta senza il parametro \textit{body} & I & S\\


TU12 & Viene verificata la correttezza del messaggio d'errore durante la modifica di un prodotto con una richiesta senza il parametro \textit{PathParameters} & I & S\\

TU13 & Viene verificata la correttezza del messaggio d'errore durante la modifica di un prodotto non esistente & I & S\\

TU14 & Viene verificata la corretta cancellazione di un prodotto & I & S\\

TU15 & Viene verificata la correttezza del messaggio d'errore durante la cancellazione di un prodotto con una richiesta senza il parametro \textit{PathParameters} & I & S\\

TU16 & Viene verificata la correttezza del messaggio d'errore durante la cancellazione di un prodotto non esistente & I & S\\

TU17 & Viene verificata la corretta visualizzazione della lista di tutti i prodotti & I & S\\

TU18 & Viene verificata la corretta ricerca di un prodotto & I & S\\

TU19 & Viene verificata la corretta ricerca di molteplici prodotti & I & S\\

TU20 & Viene verificata la corretta ricerca di un prodotto tramite prezzo massimo & I & S\\

TU21 & Viene verificata la corretta ricerca di un prodotto tramite categoria & I & S\\

TU22 & Viene verificata la corretta ricerca di un prodotto tramite prezzo minimo & I & S\\

TU23 & Viene verificata la corretta ricerca dei prodotti limitando il numero di risultati da visualizzare & I & S\\

TU24 & Viene verificata la corretta visualizzazione di un prodotto richiesto tramite il suo ID & I & S\\

TU25 & Viene verificata la correttezza del messaggio d'errore durante la richiesta di visualizzazione di tutti i prodotti con la tabella vuota & I & S\\

TU26 & Viene verificata la correttezza del messaggio d'errore durante la ricerca di un prodotto con una richiesta senza \textit{PathParameters} & I & S\\

TU27 & Viene verificata la correttezza del messaggio d'errore durante la ricerca di un prodotto con una richiesta con \textit{PathParameters} errato & I & S\\

TU28 & Viene verificata la correttezza del messaggio d'errore durante la ricerca di un prodotto non presente & I & S\\

TU29 & Viene verificata la correttezza del messaggio d'errore durante la richiesta di un prodotto non presente tramite l'ID & I & S\\

TU30 & Viene verificata la correttezza del messaggio d'errore durante una richiesta senza \textit{PathParameters} di un prodotto tramite l'ID & I & S\\

TU31 & Viene verificata la correttezza del messaggio d'errore durante una richiesta non conforme di un prodotto tramite l'ID & I & S\\

TU32 & Viene verificata la correttezza del messaggio d'errore durante la creazione di un ordine con un utente non corretto & I & S\\

TU33 & Viene verificata la correttezza del messaggio d'errore durante la creazione di un ordine non conforme & I & S\\

TU34 & Viene verificata la correttezza del messaggio d'errore durante la creazione di un ordine con un \textit{cart} non conforme & I & S\\

TU35 & Viene verificata la correttezza del messaggio d'errore durante la creazione di un ordine con un \textit{cart} non presente & I & S\\

TU36 & Viene verificata la correttezza del messaggio d'errore durante la creazione di un ordine senza un'email a cui inviare l'ordine & I & S\\

TU37 & Viene verificata la correttezza della visualizzazione dei dati degli ordini effettuati da uno specifico utente & I & S\\

TU38 & Viene verificata la correttezza del messaggio d'errore durante la visualizzazione degli ordini di uno specifico utente con una richiesta senza \textit{PathParameters} & I & S\\

TU39 & Viene verificata la correttezza del messaggio d'errore durante la visualizzazione degli ordini di un utente che non ne ha effettuati & I & S\\

TU40 & Viene verificata la correttezza della visualizzazione dei dati di un ordine tramite il proprio ID & I & S\\

TU41 & Viene verificata la correttezza del messaggio d'errore durante la visualizzazione di un ordine tramite il proprio ID con una richiesta senza senza \textit{PathParameters} & I & S\\

TU42 & Viene verificata la correttezza del messaggio d'errore durante la visualizzazione di un ordine tramite il proprio ID con una richiesta errata & I & S\\

TU43 & Viene verificata la correttezza del messaggio d'errore durante la visualizzazione di un ordine non presente tramite un ID & I & S\\

TU44 & Viene verificata la correttezza del messaggio d'errore durante la creazione di una sessione Stripe\ped{G} con una richiesta senza \textit{body} & I & S\\

TU45 & Viene verificata la correttezza del messaggio d'errore durante la creazione di una sessione Stripe\ped{G} con un \textit{cart} non presente & I & S\\


TU46 & Viene verificata la correttezza della modifica dei prodotti all'interno del carrello durante la creazione di una sessione Stripe\ped{G} con alcuni prodotti modificati durante la fase di checkout & I & S\\

TU47 & Viene verificata la correttezza della cancellazione dei prodotti all'interno del carrello durante la creazione di una sessione Stripe\ped{G} con alcuni prodotti eliminati durante la fase di checkout & I & S\\

TU48 & Viene verificata la correttezza del messaggio d'errore durante la cancellazione di un utente & I & S\\

TU49 & Viene verificata la correttezza della creazione di un carrello & I & S\\

TU50 & Viene verificata la correttezza del messaggio d'errore durante della creazione di un carrello con una richiesta senza \textit{body} & I & S\\

TU51 & Viene verificata la correttezza del messaggio d'errore durante della creazione di un carrello con una richiesta senza \textit{username} relativo all'utente & I & S\\

TU52 & Viene verificata la correttezza del messaggio d'errore durante della creazione di un carrello con parametri non conformi & I & S\\

TU53 & Viene verificata la correttezza della visualizzazione dei dati di un carrello relativi ad un utente specifico & I & S\\

TU54 & Viene verificato il corretto inserimento di un prodotto all'interno di un carrello & I & S\\

TU55 & Viene verificata la corretta visualizzazione del prezzo totale del carrello dopo che è stato inserimento un nuovo prodotto all'interno di esso & I & S\\

TU56 & Viene verificata la correttezza del messaggio d'errore dopo l'inserimento di un prodotto all'interno del carrello con una richiesta senza \textit{body} & I & S\\

TU57 & Viene verificata la correttezza del messaggio d'errore dopo l'inserimento di un prodotto all'interno del carrello con una richiesta senza \textit{PathParameters} & I & S\\

TU58 & Viene verificata la correttezza del messaggio d'errore dopo l'inserimento di un prodotto all'interno del carrello con una richiesta non conforme & I & S\\

TU59 & Viene verificato il corretto inserimento di un prodotto all'interno del carrello senza avere un istanza del carrello già creata & I & S\\

TU60 & Viene verificato il corretto inserimento di un prodotto già presente all'interno del carrello (aumento della quantità)& I & S\\

TU61 & Viene verificata la correttezza del messaggio d'errore durante l'inserimento di un prodotto con una richiesta non conforme & I & S\\

TU62 & Viene verificata la correttezza del messaggio d'errore durante l'inserimento di un prodotto non più disponibile all'interno del carrello & I & S\\

TU63 & Viene verificata la corretta eliminazione di un prodotto all'interno del carrello & I & S\\

TU64 & Viene verificata la corretta visualizzazione del prezzo totale del carrello dopo l'eliminazione di un prodotto all'interno di esso & I & S\\

TU65 & Viene verificata la corretta visualizzazione dei prodotti all'interno del carrello dopo la modifica di uno di essi & I & S\\

TU66 & Viene verificata la correttezza del messaggio d'errore durante l'eliminazione di un prodotto all'interno del carrello non presente nel medesimo & I & S\\

TU67 & Viene verificata la correttezza del messaggio d'errore dopo l'eliminazione di un prodotto all'interno del carrello con una richiesta senza \textit{body} & I & S\\

TU68 & Viene verificata la correttezza del messaggio d'errore dopo l'eliminazione di un prodotto all'interno del carrello con una richiesta senza \textit{PathParameters} & I & S\\

TU69 & Viene verificata la correttezza del messaggio d'errore dopo lo svuotamento del carrello con una richiesta senza \textit{PathParameters} & I & S\\

TU70 & Viene verificata la correttezza del messaggio d'errore dopo lo svuotamento del carrello con una richiesta non conforme & I & S\\

TU71 & Viene verificata il corretto svuotamento del carrello & I & S\\

TU72 & Viene verificata la corretta eliminazione di un carrello & I & S\\

TU73 & Viene verificata la correttezza del messaggio d'errore dopo la cancellazione di un carrello con una richiesta senza \textit{PathParameters} & I & S\\

TU74 & Viene verificata la correttezza del messaggio d'errore dopo la cancellazione di un carrello con una richiesta non conforme & I & S\\

TU75 & Viene verificata la corretta visualizzazione dei dati all'interno di un carrello di uno specifico utente dopo che un prodotto non è più disponibile & I & S\\

TU76 & Viene verificata la correttezza del messaggio d'errore della visualizzazione di un carrello di uno specifico utente dopo una richiesta senza \textit{PathParameters} & I & S\\

TU77 & Viene verificata la correttezza del messaggio d'errore della visualizzazione di un carrello di uno specifico utente dopo una richiesta non conforme & I & S\\

TU78 & Viene verificata la correttezza del messaggio d'errore della visualizzazione di un carrello di uno specifico utente non presente & I & S\\

TU79 & Viene verificata la corretta visualizzazione dei dati di un specifico ordine (dato il suo ID), per uno specifico utente (dato il suo username) & I & S\\

TU80 & Viene verificata la correttezza del messaggio d'errore della visualizzazione dei dati di un specifico ordine (dato il suo ID), per uno specifico utente (dato il suo username), dove l'ID non è presente & I & S\\

TU81 & Viene verificata la correttezza del messaggio d'errore della visualizzazione dei dati di un specifico ordine (dato il suo ID), per uno specifico utente (dato il suo username), dove l'username non è presente & I & S\\

TU82 & Viene verificata la correttezza del messaggio d'errore generale riguardante la visualizzazione dei dati di un specifico ordine (dato il suo ID), per uno specifico utente (dato il suo username) & I & S\\

TU83 & Viene verificata la correttezza del messaggio d'errore relativo ad un prezzo passato non come numero durate la modifica di un prodotto & I & S\\

TU84 & Viene verificata la correttezza del messaggio di risposta relativo alla visualizzazione della lista degli ordini effettuati & I & S\\

TU85 & Viene verificata la correttezza del messaggio d'errore relativo alla visualizzazione della lista degli ordini vuota & I & S\\

TU86 & Viene verificata la correttezza del messaggio d'errore relativo alla cancellazione di un utente con una richiesta senza \textit{PathParameters} & I & S\\

TU87 & Viene verificata la correttezza del messaggio d'errore relativo alla visualizzazione dei dati di un utente con una richiesta senza \textit{PathParameters} & I & S\\

TU88 & Viene verificata la correttezza del messaggio d'errore durante la visualizzazione dei dati di un utente & I & S\\

TU89 & Viene verificata la correttezza del messaggio d'errore relativo alla modifica dei dati di un utente con una richiesta senza \textit{PathParameters} & I & S\\

TU90 & Viene verificata la correttezza del messaggio d'errore relativo alla modifica dei dati di un utente con una richiesta senza \textit{Body} & I & S\\

TU91 & Viene verificata la correttezza del messaggio d'errore durante la modifica dei dati di un utente & I & S\\

TU92 & Viene verificata la correttezza del messaggio d'errore relativo alla modifica della password di un utente con una richiesta senza \textit{PathParameters} & I & S\\

TU93 & Viene verificata la correttezza del messaggio d'errore relativo alla modifica della password di un utente con una richiesta senza \textit{Body} & I & S\\

TU94 & Viene verificata la correttezza del messaggio d'errore durante la modifica della password di un utente & I & S\\

TU95 & Viene verificata il corretto invio dell'email contenente un ordine effettuato   & I & S\\

TU96 & Viene verificata la correttezza del messaggio d'errore durante l'invio dell'email contenente un ordine effettuato   & I & S\\ 

TU97 & Viene verificata la correttezza del messaggio d'errore relativo alla creazione di una categoria con una richiesta senza \textit{Body} & I & S\\ 

TU98 & Viene verificata la correttezza del messaggio d'errore relativo alla creazione di una categoria con una richiesta senza nome della categoria & I & S\\ 

TU99 & Viene verificata la correttezza della creazione di una categoria & I & S\\ 

TU100 & Viene verificata la correttezza della cancellazione di una categoria & I & S\\ 

TU101 & Viene verificata la correttezza del messaggio d'errore relativo alla cancellazione di una categoria con una richiesta senza \textit{PathParameters} & I & S\\

TU102 & Viene verificata la correttezza del messaggio d'errore relativo alla cancellazione di una categoria dove non si riesce a controllare le categorie esistenti & I & S\\

TU103 & Viene verificata la correttezza del messaggio d'errore relativo alla cancellazione di una categoria in uso da almeno un prodotto & I & S\\

TU104 & Viene verificata la correttezza del messaggio d'errore relativo alla creazione di una categoria già presente & I & S\\

TU105 & Viene verificata la correttezza della visualizzazione della lista di categorie & I & S\\

TU106 & Viene verificata la correttezza del messaggio d'errore della visualizzazione della lista di categorie, dove tale lista è vuota & I & S\\

TU107 & Viene verificata la correttezza del messaggio d'errore durante l'autenticazione di un API\ped{G} con un token scaduto & I & S\\

TU108 & Viene verificata la correttezza del messaggio d'errore durante l'autenticazione di un API\ped{G} con un Kid non conforme alle attese & I & S\\

TU109 & Viene verificata la correttezza del messaggio d'errore durante l'autenticazione di un API\ped{G} con un tipo di token non conforme alle attese & I & S\\

TU110 & Viene verificata la correttezza del messaggio d'errore durante l'autenticazione di un API\ped{G} con un Iss non conforme alle attese & I & S\\

TU111 & Viene verificata la correttezza del messaggio d'errore durante l'autenticazione di un API\ped{G} con un Jwt non conforme alle attese & I & S\\

TU112 & Viene verificata la correttezza del messaggio d'errore durante l'autenticazione di un API\ped{G} con una richiesta senza nessun token & I & S\\

TU113 & Viene verificata la correttezza dell'autenticazione di un API\ped{G} con una richiesta da parte di un utente di tipo \textit{admin} & I & S\\

TU114 & Viene verificata la correttezza dell'autenticazione di un API\ped{G} con una richiesta da parte di un utente di tipo \textit{utente autenticato} & I & S\\

TU115 & Viene verificata la correttezza dell'autenticazione di un API\ped{G} con una richiesta da parte di un utente di tipo \textit{utente non autenticato} & I & S\\

TU116 & Viene verificata la correttezza del messaggio d'errore durante l'autenticazione di un API\ped{G} con un errore rilevate durante la connessione al server dove sono presenti le chiavi pubbliche & I & S\\

TU117 & Viene verificato il corretto caricamento della pagina relativa al dettaglio di un prodotto & I & S\\

TU118 & Viene verificata la correttezza dei dati relativi al dettaglio di un prodotto & I & S\\

TU119 & Viene verificato il corretto funzionamento del bottone che aggiunge un prodotto al carrello & I & S\\

TU120 & Viene verificato il corretto caricamento della pagina relativa al profilo di un utente & I & S\\

TU121 & Viene verificata la correttezza dei dati relativi ad un utente all'interno del proprio profilo & I & S\\

TU122 & Viene verificato che siano presenti i bottoni \textit{editPassword, editProfilo, delete} & I & S\\

TU123 & Viene verificato che siano presenti i bottoni \textit{submit, cancel} e la form per la modifica del profilo & I & S\\

TU124 & Viene verificato che il bottone \textit{cancel} reindirizza alla pagina \textit{profile} & I & S\\

TU125 & Viene verificato che non avvenga la modifica del profilo con la form vuota & I & S\\

TU126 & Viene verificato che non avvenga la modifica del profilo con una form contente dati in forma errata & I & S\\

TU127 & Viene verificato che venga modificato un profilo utente & I & S\\

TU128 & Viene verificato che fallisca la modifica della password con un dato vuoto & I & S\\

TU129 & Viene verificato che fallisca la modifica della password con un dato con una lunghezza inferiore ad 8 & I & S\\

TU130 & Viene verificato che fallisca la modifica della password con un dato non contenente almeno un carattere maiuscolo & I & S\\


TU131 & Viene verificato che fallisca la modifica della password con un dato non contenente almeno un carattere minuscolo & I & S\\

TU132 & Viene verificato che fallisca la modifica della password con un dato non contenente almeno un numero & I & S\\

TU133 & Viene verificato che fallisca la modifica della password con un dato non contenente almeno un carattere speciale & I & S\\

TU134 & Viene verificato che venga modificata la password di un utente & I & S\\

TU135 & Viene verificato il corretto caricamento della sezione degli ordini relativo ad un utente, dove non è presente nessun ordine & I & S\\

TU136 & Viene verificato il corretto caricamento della sezione degli ordini relativo ad un utente, dove sono presenti degli ordini & I & S\\

TU137 & Viene verificato il corretto funzionamento del bottone relativo al dettaglio dell'ordine & I & S\\

TU138 & Viene verificata la corretta eliminazione di un utente & I & S\\

TU139 & Viene verificato il corretto caricamento della pagina \textit{dashboard} & I & S\\

TU140 & Viene verificato se è presente il bottone per l'inserimento di un nuovo prodotto & I & S\\

TU141 & Viene verificato se è presente la form per l'inserimento di un nuovo prodotto & I & S\\

TU142 & Viene verificata la corretta visualizzazione della lista dei prodotti, dove questa non è vuota & I & S\\

TU143 & Viene verificata la correttezza dell'errore durante la creazione di un prodotto con un input errato & I & S\\

TU144 & Viene verificata la corretta creazione di un nuovo prodotto  & I & S\\

TU145 & Viene verificata la presenta del nuovo prodotto appena creato nella lista dei prodotti & I & S\\

TU146 & Viene verificato il corretto caricamento della pagina di modifica di un prodotto & I & S\\

TU147 & Viene verificato che il bottone \textit{cancel}, nella pagina di modifica di un prodotto, reindirizza alla pagina \textit{dashboard} & I & S\\

TU148 & Viene verificata la correttezza della modifica di un prodotto & I & S\\

TU149 & Viene verificata la correttezza dell'errore nella modifica di un prodotto senza input da parte dell'utente & I & S\\

TU150 & Viene verificata la correttezza dell'errore nella modifica di un prodotto con input errati da parte dell'utente & I & S\\

TU151 & Viene verificata la correttezza dell'eliminazione di un prodotto & I & S\\

TU152 & Viene verificata la correttezza dell'errore durante l'eliminazione di un prodotto & I & S\\

TU153 & Viene verificata la presenza del bottone \textit{Add new category} nella pagina \textit{dashboard} & I & S\\

TU154 & Viene verificata la corretta visualizzazione dell'alert dopo la creazione di una nuova categoria, e la relativa chiusura & I & S\\

TU155 & Viene verificata la corretta visualizzazione dell'alert dopo l'eliminazione di una categoria & I & S\\

TU156 & Viene verificata la corretta visualizzazione della lista degli ordini all'interno della pagina \textit{dashboard} & I & S\\

TU157 & Viene verificato il corretto re-indirizzamento alla pagina degli ordini tramite il bottone \textit{details} nella pagina \textit{dashboard} & I & S\\


TU158 & Viene verificata la corretta presenza del bottone \textit{sign in}& I & S\\

TU159 & Viene verificata la corretta presenza del bottone \textit{sign out} & I & S\\

TU160 & Viene verificata la corretta presenza del bottone \textit{merchantDashboard} & I & S\\

TU161 & Viene verificato il corretto funzionamento del bottone \textit{sign in} & I & S\\

TU162 & Viene verificato il corretto funzionamento del bottone \textit{sign out} & I & S\\

TU163 & Viene verificato che sia presente la \textit{search bar} all'interno della \textit{Home} & I & S\\

TU164 & Viene verificato che la \textit{search bar} reindirizza correttamente alla pagina di ricerca & I & S\\


\end{longtable}

}

\myparagraph{Tracciamento test di unità - metodo}
Di seguito viene riportato il tracciamento fra i test di unità ed i rispettivi metodi testati.

{
\rowcolors{2}{azzurro2}{azzurro3}

\centering
\renewcommand{\arraystretch}{2}
\begin{longtable}{C{2cm} C{12.8cm}}
\caption{Tabella per il tracciamento dei test - metodi}\\
\rowcolor{azzurro1}
\textbf{Test} &
\textbf{Metodo}\\
\endhead


TU1 \newline TU2 \newline TU3 \newline TU4 \newline TU5 \newline TU6 \newline TU7 \newline TU8 \newline TU9 & EmporioLambda-BE/src/endpoints/product/create.ts:index(event)\\

TU10 \newline TU11 \newline TU12 \newline TU13 \newline TU83 & EmporioLambda-BE/src/endpoints/product/update.ts:index(event)\\

TU14 \newline TU15 \newline TU16 & EmporioLambda-BE/src/endpoints/product/delete.ts:index(event)\\

TU17 \newline TU25 & EmporioLambda-BE/src/endpoints/product/list.ts:index()\\

TU18 \newline TU19 \newline TU20 \newline TU21 \newline TU22 \newline TU23 \newline TU26 \newline TU27 \newline TU28 & EmporioLambda-BE/src/endpoints/product/search.ts:index(event)\\

TU24 \newline TU29 \newline TU30 \newline TU31 & EmporioLambda-BE/src/endpoints/product/getById.ts:index(event)\\

TU32 \newline TU33 \newline TU34 \newline TU35 \newline TU36 & EmporioLambda-BE/src/endpoints/order/create.ts:index(event)\\

TU37 \newline TU38 \newline TU39 & EmporioLambda-BE/src/endpoints/order/getByUsername.ts:index(event)\\

TU40 \newline TU41 \newline TU42 \newline TU43 & EmporioLambda-BE/src/endpoints/order/getById.ts:index(event)\\

TU44 \newline TU45 \newline TU46 \newline TU47 & EmporioLambda-BE/src/endpoints/checkout/createSessionStripe.ts:index(event)\\

TU48 \newline TU86 & EmporioLambda-BE/src/endpoints/user/delete.ts:index(event)\\

TU49 \newline TU50 \newline TU51 \newline TU52 & EmporioLambda-BE/src/endpoints/cart/create.ts:index(event)\\

TU53 \newline TU55 \newline TU64 \newline TU65 \newline TU75 \newline TU76 \newline TU77 \newline TU78 & EmporioLambda-BE/src/endpoints/cart/getByUsername.ts:index(event)\\

TU54 \newline TU56 \newline TU57 \newline TU58 \newline TU59 \newline TU60 \newline TU61 \newline TU62 & EmporioLambda-BE/src/endpoints/cart/addProduct.ts:index(event)\\

TU63 \newline TU66 \newline TU67 \newline TU68 & EmporioLambda-BE/src/endpoints/cart/removeProduct.ts:index(event)\\

TU69 \newline TU70 \newline TU71 & EmporioLambda-BE/src/endpoints/cart/toEmpty.ts:index(event)\\


TU79 \newline TU80 \newline TU81 \newline TU82 & EmporioLambda-BE/src/endpoints/order/getByUsernameAndId.ts:index(event)\\

TU84 \newline TU85 & EmporioLambda-BE/src/endpoints/order/list.ts:index()\\

TU87 \newline TU88 & EmporioLambda-BE/src/endpoints/user/getUser.ts:index(event)\\

TU89 \newline TU90 \newline TU91 & EmporioLambda-BE/src/endpoints/user/update.ts:index(event)\\

TU92 \newline TU93 \newline TU94 & EmporioLambda-BE/src/endpoints/user/updatePassword.ts:index(event)\\

TU95 \newline TU96 & EmporioLambda-BE/src/services/nodemailer/nodemailer.ts\\

TU97 \newline TU98 \newline TU99 \newline TU104 & EmporioLambda-BE/src/endpoints/category/create.ts:index(event)\\

TU100 \newline TU101 \newline TU102 \newline TU103 & EmporioLambda-BE/src/endpoints/category/delete.ts:index(event)\\

TU105 \newline TU106 & EmporioLambda-BE/src/endpoints/category/list.ts:index()\\

TU107 \newline TU108 \newline TU109 \newline TU110 \newline TU111 \newline TU112 \newline TU113 \newline TU114 \newline TU115 \newline TU116 & EmporioLambda-BE/src/lib/auth.ts:handler(event,context)\\

TU117 \newline TU118 \newline TU119 & EmporioLambda-FE/pages/pdp/[id].tsx\\

TU120 \newline TU121 \newline TU122 \newline TU135 \newline TU136 \newline TU138 & EmporioLambda-FE/pages/profile/index.tsx\\

TU123 \newline TU124 \newline TU125 \newline TU126 \newline TU127 \newline TU128 \newline TU129 \newline TU130 \newline TU131 \newline TU132 \newline TU133 \newline TU134  & EmporioLambda-FE/pages/profile/edit/[username].tsx\\

TU137 & EmporioLambda-FE/pages/profile/order/[id].tsx\\

TU139 \newline TU140 \newline TU141 \newline TU142 \newline TU143 \newline TU144 \newline TU145 \newline TU151 \newline TU152 \newline TU153 \newline TU154 \newline TU155 \newline TU156 \newline TU157 & EmporioLambda-FE/pages/dashboard/index.tsx\\

TU146 \newline TU147 \newline TU148 \newline TU149 \newline TU150 & EmporioLambda-FE/pages/dashboard/modify/[id].tsx\\



TU158 \newline TU159 \newline TU160 \newline TU161 \newline TU162 & EmporioLambda-FE/components/layout.tsx\\

TU163 & EmporioLambda-FE/pages/index.tsx\\

TU164 & EmporioLambda-FE/pages/search/index.tsx\\





\end{longtable}

}

\subsubsection{Test di integrazione}
{
\rowcolors{2}{azzurro2}{azzurro3}

\centering
\renewcommand{\arraystretch}{2}
\begin{longtable}{C{2cm} C{9.6cm} C{1.5cm} C{1cm}}
\caption{Tabella riassuntiva test di integrazione}\\
\rowcolor{azzurro1}
\textbf{Codice} &
\textbf{Descrizione}&
\textbf{Stato}&
\textbf{Esito}\\
\endhead


TI1 & Viene verificato che l'integrazione fra modulo Back-end\ped{G} e DynamoDB\ped{G} funzioni correttamente & I & S\\

TI2 & Viene verificato che l'integrazione fra modulo Back-end\ped{G} e Stripe\ped{G} funzioni correttamente & I & S\\

TI3 & Viene verificato che l'integrazione fra modulo Back-end\ped{G} e S3\ped{G} funzioni correttamente & I & S\\

TI4 & Viene verificato che l'integrazione fra modulo Back-end\ped{G} e Nodemailer\ped{G} funzioni correttamente & I & S\\

TI4 & Viene verificato che l'integrazione fra modulo Back-end\ped{G} e Cognito\ped{G} funzioni correttamente & I & S\\

TI5 & Viene verificato che l'integrazione fra modulo Front-end\ped{G} e Cognito\ped{G} funzioni correttamente & I & S\\

TI6 & Viene verificato che l'integrazione fra modulo Front-end\ped{G} e Stripe\ped{G} funzioni correttamente & I & S\\

TI7 & Viene verificato che l'integrazione fra modulo Front-end\ped{G} e modulo Back-end\ped{G} funzioni correttamente & I & S\\

TI8 & Viene verificato il corretto refresh del token di Cognito\ped{G} per l'autenticazione agli endpoint & I & S\\



\end{longtable}

}

\subsubsection{Test di sistema}
{
\rowcolors{2}{azzurro2}{azzurro3}

\centering
\renewcommand{\arraystretch}{2}
\begin{longtable}{C{2cm} C{9.6cm} C{1.5cm} C{1cm}}
\caption{Tabella riassuntiva test di sistema}\\
\rowcolor{azzurro1}
\textbf{Codice} &
\textbf{Descrizione}&
\textbf{Stato}&
\textbf{Esito}\\
\endhead


TS1F1 & Un utente deve aver la possibilità di registrasi con:
\begin{itemize}
	\item nome;
	\item cognome;
	\item indirizzo di fatturazione;
	\item email;
	\item password.
\end{itemize} & I & S\\
TS2F1 & Il sistema deve mostrare un errore se
i campi inseriti durante la fase di registrazione non sono validi & I & S\\
TS1F2 & Un utente può effettuare il login inserendo:
\begin{itemize}
	\item email;
	\item password.
\end{itemize} & I & S\\
TS2F2 & Il sistema deve mostrare un errore se le credenziali del login sono errate & I & S\\
TS1F3 & Un utente autenticato deve poter effettuare il
logout & I & S\\
TS1F4 & Un utente generico deve accedere alla pagina del carrello dalla Homepage, dalla PLP\ped{G} e dal PDP\ped{G} & I & S\\
TS1F5 & Un utente generico, dal carrello, può:
\begin{itemize}
	\item visualizzare i prodotti precedentemente aggiunti;
	\item visualizzare nome, immagine e quantità del prodotto;
	\item rimuovere i singoli prodotti dal carrello;
	\item modificare la quantità dei prodotti;
	\item visualizzare il costo delle voci del carrello;
	\item visualizzare il prezzo totale dei prodotti nel carrello;
	\item visualizzare le tasse applicate al prezzo totale dei prodotti nel carrello.
\end{itemize}
& I & S\\

TS1F6 & Un cliente può effettuare il checkout se:
\begin{itemize}
	\item ha almeno un prodotto nel carrello;
	\item si è registrato.
\end{itemize}
& I & S\\

TS1F7 & Durante la fase di checkout:
\begin{itemize}
	\item viene visualizzata l'email a cui verranno mandati i prodotti;
	\item il cliente deve inserire i dati di pagamento tramite Stripe\ped{G};
	\item dopo aver inserito i dati di pagamento, il cliente continua il pagamento effettivo tramite Stripe\ped{G}.
\end{itemize}
& I & S\\

TS2F7 & Durante la fase di checkout il cliente deve poter modificare l'email a cui verranno mandati i prodotti
& I & S\\

TS1F8 & A pagamento riuscito:
\begin{itemize}
	\item il cliente visualizza un riepilogo dell'ordine effettuato;
	\item il cliente riceve i prodotti acquistati tramite l'email usata per l'acquisto.
\end{itemize}
& NI & NS\\

TS1F9 & A pagamento fallito:
\begin{itemize}
	\item il cliente visualizza un messaggio d'errore;
	\item il cliente può riprovare il pagamento verificando i dati inseriti.
\end{itemize}
& NI & NS\\


TS1F10 & Un cliente può visualizzare il suo profilo, contenente:
\begin{itemize}
	\item nome;
	\item cognome;
	\item indirizzo di fatturazione;
	\item email;
	\item lista degli ordini effettuati.
\end{itemize}
& I & S\\

TS1F11 & Un cliente può aggiornare, nel suo profilo, le seguenti informazioni:
\begin{itemize}
	\item nome;
	\item cognome;
	\item password;
	\item indirizzo di fatturazione;
	\item email.
\end{itemize}
& I & S\\

TS2F11 & Se la modifica delle informazioni nel profilo non va a buon fine, viene visualizzato un errore & NI & NS\\

TS1F12 & Per ogni ordine effettuato vengono visualizzati:
\begin{itemize}
	\item id ordine;
	\item prodotti acquistati;
	\item quantità dei prodotti acquistati;
	\item costo singolo prodotto;
	\item costo totale;
	\item tasse applicate;
	\item data di acquisto.
\end{itemize}
& I & S\\

TS1F13 & Un cliente può eliminare il suo account & I & S\\

TS1F14 & Il venditore ha a disposizione una dashboard\ped{G} dove può inserire nuovi prodotti e gestire il proprio catalogo digitale & I & S\\

TS2F14 & Il venditore, dalla dashboard\ped{G}, può accedere agli strumenti esterni riservati per gli admin & I & S\\

TS1F15 & Durante l'inserimento di un nuovo prodotto, il venditore, deve assegnare i seguenti dati:
\begin{itemize}
	\item nome;
	\item descrizione;
	\item immagine;
	\item prezzo;
	\item categoria.
\end{itemize} & I & S\\

TS1F16 & Il venditore può:
\begin{itemize}
	\item visualizzare i prodotti da lui venduti;
	\item modificare la descrizione dei prodotti da lui venduti;
	\item eliminare i prodotti da lui venduti;
	\item visualizzare i dettagli di tutti gli ordini effettuati dai clienti.
\end{itemize} & I & S\\

TS2F16 & Il venditore può modificare il nome, il prezzo, l'immagine e la categoria dei prodotti da lui venduti & I & S\\

TS1F17 & Il venditore, durante la visualizzazione di un prodotto, visualizza:
\begin{itemize}
	\item nome;
	\item descrizione;
	\item categoria;
	\item prezzo;
	\item immagine.
\end{itemize} & I & S\\

TS1F18 & Il venditore, per ogni ordine effettuato, può visualizzare:
\begin{itemize}
	\item numero ordine;
	\item prodotti acquistati;
	\item quantità dei prodotti acquistati;
	\item costo singolo prodotto;
	\item costo totale;
	\item tasse applicate;
	\item data di acquisto.
\end{itemize}
& I & S\\

TS1F19 & Il venditore può gestire le categorie dei prodotti, ovvero:
\begin{itemize}
	\item visualizzare una lista di categorie di prodotti;
	\item inserire nuove categorie di prodotti;
	\item eliminare categorie esistenti di prodotti.
\end{itemize}
& I & S\\





TS1F20 & Un utente generico può effettuare una ricerca tra i prodotti in vendita & I & S\\

TS1F21 & L'utente generico può accedere alla PLP\ped{G} corrispondente ad una categoria di prodotti. Nella PLP\ped{G} viene visualizzata la lista di prodotti corrispondente & I & S\\

TS1F22 & Per ogni prodotto listato nella PLP\ped{G} viene visualizzato:
\begin{itemize}
	\item nome;
	\item immagine;
	\item prezzo.
\end{itemize} & I & S\\

TS1F23 & Nella PLP\ped{G} un utente generico può:
\begin{itemize}
	\item selezionare alcuni prodotti;
	\item aggiungere i prodotti al carrello.
\end{itemize} & I & S\\

TS2F23 & Nella PLP\ped{G} un utente generico può filtrare l'insieme dei prodotti visualizzati in base al prezzo & NI & NS\\

TS1F24 & Un utente generico può accedere alla PDP\ped{G} cliccando su un prodotto nella PLP\ped{G} & I & S\\

TS1F25 & Un utente generico, nella PDP\ped{G} può:
\begin{itemize}
	\item visualizzare nome, immagine, descrizione e prezzo del prodotto;
	\item aggiungere il prodotto al carrello scegliendone la quantità.
\end{itemize} & I & S\\



\end{longtable}


}
\subsubsection{Test di accettazione o test di collaudo}
{
\rowcolors{2}{azzurro2}{azzurro3}

\centering
\renewcommand{\arraystretch}{2}
\begin{longtable}{C{3cm} C{10.2cm} C{1.5cm}}
\caption{Tabella riassuntiva test di accettazione o di collaudo}\\
\rowcolor{azzurro1}
\textbf{Codice} &
\textbf{Descrizione}&
\textbf{Stato}\\
\endhead


TA1F1 & Un utente deve aver la possibilità di registrasi con:
\begin{itemize}
	\item nome;
	\item cognome;
	\item indirizzo di fatturazione;
	\item email;
	\item password.
\end{itemize} & NI\\
TA2F1 & Il sistema deve mostrare un errore se
i campi inseriti durante la fase di registrazione non sono validi & NI\\
TA1F2 & Un utente può effettuare il login inserendo:
\begin{itemize}
	\item email;
	\item password.
\end{itemize} &  NI\\
TA2F2 & Il sistema deve mostrare un errore se le credenziali del login sono errate & NI\\
TA1F3 & Un utente autenticato deve poter effettuare il
logout & NI\\
TA1F4 & Un utente generico deve accedere alla pagina del carrello dalla Homepage, dalla PLP\ped{G} e dal PDP\ped{G} & NI\\
TA1F5 & Un utente generico, dal carrello, può:
\begin{itemize}
	\item visualizzare i prodotti precedentemente aggiunti;
	\item visualizzare nome, immagine e quantità del prodotto;
	\item rimuovere i singoli prodotti dal carrello;
	\item modificare la quantità dei prodotti;
	\item visualizzare il costo delle voci del carrello;
	\item visualizzare il prezzo totale dei prodotti nel carrello;
	\item visualizzare le tasse applicate al prezzo totale dei prodotti nel carrello.
\end{itemize}
& NI\\

TA1F6 & Un cliente può effettuare il checkout se:
\begin{itemize}
	\item ha almeno un prodotto nel carrello;
	\item si è registrato.
\end{itemize}
& NI\\

TA1F7 & Durante la fase di checkout:
\begin{itemize}
	\item viene visualizzata l'email a cui verranno mandati i prodotti;
	\item il cliente deve inserire i dati di pagamento tramite Stripe\ped{G};
	\item dopo aver inserito i dati di pagamento, il cliente continua il pagamento effettivo tramite Stripe\ped{G}.
\end{itemize}
& NI\\

TA2F7 & Durante la fase di checkout il cliente deve poter modificare l'email a cui verranno mandati i prodotti
& NI\\

TA1F8 & A pagamento riuscito:
\begin{itemize}
	\item il cliente visualizza un riepilogo dell'ordine effettuato;
	\item il cliente riceve i prodotti acquistati tramite l'email usata per l'acquisto.
\end{itemize}
& NI\\

TA1F9 & A pagamento fallito:
\begin{itemize}
	\item il cliente visualizza un messaggio d'errore;
	\item il cliente può riprovare il pagamento verificando i dati inseriti.
\end{itemize}
& NI\\


TA1F10 & Un cliente può visualizzare il suo profilo, contenente:
\begin{itemize}
	\item nome;
	\item cognome;
	\item indirizzo di fatturazione;
	\item email;
	\item lista degli ordini effettuati.
\end{itemize}
& NI\\

TA1F11 & Un cliente può aggiornare, nel suo profilo, le seguenti informazioni:
\begin{itemize}
	\item nome;
	\item cognome;
	\item password;
	\item indirizzo di fatturazione;
	\item email.
\end{itemize}
& NI\\

TA2F11 & Se la modifica delle informazioni nel profilo non va a buon fine, viene visualizzato un errore & NI\\

TA1F12 & Per ogni ordine effettuato vengono visualizzati:
\begin{itemize}
	\item id ordine;
	\item prodotti acquistati;
	\item quantità dei prodotti acquistati;
	\item costo singolo prodotto;
	\item costo totale;
	\item tasse applicate;
	\item data di acquisto.
\end{itemize}
& NI\\

TA1F13 & Un cliente può eliminare il suo account & NI\\

TA1F14 & Il venditore ha a disposizione una dashboard\ped{G} dove può inserire nuovi prodotti e gestire il proprio catalogo digitale & NI\\

TA2F14 & Il venditore, dalla dashboard\ped{G}, può accedere agli strumenti esterni riservati per gli admin & NI\\

TA1F15 & Durante l'inserimento di un nuovo prodotto, il venditore, deve assegnare i seguenti dati:
\begin{itemize}
	\item nome;
	\item descrizione;
	\item immagine;
	\item prezzo;
	\item categoria.
\end{itemize} & NI\\

TA1F16 & Il venditore può:
\begin{itemize}
	\item visualizzare i prodotti da lui venduti;
	\item modificare la descrizione dei prodotti da lui venduti;
	\item eliminare i prodotti da lui venduti;
	\item visualizzare i dettagli di tutti gli ordini effettuati dai clienti.
\end{itemize} & NI\\

TA2F16 & Il venditore può modificare il nome, il prezzo, l'immagine e la categoria dei prodotti da lui venduti & NI\\

TA1F17 & Il venditore, durante la visualizzazione di un prodotto, visualizza:
\begin{itemize}
	\item nome;
	\item descrizione;
	\item categoria;
	\item prezzo;
	\item immagine.
\end{itemize} & NI\\

TA1F18 & Il venditore, per ogni ordine effettuato, può visualizzare:
\begin{itemize}
	\item numero ordine;
	\item prodotti acquistati;
	\item quantità dei prodotti acquistati;
	\item costo singolo prodotto;
	\item costo totale;
	\item tasse applicate;
	\item data di acquisto.
\end{itemize}
& NI\\

TA1F19 & Il venditore può gestire le categorie dei prodotti, ovvero:
\begin{itemize}
	\item visualizzare una lista di categorie di prodotti;
	\item inserire nuove categorie di prodotti;
	\item eliminare categorie esistenti di prodotti.
\end{itemize}
& NI\\





TA1F20 & Un utente generico può effettuare una ricerca tra i prodotti in vendita & NI\\

TA1F21 & L'utente generico può accedere alla PLP\ped{G} corrispondente ad una categoria di prodotti. Nella PLP\ped{G} viene visualizzata la lista di prodotti corrispondente & NI\\

TA1F22 & Per ogni prodotto listato nella PLP\ped{G} viene visualizzato:
\begin{itemize}
	\item nome;
	\item immagine;
	\item prezzo.
\end{itemize} & NI\\

TA1F23 & Nella PLP\ped{G} un utente generico può:
\begin{itemize}
	\item selezionare alcuni prodotti;
	\item aggiungere i prodotti al carrello.
\end{itemize} & NI\\

TA2F23 & Nella PLP\ped{G} un utente generico può filtrare l'insieme dei prodotti visualizzati in base al prezzo & NI\\

TA1F24 & Un utente generico può accedere alla PDP\ped{G} cliccando su un prodotto nella PLP\ped{G} & NI\\

TA1F25 & Un utente generico, nella PDP\ped{G} può:
\begin{itemize}
	\item visualizzare nome, immagine, descrizione e prezzo del prodotto;
	\item aggiungere il prodotto al carrello scegliendone la quantità.
\end{itemize} & NI\\



\end{longtable}


}