\section{Introduzione}
\subsection{Scopo del documento}
L'attuale documento ha lo scopo di descrivere le caratteristiche tecniche del prodotto \nameproject{} sviluppato dal team \Omicron. Il documento descriverà le specifiche tecniche relative sia al modulo Front-end\ped{G} che al modulo Back-end\ped{G}
\subsection{Scopo del prodotto}
Lo scopo del prodotto \textit{EmporioLambda} (EML) di \textit{Red Babel} consiste nella creazione di una piattaforma E-commerce\ped{G} generica, la quale può essere mostrata come concetto software vendibile a un commerciante in modo che esso abbia la possibilità di utilizzarlo per vendere la propria merce. L'applicativo sarà implementato utilizzando esclusivamente tecnologie serverless\ped{G} e i servizi AWS\ped{G}.
\subsection{Glossario}
Al fine di migliorare la chiarezza del documento ed evitare possibili ambiguità, viene fornito un Glossario contenente i termini più critici scelti dai membri del gruppo, e una loro spiegazione. In questo documento, tali termini verranno indicati con la lettera `G' a pedice della parola.
\subsection{Riferimenti}
\subsubsection{Normativi}
\begin{itemize}
	\item \textbf{\NdP}: \NdPv{2.0.0};
	\item \textbf{Capitolato d'appalto C2 - EmporioLambda}: \\ \url{https://www.math.unipd.it/~tullio/IS-1/2020/Progetto/C2.pdf}.
\end{itemize}

\subsubsection{Informativi}
\begin{itemize}
	\item \textbf{Diagrammi delle classi}:\\  \url{https://www.math.unipd.it/~rcardin/swea/2021/Diagrammi\%20delle\%20Classi_4x4.pdf};
	\item \textbf{Diagrammi dei package}:\\  \url{https://www.math.unipd.it/~rcardin/swea/2021/Diagrammi\%20dei\%20Package_4x4.pdf};
	\item \textbf{Diagrammi di sequenza}:\\  \url{https://www.math.unipd.it/~rcardin/swea/2021/Diagrammi\%20di\%20Sequenza_4x4.pdf};
	
	\item \textbf{Design pattern creazionali}:\\  \url{https://www.math.unipd.it/~rcardin/swea/2021/Design\%20Pattern\%20Creazionali_4x4.pdf};
	
	\item \textbf{Software Architecture Patterns}:\\  \url{https://www.math.unipd.it/~rcardin/sweb/2021/L03.pdf}.
	
	
	
	
	
	
	
\end{itemize}

