\subsection{Validazione e collaudo}
\textit{\textbf{Periodo}: dal 2021-04-09 al 2021-05-10}

L'inizio di questa fase coincide con data della Revisione di Qualifica e si conclude con la scadenza della Revisione di Accettazione.

\subsubsection{Attività}

\begin{itemize}
\item \textbf{Incremento e verifica documenti}: vengono realizzati gli incrementi necessari ai documenti. I documenti in questione sono:
\begin{itemize}
\item \NdP{};
\item \AdR{};
\item \PdQ{};
\item \PdP{};
\item Glossario;
\item \MU{};
\item \MM{}.
\end{itemize}
\item \textbf{Validazione e collaudo}: viene completato il prodotto e i documenti in base a requisiti mancanti, indicazioni del proponente o analisi interne. Vengono inoltre eseguiti tutti i test per validare e collaudare il prodotto finale.\\ Gli incrementi individuati nella sezione \S{3.2} vengono, in questa fase, raggruppati a due a due. Il motivo è dato dal fatto che essi prevedono solo correzioni o aggiunte minime di funzionalità, il che risulta in un carico di lavoro minore rispetto alle due fasi precedenti. Inoltre gli incrementi, in questo caso, possono essere svolti in contemporanea da diversi membri del gruppo (per lo stesso motivo notato precedentemente);
\item \textbf{Consolidamento}: viene realizzata la presentazione da esporre in sede di Revisione di Accettazione.
\end{itemize}

\subsubsection{Incremento 11}
\myparagraph{Consuntivo}

{

\rowcolors{2}{azzurro2}{azzurro3}

\centering
\renewcommand{\arraystretch}{1.8}
\begin{longtable}{C{4cm} C{1.5cm} C{4cm} }

\rowcolor{azzurro1}
\textbf{Ruolo} &
\textbf{Ore}&
\textbf{Costo}\\
\endhead

\textit{Responsabile} & 5 (+0) & 150 (+0\euro{}) \\
\ammProg & 6 (+0) & 120\euro{} (+0\euro{}) \\
\analProg & 0 (+0) & 0\euro{} (+0\euro{}) \\
\progetProg & 11 (+0) & 242\euro{} (+0\euro{}) \\
\programProg & 13 (+2) & 195\euro{} (+30\euro{}) \\
\verifProg & 16 (+0) & 240\euro{} (+0\euro{})\\
\textbf{Totale Preventivo} & \textbf{51} & \textbf{947\euro{}} \\
\textbf{Totale Consuntivo} & \textbf{53} & \textbf{977\euro{}} \\
\textbf{Differenza} & \textbf{+2} & \textbf{+30\euro{}} \\


\rowcolor{white}
\caption{Consuntivo di periodo dell'incremento 11}\\

\end{longtable}
}

\myparagraph{Considerazioni}
In questo incremento sono state implementate le ultime funzionalità previste per il nostro prodotto, che ha però portato ad un ritardo e un impegno maggiore di quanto previsto. In particolare sono state trovate alcune problematiche nello sviluppo delle pagine statiche incrementali e nell'autenticazione delle API\ped{G}, ma che sono state risolte tramite ricerche.\\
Nonostante ciò, la produzione degli incrementi documentali è risultata come prevista inizialmente.
Dato il ritardo di completamento di questo incremento ed una discussione sugli impegni individuali dei componenti del gruppo durante la fase corrente, è stata effettuata una ripianificazione temporale degli incrementi successivi, lasciando comunque invariato l'impegno orario. Rimangono invariati gli obiettivi dei successivi incrementi.



\myparagraph{Preventivo a finire}
Il bilancio risulta leggermente negativo rispetto al preventivo dell'incremento, con una perdita di 30\euro{}. Esso viene però sanato dal risparmio precedente di 36\euro{}, risultando in un risparmio totale di 6\euro{}.\\ 
Date le considerazioni precedenti e con il bilancio totale ancora in positivo, riteniamo di essere in linea con il preventivo e non prevediamo cambiamenti drastici di esso.


\subsubsection{Incremento 12}

\myparagraph{Prospetto orario}
In questo incremento, la distribuzione oraria dei componenti del gruppo è la seguente:

{

\rowcolors{2}{azzurro2}{azzurro3}

\centering
\renewcommand{\arraystretch}{1.8}
\begin{longtable}{C{4cm} C{1cm} C{1cm} C{1cm} C{1cm} C{1cm} C{1cm} C{2cm}}

\rowcolor{azzurro1}
\textbf{Nominativo} &
\textbf{RE}&
\textbf{AM}&
\textbf{AN}&
\textbf{PT}&
\textbf{PR}&
\textbf{VE}&
\textbf{Ore totali}\\
\endhead

\MB & 0 & 0 & 0 & 1 & 1 & 5 & 7 \\
\VAS & 5 & 0 & 0 & 0 & 3 & 6 & 14 \\
\FD & 0 & 0 & 0 & 2 & 3 & 5 & 10 \\
\NM & 0 & 1 & 0 & 0 & 4 & 5 & 10 \\
\SB & 0 & 2 & 0 & 0 & 2 & 5 & 9 \\
\GB & 0 & 1 & 0 & 0 & 5 & 4 & 10 \\
\MDI & 0 & 0 & 0 & 2 & 0 & 2 & 4 \\
\textbf{Ore Totali} & 5 & 4 & 0 & 5 & 18 & 32 & 64 \\

\rowcolor{white}
\caption{Distribuzione oraria nell'incremento 12}\\

\end{longtable}
}
\newpage
Il seguente istogramma riassume i dati ottenuti:

\begin{figure}[H]
\centering
\includegraphics[scale=0.90]{res/Preventivo/Fasi/VerificaIncrementi/istogramma12}\\
\caption{Istogramma della ripartizione dei ruoli nell'incremento 12}
\end{figure}


\myparagraph{Prospetto economico}

In questo incremento, il costo per ogni ruolo è il seguente:

{

\rowcolors{2}{azzurro2}{azzurro3}

\centering
\renewcommand{\arraystretch}{1.8}
\begin{longtable}{C{3cm} C{1cm} C{2cm} }

\rowcolor{azzurro1}
\textbf{Ruolo} &
\textbf{Ore}&
\textbf{Costo}\\
\endhead

\textit{Responsabile} & 5 & 150\euro{} \\
\ammProg & 4 & 80\euro{} \\
\analProg & 0 & 0\euro{} \\
\progetProg & 5 & 110\euro{} \\
\programProg & 18 & 270\euro{} \\
\verifProg & 32 & 480\euro{} \\
\textbf{Totale} & 64 & 1090\euro{} \\

\rowcolor{white}
\caption{Prospetto dei costi per ruolo nell'incremento 12}\\

\end{longtable}
}
\newpage
Il seguente areogramma riassume i dati ottenuti:

\begin{figure}[H]
\centering
\includegraphics[scale=0.90]{res/Preventivo/Fasi/VerificaIncrementi/torta12}\\
\caption{Areogramma della distribuzione economica nell'incremento 12}
\end{figure}






\subsubsection{Incremento 13}
\textit{\textbf{Periodo}: dal 2021-05-13 al 2021-05-26}

\myparagraph{Obiettivi}
Gli obiettivi definiti per questo incremento sono i seguenti:
\begin{itemize}

\item preparazione alla presentazione della Revisione di Accettazione;
\item preparazione al collaudo finale con il proponente;
\item incremento della documentazione.
\end{itemize}

\myparagraph{Attività}
Per raggiungere gli obiettivi, vengono svolte le seguenti attività:
\begin{itemize}
\item \textbf{presentazione Revisione di Accettazione}: creazione e preparazione della presentazione con il \VT{};
\item \textbf{collaudo}: controlli e verifiche ultime al prodotto e ai documenti per il collaudo finale con il proponente;
\item \textbf{ampliamento documentazione e verifiche}:
\begin{itemize}
\item incremento del \Glossariov{4.0.0};
\item rilevazione e registrazione di metriche, esiti di verifica e obiettivi di qualità;
\item aggiornamento dei rischi rilevati;
\item calcolo e registrazione del consuntivo di periodo.
\end{itemize}

\end{itemize}
\myparagraph{Diagramma di Gantt}
\begin{figure}[H]
\centering

\centerline{\includegraphics[scale=0.6]{res/Pianificazione/Fasi/VerificaIncrementi/ganttIncremento13}}
\caption{Diagramma di Gantt per l'incremento 13}
\end{figure}

\subsubsection{Diagramma di Gantt}

\begin{figure}[H]
\centering

\centerline{\includegraphics[scale=0.6]{res/Pianificazione/Gantt/verifica}}
\caption{Diagramma di Gantt per il periodo di validazione e collaudo}
\end{figure}