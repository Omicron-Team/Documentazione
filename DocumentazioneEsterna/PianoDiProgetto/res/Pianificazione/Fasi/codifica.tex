\subsection{Progettazione di dettaglio e codifica}
\textit{\textbf{Periodo}: dal 2021-03-08 al 2021-04-19}

L'inizio di questa fase coincide con la data della Revisione di Progettazione e si conclude con la scadenza della Revisione di Qualifica.

\subsubsection{Obiettivi generali}

\begin{itemize}
\item \textbf{Incremento e verifica documenti}: vengono realizzati gli incrementi necessari ai documenti. I documenti in questione sono:
\begin{itemize}
\item \NdP{};
\item \AdR{};
\item \PdQ{};
\item \PdP{};
\item Glossario.
\end{itemize}
\item \textbf{Product Baseline\ped{G}}: viene realizzata la baseline architetturale del prodotto, in base alla Technology Baseline\ped{G}. Il codice sviluppato precedentemente nel Proof of Concept\ped{G} può essere riutilizzato, se ritenuto corretto per il design architetturale individuato. Viene inoltre redatto l'\textit{Allegato tecnico}\ped{G} per essere inviato e presentato al \CR{} in un colloquio agile;
\item \textbf{Manuali}: Durante lo sviluppo della Product Baseline\ped{G} verranno redatti il \MU{} e il \MM. Il primo servirà per fornire istruzioni per l'utilizzo dell'applicazione, il secondo per fornire informazioni necessarie al mantenimento e l'ampliamento del prodotto;
\item \textbf{Consolidamento}: viene realizzata la presentazione da esporre in sede di Revisione di Qualifica e Product baseline\ped{G} e si approfondiscono aspetti lacunari riguardo il progetto.
\end{itemize}

La fase viene suddivisa in sei incrementi, di seguito riportati e spiegati.

\subsubsection{Incremento 5}
...
\paragraph{Obiettivi}
...
\paragraph{Attività}
...
\paragraph{Diagramma di Gantt}
...


\newpage
\subsubsection{Incremento 6}
\textit{\textbf{Periodo}: dal 2021-03-13 al 2021-03-17}

\paragraph{Obiettivi}\\
Gli obiettivi definiti per questo incremento sono i seguenti:
\begin{itemize}
\item implementazione autenticazione del sito e visualizzazione pagina del profilo;
\item incremento della documentazione, con correzione in base a segnalazioni dei committenti;
\item inizio stesura di manuali e allegato tecnico.
\end{itemize}

\paragraph{Attività}\\
Per raggiungere gli obiettivi, vengono svolte le seguenti attività:
\begin{itemize}

\item \textbf{codifica}:
\begin{itemize}
\item implementazione UC1 - Registrazione cliente
\item implementazione UC2 - Visualizzazione errori di registrazione
\item implementazione UC3 - Login
\item implementazione UC4 - Visualizzazione errori dati login errati
\item implementazione UC5 - Logout
\item implementazione UC9 - Visualizzazione 
\end{itemize}

\item \textbf{progettazione}:
\begin{itemize}
\item ricerca best practices per le tecnologie da utilizzare;
\item studio di tecnologie mancanti nel Proof of Concept\ped{G}.
\end{itemize}

\item \textbf{ampliamento documentazione}:
\begin{itemize}
\item aggiornamento \PdPv{3.0.0} per pianificazione dei successivi incrementi;
\item incremento del \Glossariov{3.0.0};
\item rilevazione e registrazione degli esiti di verifica e obiettivi di qualità;
\item aggiornamento dei rischi rilevati;
\item calcolo e registrazione del consuntivo di periodo.
\end{itemize}

\end{itemize}
\paragraph{Diagramma di Gantt}\\
\newpage
\subsubsection{Incremento 7}
...
\paragraph{Obiettivi}
...
\paragraph{Attività}
...
\paragraph{Diagramma di Gantt}
...


\newpage
\subsubsection{Incremento 8}
\textit{\textbf{Periodo}: dal 2021-03-26 al 2021-03-31}

\myparagraph{Obiettivi}
Gli obiettivi definiti per questo incremento sono i seguenti:
\begin{itemize}
\item implementazione acquisto prodotti, compreso di gestione del carrello e checkout;
\item incremento della documentazione.
\end{itemize}

\myparagraph{Attività}
Per raggiungere gli obiettivi, vengono svolte le seguenti attività:
\begin{itemize}

\item \textbf{codifica e progettazione}:
\begin{itemize}
\item implementazione UC6 - Visualizzazione prodotti nel carrello;
\item implementazione UC7 - Rimozione prodotto nel carrello;
\item implementazione UC8 - Modifica quantità di un prodotto nel carrello;
\item implementazione UC9 -  Visualizzazione costo totale e tasse applicate;
\item implementazione UC10 - Acquisto prodotti;
\item implementazione UC11 - Visualizzazione riepilogo ordine;
\item implementazione UC14 - Visualizzazione ordini effettuati dall’utente;
\item implementazione UC20 - Inserimento nel carrello dei prodotti selezionati;
\item implementazione UC24 - Modifica della quantità selezionata del prodotto;
\item implementazione UC25 - Inserimento nel carrello del prodotto visualizzato;
\item implementazione UC34 -  Visualizzazione lista ordini;
\item creazione diagrammi UML inerenti.

\end{itemize}

\item \textbf{ampliamento documentazione e verifiche}:
\begin{itemize}
\item incremento del \MU per i casi d'uso implementati;
\item incremento del \MM per i casi d'uso implementati;
\item incremento dell'\textit{Allegato tecnico} per i casi d'uso implementati;
\item incremento del \Glossariov{3.0.0};
\item rilevazione e registrazione di metriche, esiti di verifica e obiettivi di qualità;
\item aggiornamento dei rischi rilevati;
\item calcolo e registrazione del consuntivo di periodo.
\end{itemize}

\end{itemize}
\myparagraph{Diagramma di Gantt}
\newpage
\subsubsection{Incremento 9}
\myparagraph{Consuntivo}

{

\rowcolors{2}{azzurro2}{azzurro3}

\centering
\renewcommand{\arraystretch}{1.8}
\begin{longtable}{C{4cm} C{1.5cm} C{4cm} }

\rowcolor{azzurro1}
\textbf{Ruolo} &
\textbf{Ore}&
\textbf{Costo}\\
\endhead

\textit{Responsabile} & 4 (+0) & 120 (+0\euro{}) \\
\ammProg & 4 (+0) & 80\euro{} (+0\euro{}) \\
\analProg & 0 (+0) & 0\euro{} (+0\euro{}) \\
\progetProg & 3 (+0) & 66\euro{} (+0\euro{}) \\
\programProg & 0 (+0) & 0\euro{} (+0\euro{}) \\
\verifProg & 17 (+0) & 225\euro{} (+0\euro{})\\
\textbf{Totale Preventivo} & \textbf{28} & \textbf{521\euro{}} \\
\textbf{Totale Consuntivo} & \textbf{28} & \textbf{521\euro{}} \\
\textbf{Differenza} & \textbf{+0} & \textbf{+0\euro{}} \\


\rowcolor{white}
\caption{Consuntivo di periodo dell'incremento 9}\\

\end{longtable}
}

\myparagraph{Considerazioni}
Il gruppo ha raggiunto gli obiettivi pianificati per questo incremento, preparandosi al meglio per le varie presentazioni.\\
Le attività sono state portate a termine senza ritardi, nonostante alcuni impegni didattici da parte di ogni membro del gruppo. Sono state inoltre aggiornate le date dei prossimi incrementi in base alla data dell'incontro con il \CR{} per la Product baseline\ped{G}.

\myparagraph{Preventivo a finire}
Il bilancio è in pari rispetto al preventivo dell'incremento. Il risparmio totale resta di 21\euro{}.
Gli obiettivi preimpostati sono stati raggiunti a pieno, senza alcun rallentamento rispetto a quanto pianificato.
Alla luce dei costi e degli obiettivi raggiunti, non si ritiene necessaria alcuna ripianificazione degli incrementi (eccetto l'aggiornamento delle date) o modifica del preventivo.



\newpage
\subsubsection{Incremento 10}

\subsubsection{Prospetto orario}
In questo incremento, la distribuzione oraria dei componenti del gruppo è la seguente:

{

\rowcolors{2}{azzurro2}{azzurro3}

\centering
\renewcommand{\arraystretch}{1.8}
\begin{longtable}{C{4cm} C{1cm} C{1cm} C{1cm} C{1cm} C{1cm} C{1cm} C{2cm}}

\rowcolor{azzurro1}
\textbf{Nominativo} &
\textbf{RE}&
\textbf{AM}&
\textbf{AN}&
\textbf{PT}&
\textbf{PR}&
\textbf{VE}&
\textbf{Ore totali}\\
\endhead

\MB & 0 & 0 & 0 & 1 & 2 & 1 & 4 \\
\VAS & 0 & 0 & 0 & 2 & 0 & 2 & 4 \\
\FD & 0 & 0 & 0 & 0 & 3 & 3 & 6 \\
\NM & 0 & 0 & 0 & 0 & 4 & 1 & 5 \\
\SB & 1 & 0 & 0 & 0 & 0 & 3 & 4 \\
\GB & 2 & 3 & 0 & 0 & 0 & 0 & 5 \\
\MDI & 0 & 0 & 0 & 1 & 1 & 3 & 5 \\
\textbf{Ore Totali} & 3 & 3 & 0 & 4 & 10 & 13 & 33 \\

\rowcolor{white}
\caption{Distribuzione oraria nell'incremento 10}\\

\end{longtable}
}
\newpage
Il seguente istogramma riassume i dati ottenuti:

\begin{figure}[H]
\centering
\includegraphics[scale=0.90]{res/Preventivo/Fasi/CodificaIncrementi/istogramma10}\\
\caption{Istogramma della ripartizione dei ruoli nell'incremento 10}
\end{figure}


\subsubsection{Prospetto economico}

In questo incremento, il costo per ogni ruolo è il seguente:

{

\rowcolors{2}{azzurro2}{azzurro3}

\centering
\renewcommand{\arraystretch}{1.8}
\begin{longtable}{C{3cm} C{1cm} C{2cm} }

\rowcolor{azzurro1}
\textbf{Ruolo} &
\textbf{Ore}&
\textbf{Costo}\\
\endhead

\textit{Responsabile} & 3 & 90\euro{} \\
\ammProg & 3 & 60\euro{} \\
\analProg & 0 & 0\euro{} \\
\progetProg & 4 & 88\euro{} \\
\programProg & 10 & 150\euro{} \\
\verifProg & 13 & 195\euro{} \\
\textbf{Totale} & 33 & 583\euro{} \\

\rowcolor{white}
\caption{Prospetto dei costi per ruolo nell'incremento 10}\\

\end{longtable}
}
\newpage
Il seguente areogramma riassume i dati ottenuti:

\begin{figure}[H]
\centering
\includegraphics[scale=0.90]{res/Preventivo/Fasi/CodificaIncrementi/torta10}\\
\caption{Areogramma della distribuzione economica nell'incremento 10}
\end{figure}






\newpage
