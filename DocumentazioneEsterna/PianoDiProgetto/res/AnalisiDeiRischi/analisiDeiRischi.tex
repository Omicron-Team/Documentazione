\section{Analisi dei rischi}

Nel corso dello sviluppo di un progetto di una certa complessità è possibile incorrere in alcuni problemi, che possono venire evitati o attenuati da un'attenta attività di analisi dei rischi. La procedura per individuare e risolvere i vari rischi è la seguente:
\begin{itemize}
\item \textbf{Individuazione dei rischi:} attività di identificazione dei rischi che possono portare a problemi durante l'avanzamento del progetto;
\item \textbf{Analisi dei rischi:} attività di controllo dei rischi (individuati nel punto precedente) con valutazione di:
\begin{itemize}
\item probabilità che si verifichi;
\item indice di gravità;
\item conseguenze per il progetto.
\end{itemize}
\item \textbf{Pianificazione di controllo:} attività di pianificazione di eventuali soluzioni per prevenire i problemi o risolverli nel caso si verifichino;
\item \textbf{Monitoraggio dei rischi:} attività periodica svolta per evitare che avvengano i problemi individuati o che permetta di agire per contenerli il prima possibile nel caso si verifichino.
\end{itemize}

\subsection{Tipologia dei rischi}
Le varie tipologie di rischi riscontrabili sono state, inoltre, classificate e assegnate ad una etichetta univoca nel modo seguente:
\begin{itemize}
\item \textbf{RT:} Rischi tecnologici;
\item \textbf{RO:} Rischi Organizzativi;
\item \textbf{RP:} Rischi Personali;
\item \textbf{RI:} Rischi Interpersonali.
\end{itemize} 

\subsection{Tabella dei rischi}
I rischi riscontrati sono riportati nella seguente tabella con i seguenti campi:
\begin{itemize}
\item \textbf{Codice e Nome:} indicano il codice del rischio (composto dalla sua etichetta e da un numero incrementale) e il suo nome;
\item \textbf{Descrizione:} fornisce una breve descrizione del rischio con relative conseguenze nel caso si verificasse;
\item \textbf{Precauzioni:} indica il modo in cui tale rischio possa essere rilevato e prevenuto;
\item \textbf{Grado di rischio:} indica la possibilità che il rischio avvenga (\textbf{F}) e la sua relativa pericolosità (\textbf{P});
\item \textbf{Piano di contingenza:} indica come risolvere un problema nel caso si verifichi.
\end{itemize}

\setcounter{table}{-1}
{

\rowcolors{2}{azzurro2}{azzurro3}

\centering
\renewcommand{\arraystretch}{1.5}
\begin{longtable}{C{3.5cm} | C{11cm}}
\rowcolor{azzurro1}\textbf{Codice: Nome} & \textbf{RT1: }{Inesperienza tecnologica}\\
\textbf{Descrizione} & Le tempistiche scelte potrebbero non venire rispettate a causa della presenza di tecnologie nuove per molti componenti del gruppo.\\ 
\textbf{Precauzione} & Chi ha problemi su una determinata tecnologia lo comunicherà al \respProg{}.\\
\textbf{Grado di rischio} & \textbf{F: }  Alta \textbf{P: } Alta\\   
\textbf{Piano di contingenza} & Chi ha già conoscenze sulla tecnologia cercherà di aiutare i componenti con più difficoltà. Sarà comunque responsabilità di quest'ultimo riuscire a comprendere, anche tramite una formazione personale se necessario, la tecnologia in questione.\\
\hline

\rowcolor{azzurro1}\textbf{Codice: Nome} & \textbf{RT2: }{Problemi hardware}\\
\textbf{Descrizione} & Alcuni membri del gruppo potrebbero essere soggetti a malfunzionamenti dal punto di vista hardware.\\ 
\textbf{Precauzione} & I dati del progetto vengono salvati in remoto (su GitHub) al fine di evitarne la perdita completa.\\
\textbf{Grado di rischio} & \textbf{F: }Bassa \textbf{P: }Media\\ 
\textbf{Piano di contingenza} & In caso di malfunzionamento il membro si impegna a risolvere il guasto il prima possibile. Tutti i membri inoltre caricheranno eventuali modifiche in remoto appena esse siano concluse per evitarne eventuali perdite.\\
\hline

\rowcolor{azzurro1}\textbf{Codice: Nome} & \textbf{RP1: }{Impegni Personali}\\
\textbf{Descrizione} & Ogni membro del gruppo ha impegni personali per la quale potrebbe risultare poco disponibile in certi periodi, causando ritardi nelle consegne.\\ 
\textbf{Precauzione} & Eventuali impegni personali imprevisti che possono portare a ritardi del progetto dovranno venire comunicati al \respProg{}.\\
\textbf{Grado di rischio} & \textbf{F: }Media \textbf{P: }Alta\\ 
\textbf{Piano di contingenza} & Gli incarichi e le risorse verranno riassegnati rispettando gli impegni personali dei membri.\\
\hline
\hline
\rowcolor{azzurro1}\textbf{Codice: Nome} & \textbf{RP2: }{Impegni accademici}\\ 
\textbf{Descrizione} & Ogni membro del gruppo ha impegni accademici per la quale potrebbe risultare poco disponibile in certi periodi, causando ritardi nelle consegne.\\ 
\textbf{Precauzione} & A inizio progetto ognuno ha elencato eventuali corsi mancanti con possibili periodi in cui potrebbero dover sostenerne gli esami.\\
\textbf{Grado di rischio} & \textbf{F: }Alta \textbf{P: }Media\\
\textbf{Piano di contingenza} & Gli incarichi verranno assegnati rispettando gli impegni accademici di ciascun membro.\\
\hline

\rowcolor{azzurro1}\textbf{Codice: Nome} & \textbf{RI1: }{Comunicazioni interne}\\
\textbf{Descrizione} & Potrebbe risultare difficile trovare un giorno dove tutti i membri del gruppo siano disponibili per un incontro, ritardando eventuali attività del progetto.\\ 
\textbf{Precauzione} & A inizio progetto ognuno ha avvisato gli altri membri riguardo i giorni della settimana in cui probabilmente non potrà essere reperibile. La data di un incontro si decide comunque con largo anticipo (4-5 giorni prima) al fine di assicurarsi la disponibilità di tutti i membri.\\
\textbf{Grado di rischio} & \textbf{F: }Media \textbf{P: }Bassa\\
\textbf{Piano di contingenza} & Nel caso la presenza di un membro non fosse necessaria si può comunque procedere all'incontro con una eventuale registrazione di quest'ultimo.\\
\hline

\rowcolor{azzurro1}\textbf{Codice: Nome} & \textbf{RI2: }{Comunicazioni esterne}\\ 
\textbf{Descrizione} & Potrebbero risultare difficoltose le comunicazioni con il proponente esterno, rallentando così il lavoro del gruppo.\\ 
\textbf{Precauzione} & Eventuali conferenze con il proponente esterno saranno organizzate e comunicate con il dovuto preavviso, assicurandosi la disponibilità dei membri del gruppo.\\
\textbf{Grado di rischio} & \textbf{F: }Bassa \textbf{P: }Media\\
\textbf{Piano di contingenza} & Eventuali domande per il proponente vengono registrare in un documento Google condiviso all'interno del gruppo.\\
\hline

\rowcolor{azzurro1}\textbf{Codice: Nome} & \textbf{RI3: }{Contrasti interni}\\ 
\textbf{Descrizione} & Potrebbero verificarsi, nel corso del progetto, dei disaccordi sulle decisioni prese che possono portare a tensioni fra i membri del gruppo.\\
\textbf{Precauzione} & Ogni componente del gruppo ha il compito di comunicare eventuali problemi riguardo le decisioni prese.\\
\textbf{Grado di rischio} & \textbf{F: }Media {\textbf{P: }}Media\\
\textbf{Piano di contingenza} & Si cercherà di trovare una soluzione che metta d'accordo, per quanto possibile, tutti i membri del gruppo.\\
\hline

\rowcolor{azzurro1}\textbf{Codice: Nome} & \textbf{RO1: }{Calcolo Tempistiche}\\
\textbf{Descrizione} & Come indicato sopra l'inesperienza nelle tecnologie e nel lavorare in un gruppo di 7 persone potrebbe portare a errori nel calcolo delle tempistiche effettuato, causando il non rispetto delle scadenze imposte.\\
\textbf{Precauzione} & Le scadenze dei documenti terranno conto anche di eventuali ritardi.\\
\textbf{Grado di rischio} & \textbf{F: }Media \textbf{P: }Alta\\
\textbf{Piano di contingenza} & Il \respProg{} riassegnerà le risorse, mettendone di più per l'attività che rischia di essere completata in ritardo e, nel peggiore dei casi, spostandone la scadenza.\\
\hline

\rowcolor{azzurro1}\textbf{Codice: Nome} & \textbf{RO2: }{Ritardi}\\
\textbf{Descrizione} & Per problemi di vario tipo (compresi i rischi sopra indicati) si potrebbe incorrere in ritardi e non rispetto delle milestone imposte.\\ 
\textbf{Precauzione} & Ogni membro del gruppo è tenuto a segnalare eventuali problemi personali che possono portare al non completamento del progetto entro i tempi prestabiliti.\\
\textbf{Grado di rischio} & \textbf{F: }Alta {\textbf{P: }}Alta\\
\textbf{Piano di contingenza} & Il \respProg{} ha il compito di riassegnare risorse e attività al fine di assicurare il rispetto dei tempi prestabiliti.\\
\hline

\rowcolor{azzurro1}\textbf{Codice: Nome} & \textbf{RO3: }{Calcolo Costi}\\
\textbf{Descrizione} & Alcuni costi stimati potrebbero non venire rispettati (per esempio a causa di eventuali ritardi e, di conseguenza, di ore di lavoro aggiuntive richieste) causandone una sovrastima o una sottostima.\\
\textbf{Precauzione} & Nel caso uno dei membri non riuscirà a rispettare i tempi, e di conseguenza i costi preventivati, dovrà avvisare il \respProg{}.\\
\textbf{Grado di rischio} & \textbf{F: }Alta \textbf{P: }Media\\
\textbf{Piano di contingenza} & Il \respProg{} ha il compito di assicurarsi che il preventivo sia il più realistico possibile al fine di evitare riassegnamenti di risorse.\\
\hline
\end{longtable}
}