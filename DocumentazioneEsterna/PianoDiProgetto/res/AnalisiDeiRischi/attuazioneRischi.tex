\section{Attualizzazione dei rischi}

Nella seguente tabella sono stati riportati i rischi in cui il gruppo \Omicron{} è incorso durante lo sviluppo del progetto \nameproject{}.
Per ogni voce nella tabella vengono indicati:
\begin{itemize}
\item \textbf{Codice, Nome e numero occorrenza:} codice e nome risultano essere uguali a quelli riportati nell'analisi dei rischi. L'occorrenza serve per distinguere le varie tipologie di un rischio riscontrate, in quanto tale rischio può presentarsi in diversi modi;
\item \textbf{Descrizione:} indica nello specifico il problema che è avvenuto;
\item \textbf{Contromisura adottata:} indica come il gruppo ha cercato di risolvere il problema o di ridurre i danni causati. Vengono inoltre riportati, se individuati, eventuali miglioramenti che si possono applicare alla contromisura adottata, nel caso si dovesse incorrere nuovamente nello stesso problema.
\end{itemize}

\setcounter{table}{-1}
{

\rowcolors{2}{azzurro2}{azzurro3}

\centering
\renewcommand{\arraystretch}{1.5}
\begin{longtable}{C{3.5cm} | C{11cm}}
\rowcolor{azzurro1}\textbf{Codice: Nome (occorrenza)} & \textbf{RT1: }{Inesperienza tecnologica (1)}\\
\textbf{Descrizione} & Alcuni membri del gruppo hanno riscontrato diverse difficoltà nel comprendere le tecnologie nuove da utilizzare, portando ad un rallentamento del progetto e ad un rischio di consegna in ritardo.\\
\textbf{Contromisura adottata} & I membri del gruppo che avevano compreso le nuove tecnologie hanno aiutato chi era in difficoltà. Così facendo si è riusciti ad iniziare a sviluppare il Proof of Concept\ped{G} rispettando le scadenze stabilite, senza causare ritardi.\\
\hline
\rowcolor{azzurro1}\textbf{Codice: Nome (occorrenza)} & \textbf{RT1: }{Inesperienza tecnologica (2)}\\
\textbf{Descrizione} & Sono stati riscontrati dei problemi nella scelta del design pattern architetturale da usare a causa della mancata esperienza da parte del gruppo. Ciò ha portato ad un rallentamento generale del progetto.\\
\textbf{Contromisura adottata} & A seguito del rilevamento del problema si è deciso di chiedere un parere al proponente quanto prima possibile, in modo da evitare ritardi significativi.\\
\hline
\rowcolor{azzurro1}\textbf{Codice: Nome (occorrenza)} & \textbf{RT3: }{Problemi software (1)}\\
\textbf{Descrizione} & Si è verificato un malfunzionamento con la generazione dei PDF durante il processo di build su GitHub Actions\ped{G}.\\
\textbf{Contromisura adottata} & Uno dei membri del gruppo ha risolto il problema quanto prima possibile. Ciò ha portato a perdite minimali di tempo che non hanno causato ritardi. Nel caso il problema si dovesse verificare nuovamente si cercherà di  lasciare tale compito a più membri, così da ridurre il tempo perso.\\
\hline
\rowcolor{azzurro1}\textbf{Codice: Nome (occorrenza)} & \textbf{RP1: }{Impegni personali (1)}\\
\textbf{Descrizione} & Diversi membri del gruppo sono risultati impossibilitati alla partecipazione di alcuni incontri causa di impegni personali.\\ 
\textbf{Contromisura adottata} & I membri hanno indicato in anticipo eventuali problemi di partecipazione, così da far emergere nel gruppo eventuali necessità di modificare le date degli incontri, nel caso la loro presenza risultasse necessaria.\\
\hline
\rowcolor{azzurro1}\textbf{Codice: Nome (occorrenza)} & \textbf{RP2: }{Impegni accademici (1)}\\ 
\textbf{Descrizione} & Molti membri del gruppo sono risultati poco disponibili nel periodo tra il 20 Gennaio e il 1 Febbraio a causa di esami universitari da sostenere.\\
\textbf{Contromisura adottata} & Grazie al calendario che è stato creato per segnare le date degli esami dei membri del gruppo, si è reso possibile riorganizzare il lavoro rispetto alle diverse disponibilità dei singoli, riuscendo ad impedire così eventuali ritardi di consegna.\\
\hline
\hline
\rowcolor{azzurro1}\textbf{Codice: Nome (occorrenza)} & \textbf{RP2: }{Impegni accademici (2)}\\ 
\textbf{Descrizione} & Tutti i membri del gruppo sono risultati poco disponibili nella prima settimana di Aprile a causa di un'esame universitario da sostenere.\\
\textbf{Contromisura adottata} & Come nella prima occorrenza si è avvisato in anticipo dell'indisponibilità (in quanto era già risultata una contromisura valida). 
In questo caso non si è fatto uso del calendario di gruppo creato precedentemente, in quanto l'esame doveva essere sostenuto da tutti i membri del gruppo ma, nel caso il rischio si verificasse nuovamente, si cercherà di usarlo per motivi di tracciamento.\\
\hline
\rowcolor{azzurro1}\textbf{Codice: Nome (occorrenza)} & \textbf{RP2: }{Impegni accademici (3)}\\ 
\textbf{Descrizione} & Alcuni membri del gruppo sono risultati poco disponibili nella prima settimana di Maggio a causa di un'esame universitario da sostenere.\\
\textbf{Contromisura adottata} & In questo caso i membri che dovevano sostenere la prova hanno avvisato in anticipo della loro indisponibilità e, inoltre, hanno segnato la data della prova nell'apposito calendario. 
Così facendo è stato possibile riorganizzare il lavoro e ridurre l'impatto che tale imprevisto ha avuto sul ritardo della consegna.\\
\hline
\rowcolor{azzurro1}\textbf{Codice: Nome (occorrenza)} & \textbf{RI1: }{Comunicazioni interne (1)}\\
\textbf{Descrizione} & È risultato complesso decidere dei giorni in cui tutti i membri del gruppo fossero disponibili per un incontro interno, a causa di impegni personali da parte dei singoli membri (rischio RP1).\\ 
\textbf{Contromisura adottata} & Per i membri che non hanno potuto partecipare ad un dato incontro interno si sono fatte delle registrazioni di quest'ultimo nel caso risultasse di una certa importanza. I membri che non hanno partecipato all'incontro, dopo aver ascoltato la registrazione, hanno dato una propria opinione riguardo ad eventuali decisioni importanti da prendere, così da aiutare a prevenire il rischio RI3 (contrasti interni).\\
\hline
\hline
\rowcolor{azzurro1}\textbf{Codice: Nome (occorrenza)} & \textbf{RI2: }{Comunicazioni esterne (1)}\\ 
\textbf{Descrizione} & È risultato complesso decidere dei giorni in cui tutti i membri del gruppo fossero disponibili per un incontro con il proponente, a causa di impegni personali da parte dei singoli membri (rischio RP1).\\ 
\textbf{Contromisura adottata} & Per i membri che non hanno potuto partecipare ad un dato incontro esterno è stato fatto un riassunto di ciò che si è discusso al successivo incontro interno del team.\\
\hline
\hline
\rowcolor{azzurro1}\textbf{Codice: Nome (occorrenza)} & \textbf{RO1: }{Calcolo tempistiche (1)}\\
\textbf{Descrizione} & I diversi problemi riscontrati hanno portato alla possibilità di ritardi riducendo così l'affidabilità del calcolo delle tempistiche che era stato effettuato.\\
\textbf{Contromisura adottata} & Grazie a: 
\begin{itemize}
\item contromisure effettuate per i vari problemi riscontrati;
\item una riorganizzazione dei vari compiti;
\item una valutazione preventiva di eventuali ritardi.
\end{itemize} 
si è riusciti a rispettare le varie scadenze imposte.\\
\hline
\hline
\rowcolor{azzurro1}\textbf{Codice: Nome (occorrenza)} & \textbf{RO1: }{Calcolo tempistiche (2)}\\
\textbf{Descrizione} & I problemi riscontrati durante la fase per la consegna della revisione di accettazione hanno portato a dei ritardi, riducendo l'affidabilità del calcolo delle tempistiche effettuato.\\
\textbf{Contromisura adottata} & Grazie alle contromisure adottate il gruppo è riuscito a ridurre il ritardo, ma non a prevenirlo. Questo principalmente è stato causato dall'aver sottovalutato la complessità delle funzionalità mancanti. Un migliore studio iniziale delle funzionalità mancanti avrebbe reso più semplice la prevenzione del ritardo.\\
\hline
\hline
\rowcolor{azzurro1}\textbf{Codice: Nome (occorrenza)} & \textbf{RO2: }{Ritardi (1)}\\
\textbf{Descrizione} & A causa dei problemi RP2(3) e RO1(2) il gruppo non è riuscito a rispettare le milestone imposte, portando a ritardi.\\
\textbf{Contromisura adottata} & Il \respProg{}, a seguito delle segnalazioni di eventuali indisponibilità e problemi ha provato a riassegnare le risorse a disposizione. Ciò è risultato più complesso del previsto, in quanto lo spostare delle risorse da un'attività ad un'altra causava dei nuovi ritardi. Tuttavia, dando priorità ad attività che risultavano più complesse rispetto ad altre, il gruppo è riuscito a ridurre il ritardo conseguito durante la fase di revisione di accettazione.\\
\hline
\hline
\end{longtable}
}
