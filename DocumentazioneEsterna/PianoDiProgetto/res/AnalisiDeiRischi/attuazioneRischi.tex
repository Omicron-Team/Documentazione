\section{Attuazione dei rischi}

Nella seguente tabella sono stati riportati i rischi in cui il gruppo \Omicron{} è incorso durante lo sviluppo del progetto \nameproject{}.
Per ogni voce nella tabella vengono indicati:
\begin{itemize}
\item \textbf{Codice, Nome e numero occorrenza:} codice e nome risultano essere uguali a quelli riportati nell'analisi dei rischi. L'occorrenza serve per distinguere i vari problemi che si sono riscontrati a causa di uno stesso rischio;
\item \textbf{Descrizione:} indica nello specifico il problema che è avvenuto;
\item \textbf{Contromisura adottata:} indica come il gruppo ha cercato di risolvere il problema o di ridurre i danni causati.
\end{itemize}

\setcounter{table}{-1}
{

\rowcolors{2}{azzurro2}{azzurro3}

\centering
\renewcommand{\arraystretch}{1.5}
\begin{longtable}{C{3.5cm} | C{11cm}}
\rowcolor{azzurro1}\textbf{Codice: Nome (occorrenza)} & \textbf{RT1: }{Inesperienza tecnologica (1)}\\
\textbf{Descrizione} & Alcuni membri del gruppo hanno riscontrato diverse difficoltà nel comprendere le tecnologie nuove da utilizzare, portando ad un rallentamento del progetto e ad un rischio di consegna in ritardo.\\
\textbf{Contromisura adottata} & I membri del gruppo che avevano compreso le nuove tecnologie hanno aiutato chi era in difficoltà. Così facendo si è riusciti ad iniziare a sviluppare il Proof of Concept rispettando le scadenze stabilite, senza causare ritardi.\\
\hline
\rowcolor{azzurro1}\textbf{Codice: Nome (occorrenza)} & \textbf{RP1: }{Impegni Personali (1)}\\
\textbf{Descrizione} & Diversi membri del gruppo sono risultati poco disponibili ad incontri in certi periodi a causa di impegni personali.\\ 
\textbf{Contromisura adottata} & I membri hanno indicato in anticipo eventuali problemi di partecipazione, così da far emergere nel gruppo eventuali necessità di modificare le date degli incontri, nel caso la loro presenza risultasse necessaria.\\
\hline
\hline
\rowcolor{azzurro1}\textbf{Codice: Nome (occorrenza)} & \textbf{RP2: }{Impegni accademici (1)}\\ 
\textbf{Descrizione} & Molti membri del gruppo sono risultati poco disponibili nel periodo tra il 20 Gennaio e il 1 Febbraio a causa di esami universitari da sostenere.\\
\textbf{Contromisura adottata} & Grazie al calendario che era stato creato per segnare le date degli esami dei membri del gruppo è stato possibile riorganizzare il lavoro rispetto alle diverse disponibilità in anticipo, riuscendo ad impedire così eventuali ritardi di consegna.\\
\hline
\hline
\rowcolor{azzurro1}\textbf{Codice: Nome (occorrenza)} & \textbf{RI1: }{Comunicazioni interne (1)}\\
\textbf{Descrizione} & È risultato complesso decidere dei giorni in cui tutti i membri del gruppo fossero disponibili per un incontro interno, a causa di impegni personali da parte dei singoli membri(rischio RP1).\\ 
\textbf{Contromisura adottata} & Per i membri che non hanno potuto partecipare ad un dato incontro interno si sono fatte delle registrazioni di quest'ultimo nel caso risultasse di una certa importanza. I membri che non hanno partecipato all'incontro, dopo aver ascoltato la registrazione, hanno dato una propria opinione riguardo ad eventuali decisioni importanti da prendere così da aiutare a prevenire il rischio RI3(contrasti interni).\\
\hline
\hline
\rowcolor{azzurro1}\textbf{Codice: Nome (occorrenza)} & \textbf{RI2: }{Comunicazioni esterne (1)}\\ 
\textbf{Descrizione} & È risultato complesso decidere dei giorni in cui tutti i membri del gruppo fossero disponibili per un incontro con il proponente, a causa di impegni personali da parte dei singoli membri(rischio RP1).\\ 
\textbf{Contromisura adottata} & Per i membri che non hanno potuto partecipare ad un dato incontro esterno(come nel caso degli incontri interni) si sono fatte delle registrazioni. I membri che non erano riusciti a partecipare hanno ascoltato le registrazioni effettuate.\\
\hline
\hline
\rowcolor{azzurro1}\textbf{Codice: Nome (occorrenza)} & \textbf{RO1: }{Calcolo Tempistiche (1)}\\
\textbf{Descrizione} & I diversi problemi riscontrati hanno portato alla possibilità di ritardi riducendo così l'affidabilità del calcolo delle tempistiche che era stato effettuato.\\
\textbf{Contromisura adottata} & Grazie a: 
\begin{itemize}
\item contromisure effettuate per i vari problemi riscontrati;
\item una riorganizzazione dei vari compiti;
\item una valutazione preventiva di eventuali ritardi.
\end{itemize} 
si è riusciti a rispettare le varie scadenze imposte.\\
\hline
\hline
\end{longtable}
}