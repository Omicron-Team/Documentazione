\subsection{Fase di progettazione architetturale}
\subsubsection{Consuntivo}

{

\rowcolors{2}{azzurro2}{azzurro3}

\centering
\renewcommand{\arraystretch}{1.8}
\begin{longtable}{C{4cm} C{1.5cm} C{4cm} }

\rowcolor{azzurro1}
\textbf{Ruolo} &
\textbf{Ore}&
\textbf{Costo}\\
\endhead

\textit{Responsabile} & 16 (+0) & 480\euro{} (+0\euro{}) \\
\ammProg & 28 (+0) & 560\euro{} (+0\euro{}) \\
\analProg & 28 (+0) & 700\euro{} (+0\euro{}) \\
\progetProg & 66 (-5) & 1452\euro{} (-110\euro{}) \\
\programProg & 24 (+6) & 360\euro{} (+90\euro{}) \\
\verifProg & 62 (-3) & 930\euro{} (-45\euro{})\\
\textbf{Totale Preventivo} & \textbf{224} & \textbf{4482\euro{}} \\
\textbf{Totale Consuntivo} & \textbf{222} & \textbf{4417\euro{}} \\
\textbf{Differenza} & \textbf{-2} & \textbf{-65\euro{}} \\


\rowcolor{white}
\caption{Consuntivo di periodo della fase di progettazione architetturale}\\

\end{longtable}
}

\subsubsection{Conclusioni}
Come viene riportato dalla tabella precedente, il bilancio risulta positivo ed economicamente sono stati risparmiati 65.00\euro{}, grazie ad una attenta analisi realizzata nella fase precedente, che ha portato a meno ore dedicate alla progettazione; inoltre, vari membri del gruppo erano occupati con impegni accademici, che hanno rimosso alcune ore dallo sviluppo del progetto.\\ Nonostante ciò sono state utilizzate più ore per il ruolo di \programProg{}, dato la difficoltà di codifica di tecnologie nuove e per la pratica, spesso usata, di trial and error\ped{G}.\\

\subsubsection{Preventivo a finire}
Gli obiettivi da noi pre-impostati per la realizzazione del Proof of Concept\ped{G} sono stati completamente soddisfatti. Inoltre, sono stati implementati ulteriori requisiti, in modo che tutte le tecnologie coinvolte nella realizzazione del prodotto fossero testate e provate fattibili prima della fase di progettazione di dettaglio e codifica.\\

Per questo motivo, oltre all'esito positivo del consuntivo e la differenza minima tra ore preventivate ed effettive, manteniamo invariato il nostro preventivo inizialmente calcolato. Inoltre, saranno a nostra disposizione 65.00\euro{} da utilizzare per eventuali ritardi.\\

Nonostante la discrepanza di ore effettive dei ruoli di \programProg{} e \progetProg{}, riteniamo che la ripartizione effettuata per la fase di progettazione di dettaglio e codifica sia più coerente rispetto alla fase appena conclusa, che infatti include molte più ore per lo sviluppo del prodotto, cosa che a noi mancava per lo sviluppo della Technology Baseline\ped{G}. 

