\subsubsection{Incremento 9}
\myparagraph{Consuntivo}

{

\rowcolors{2}{azzurro2}{azzurro3}

\centering
\renewcommand{\arraystretch}{1.8}
\begin{longtable}{C{4cm} C{1.5cm} C{4cm} }

\rowcolor{azzurro1}
\textbf{Ruolo} &
\textbf{Ore}&
\textbf{Costo}\\
\endhead

\textit{Responsabile} & 4 (+0) & 120 (+0\euro{}) \\
\ammProg & 4 (+0) & 80\euro{} (+0\euro{}) \\
\analProg & 0 (+0) & 0\euro{} (+0\euro{}) \\
\progetProg & 3 (+0) & 66\euro{} (+0\euro{}) \\
\programProg & 0 (+0) & 0\euro{} (+0\euro{}) \\
\verifProg & 17 (+0) & 225\euro{} (+0\euro{})\\
\textbf{Totale Preventivo} & \textbf{28} & \textbf{521\euro{}} \\
\textbf{Totale Consuntivo} & \textbf{28} & \textbf{521\euro{}} \\
\textbf{Differenza} & \textbf{+0} & \textbf{+0\euro{}} \\


\rowcolor{white}
\caption{Consuntivo di periodo dell'incremento 9}\\

\end{longtable}
}

\myparagraph{Considerazioni}
Il gruppo ha raggiunto gli obiettivi pianificati per questo incremento, preparandosi al meglio per le varie presentazioni.\\
Le attività sono state portate a termine senza ritardi, nonostante alcuni impegni didattici da parte di ogni membro del gruppo. Sono state inoltre aggiornate le date dei prossimi incrementi in base alla data dell'incontro con il \CR{} per la Product baseline\ped{G}.

\myparagraph{Preventivo a finire}
Il bilancio è in pari rispetto al preventivo dell'incremento. Il risparmio totale resta di 21\euro{}.
Gli obiettivi preimpostati sono stati raggiunti a pieno, senza alcun rallentamento rispetto a quanto pianificato.
Alla luce dei costi e degli obiettivi raggiunti, non si ritiene necessaria alcuna ripianificazione degli incrementi (eccetto l'aggiornamento delle date) o modifica del preventivo.


