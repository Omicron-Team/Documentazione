\subsubsection{Incremento 10}
\myparagraph{Consuntivo}

{

\rowcolors{2}{azzurro2}{azzurro3}

\centering
\renewcommand{\arraystretch}{1.8}
\begin{longtable}{C{4cm} C{1.5cm} C{4cm} }

\rowcolor{azzurro1}
\textbf{Ruolo} &
\textbf{Ore}&
\textbf{Costo}\\
\endhead

\textit{Responsabile} & 3 (+0) & 90 (+0\euro{}) \\
\ammProg & 3 (+0) & 60\euro{} (+0\euro{}) \\
\analProg & 0 (+0) & 0\euro{} (+0\euro{}) \\
\progetProg & 4 (+0) & 88\euro{} (+0\euro{}) \\
\programProg & 10 (-2) & 150\euro{} (-30\euro{}) \\
\verifProg & 13 (+1) & 195\euro{} (+15\euro{})\\
\textbf{Totale Preventivo} & \textbf{33} & \textbf{583\euro{}} \\
\textbf{Totale Consuntivo} & \textbf{32} & \textbf{568\euro{}} \\
\textbf{Differenza} & \textbf{-1} & \textbf{-15\euro{}} \\


\rowcolor{white}
\caption{Consuntivo di periodo dell'incremento 10}\\

\end{longtable}
}

\myparagraph{Considerazioni}
Il gruppo ha ricevuto un esito positivo dal colloquio con proponente e committente per la product baseline. Di conseguenza, il gruppo ha potuto prepararsi al meglio per la Revisione di Qualifica, finalizzando una demo del prodotto e completando la documentazione in base alle segnalazioni.\\
Sono quindi stati raggiunti gli obiettivi di questo incremento, senza alcun rallentamento.\\
Sono state inoltre richieste meno ore per il ruolo di \programProg{}, dato le minori correzioni al codice. Il ruolo di \verifProg{} invece ha richiesto un'ora in più per le verifiche ai vari manuali in lingua inglese.


\myparagraph{Preventivo a finire}
Il bilancio è positivo rispetto al preventivo dell'incremento, con un risparmio di 15\euro{}, che si va ad aggiungere al bilancio della precedente fase, risultando in un risparmio totale di 36\euro{} da poter usare nei prossimi incrementi in caso di ritardi.
Gli obiettivi preimpostati sono stati raggiunti a pieno, senza alcun rallentamento rispetto a quanto pianificato.\\
Alla luce dei costi e degli obiettivi raggiunti, non si ritiene necessaria alcuna ripianificazione degli incrementi o modifica del preventivo.\\


