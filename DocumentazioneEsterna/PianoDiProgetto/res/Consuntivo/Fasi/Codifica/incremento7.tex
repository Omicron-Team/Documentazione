\subsubsection{Incremento 7}
\myparagraph{Consuntivo}

{

\rowcolors{2}{azzurro2}{azzurro3}

\centering
\renewcommand{\arraystretch}{1.8}
\begin{longtable}{C{4cm} C{1.5cm} C{4cm} }

\rowcolor{azzurro1}
\textbf{Ruolo} &
\textbf{Ore}&
\textbf{Costo}\\
\endhead

\textit{Responsabile} & 4 (+0) & 120 (+0\euro{}) \\
\ammProg & 3 (+0) & 60\euro{} (+0\euro{}) \\
\analProg & 0 (+0) & 0\euro{} (+0\euro{}) \\
\progetProg & 13 (+0) & 286\euro{} (+0\euro{}) \\
\programProg & 48 (-2) & 720\euro{} (-30\euro{}) \\
\verifProg & 14 (+0) & 255\euro{} (+0\euro{})\\
\textbf{Totale Preventivo} & \textbf{85} & \textbf{1441\euro{}} \\
\textbf{Totale Consuntivo} & \textbf{83} & \textbf{1411\euro{}} \\
\textbf{Differenza} & \textbf{-2} & \textbf{-30\euro{}} \\


\rowcolor{white}
\caption{Consuntivo di periodo dell'incremento 7}\\

\end{longtable}
}

\myparagraph{Considerazioni}
Il gruppo ha raggiunto gli obiettivi pianificati per questo incremento e l'andamento delle attività non ha subito rallentamenti.\\
Lo sviluppo delle funzionalità non ha trovato ostacoli grossi e ci ha permesso di risparmiare due ore per il ruolo di \programProg{}, soprattutto per quanto riguarda i casi d'uso UC17, UC18, UC19 e UC23, che sono stati più semplici del previsto.

\myparagraph{Preventivo a finire}
Il bilancio è positivo rispetto al preventivo dell'incremento, con un risparmio di 30\euro{}, che si va ad aggiungere al bilancio della precedente fase, risultando in un risparmio totale di 6\euro{}.
Gli obiettivi preimpostati sono stati raggiunti a pieno, senza alcun rallentamento rispetto a quanto pianificato.
Alla luce dei costi e degli obiettivi raggiunti, non si ritiene necessaria alcuna ripianificazione degli incrementi o modifica del preventivo.


