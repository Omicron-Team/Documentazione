\subsubsection{Incremento 8}
\myparagraph{Consuntivo}

{

\rowcolors{2}{azzurro2}{azzurro3}

\centering
\renewcommand{\arraystretch}{1.8}
\begin{longtable}{C{4cm} C{1.5cm} C{4cm} }

\rowcolor{azzurro1}
\textbf{Ruolo} &
\textbf{Ore}&
\textbf{Costo}\\
\endhead

\textit{Responsabile} & 4 (+0) & 120 (+0\euro{}) \\
\ammProg & 3 (+0) & 60\euro{} (+0\euro{}) \\
\analProg & 0 (+0) & 0\euro{} (+0\euro{}) \\
\progetProg & 13 (+0) & 286\euro{} (+0\euro{}) \\
\programProg & 44 (+0) & 660\euro{} (+0\euro{}) \\
\verifProg & 19 (-1) & 285\euro{} (-15\euro{})\\
\textbf{Totale Preventivo} & \textbf{83} & \textbf{1411\euro{}} \\
\textbf{Totale Consuntivo} & \textbf{82} & \textbf{1396\euro{}} \\
\textbf{Differenza} & \textbf{-1} & \textbf{-15\euro{}} \\


\rowcolor{white}
\caption{Consuntivo di periodo dell'incremento 8}\\

\end{longtable}
}

\myparagraph{Considerazioni}
Il gruppo ha raggiunto gli obiettivi pianificati per questo incremento e l'andamento delle attività non ha subito rallentamenti.\\
Lo sviluppo delle funzionalità, come nell'incremento precedente, non ha trovato ostacoli grossi. Anche la documentazione non ha avuto problemi, permettendo inoltre di utilizzare un'ora in meno per il ruolo di \verifProg{}, data anche l'esperienza accumulata durante il progetto.

\myparagraph{Preventivo a finire}
Il bilancio è positivo rispetto al preventivo dell'incremento, con un risparmio di 15\euro{}, che si va ad aggiungere al bilancio della precedente fase, risultando in un risparmio totale di 21\euro{}.
Gli obiettivi preimpostati sono stati raggiunti a pieno, senza alcun rallentamento rispetto a quanto pianificato.
Alla luce dei costi e degli obiettivi raggiunti, non si ritiene necessaria alcuna ripianificazione degli incrementi o modifica del preventivo.

