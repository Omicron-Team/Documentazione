\subsubsection{Incremento 5}
\myparagraph{Consuntivo}

{

\rowcolors{2}{azzurro2}{azzurro3}

\centering
\renewcommand{\arraystretch}{1.8}
\begin{longtable}{C{4cm} C{1.5cm} C{4cm} }

\rowcolor{azzurro1}
\textbf{Ruolo} &
\textbf{Ore}&
\textbf{Costo}\\
\endhead

\textit{Responsabile} & 4 (+0) & 120 (+0\euro{}) \\
\ammProg & 10 (-2) & 200\euro{} (-40\euro{}) \\
\analProg & 0 (+0) & 0\euro{} (+0\euro{}) \\
\progetProg & 21 (+0) & 462\euro{} (+0\euro{}) \\
\programProg & 0 (+0) & 0\euro{} (+0\euro{}) \\
\verifProg & 15 (+0) & 225\euro{} (+0\euro{})\\
\textbf{Totale Preventivo} & \textbf{50} & \textbf{1007\euro{}} \\
\textbf{Totale Consuntivo} & \textbf{49} & \textbf{967\euro{}} \\
\textbf{Differenza} & \textbf{-2} & \textbf{-40\euro{}} \\


\rowcolor{white}
\caption{Consuntivo di periodo dell'incremento 5}\\

\end{longtable}
}

\myparagraph{Considerazioni}
Il gruppo ha raggiunto gli obiettivi pianificati per questo incremento e l'andamento delle attività non ha subito rallentamenti.\\
La configurazione delle nuove repository\ped{G} e degli ambienti di sviluppo ha richiesto meno ore del previsto per il ruolo di \ammProg{}, dato che non sono stati riscontrati particolari problemi dagli incontri effettuati con il proponente e con il committente.


\myparagraph{Preventivo a finire}
Il bilancio è positivo rispetto al preventivo dell'incremento, con un risparmio di 40\euro{}, che si va ad aggiungere al bilancio della precedente fase, risultando in un risparmio totale di 105\euro{}.
Inoltre sono stati raggiunti tutti gli obiettivi pianificati per questo incremento, senza alcun rallentamento.
Alla luce dei costi e degli obiettivi raggiunti, non si ritiene necessaria alcuna ripianificazione degli incrementi o modifica del preventivo.

