\subsubsection{Incremento 13}
\myparagraph{Consuntivo}

{

\rowcolors{2}{azzurro2}{azzurro3}

\centering
\renewcommand{\arraystretch}{1.8}
\begin{longtable}{C{4cm} C{1.5cm} C{4cm} }

\rowcolor{azzurro1}
\textbf{Ruolo} &
\textbf{Ore}&
\textbf{Costo}\\
\endhead

\textit{Responsabile} & 5 (+0) & 150 (+0\euro{}) \\
\ammProg & 5 (+0) & 100\euro{} (+0\euro{}) \\
\analProg & 0 (+0) & 0\euro{} (+0\euro{}) \\
\progetProg & 0 (+0) & 0\euro{} (+0\euro{}) \\
\programProg & 0 (+0) & 0\euro{} (+0\euro{}) \\
\verifProg & 15 (+0) & 225\euro{} (+0\euro{})\\
\textbf{Totale Preventivo} & \textbf{25} & \textbf{225\euro{}} \\
\textbf{Totale Consuntivo} & \textbf{25} & \textbf{225\euro{}} \\
\textbf{Differenza} & \textbf{+0} & \textbf{+0\euro{}} \\


\rowcolor{white}
\caption{Consuntivo di periodo dell'incremento 13}\\

\end{longtable}
}

\myparagraph{Considerazioni}
In questo ultimo incremento sono state effettuate le ultime preparazioni alla Revisione di Accettazione. \\
Sono stati inseriti gli ultimi test per migliorare il prodotto e stesi conseguentemente nel \PdQv{4.0.0}, insieme alla verifica finale dei test di accettazione.
Inoltre, I documenti sono stati completati come previsto.\\
Gli obiettivi sono stati soddisfatti; il prodotto, insieme alla relativa documentazione, è pronto per il rilascio.



\myparagraph{Preventivo a finire}
Il bilancio è in pari rispetto al preventivo dell'incremento. Il risparmio totale resta di 6\euro{}.
Sono state quindi concluse tutte le attività di progetto senza problemi riguardanti il bilancio.


