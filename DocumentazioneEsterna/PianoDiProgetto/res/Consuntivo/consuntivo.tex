\section{Consuntivo di periodo}

Di seguito vengono indicate le spese effettivamente sostenute per ogni ruolo, confrontate con quanto preventivato per ogni incremento. Il bilancio potrà risultare:

\begin{itemize}
\item \textbf{positivo:} se il consuntivo è minore di quanto preventivato;
\item \textbf{pari:} se il consuntivo è uguale a quanto preventivato;
\item \textbf{negativo:} se il consuntivo è maggiore di quanto preventivato.
\end{itemize}

Verrà inoltre analizzato l'effettivo andamento del gruppo nei vari incrementi, tramite gli obiettivi pianificati per ognuno di essi. Infine verrà fatta un'analisi del preventivo a finire per ogni incremento.

\subsection{Fase di analisi}
\subsubsection{Consuntivo}

Le seguenti ore di lavoro indicate sono considerate di investimento, quindi esse non saranno rendicontate.

\subsubsection{Conclusioni}
\subsubsection{Preventivo a finire}

\subsection{Progettazione architetturale}
\textit{\textbf{Periodo}: dal 2021-01-18 al 2021-03-08}

L'inizio di questa fase coincide con la data della Revisione dei Requisiti e si conclude con la scadenza della Revisione di Progettazione.

\subsubsection{Attività}

\begin{itemize}
\item \textbf{Incremento e verifica documenti}: vengono realizzate le aggiunte necessarie ai documenti e le eventuali correzioni provenienti dalle segnalazioni del committente o da analisi interne. I documenti in questione sono:
\begin{itemize}
\item \NdP{};
\item \AdR{};
\item \PdQ{};
\item \PdP{};
\item Glossario.
\end{itemize}
\item \textbf{Technology Baseline\ped{G}}: viene realizzato un \textit{Proof of Concept}\ped{G} per testare le tecnologie coinvolte e per provare che le funzionalità base del prodotto possano effettivamente essere implementate, soddisfacendo i requisiti collegati. Esso verrà condiviso con il proponente, per verificare che sia soddisfacente, e con il \CR{}.\\ Come precedentemente notato, nella sezione \S{3.2}, per la realizzazione del \textit{Proof of Concept}\ped{G} sono stati individuati quattro incrementi, che suddividono lo sviluppo in  aree di interesse distinte. Per ogni incremento vengono riportati i requisiti che ci impegniamo a soddisfare nel \textit{Proof of Concept}\ped{G} (non comprendendo i requisiti figlio):
\begin{itemize}
\item \textbf{Incremento 1}: R1F1;
\item \textbf{Incremento 2}: R1F7.1, R1F9.1;
\item \textbf{Incremento 3}: R1F4;
\item \textbf{Incremento 4}: R1F11.
\end{itemize}
Questa lista di requisiti non esclude il fatto che possano essere soddisfatti altri requisiti aggiuntivi.
\item \textbf{Consolidamento}: viene realizzata la presentazione da esporre in sede di Revisione di Progettazione e si approfondiscono aspetti lacunari riguardo il progetto.
\end{itemize}

\subsubsection{Periodi}

\begin{itemize}
\item \textbf{Periodo 1}: \textit{dal 2021-01-18 al 2021-01-31}. \\
Viene svolto un approfondimento personale da ogni membro del gruppo riguardo le tecnologie da utilizzare per lo sviluppo del prodotto. Inoltre, se necessario, verranno corretti i documenti realizzati nella fase di analisi.
\item \textbf{Periodo 2}: \textit{dal 2021-01-31 al 2021-03-01}. \\
Viene realizzata la Technology Baseline\ped{G}, compresa di un'analisi iniziale degli incrementi e delle tecnologie, la realizzazione degli incrementi e una verifica finale di integrazione. Di seguito vengono riportati i periodi individuati per i singoli incrementi:
\begin{itemize}
\item \textbf{Incremento 1}: \textit{dal 2021-02-04 al 2021-02-09};
\item \textbf{Incremento 2}: \textit{dal 2021-02-09 al 2021-02-14};
\item \textbf{Incremento 3}: \textit{dal 2021-02-14 al 2021-02-19};
\item \textbf{Incremento 4}: \textit{dal 2021-01-19 al 2021-02-24}.
\end{itemize}
Il periodo si conclude con la consegna del materiale per la Revisione di Progettazione;
\item \textbf{Periodo 3}: \textit{dal 2021-03-01 al 2021-03-08}. \\
Viene svolta l'attività di consolidamento. Il periodo si conclude con la Revisione di Progettazione;
\end{itemize}

\subsubsection{Diagramma di Gantt}

\begin{figure}[H]
\centering

\centerline{\includegraphics[scale=0.5]{res/Pianificazione/Gantt/progettazione}}
\caption{Diagramma di Gantt per il periodo di progettazione architetturale}
\end{figure}
\subsection{Progettazione di dettaglio e codifica}
\textit{\textbf{Periodo}: dal 2021-03-08 al 2021-04-09}

L'inizio di questa fase coincide con data della Revisione di Progettazione e conclude con la scadenza della Revisione di Qualifica.

\subsubsection{Attività}

\begin{itemize}
\item \textbf{Incremento e verifica documenti}: vengono realizzate le aggiunte necessarie ai documenti e le eventuali correzioni provenienti dalle segnalazioni dal committente o da analisi interne. I documenti in questione sono:
\begin{itemize}
\item \NdP{};
\item \AdR{};
\item \PdQ{};
\item \PdP{};
\item \textit{Glossario}.
\end{itemize}
\item \textbf{Product Baseline\ped{G}}: viene realizzata la baseline architetturale del prodotto, in base alla Technology Baseline\ped{G}. Il codice sviluppato precedentemente nel Proof of Concept può essere riutilizzato, se ritenuto corretto per il design architetturale individuato. Viene inoltre redatto l'\textit{Allegato tecnico}\ped{G} per essere inviato e presentato al \CR{}.\\ Come precedentemente notato, nella sezione \S{3.2}, per la realizzazione del prodotto sono stati individuati quattro incrementi, che suddividono lo sviluppo in aree di interesse distinte. I requisiti obbligatori di tali incrementi saranno coloro che ci impegneremo a soddisfare in questa fase.
\item \textbf{Manuali}: Durante lo sviluppo della Product Baseline\ped{G} verranno redatti il \MU e il \MM. Il primo servirà per fornire istruzioni per l'utilizzo dell'applicazione, il secondo per fornire informazioni necessarie per il mantenimento e l'ampliamento del prodotto;
\item \textbf{Consolidamento}: viene realizzata la presentazione da esporre in sede di Revisione di Qualifica e si approfondiscono aspetti lacunari riguardo il progetto.
\end{itemize}

\subsubsection{Periodi}

\begin{itemize}
\item \textbf{Periodo 1}: \textit{dal 2021-03-8 al 2021-03-11}. \\
Viene svolto un ulteriore approfondimento personale per lo sviluppo dei requisiti non visti dal Proof of Concept\ped{G}. Inoltre, se necessario, verranno corretti i documenti realizzati nella fase di progettazione.
\item \textbf{Periodo 2}: \textit{dal 2021-03-11 al 2021-04-02}. \\
Viene realizzata la Product Baseline\ped{G}, compresa di \textit{Allegato Tecnico} e di manuali. Di seguito vengono riportati i periodi individuati per i singoli incrementi:
\begin{itemize}
\item \textbf{Incremento 1}: \textit{dal 2021-03-12 al 2021-03-17};
\item \textbf{Incremento 2}: \textit{dal 2021-03-17 al 2021-03-22};
\item \textbf{Incremento 3}: \textit{dal 2021-03-22 al 2021-03-27};
\item \textbf{Incremento 4}: \textit{dal 2021-03-27 al 2021-03-31}.
\end{itemize}
Il periodo conclude con la consegna del materiale per la Revisione di Qualifica;
\item \textbf{Periodo 3}: \textit{dal 2021-04-02 al 2021-04-09}. \\
Viene svolta l'attività di consolidamento. Il periodo conclude con la Revisione di Qualifica;
\end{itemize}

\subsubsection{Diagramma di Gantt}

\begin{figure}[H]
\centering

\centerline{\includegraphics[scale=0.6]{res/Pianificazione/Gantt/codifica}}
\caption{Diagramma di Gantt per il periodo di progettazione di dettaglio e codifica}
\end{figure}

Per migliorare la visualizzazione del diagramma, la pianificazione dei singoli incrementi viene rappresentata dal successivo diagramma, che ha una durata di 5 giorni. L'ultimo incremento è pianificato per solo 4 giorni poiché prevediamo non sia così impegnativo rispetto agli altri, che richiedono molta più codifica.\\

\begin{figure}[H]
\centering

\centerline{\includegraphics[scale=1]{res/Pianificazione/Gantt/incrementoCodifica}}
\caption{Diagramma di Gantt per i singoli incrementi nel periodo di progettazione di dettaglio e codifica}
\end{figure}

