\section{Introduzione}
\subsection{Scopo del documento}
Questo documento ha lo scopo di definire la modalità con cui il progetto \nameproject{} verrà svolto dal gruppo \Omicron{}. I punti trattati nel documento sono i seguenti:
\begin{itemize}
\item analisi dei rischi;
\item descrizione breve del modello di sviluppo adottato;
\item pianificazione delle attività e suddivisione dei ruoli;
\item stima dei costi e delle risorse necessarie.
\end{itemize}
\subsection{Scopo del prodotto}
Lo scopo del capitolato C2 è la realizzazione di una generica piattaforma di e-commerce\ped{G}, chiamata \nameproject{}, basata su tecnologia Serverless\ped{G} da vendere a mercanti. \nameproject{} deve essere distribuibile usando l'account AWS\ped{G} del mercante con una configurazione manuale minima. Deve inoltre essere prodotta una piattaforma dimostrativa per l'utilizzo di \nameproject{}.
\subsection{Glossario}
Viene fornito un glossario il quale scopo è quello di evitare ambiguità nel linguaggio utilizzato fornendo una definizione ai vari termini usati nella documentazione. Il glossario può essere trovato nell'apposito documento \Glossario{}.pdf.
\subsection{Riferimenti}
\subsubsection{Normativi}
\begin{itemize}
\item \textbf{\NdP}: \NdPv{4.0.0};
\item \textbf{Regolamento organigramma}:\\ \url{https://www.math.unipd.it/~tullio/IS-1/2020/Progetto/RO.html};
\item \textbf{Regolamento didattico del progetto}:\\ \url{https://www.math.unipd.it/~tullio/IS-1/2020/Dispense/P1.pdf};
\end{itemize}

\subsubsection{Informativi}
\begin{itemize}
\item \textbf{\AdR}: \AdRv{3.0.0};
\item \textbf{Modello incrementale}:\\ \url{https://it.wikipedia.org/wiki/Modello_incrementale} \\ \url{https://www.math.unipd.it/~tullio/IS-1/2020/Dispense/L05.pdf} - p.18-21;
\item \textbf{Regolamento didattico del progetto - Baselines e allegato tecnico(p.8-9)}:\\ \url{https://www.math.unipd.it/~tullio/IS-1/2020/Dispense/P1.pdf};
\item \textbf{Gestione di progetto}:\\ \url{https://www.math.unipd.it/~tullio/IS-1/2020/Dispense/L06.pdf};

\end{itemize}

\subsection{Scadenze}
Il gruppo \Omicron{} si impegna a rispettare le seguenti scadenze per lo svolgimento del progetto \nameproject{}:
\begin{itemize}
\item \textbf{Revisione dei Requisiti:} 2021-01-18;
\item \textbf{Revisione di Progettazione:} 2021-03-08;
\item \textbf{Revisione di Qualifica:} 2021-04-23;
\item \textbf{Revisione di Accettazione:} 2021-06-01.
\end{itemize}