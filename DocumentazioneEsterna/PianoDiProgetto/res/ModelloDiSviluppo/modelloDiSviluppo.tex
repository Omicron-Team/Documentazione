\section{Modello di sviluppo}
Come modello di sviluppo si è deciso di adottare il \textbf{modello incrementale}, al fine di garantire qualità e conformità del prodotto.

\subsection{Modello incrementale}
Nel modello di sviluppo incrementale si suddivide lo sviluppo del prodotto in incrementi. Ogni incremento porta a nuove funzionalità rispetto al precedente riducendo, inoltre, il rischio di fallimento. Questo ciclo viene ripetuto finché non si ha un numero di requisiti soddisfatti sufficiente.\\
Eventuali modifiche, aggiunte o rimozioni dei requisiti non sono concesse durante la fase di sviluppo dell'incremento corrente e devono essere consentite dal proponente.\\
Le motivazioni che ci hanno spinto ad adottare il modello incrementale sono i suoi seguenti vantaggi:
\begin{itemize}
\item alla fine di ogni incremento viene effettuata una fase di verifica. Così facendo eventuali errori sono vincolati nell'incremento corrente;
\item eventuali modifiche sono limitate al singolo incremento, rendendole più economiche dal punto di vista del tempo di codifica;
\item alla fine di ogni incremento vi è la possibilità di avere un riscontro con il proponente, il quale può valutare le funzionalità aggiunte e le modifiche effettuate;
\item si dà priorità allo sviluppo delle funzionalità primarie, cosicché il proponente possa subito valutarle;
\item risulta un ottimo modello da utilizzare nel nostro progetto, in quanto fin dall'inizio abbiamo i requisiti che devono venire soddisfatti.
\end{itemize}

\subsection{Incrementi}
Per attuare il modello incrementale in modo corretto, è necessario individuare degli incrementi che suddivideranno il progetto in aree distinte. Essi vengono collegati ai requisiti individuati nel documento \AdRv{2.0.0}, in modo da sapere che funzionalità ogni incremento prevede di implementare. Per ogni requisito riportato nella seguente tabella, si prevede di implementare anche i corrispondenti requisiti figli. \\
Per quanto riguarda la fase di analisi, gli unici incrementi previsti sono i documenti da realizzare. In realtà, data l'inesperienza, si riveleranno probabilmente imprecisi inizialmente ed essi verranno eventualmente corretti nelle fasi successive.
Per il resto della realizzazione del progetto, sono stati individuati quattro incrementi:

{

\rowcolors{2}{azzurro2}{azzurro3}

\centering
\renewcommand{\arraystretch}{1.8}
\begin{longtable}{C{3cm} C{8cm} C{3cm} }

\rowcolor{azzurro1}
\textbf{Nome} &
\textbf{Descrizione}&
\textbf{Requisiti}\\
\endhead

\textbf{Incremento 1} & Ci si occuperà dell'autenticazione nel sito, compresa di registrazione, login, logout e visualizzazione del profilo. & R1F1 \newline R1F2 \newline R1F3 \newline R1F6 \\
\textbf{Incremento 2} & Ci si occuperà della visualizzazione dei prodotti nel sito e della dashboard del venditore, per gestire i prodotti. & R1F7 \newline R1F8 \newline R1F9 \newline R1F10 \\
\textbf{Incremento 3} & Ci si occuperà dell'acquisto dei prodotti nel sito, compreso di gestione del carrello e del checkout degli ordini. & R1F4 \newline R1F5 \\
\textbf{Incremento 4} & Ci si occuperà delle funzionalità riservate agli admin e alla realizzazione di eventuali requisiti mancanti dagli incrementi precedenti. & R1V4 \\

\rowcolor{white}
\caption{Lista di incrementi individuati}

\end{longtable}
}

Per ognuno di questi verranno inoltre incrementati i documenti coinvolti. In particolare l'Allegato tecnico\ped{G}, il \MU{} ed il \MM{}, senza però escludere i documenti precedentemente realizzati nella fase di analisi.\\
Viene inoltre realizzato un incremento iniziale per ogni fase (diversa da quella di analisi) riservato ai documenti, per l'aggiornamento di essi.
