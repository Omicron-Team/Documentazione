\section{Modello di sviluppo}
Come modello di sviluppo si è deciso di adottare il \textbf{modello incrementale}, al fine di garantire qualità e conformità del prodotto.

\subsection{Modello incrementale}
Nel modello di sviluppo incrementale si suddivide lo sviluppo del prodotto in incrementi. Ogni incremento porta a nuove funzionalità rispetto al precedente riducendo, inoltre, il rischio di fallimento. Questo ciclo viene ripetuto finché non si ha un numero di requisiti soddisfatti sufficiente.\\
Eventuali modifiche, aggiunte o rimozioni dei requisiti non sono concesse durante la fase di sviluppo dell'incremento corrente e devono essere consentite dal proponente.\\
Le motivazioni che ci hanno spinto ad adottare il modello incrementale sono i suoi seguenti vantaggi:
\begin{itemize}
\item alla fine di ogni incremento viene effettuata una fase di verifica. Così facendo eventuali errori sono vincolati nell'incremento corrente;
\item eventuali modifiche sono limitate al singolo incremento, rendendole più economiche dal punto di vista del tempo di codifica;
\item alla fine di ogni incremento vi è la possibilità di avere un riscontro con il proponente, il quale può valutare le funzionalità aggiunte e le modifiche effettuate;
\item si dà priorità allo sviluppo delle funzionalità primarie, cosicché il proponente possa subito valutarle;
\item risulta un ottimo modello da utilizzare nel nostro progetto, in quanto fin dall'inizio abbiamo i requisiti che devono venire soddisfatti.
\end{itemize}

\subsection{Incrementi}
Per attuare il modello incrementale in modo corretto, è necessario individuare degli incrementi che suddivideranno il progetto in aree distinte. Essi vengono collegati ai requisiti individuati nel documento \AdRv{3.0.0}, in modo da sapere che funzionalità ogni incremento prevede di implementare. Per ogni requisito riportato nella seguente tabella, si prevede di implementare anche i corrispondenti requisiti figli. \\
Per quanto riguarda la fase di analisi, gli unici incrementi previsti sono i documenti da realizzare. In realtà, data l'inesperienza, si riveleranno probabilmente imprecisi inizialmente ed essi verranno eventualmente corretti nelle fasi successive.
Per il resto della realizzazione del progetto, sono stati individuati i seguenti incrementi:

{

\rowcolors{2}{azzurro2}{azzurro3}

\centering
\renewcommand{\arraystretch}{1.8}
\begin{longtable}{C{3cm} C{8cm} C{3cm} }

\rowcolor{azzurro1}
\textbf{Nome} &
\textbf{Obiettivi}&
\textbf{Requisiti}\\
\endhead

\rowcolor{azzurro1}
\multicolumn{3}{c}{\textbf{Fase di progettazione architetturale}}\\
\textbf{Incremento 1} & Implementazione funzionalità base per l'autenticazione e per la visualizzazione della pagina del profilo per Proof of Concept\ped{G}, incremento della documentazione & R1F1 \newline R1F2 \newline R1F3 \newline R1F6 \\
\textbf{Incremento 2} & Implementazione funzionalità base per la visualizzazione dei prodotti e dashboard venditore per Proof of Concept\ped{G}, incremento della documentazione & R1F7 \newline R1F9 \\
\textbf{Incremento 3} & Implementazione funzionalità base per l'acquisto dei prodotti, compreso di gestione carrello e checkout per Proof of Concept\ped{G}, incremento della documentazione & R1F4 \newline R1F5 \\
\textbf{Incremento 4} & Implementazione funzionalità riservate agli admin, incremento della documentazione & R1V4 \\

\rowcolor{azzurro1}
\multicolumn{3}{c}{\textbf{Fase di progettazione di dettaglio e codifica}}\\
\textbf{Incremento 5} & Incremento della documentazione, preparazione attività di progettazione e codifica, approfondimento personale per requisiti non visti dal Proof of Concept\ped{G} & Nessun requisito da implementare \\
\textbf{Incremento 6} & Implementazione autenticazione del sito e pagina del profilo, incremento della documentazione (con correzione in base a segnalazioni dei committenti), inizio stesura manuali e allegato tecnico & R1F1 \newline R1F2 \newline R1F3 \newline R1F6 \\
\textbf{Incremento 7} & Implementazione della visualizzazione prodotti (pagine di listino) e dashboard venditore, incremento della documentazione, correzione dei documenti in base a segnalazioni dei committenti & R1F7 \newline R1F8 \newline R1F9 \newline R1F10 \\
\textbf{Incremento 8} & Implementazione dell'acquisto dei prodotti, compreso di gestione carrello e checkout, incremento della documentazione & R1F4 \newline R1F5\\
\textbf{Incremento 9} & Incremento della documentazione, preparazione alla presentazione della Product baseline\ped{G} & Nessun requisito da implementare \\
\textbf{Incremento 10} & Incremento del codice con correzioni ricevute dalla Product baseline\ped{G} e dal proponente, incremento della documentazione, preparazione alla Revisione di Qualifica & Nessun requisito da implementare \\

\rowcolor{azzurro1}
\multicolumn{3}{c}{\textbf{Fase di validazione e collaudo}}\\
\textbf{Incremento 11} & Implementazione di feedback all'utente e visualizzazione di errori, implementazione requisiti obbligatori mancanti, incremento della documentazione & R2F6.2.1 \newline R1F6.4 \\
\textbf{Incremento 12} & Implementazione di test mancanti, eventuale correzione dei documenti in base a segnalazioni dei committenti, incremento della documentazione & Nessun requisito da implementare \\
\textbf{Incremento 13} & Preparazione all'esposizione della Revisione di Accettazione, collaudo finale con il proponente, incremento della documentazione & Nessun requisito da implementare \\



\rowcolor{white}
\caption{Lista di incrementi individuati}

\end{longtable}
}
