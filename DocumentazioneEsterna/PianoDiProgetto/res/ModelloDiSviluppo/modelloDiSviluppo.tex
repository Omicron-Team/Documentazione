\section{Modello di sviluppo}
Come modello di sviluppo si è deciso di adottare il \textbf{modello incrementale}, al fine di garantire qualità e conformità del prodotto.

\subsection{Modello incrementale}
Nel modello di sviluppo incrementale si suddivide lo sviluppo del prodotto in incrementi. Ogni incremento porta a nuove funzionalità rispetto al precedente riducendo, inoltre, il rischio di fallimento. Questo ciclo viene ripetuto finché non si ha un numero di requisiti soddisfatti sufficiente.\\
Eventuali modifiche, aggiunte o rimozione dei requisiti non sono concesse durante la fase di sviluppo dell'incremento corrente e devono venire consentite dal proponente.\\
Le motivazioni che ci hanno spinto ad adottare il modello incrementale sono i suoi seguenti vantaggi:
\begin{itemize}
\item alla fine di ogni incremento viene effettuata una fase di verifica. Così facendo eventuali errori sono vincolati nell'incremento corrente;
\item eventuali modifiche sono limitate al singolo incremento, rendendole più economiche dal punto di vista del tempo di codifica;
\item alla fine di ogni incremento vi è la possibilità di avere un riscontro con il proponente, il quale può valutare le funzionalità aggiunte e le modifiche effettuate;
\item si dà priorità allo sviluppo delle funzionalità primarie, cosicché il proponente possa subito valutarle;
\item risulta un ottimo modello da utilizzare nel nostro progetto, in quanto fin dall'inizio abbiamo i requisiti che devono venire soddisfatti.
\end{itemize}

