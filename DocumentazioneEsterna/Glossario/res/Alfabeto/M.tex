\section{M}

\TermineGlossario{Machine learning}
\DefinizioneGlossario{L'apprendimento automatico è strettamente legato al riconoscimento di pattern e alla teoria computazionale dell'apprendimento ed esplora lo studio e la costruzione di algoritmi che possano apprendere da un insieme di dati e fare delle predizioni su questi, costruendo in modo induttivo un modello basato su dei campioni.}
\\
\TermineGlossario{Message Queue Telemetry Transport (MQTT)}
\DefinizioneGlossario{Protocollo ISO standard (ISO/IEC PRF 20922) di messaggistica leggero di tipo publish/subscribe posizionato in cima a TCP/IP. È stato progettato per le situazioni in cui è richiesto un basso impatto e dove la banda è limitata.}
\\
\TermineGlossario{Metrica}
\DefinizioneGlossario{Standard per la misura di alcune proprietà del software o delle sue specifiche.}
\\
\TermineGlossario{Microservizi}
\DefinizioneGlossario{I microservizi sono un approccio architetturale alla realizzazione di applicazioni. Quello che distingue l'architettura basata su microservizi dagli approcci monolitici tradizionali è la suddivisione dell'app nelle sue funzioni di base. Ciascuna funzione, denominata servizio, può essere compilata e implementata in modo indipendente.}
\\
\TermineGlossario{MikTex}
\DefinizioneGlossario{Distribuzione gratuita di TeX/LaTeX per Microsoft Windows.}
\\
\TermineGlossario{Milestone}
\DefinizioneGlossario{Termine inglese che si traduce in 'pietra miliare'. Nelle discipline tecniche indica il raggiungimento di obiettivi stabiliti in precedenza. Una milestone molto spesso è rappresentata da un evento importante di durata zero o un giorno, come il termine di consegna di un progetto.}
\\
\TermineGlossario{Model-View-Controller (MVC)}
\DefinizioneGlossario{Pattern architetturale di programmazione che permette di separare la logica di presentazione dei dati(View) dalla logica di business(Model).}
\\
\TermineGlossario{Modulo}
\DefinizioneGlossario{File o porzione di codice sorgente che contiene istruzioni scritte per essere riutilizzate più volte nello stesso programma o in più programmi diversi.}
\\
\TermineGlossario{Motion planning}
\DefinizioneGlossario{Problema computazionale che consiste nel trovare una sequenza di movimenti validi che permettono di spostare un oggetto da un punto iniziale a una destinazione finale.}
\\
\TermineGlossario{Multiplayer}
\DefinizioneGlossario{Dall'inglese multigiocatore, nell'ambito dei videogiochi è la modalità di utilizzo in cui più persone partecipano al gioco nello stesso tempo.}
\\
