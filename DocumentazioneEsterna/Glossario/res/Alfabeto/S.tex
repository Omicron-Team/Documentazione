\section{S}
\TermineGlossario{Scalabile}
\DefinizioneGlossario{Nel campo dell'informatica, scalabilità denota la capacità di un sistema di aumentare o diminuire di scala in funzione delle necessità e disponibilità.}
\\
\TermineGlossario{Scikit-learn}
\DefinizioneGlossario{Libreria open-source di apprendimento automatico per il linguaggio di programmazione Python.}
\\
\TermineGlossario{Script}
\DefinizioneGlossario{Il termine script, in informatica, designa un tipo particolare di programma, scritto in una particolare classe di linguaggi di programmazione, detti linguaggi di scripting.}
\\
\TermineGlossario{Sentry}
\DefinizioneGlossario{Servizio di monitoraggio di applicazioni. Si specializza nella diagnosi diretta dal codice in esecuzione, come per esempio reporting di condizioni di errore. Tramite questo servizio è possibile quindi unire la reportistica proviente da piattaforme diverse e standardizzarla.}
\\
\TermineGlossario{Server-side rendering}
\DefinizioneGlossario{Processo di rendering di pagine HTML e CSS a partire dal server, e non dal client. Il server renderizza la pagina ad ogni richiesta di un client.}
\\
\TermineGlossario{Serverless (architettura)}
\DefinizioneGlossario{Architettura di rete che sfrutta la dislocazione tra i vari utenti che utilizzano la rete stessa piuttosto di una sua centralizzazione. In questo modo il lavoro di gestione del network viene eseguito dagli stessi utilizzatori.}
\\
\TermineGlossario{Serverless (framework)}
\DefinizioneGlossario{Framework Web gratuito e open-source scritto utilizzando Node.js. Serverless è il primo framework sviluppato per la creazione di applicazioni su AWS Lambda.}
\\
\TermineGlossario{Singleplayer}
\DefinizioneGlossario{Dall'inglese significa giocatore singolo, nell'ambito dei videogiochi è la modalità di gioco in cui una sola persona prende parte al gioco per tutta la durata della partita.}
\\
\TermineGlossario{Slack}
\DefinizioneGlossario{Software che rientra nella categoria degli strumenti di collaborazione aziendale utilizzato per inviare messaggi in modo istantaneo ai membri del team.}
\\
\TermineGlossario{Staging area}
\DefinizioneGlossario{L'area di stage rappresenta una sorta di entità intermedia tra la directory di lavoro e la directory del DVCS; una volta eseguite le operazioni relative allo staging, sarà possibile effettuare il commit delle modifiche apportate al proprio progetto.}
\\
\TermineGlossario{Standard}
\DefinizioneGlossario{Insieme di norme, raccomandazioni o specifiche puramente convenzionali, prestabilite da un'autorità e riconosciute tali con lo scopo di rappresentare una base di riferimento per la realizzazione di tecnologie fra loro compatibili e interoperabili.}
\\
\TermineGlossario{Static generation}
\DefinizioneGlossario{Processo di rendering di pagine HTML e CSS a partire dal server, e non dal client. Il server renderizza le pagine prima delle richieste, per esempio nel momento in cui viene eseguita la build. Le pagine HTML già renderizzate vengono salvate in cache e sono pronte per essere spedite al momento delle richieste.}
\\
\TermineGlossario{Stream}
\DefinizioneGlossario{Canale tra la sorgente e la destinazione attraverso il quale fluiscono i dati.}
\\
\TermineGlossario{Stripe}
\DefinizioneGlossario{Azienda statunitense che fornisce un'infrastruttura software che permette a privati e aziende di inviare e ricevere pagamenti via internet.}
\\
\TermineGlossario{Structured Query Language (SQL)}
\DefinizioneGlossario{Linguaggio standardizzato per database basati sul modello relazionale (RDBMS). Permette di creare e modificare database, oltre a gestire i dati al loro interno.}
\\
\TermineGlossario{Swift}
\DefinizioneGlossario{Linguaggio di programmazione object-oriented per sistemi macOS, iOS, watchOS, tvOS e Linux.}
\\
