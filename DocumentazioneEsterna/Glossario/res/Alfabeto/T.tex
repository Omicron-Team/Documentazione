\section{T}
\TermineGlossario{\TeX}
\DefinizioneGlossario{\TeX{} è un programma di tipografia digitale, adatto alla stesura di testi scientifici e matematici.}
\\
\TermineGlossario{Technology baseline}
\DefinizioneGlossario{Baseline che motiva le tecnologie, i framework e le librerie utilizzate per la realizzazione di un progetto. Tramite il Proof of Concept ne dimostra l'adeguatezza e la fattibilità.}
\\
\TermineGlossario{Telegram}
\DefinizioneGlossario{Software di messaggistica instantanea. Permette di creare chat private e di gruppo ed è disponibile su dispositivi mobili, fissi e direttamente nel web.}
\\
\TermineGlossario{Template}
\DefinizioneGlossario{Il termine inglese template in informatica indica un documento o programma nel quale, come in un foglio semicompilato cartaceo, su una struttura generica o standard esistono spazi temporaneamente bianchi da riempire successivamente. In questo ambito, la parola in italiano è traducibile come ``modello'', ``semicompilato'', ``schema'', ``struttura base'', ``ossatura generale'' o ``scheletro'', o più correntemente ``modulo'', anche se di solito non così elaborato e sofisticato.}
\\
\TermineGlossario{TensorFlow}
\DefinizioneGlossario{Libreria software open-source per l'apprendimento automatico.}
\\
\TermineGlossario{TexMaker}
\DefinizioneGlossario{Editor open-source per LaTeX, con incluso un visualizzatore di file PDF per migliorare la scrittura di documenti.}
\\
\TermineGlossario{Tomcat}
\DefinizioneGlossario{Server web open-source che permette l'esecuzione di applicazioni web sviluppate in Java.}
\\
\TermineGlossario{Typescript}
\DefinizioneGlossario{Linguaggio di programmazione open-source sviluppato da Microsoft. E' stato progettato per lo sviluppo di grandi applicazioni ed è destinato a essere compilato in JavaScript per poter essere interpretato da qualunque web browser o app.}
\\
