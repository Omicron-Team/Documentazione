\section{G}

\TermineGlossario{Git}
\DefinizioneGlossario{Git è un software di controllo di versione distribuito utilizzabile da interfaccia a riga di comando, creato da Linus Torvalds nel 2005.}
\\
\TermineGlossario{GitHub}
\DefinizioneGlossario{Servizio di hosting per progetti software. Il nome deriva dal fatto che è una implementazione dello strumento di controllo versione distribuito Git.}
\\
\TermineGlossario{GitHub Actions}
\DefinizioneGlossario{Github Actions è una feature di Github che permette di creare dei veri e propri workflow di CI/CD (continuous integration e continuos delivery) all'interno di ogni repository.}
\\
\TermineGlossario{GitLab}
\DefinizioneGlossario{Piattaforma web open source, appartenente a GitLab Inc., che permette la gestione di repository Git.}
\\
\TermineGlossario{Google Meet}
\DefinizioneGlossario{Applicazione di teleconferenza sviluppata da Google.}
\\
