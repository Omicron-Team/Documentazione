%% INSERIRE QUI IL NOME DEL DOCUMENTO (INSERITE SEMPRE UNO SPAZIO ALLA FINE DEL NOME)
\newcommand{\doctitle}{Glossario }

%% INSERIRE QUI LA VERSIONE ATTUALE DEL DOCUMENTO (INSERITE SEMPRE UNO SPAZIO ALLA FINE DELLA VERSIONE)
\newcommand{\versiondoc}{1.0.0 }

%%INSERITE QUI LA DATA DI COMPILAZIONE FINALE DEL DOCUMENTO
\newcommand{\datared}{2021-01-10}

%%INSERIRE QUI IL/I REDATTORI
\newcommand{\redattore}{
\GB\\
\MDI
\vspace{0.1cm}}

%%INSERIRE IL/I NOME DEI VERIFICATORI CHE HANNO VERIFICATO IL DOCUMENTO
\newcommand{\verificatori}{\FD}

%%INSERIRE IL NOME DI CHI HA APPROVATO IL DOCUMENTO
\newcommand{\approvazione}{\MB}

%%INSERIRE LA TIPOLOGIA DI USO DEL DOCUMENTO [Interno/Esterno]
\newcommand{\usodoc}{Esterno}

%%INSERIRE LA LISTA DI DISTRIBUZIONE DEL DOCUMENTO
\newcommand{\listadistr}{
    \Omicron\\
    \emph{\VT}\\
    \emph{\CR}\\
    \emph{\Proponente}
}

%%INSERIRE IL SOMMARIO DEL DOCUMENTO
\newcommand{\testosommario}{Questo documento si occupa di chiarire una serie di vocaboli meno comuni utilizzati nella documentazione.}

%INSERIRE IL PATH RELATIVO ALL'IMMAGINE IN BASE ALLA CARTELLA DI DOVE CI SI TROVA
\newcommand{\relativePathImg}{../../Utilita/img/}

%INSERIRE UN NUOVO TERMINE DA INSERIRE NEL GLOSSARIO
\newcommand{\TermineGlossario}[1]{\textbf{#1}\\}

%INSERIRE LA DEFINIZIONE DEL NUOVO TERMINE INSERITO NEL GLOSSARIO
\newcommand{\DefinizioneGlossario}[1]{#1\\}