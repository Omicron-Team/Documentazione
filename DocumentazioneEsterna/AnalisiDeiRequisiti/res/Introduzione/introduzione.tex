\section{Introduzione}
\subsection{Scopo del documento}
L'attuale documento ha lo scopo di descrivere in maniera dettagliata i requisiti e i casi d'uso identificati dopo un'attenta analisi del capitolato C2 e degli incontri con il proponente.
\subsection{Scopo del prodotto}
Lo scopo del prodotto \textit{EmporioLambda} (EML) di \textit{Red Babel} consiste nella creazione di una piattaforma E-commerce\ped{G} generica, la quale può essere mostrata come concetto software vendibile a un commerciante in modo che esso abbia la possibilità di utilizzarlo per vendere la propria merce. L'applicativo sarà implementato utilizzando esclusivamente tecnologie serverless\ped{G} e i servizi AWS\ped{G}.
\subsection{Glossario}
Al fine di migliorare la chiarezza del documento ed evitare possibili ambiguità, viene fornito un Glossario contenente i termini più critici scelti dai membri del gruppo, e una loro spiegazione. In questo documento, tali termini verranno indicati con la lettera `G' a pedice della parola.
\subsection{Riferimenti}
\subsubsection{Normativi}
\begin{itemize}
\item \textbf{\NdP}: \NdPv{2.0.0} ;
\item \textbf{Capitolato d'appalto C2 - EmporioLambda}: \\ \url{https://www.math.unipd.it/~tullio/IS-1/2020/Progetto/C2.pdf} ;
\item \textbf{Verbale}: \textit{Verbale esterno 2020-12-11} ;
\item \textbf{Verbale}: \textit{Verbale esterno 2020-12-22} ;
\item \textbf{Verbale}: \textit{Verbale esterno 2021-01-05} .
\end{itemize}
\subsubsection{Informativi}
\begin{itemize}
\item \textbf{\SdF}: \SdFv{1.0.0} ;
\item \textbf{Capitolato d'appalto C2 - EmporioLambda}: \\ \url{https://www.math.unipd.it/~tullio/IS-1/2020/Progetto/C2.pdf} .
\end{itemize}