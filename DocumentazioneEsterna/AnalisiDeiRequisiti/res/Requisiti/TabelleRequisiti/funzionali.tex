\subsection{Requisiti funzionali} 


{

\rowcolors{2}{azzurro2}{azzurro3}

\centering
\renewcommand{\arraystretch}{2}
\begin{longtable}{C{2.5cm} C{2.8cm} C{6.7cm} C{2.5cm}}
\caption{Tabella dei Requisiti funzionali}\\
\rowcolor{azzurro1}
\textbf{Codice} &
\textbf{Classificazione}&
\textbf{Descrizione}&
\textbf{Fonti}\\
\endfirsthead
\rowcolor{white}
\caption*{Tabella dei Requisiti funzionali (continuazione)}\\
\rowcolor{azzurro1}
\textbf{Codice} &
\textbf{Classificazione}&
\textbf{Descrizione}&
\textbf{Fonti}\\
\endhead

%REGISTRAZIONE

R1F1 & Obbligatorio & Un utente non autenticato può registrarsi & Capitolato\newline UC1, UC1.2 \\
R1F1.1 & Obbligatorio & La registrazione necessita l'inserimento dei seguenti dati personali: nome, cognome e indirizzo di fatturazione & Interno\newline UC1.1, UC1.1.1, UC1.1.2, UC1.1.3 \\
R1F1.2 & Obbligatorio & La registrazione necessita l'inserimento della email & VE\_2020-12-22 \newline UC1.1.4 \\
R1F1.3 & Obbligatorio & La registrazione necessita l'inserimento di una password & Interno\newline UC1.1.5 \\
R2F1.4 & Desiderabile & Il sistema deve mostrare un errore se i campi inseriti nella registrazione non sono validi & Interno \newline UC2 \\

%LOGIN

R1F2 & Obbligatorio & Un utente non autenticato può effettuare il login & Capitolato\newline UC3, UC3.3 \\
R1F2.1 & Obbligatorio & Il login per il cliente e per il venditore sono distinti & VE\_2021-01-05 \newline UC3.1 \\
R1F2.2 & Obbligatorio & Il login cliente necessita l'inserimento della email & Interno \newline UC3.2, UC3.2.2, UC3.2.3 \\
R1F2.3 & Obbligatorio & Il login cliente e il login venditore necessita l'inserimento della password & Interno \newline UC3.2, UC3.2.1, UC3.2.3, UC3.2.4  \\
R2F2.4 & Desiderabile & Il sistema deve mostrare un errore se le credenziali del login sono errate & Interno \newline UC4 \\

%LOGOUT

R1F3 & Obbligatorio & Un utente autenticato può effettuare il logout & Interno \newline UC5 \\

%CARRELLO

R1F4 & Obbligatorio & Un utente generico può accedere alla pagina del carrello dalla Homepage, dalla PLP\ped{G} e dalla PDP\ped{G} & Capitolato \newline UC6 \\
R1F4.1 & Obbligatorio & Un utente generico nel carrello può visualizzare tutti i prodotti che ha precedentemente aggiunto al carrello & Capitolato \newline UC6.1 \\
R1F4.1.1 & Obbligatorio & Un utente generico può visualizzare per ogni prodotto del carrello il nome, l'immagine e la quantità & Interno \newline UC6.1 \\
R1F4.2 & Obbligatorio & Un utente generico può rimuovere i singoli prodotti dal carrello & Interno \newline UC6.2 \\
R1F4.3 & Obbligatorio & Un utente generico può modificare le quantità di ogni prodotto nel carrello & Interno \newline UC6.3\\
R1F4.4 & Obbligatorio & Un utente generico può visualizzare il costo totale dei prodotti nel carrello & Capitolato \newline UC6.4\\
R1F4.4.1 & Obbligatorio & Un utente generico può visualizzare il costo di ogni voce del carrello & Interno \newline UC6.1\\
R1F4.5 & Obbligatorio & Un utente generico può visualizzare le tasse applicate al costo totale dei prodotti nel carrello & Capitolato \newline UC6.5\\

%CHECKOUT

R1F5 & Obbligatorio & Un cliente registrato può procedere al checkout dei prodotti presenti nel carrello & Capitolato \newline UC7, UC7.1\\
R1F5.1 & Obbligatorio & Per iniziare il processo di checkout il cliente deve avere almeno un prodotto nel carrello & Interno \newline UC7.2\\
R1F5.2 & Obbligatorio & Nel processo di checkout il cliente deve poter visualizzare l'email a cui verranno mandati i prodotti & VE\_2021-01-05 \newline UC7.3.1 \\
R2F5.2.1 & Desiderabile & Nel processo di checkout il cliente deve poter modificare l'email a cui verranno mandati i prodotti & VE\_2021-01-05 \newline UC7.3, UC7.3.2 \\
R1F5.3 & Obbligatorio & Un cliente che ha iniziato il processo di checkout deve inserire i dati del pagamento tramite Stripe\ped{G} & Capitolato \newline UC7.3, UC7.3.3, UC7.3.3.1, UC7.3.3.2, UC7.3.3.3, UC7.3.3.4, UC7.3.3.5\\
R1F5.4 & Obbligatorio & Un cliente, dopo aver inserito i dati del pagamento, può continuare al pagamento effettivo tramite Stripe\ped{G} & Capitolato \newline UC7.4\\
R1F5.5 & Obbligatorio & A pagamento riuscito il cliente può visualizzare un riepilogo dell'ordine effettuato & Capitolato \newline UC7.7\\
R1F5.6 & Obbligatorio & A pagamento riuscito il cliente riceve i prodotti acquistati tramite l'email usata per l'acquisto & Capitolato e VE\_2020-12-22 \\
R1F5.7 & Obbligatorio & A pagamento fallito il cliente visualizza un messaggio di errore, e può riprovare il pagamento verificando i dati inseriti & Capitolato \newline UC7.5, UC7.6\\

%PROFILO

R1F6 & Obbligatorio & Un cliente può visualizzare il suo profilo & Capitolato \newline UC8 \\
R1F6.1 & Obbligatorio & Un cliente può visualizzare le seguenti informazioni nel suo profilo: nome, cognome, indirizzo di fatturazione e indirizzo email & Interno \newline UC8.1\\
R1F6.2 & Obbligatorio & Un cliente può aggiornare le seguenti informazioni nel suo profilo: nome, cognome, password, indirizzo di fatturazione e indirizzo email & VE\_2020-12-22 \newline UC8.2, UC8.2.1, UC8.2.2, UC8.2.3, UC8.2.4, UC8.2.5, UC8.2.6 \\
R2F6.2.1 & Desiderabile & Se la modifica non va a buon fine, viene mostrato un errore & Interno UC8.2.7\\
R1F6.3 & Obbligatorio & Un cliente può visualizzare nel suo profilo la lista degli ordini che ha effettuato & Capitolato \newline UC8.3\\
R1F6.3.1 & Obbligatorio & Per ogni ordine effettuato si deve visualizzare i seguenti dati: id ordine, prodotti acquistati, quantità prodotti acquistati, costo per singola voce, costo totale, tasse applicate, data acquisto & Interno \newline UC8.3.1\\
R1F6.4 & Obbligatorio & Un cliente può eliminare il suo account & Capitolato \newline UC8.4\\

%MERCHANT DASHBOARD

R1F7 & Obbligatorio & Il venditore può accedere ad una dashboard\ped{G} per gestire il proprio catalogo digitale & Capitolato \newline UC12\\
R1F7.1 & Obbligatorio & Il venditore può inserire nuovi prodotti nella dashboard\ped{G} & Capitolato \newline UC12.1, UC12.1.1, UC12.1.1.6, UC12.1.1.7\\
R1F7.1.1 & Obbligatorio & Il venditore, nell'inserimento di nuovi prodotti, deve assegnare al prodotto i seguenti dati: nome, descrizione e immagine & Capitolato \newline UC12.1.1.1, UC12.1.1.2, UC12.1.1.4\\
R1F7.1.2 & Obbligatorio & Il venditore, nell'inserimento di nuovi prodotti, deve assegnare al prodotto un prezzo & Interno \newline UC12.1.1.3\\ 
R1F7.1.3 & Obbligatorio & Il venditore, nell'inserimento di nuovi prodotti, deve assegnare al prodotto una categoria & VE\_2021-01-05 \newline UC12.1.1.5\\ 
R1F7.2 & Obbligatorio & Il venditore può vedere tutti i prodotti da lui venduti & Capitolato \newline UC12.1.2 \\
R1F7.2.1 & Obbligatorio & Il venditore può vedere nome, immagine, descrizione, prezzo e categoria di tutti i prodotti da lui venduti & Interno \newline UC12.1.2\\
R1F7.3 & Obbligatorio & Il venditore può modificare alcuni dati di tutti i prodotti da lui venduti & Capitolato \newline UC12.1.3, UC12.1.3.6, UC12.1.3.7 \\
R1F7.3.1 & Obbligatorio & Il venditore può modificare la descrizione di tutti i prodotti da lui venduti & Capitolato \newline UC12.1.3.2\\
R2F7.3.2 & Desiderabile & Il venditore può modificare il prezzo, il nome, l'immagine e la categoria di tutti i prodotti da lui venduti & Interno \newline UC12.1.3.1, UC12.1.3.3, UC12.1.3.4, UC12.1.3.5\\
R1F7.4 & Obbligatorio & Il venditore può eliminare i prodotti da lui venduti & Capitolato \newline UC12.1.4 \\
R1F7.5 & Obbligatorio & Il venditore può visualizzare i dettagli di tutti gli ordini effettuati dai clienti & Capitolato \newline UC12.3\\
R1F7.5.1 & Obbligatorio & Il venditore può visualizzare per ogni ordine effettuato i seguenti dati: numero ordine, prodotti acquistati, quantità prodotti acquistati, costo per singola voce, costo totale, tasse applicate, data acquisto & Interno \newline UC12.3\\
R2F7.6 & Desiderabile & Il venditore dalla dashboard\ped{G} può accedere agli strumenti esterni riservati per gli admin & VE\_2020-12-22 \newline UC12.4\\
R1F7.7 & Obbligatorio & Il venditore può gestire le categorie dei prodotti & VE\_2021-01-05 \newline UC12.2\\
R1F7.7.1 & Obbligatorio & Il venditore può visualizzare una lista di categorie di prodotti & VE\_2021-01-05 \newline UC12.2.1\\
R1F7.7.2 & Obbligatorio & Il venditore può inserire nuove categorie di prodotti & VE\_2021-01-05 \newline UC12.2.2, UC12.2.2.1, UC12.2.2.2, UC12.2.2.3\\
R1F7.7.3 & Obbligatorio & Il venditore può eliminare categorie esistenti di prodotti & Interno \newline UC12.2.3\\

%RESTO

R1F8 & Obbligatorio & Un utente generico può effettuare una ricerca tra i prodotti in vendita & Capitolato \newline UC9, UC10\\

%PLP

R1F9 & Obbligatorio & Un utente generico può accedere alla PLP\ped{G} corrispondente ad una categoria di prodotti & Capitolato \newline UC11 \\
R1F9.1 & Obbligatorio & Un utente generico nella PLP\ped{G} può visualizzare un insieme di prodotti listati & Capitolato \newline UC11.1\\
R1F9.1.1 & Obbligatorio & Per ogni prodotto listato nella PLP\ped{G} deve venir mostrato un nome, una immagine e un prezzo & Interno \newline UC11.1\\
R1F9.2 & Obbligatorio & Un utente generico può selezionare alcuni prodotti listati & Capitolato \newline UC11.2\\
R1F9.2.1 & Obbligatorio & Un utente generico può aggiungere al carrello i prodotti selezionati & Capitolato \newline UC11.3\\
R2F9.3 & Desiderabile & Un utente generico può filtrare, tramite il range di prezzo, l'insieme dei prodotti & Interno \newline UC11.4, UC11.5\\

%PDP

R1F10 & Obbligatorio & Un utente generico può accedere alla PDP\ped{G} cliccando un prodotto nella PLP\ped{G} & Capitolato \newline UC11.6\\
R1F10.1 & Obbligatorio & Un utente generico può visualizzare le seguenti informazioni relative al prodotto: nome, immagine, descrizione, prezzo e tasse applicate & Capitolato \newline UC11.6.1\\
R1F10.2 & Obbligatorio & Un utente generico può aggiungere il prodotto al carrello & Capitolato \newline UC11.6.3\\
R1F10.2.1 & Obbligatorio & Un utente generico può aggiungere il prodotto al carrello, scegliendo la quantità & Interno \newline UC11.6.2 \\

%ADMIN

R1F11 & Obbligatorio & L'amministratore può eseguire il deploy\ped{G} dell'applicazione & Capitolato \newline UC13\\
R1F12 & Obbligatorio & L'amministratore può gestire e configurare le integrazioni di terze parti dell'applicazione & Capitolato \newline UC14\\

\end{longtable}

}