\section{Descrizione generale}
\subsection{Contesto d'uso del prodotto}
Il progetto \textit{EmporioLambda} si pone come obiettivo finale lo sviluppo di un'applicazione Web\ped{G} che possa offrire tutte le funzionalità necessarie ai clienti per navigare e comprare, attraverso il servizio di pagamento esterno Stripe\ped{G}, i prodotti contenuti nel catalogo digitale, il quale sarà gestibile da appositi strumenti disponibili al venditore. Sarà data inoltre la possibilità all'amministratore (chiamato anche admin) della piattaforma di distribuire l'applicativo nel cloud\ped{G} e di configurare al meglio le integrazioni dei servizi esterni.
\subsection{Funzioni del prodotto}
\begin{itemize}
\item \textbf{Clienti:} l'applicazione Web\ped{G} deve fornire le seguenti funzionalità ai clienti:
	\begin{itemize}
	\item autenticazione e registrazione alla piattaforma;
	\item effettuare una ricerca tra i prodotti presenti nel catalogo digitale;
	\item visualizzare un insieme di prodotti listati sulla base della loro categoria, con eventualmente la possibilità di filtraggio attraverso opportuni parametri;
	\item visualizzare singoli prodotti con la possibilità di valutarne tutte le caratteristiche;
	\item aggiungere i prodotti al carrello;
	\item accedere al carrello con la possibilità di visualizzare i prodotti precedentemente inseriti e il relativo prezzo, modificarne la quantità, eliminarli o proseguire con il checkout;
	\item effettuare il checkout visualizzando inizialmente un sommario dei prodotti da acquistare e il prezzo effettivo da pagare, proseguendo poi con il versamento attraverso i servizi Stripe\ped{G}. In seguito ad un esito positivo della corresponsione il cliente deve poter visualizzare il riepilogo dell'ordine e ricevere i prodotti tramite l'email personale, altrimenti deve poter visualizzare un messaggio di errore, avendo inoltre la possibilità di ritentare l'acquisto ricontrollando i dati precedentemente inseriti;
	\item accedere, se il cliente è autenticato, al proprio profilo per visualizzare, ed eventualmente modificare, le informazioni personali. Deve essere possibile inoltre visualizzare i dettagli degli ordini effettuati ed eliminare il proprio profilo dalla piattaforma.
	\end{itemize}
\item \textbf{Venditore:} l'applicazione Web\ped{G} deve fornire le seguenti funzionalità al venditore:
	\begin{itemize}
	\item accedere alla dashboard\ped{G} per la gestione dei prodotti presenti nel catalogo digitale;
	\item visualizzare, modificare o eliminare tutti i prodotti in vendita;
	\item visualizzare i dettagli di tutti gli ordini effettuati dai clienti;
	\item accedere agli strumenti esterni per la gestione della piattaforma riservati agli amministratori.
	\end{itemize}
\item \textbf{Amministratore (admin):} utilizzando gli strumenti presenti sul sistema di monitoraggio l'amministratore ha a disposizione le seguenti funzionalità:
	\begin{itemize}
	\item distribuire l'applicativo nel cloud\ped{G} di Amazon\ped{G};
	\item monitorare lo stato dell'applicativo;
	\item gestire la configurazione dei servizi esterni.
	\end{itemize}
\end{itemize}
\subsection{Moduli ad alto livello dell'architettura}
L'architettura di \textit{EmporioLambda} è costituita da quattro moduli software ad alto livello.
\subsubsection{EmporioLambda-frontend (EML-FE)}
Modulo che, oltre ad offrire all'utente le pagine web del sito accessibili attraverso qualsiasi browser, deve agire anche come Backend for Frontend (BFF), ovvero come strato intermedio tra l'user experience\ped{G} e le risorse chiamate.
\subsubsection{EmporioLambda-backend (EML-BE)}
Il backend\ped{G} consiste nel modulo che espone i servizi dell'applicativo ed è responsabile dell'implementazione della logica di business, gestione dei dati della piattaforma e dello stato del carrello, integrazione dei servizi esterni (come il servizio di pagamento).
\subsubsection{EmporioLambda-integration (EML-I)}
Modulo che rappresenta l'insieme dei servizi esterni integrati nella piattaforma attraverso il EML-BE.
\subsubsection{EmporioLambda-monitoring (EML-MON)}
Modulo che comprende tutti gli strumenti, forniti dal sistema di monitoraggio, utilizzati dall'amministratore per il monitoraggio, la distribuzione e la configurazione dei servizi esterni della piattaforma.
\subsection{Vincoli generali}
Un utente, per usufruire dei servizi offerti dalla piattaforma, deve avere a disposizione un browser web con Javascript\ped{G} abilitato ed una connessione ad Internet. L'amministratore dell'applicativo, inoltre, per poter utilizzare gli strumenti di monitoraggio e distribuzione deve essere in possesso di un account AWS\ped{G}.