\section{Casi d'uso}
\subsection{Attori dei casi d'uso}
Primari: Utente generico, utente autenticato, utente non autenticato, cliente, venditore, amministratore \\
Secondari: Cognito, CloudWatch, AWS Lambda, ...
\subsection{Elenco dei casi d'uso}
In questa sezione vi sono elencati tutti i casi d'uso (UC) individuati. Ogni UC rappresenta uno scenario per uno o più attori e viene descritto tramite diagrammi dei casi d'uso. Ogni UC, infine, possiede una precondizione e una postcondizione. \\
\subsubsection{UC - Caso d'uso generico}
Figura \\
\begin{itemize}
\item \textbf{Attori primari}:
\item \textbf{Attori secondari}:
\item \textbf{Descrizione}:
\item \textbf{Scenario Principale}:
\item \textbf{Estensioni}:
\item \textbf{Specializzazioni}:
\item \textbf{Precondizione}:
\item \textbf{Postcondizione}:
\end{itemize}

\subsubsection{UCX - Login}
Figura \\
\begin{itemize}
\item \textbf{Attori primari}: utente non autenticato;
\item \textbf{Descrizione}: l'utente, inserendo le proprie credenziali, viene autenticato alla piattaforma;
\item \textbf{Scenario Principale}: l'utente non ancora autenticato richiede l'accesso alla piattaforma dopo aver inserito la propria email e password negli appositi campi dedicati al login;
\item \textbf{Estensioni}:
\begin{itemize}
\item \textbf{UCX}: se le credenziali inserite non vengono riconosciute, viene visualizzato un messaggio di errore e l'utente viene indirizzato alle pagine di registrazione e di recupero password;
\end{itemize}
\item \textbf{Precondizione}: l'utente prova ad autenticarsi alla piattaforma;
\item \textbf{Postcondizione}: l'utente viene autenticato ed identificato come cliente o venditore/amministratore.
\end{itemize}

\subsubsection{UCX - Visualizzazione errore dati login non errati}
Figura \\
\begin{itemize}
\item \textbf{Attori primari}: utente non autenticato;
\item \textbf{Descrizione}: l'utente visualizza un messaggio di errore che lo informa che i dati da lui inseriti durante il login non sono riconosciuti dal sistema, indirizzandolo verso le pagine di registrazione e recupero password;
\item \textbf{Scenario Principale}: l'utente tenta di effettuare il login usando credenziali non presenti nel sistema;
\item \textbf{Precondizione}: l'utente prova ad autenticarsi alla piattaforma;
\item \textbf{Postcondizione}: viene visualizzato un messaggio che informa l'utente dell'errore di riconoscimento delle credenziali, e che lo indirizza alle pagine di registrazione e recupero password.
\end{itemize}

\subsubsection{UCX - Logout}
Figura \\
\begin{itemize}
\item \textbf{Attori primari}: utente autenticato;
\item \textbf{Descrizione}: l'utente viene sloggato dalla piattaforma;
\item \textbf{Scenario Principale}: l'utente richiede il logout tramite il bottone dedicato;
\item \textbf{Precondizione}: l'utente ha precedentemente effettuato il login ed è attualmente autenticato;
\item \textbf{Postcondizione}: l'utente non è più autenticato.
\end{itemize}

\subsubsection{UCX - Acquisto prodotti}
Figura \\
\begin{itemize}
\item \textbf{Attori primari}: cliente;
\item \textbf{Attori secondari}: Stripe;
\item \textbf{Descrizione}: il cliente può acquistare i prodotti presenti nel suo carrello.
\item \textbf{Scenario Principale}: 
\begin{enumerate}
	\item il cliente inizia il checkout dei prodotti nel carrello [UCX];
	\item il cliente inserisce l'indirizzo di spedizione e i dati per il pagamento [UCX];
	\item il cliente procede al pagamento [UCX];
	\item il cliente visualizza un riepilogo dell'ordine effettuato;
\end{enumerate}
\item \textbf{Precondizione}: l'utente è autenticato come cliente e ha precedentemente inserito dei prodotti nel carrello.
\item \textbf{Postcondizione}: il costo totale dei prodotti acquistati è stato prelevato dal conto specificato dal cliente. È stata inviata un'email al cliente per confermare l'acquisto.
\end{itemize}
