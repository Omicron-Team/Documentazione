\section{Casi d'uso}
\subsection{Attori dei casi d'uso}
\begin{figure}[H]
\centering
\includegraphics[scale=0.6]{res/UseCase/Immagini/DiagrammaAttoriPrimari}
\caption{Diagramma UML\ped{G} per la gerarchia degli attori primari}
\end{figure}
\subsubsection{Attori primari}
\begin{itemize}
\item \textbf{Utente generico:} corrisponde ad un utente generico che accede alla piattaforma attraverso l'applicazione Web\ped{G};
\item \textbf{Utente autenticato:} corrisponde ad un utente che ha effettuato la procedura di login per autenticarsi. L'utente risulta quindi in possesso delle credenziali valide per accedere ad apposite funzionalità;
\item \textbf{Utente non autenticato:} corrisponde ad un utente che non ha ancora effettuato la procedura di login alla piattaforma;
\item \textbf{Cliente:} corrisponde ad un utente che ha la possibilità di acquistare i prodotti presenti nel catalogo digitale;
\item \textbf{Venditore:} corrisponde all'utente che possiede la piattaforma, in quanto ha la possibilità vendere e gestire i propri prodotti presenti nell'applicativo.
\end{itemize}
\subsubsection{Attori secondari}
\begin{itemize}
\item \textbf{Stripe\ped{G}:} infrastruttura software che fornisce i servizi di pagamento necessari agli acquisti nella piattaforma;
\end{itemize}
\subsection{Elenco dei casi d'uso}
In questa sezione vi sono elencati tutti i casi d'uso (UC) individuati. Ogni UC rappresenta uno scenario per uno o più attori e viene descritto tramite diagrammi dei casi d'uso. Ogni UC, infine, possiede una precondizione e una postcondizione. \\

\begin{figure}[H]
\centering
\includegraphics[scale=0.6]{res/UseCase/Immagini/RegistrazioneGenerale}
\caption{Diagramma UML per modulo di registrazione}
\end{figure}

\subsubsection{UCX - Registrazione}
\begin{itemize}
\item \textbf{Attori primari}:
\item \textbf{Attori secondari}:
\item \textbf{Descrizione}:
\item \textbf{Scenario Principale}:
\item \textbf{Estensioni}:
\item \textbf{Specializzazioni}:
\item \textbf{Precondizione}:
\item \textbf{Postcondizione}:
\end{itemize}

\begin{figure}[H]
\centering
\includegraphics[scale=0.6]{res/UseCase/Immagini/Registrazione}
\caption{Diagramma UML per UCX - Registrazione}
\end{figure}

\subsubsection{UCX.1 - Inserimento dati}
\begin{itemize}
\item \textbf{Attori primari}:
\item \textbf{Attori secondari}:
\item \textbf{Descrizione}:
\item \textbf{Scenario Principale}:
\item \textbf{Estensioni}:
\item \textbf{Specializzazioni}:
\item \textbf{Precondizione}:
\item \textbf{Postcondizione}:
\end{itemize}

\begin{figure}[H]
\centering
\includegraphics[scale=0.6]{res/UseCase/Immagini/InserimentoDatiRegistrazione}
\caption{Diagramma UML per UCX.1 - Inserimento dati}
\end{figure}


\subsubsection{UCX.1.1 - Inserimento nome}
\begin{itemize}
\item \textbf{Attori primari}:
\item \textbf{Attori secondari}:
\item \textbf{Descrizione}:
\item \textbf{Scenario Principale}:
\item \textbf{Estensioni}:
\item \textbf{Specializzazioni}:
\item \textbf{Precondizione}:
\item \textbf{Postcondizione}:
\end{itemize}

\subsubsection{UCX.1.2 - Inserimento cognome}
\begin{itemize}
\item \textbf{Attori primari}:
\item \textbf{Attori secondari}:
\item \textbf{Descrizione}:
\item \textbf{Scenario Principale}:
\item \textbf{Estensioni}:
\item \textbf{Specializzazioni}:
\item \textbf{Precondizione}:
\item \textbf{Postcondizione}:
\end{itemize}

\subsubsection{UCX.1.3 - Inserimento indirizzo di fatturazione}
\begin{itemize}
\item \textbf{Attori primari}:
\item \textbf{Attori secondari}:
\item \textbf{Descrizione}:
\item \textbf{Scenario Principale}:
\item \textbf{Estensioni}:
\item \textbf{Specializzazioni}:
\item \textbf{Precondizione}:
\item \textbf{Postcondizione}:
\end{itemize}

\subsubsection{UCX.1.4 - Inserimento email}
\begin{itemize}
\item \textbf{Attori primari}:
\item \textbf{Attori secondari}:
\item \textbf{Descrizione}:
\item \textbf{Scenario Principale}:
\item \textbf{Estensioni}:
\item \textbf{Specializzazioni}:
\item \textbf{Precondizione}:
\item \textbf{Postcondizione}:
\end{itemize}

\subsubsection{UCX.1.5 - Inserimento password}
\begin{itemize}
\item \textbf{Attori primari}:
\item \textbf{Attori secondari}:
\item \textbf{Descrizione}:
\item \textbf{Scenario Principale}:
\item \textbf{Estensioni}:
\item \textbf{Specializzazioni}:
\item \textbf{Precondizione}:
\item \textbf{Postcondizione}:
\end{itemize}

\subsubsection{UCX - Visualizzazione errori di registrazione}
\begin{itemize}
\item \textbf{Attori primari}:
\item \textbf{Attori secondari}:
\item \textbf{Descrizione}:
\item \textbf{Scenario Principale}:
\item \textbf{Estensioni}:
\item \textbf{Specializzazioni}:
\item \textbf{Precondizione}:
\item \textbf{Postcondizione}:
\end{itemize}


\begin{figure}[H]
\centering
\includegraphics[scale=0.6]{res/UseCase/Immagini/Login}
\caption{Diagramma UML per modulo di login}
\end{figure}

\subsubsection{UCX - Login}
\begin{itemize}
\item \textbf{Attori primari}: utente non autenticato;
\item \textbf{Attori secondari}: identity manager;
\item \textbf{Descrizione}: l'utente, inserendo le proprie credenziali, viene autenticato alla piattaforma;
\item \textbf{Scenario Principale}: l'utente non ancora autenticato inserisce le proprie credenziali negli appositi campi dati e richiede il login
\item \textbf{Specializzazioni}:
\begin{itemize}
	\item login cliente \textbf{[UCX.1]}
	\item login venditore \textbf{[UCX.2]}
\end{itemize}
\item \textbf{Estensioni}:
\begin{itemize}
	\item \textbf{UCX}: se le credenziali inserite non vengono riconosciute dall'identity manager, viene visualizzato un messaggio che informa l'utente dell'errore;
\end{itemize}
\item \textbf{Precondizione}: l'utente prova ad autenticarsi alla piattaforma;
\item \textbf{Postcondizione}: l'utente viene autenticato.
\end{itemize}

\subsubsection{UCX - Visualizzazione errore dati login errati}
\begin{itemize}
\item \textbf{Attori primari}: utente non autenticato;
\item \textbf{Attori secondari}: identity manager;
\item \textbf{Descrizione}: l'utente visualizza un messaggio di errore che lo informa che i dati da lui inseriti durante il login non sono riconosciuti dall'identity manager;
\item \textbf{Scenario Principale}: l'utente tenta di effettuare il login usando credenziali non presenti nel sistema;
\item \textbf{Precondizione}: l'utente prova ad autenticarsi alla piattaforma;
\item \textbf{Postcondizione}: viene visualizzato un messaggio che informa l'utente dell'errore di riconoscimento delle credenziali.
\end{itemize}

\subsubsection{UCX - Inserimento dati login}
\begin{itemize}
\item \textbf{Attori primari}: utente non autenticato;
\item \textbf{Descrizione}: l'utente compila il form per il login;
\item \textbf{Scenario Principale}: l'utente inserisce i seguenti dati personali:
\begin{itemize}
	\item password \textbf{[UCX.1.1.2]}
\end{itemize}
\item \textbf{Precondizione}: il form di inserimento dati per il login è disponibile;
\item \textbf{Postcondizione}: i dati necessari al login sono stati compilati.
\end{itemize}

\subsubsection{UCX.1 - Inserimento password}
\begin{itemize}
\item \textbf{Attori primari}: utente non autenticato;
\item \textbf{Descrizione}: l'utente deve compilare il campo "Password" per procedere al login;
\item \textbf{Scenario Principale}: l'utente inserisce la sua password nell'apposito campo;
\item \textbf{Precondizione}: il campo "Password" risulta vuoto;
\item \textbf{Postcondizione}: il campo "Password" è stato compilato.
\end{itemize}

\subsubsection{UCX - Richiesta di login}
\begin{itemize}
\item \textbf{Attori primari}: utente non autenticato;
\item \textbf{Descrizione}: l'utente richiede l'autenticazione con i dati inseriti;
\item \textbf{Scenario Principale}: l'utente preme il tasto di login e manda la richiesta al sistema con i dati presenti nel form;
\item \textbf{Precondizione}: l'utente prova ad autenticarsi alla piattaforma;
\item \textbf{Postcondizione}: la richiesta di login è stata mandata al sistema;
\end{itemize}

\subsubsection{UCX.1 - Login cliente}
\begin{itemize}
\item \textbf{Attori primari}: utente non autenticato;
\item \textbf{Attori secondari}: identity manager;
\item \textbf{Descrizione}: l'utente, inserendo le proprie credenziali, viene autenticato alla piattaforma come cliente;
\item \textbf{Scenario Principale}: l'utente non ancora autenticato richiede l'accesso alla piattaforma dopo aver inserito nel form apposito i dati richiesti \textbf{[UCX.1.1]}, premendo il pulsante di login \textbf{[UCX.1.2]}
\item \textbf{Precondizione}: l'utente prova ad autenticarsi alla piattaforma;
\item \textbf{Postcondizione}: l'utente viene autenticato come cliente.
\end{itemize}

\subsubsection{UCX.1.1 - Inserimento dati login cliente}
\begin{itemize}
\item \textbf{Attori primari}: utente non autenticato;
\item \textbf{Descrizione}: l'utente compila il form per il login;
\item \textbf{Scenario Principale}: l'utente inserisce i seguenti dati personali:
\begin{itemize}
	\item email \textbf{[UCX.1.1.1]};
	\item password \textbf{[UCX.1.1.2]}
\end{itemize}
\item \textbf{Precondizione}: il form di inserimento dati per il login è disponibile;
\item \textbf{Postcondizione}: i dati necessari al login sono stati compilati.
\end{itemize}

\subsubsection{UCX.1.1.1 - Inserimento email}
\begin{itemize}
\item \textbf{Attori primari}: utente non autenticato;
\item \textbf{Descrizione}: l'utente deve compilare il campo "Email" per procedere al login;
\item \textbf{Scenario Principale}: l'utente inserisce il suo indirizzo email nell'apposito campo;
\item \textbf{Precondizione}: il campo "Email" risulta vuoto;
\item \textbf{Postcondizione}: il campo "Email" è stato compilato.
\end{itemize}

\subsubsection{UCX.2 - Login venditore}
\begin{itemize}
\item \textbf{Attori primari}: utente non autenticato;
\item \textbf{Attori secondari}: identity manager;
\item \textbf{Descrizione}: l'utente, inserendo le proprie credenziali, viene autenticato alla piattaforma come venditore;
\item \textbf{Scenario Principale}: l'utente non ancora autenticato si trova nella pagina di login dedicata al venditore. Dopo aver inserito nel form apposito i dati richiesti \textbf{[UCX.2.1]}, preme il pulsante di login per autenticarsi \textbf{[UCX.1.2]}
\item \textbf{Precondizione}: l'utente prova ad autenticarsi alla piattaforma nella pagina dedicata al login per il venditore;
\item \textbf{Postcondizione}: l'utente viene autenticato come venditore.
\end{itemize}



\begin{figure}[H]
\centering
\includegraphics[scale=0.6]{res/UseCase/Immagini/Logout}
\caption{Schema generale: modulo di logout}
\end{figure}

\subsubsection{UC5 - Logout}
\begin{itemize}
\item \textbf{Attori primari}: utente autenticato;
\item \textbf{Descrizione}: l'utente viene sloggato dalla piattaforma;
\item \textbf{Scenario Principale}: l'utente richiede il logout tramite il bottone dedicato;
\item \textbf{Precondizione}: l'utente ha precedentemente effettuato il login ed è attualmente autenticato;
\item \textbf{Postcondizione}: l'utente non è più autenticato.
\end{itemize}
\begin{figure}[H]
\centering
\includegraphics[scale=0.6]{res/UseCase/Immagini/CarrelloGenerale}
\caption{Schema generale: modulo del carrello}
\end{figure}

\subsubsection{UC6 - Gestione Carrello}
\begin{itemize}
\item \textbf{Attori primari}: utente generico;
\item \textbf{Descrizione}: per tutti gli utenti è disponibile una pagina del carrello in cui è possibile visualizzare e gestire i prodotti inseriti al suo interno. È successivamente possibile acquistare i prodotti presenti nel carrello;
\item \textbf{Scenario Principale}: un utente generico si trova all'interno della pagina del carrello ed ha a disposizione queste funzionalità:
\begin{itemize}
\item visualizzazione prodotti nel carrello [\textbf{UC6.1}];
\item rimozione prodotto dal carrello [\textbf{UC6.2}];
\item modifica quantità di un prodotto nel carrello [\textbf{UC6.3}];
\item visualizzazione costo totale del carrello [\textbf{UC6.4}];
\item visualizzazione tasse applicate nel carrello [\textbf{UC6.5}].
\end{itemize}
\item \textbf{Precondizione}: l'utente generico si trova nel sito ed ha a disposizione la pagina del carrello;
\item \textbf{Postcondizione}: l'utente ha a disposizione varie funzionalità per gestire il carrello e può procedere all'acquisto dei prodotti al suo interno.
\end{itemize}

\begin{figure}[H]
\centering
\includegraphics[scale=0.6]{res/UseCase/Immagini/GestioneCarrello}
\caption{Diagramma UML\ped{G} per UC6 - Gestione carrello}
\end{figure}

\subsubsection{UC6.1 - Visualizzazione prodotti nel carrello}
\begin{itemize}
\item \textbf{Attori primari}: utente generico;
\item \textbf{Descrizione}: l'utente generico visualizza una lista di prodotti precedentemente inseriti nel carrello;
\item \textbf{Scenario Principale}: l'utente generico si trova nella pagina del carrello e visualizza i seguenti dati per ogni prodotto inserito nel carrello:
\begin{itemize}
\item nome del prodotto;
\item immagine del prodotto;
\item quantità presente nel carrello;
\item costo del prodotto.
\end{itemize}
\item \textbf{Precondizione}: l'utente generico si trova nella pagina del carrello;
\item \textbf{Postcondizione}: viene visualizzata la lista di prodotti presenti nel carrello.
\end{itemize}

\subsubsection{UC6.2 - Rimozione prodotto nel carrello}
\begin{itemize}
\item \textbf{Attori primari}: utente generico;
\item \textbf{Descrizione}: l'utente generico può rimuovere un prodotto precedentemente inserito nel carrello;
\item \textbf{Scenario Principale}: l'utente generico si trova nella pagina del carrello e clicca un tasto dedicato per rimuovere un prodotto dal carrello;
\item \textbf{Precondizione}: l'utente generico si trova nella pagina del carrello e ha almeno un prodotto al suo interno;
\item \textbf{Postcondizione}: il prodotto rimosso non è più presente nel carrello.
\end{itemize}

\subsubsection{UC6.3 - Modifica quantità di un prodotto nel carrello}
\begin{itemize}
\item \textbf{Attori primari}: utente generico;
\item \textbf{Descrizione}: l'utente generico può modificare la quantità di un prodotto precedentemente inserito nel carrello;
\item \textbf{Scenario Principale}: l'utente generico si trova nella pagina del carrello e clicca dei tasti dedicati per modificare la quantità di un prodotto, incrementandola o decrementandola;
\item \textbf{Precondizione}: l'utente generico si trova nella pagina del carrello e ha almeno un prodotto al suo interno;
\item \textbf{Postcondizione}: viene modificata la quantità di un prodotto del carrello in base all'intento dell'utente.
\end{itemize}

\subsubsection{UC6.4 - Visualizzazione costo totale del carrello}
\begin{itemize}
\item \textbf{Attori primari}: utente generico;
\item \textbf{Descrizione}: l'utente generico visualizza il costo totale dei prodotti nel suo carrello;
\item \textbf{Scenario Principale}: l'utente generico si trova nella pagina del carrello e, se ha prodotti al suo interno, visualizza il costo totale di essi;
\item \textbf{Precondizione}: l'utente generico si trova nella pagina del carrello e ha almeno un prodotto al suo interno;
\item \textbf{Postcondizione}: viene visualizzato il costo totale dei prodotti del carrello.
\end{itemize}

\subsubsection{UC6.5 - Visualizzazione tasse applicate nel carrello}
\begin{itemize}
\item \textbf{Attori primari}: utente generico;
\item \textbf{Descrizione}: l'utente generico visualizza le tasse applicate ai prodotti nel suo carrello;
\item \textbf{Scenario Principale}: l'utente generico si trova nella pagina del carrello e, se ha prodotti al suo interno, visualizza le tasse applicate ad essi;
\item \textbf{Precondizione}: l'utente generico si trova nella pagina del carrello e ha almeno un prodotto al suo interno;
\item \textbf{Postcondizione}: vengono visualizzate le tasse applicate ai prodotti del carrello.
\end{itemize}





\subsubsection{UCX - Acquisto prodotti}
Figura \\
\begin{itemize}
\item \textbf{Attori primari}: cliente;
\item \textbf{Attori secondari}: Stripe;
\item \textbf{Descrizione}: il cliente può acquistare i prodotti presenti nel suo carrello;
\item \textbf{Scenario Principale}: 
\begin{enumerate}
	\item il cliente inizia il checkout dei prodotti nel carrello \textbf{[UCX.1]};
	\item il cliente inserisce l'indirizzo di spedizione e i dati per il pagamento \textbf{[UCX.3]};
	\item il cliente procede al pagamento \textbf{[UCX.4]};
	\item il cliente visualizza un riepilogo dell'ordine effettuato \textbf{[UCX.6]};
\end{enumerate}
\item \textbf{Precondizione}: l'utente è autenticato come cliente, si trova nella pagina del carrello e ha precedentemente inserito dei prodotti nel carrello;
\item \textbf{Postcondizione}: il costo totale dei prodotti acquistati è stato prelevato dal conto specificato dal cliente. È stata inviata un'email al cliente per confermare l'acquisto.
\end{itemize}

\subsubsection{UCX.1 - Checkout}
Figura \\
\begin{itemize}
\item \textbf{Attori primari}: cliente;
\item \textbf{Descrizione}: il cliente effettua il checkout per iniziare la procedura di acquisto prodotti;
\item \textbf{Scenario Principale}: il cliente preme il bottone dedicato al checkout, nella pagina del carrello;
\item \textbf{Estensioni}:
\begin{itemize}
\item se non è presente alcun prodotto nel carrello viene visualizzato un messaggio di errore \textbf{[UCX.2]}:;
\end{itemize}
\item \textbf{Precondizione}: l'utente è autenticato come cliente, si trova nella pagina del carrello e ha premuto il bottone dedicato al checkout;
\item \textbf{Postcondizione}: il cliente viene reindirizzato alla pagina di selezione indirizzo di spedizione e di pagamento, dove può continuare l'acquisto dei prodotti.
\end{itemize}

\subsubsection{UCX.2 - Visualizzazione errore carrello vuoto}
Figura \\
\begin{itemize}
\item \textbf{Attori primari}: cliente;
\item \textbf{Descrizione}: il cliente visualizza un messaggio di errore che lo informa che per procedere al checkout è necessario avere almeno un prodotto nel carrello;
\item \textbf{Scenario Principale}: il cliente prova ad effettuare il checkout senza aver precedentemente inserito prodotti nel suo carrello.
\item \textbf{Precondizione}: il cliente si trova nella pagina del carrello e ha premuto il bottone dedicato al checkout;
\item \textbf{Postcondizione}: viene visualizzato un messaggio che informa il cliente della necessità di avere almeno un prodotto nel carrello per iniziare il checkout.
\end{itemize}

\subsubsection{UCX.3 - Inserimento dati necessari all'acquisto}
Figura \\
\begin{itemize}
\item \textbf{Attori primari}: cliente;
\item \textbf{Descrizione}: il cliente compila i campi dati necessari all'acquisto dei prodotti selezionati;
\item \textbf{Scenario Principale}: il cliente si trova nella pagina di pagamento e, per concludere l'acquisto, inserisce i dati richiesti;
\item \textbf{Specializzazioni}:
\begin{itemize}
	\item TODO il cliente verifica l'indirizzo di spedizione indicato in fase di registrazione o può indicarne uno diverso \textbf{[UCX.3.1]};
	\item il cliente inserisce i dati relativi al metodo di pagamento, come nome, cognome e numero di carta di credito \textbf{[UCX.3.2]};
\end{itemize}
\item \textbf{Precondizione}: il cliente ha superato il checkout e si trova nella pagina di pagamento;
\item \textbf{Postcondizione}: il cliente può continuare il pagamento e l'acquisto.
\end{itemize}

\subsubsection{UCX.3.1 - TODO Inserimento indirizzo di spedizione}
Figura \\
\begin{itemize}
\item \textbf{Attori primari}: cliente;
\item \textbf{Descrizione}: il cliente inserisce i dati dell'indirizzo di spedizione;
\item \textbf{Scenario Principale}: 
\item \textbf{Estensioni}:
\item \textbf{Specializzazioni}:
\item \textbf{Precondizione}: il cliente ha superato il checkout e si trova nella pagina di pagamento;
\item \textbf{Postcondizione}: il cliente ha compilato i dati relativi all'indirizzo di spedizione.
\end{itemize}

\subsubsection{UCX.3.2 - Inserimento dati di pagamento}
Figura \\
\begin{itemize}
\item \textbf{Attori primari}: cliente;
\item \textbf{Descrizione}: il cliente inserisce i dati relativi alla sua carta di pagamento;
\item \textbf{Scenario Principale}: il cliente inserisce nel form relativo ai dati di pagamento le seguenti informazioni:
\begin{itemize}
	\item nome collegato al conto;
	\item cognome collegato al conto;
	\item numero di carta;
	\item mese e anno di scadenza della carta;
	\item codice di sicurezza;
\end{itemize}
\item \textbf{Precondizione}: il cliente ha superato il checkout e si trova nella pagina di pagamento;
\item \textbf{Postcondizione}: il cliente ha compilato i dati di pagamento.
\end{itemize}

\subsubsection{UCX.4 - Pagamento}
Figura \\
\begin{itemize}
\item \textbf{Attori primari}: cliente;
\item \textbf{Attori secondari}: Stripe;
\item \textbf{Descrizione}: il cliente, dopo aver inserito i dati richiesti, procede all'ordine. Il pagamento viene effettuato tramite Stripe, che riporterà l'esito ed eventuali errori;
\item \textbf{Scenario Principale}: il cliente conferma l'ordine tramite l'apposito bottone;
\item \textbf{Estensioni}:
\begin{itemize}
	\item il pagamento non va a buon fine. Il cliente visualizza un messaggio di errore contenente le cause del fallimento \textbf{[UCX.5]};
\end{itemize}
\item \textbf{Precondizione}: il cliente si trova nella pagina di pagamento e ha compilato i campi dati necessari per procedere all'acquisto;
\item \textbf{Postcondizione}: il costo totale dei prodotti acquistati è stato prelevato dal conto specificato dal cliente. È stata inviata un'email al cliente per confermare l'acquisto. Al cliente viene mostrato un riepilogo dell'ordine appena effettuato.
\end{itemize}

\subsubsection{UCX.5 - Visualizzazione errore pagamento}
Figura \\
\begin{itemize}
\item \textbf{Attori primari}: cliente;
\item \textbf{Attori secondari}: Stripe;
\item \textbf{Descrizione}: il cliente visualizza un messaggio di errore che lo informa del motivo per il fallimento del pagamento. L'utente può poi controllare i dati inseriti e ritentare il pagamento;
\item \textbf{Scenario Principale}: il cliente prova ad effettuare il pagamento inserendo dati di una carta che per qualche motivo rifiuta l'addebito.
\item \textbf{Precondizione}: il cliente si trova nella pagina di pagamento e ha compilato i campi dati necessari per procedere all'acquisto;
\item \textbf{Postcondizione}: viene visualizzato un messaggio che informa il cliente dell'errore avvenuto nel processo di pagamento, consigliandogli di controllare i dati inseriti e riprovare.
\end{itemize}

\subsubsection{UCX.6 - Visualizzazione riepilogo ordine}
Figura \\
\begin{itemize}
\item \textbf{Attori primari}: cliente;
\item \textbf{Attori secondari}:
\item \textbf{Descrizione}: il cliente visualizza un riepilogo dell'ordine appena effettuato contenente l'elenco di prodotti, ognuno con prezzo e quantità, il costo totale, le tasse, l'indirizzo di spedizione e la data.
\item \textbf{Scenario Principale}: il cliente ha effettuato un ordine e ne visualizza un riepilogo.
\item \textbf{Precondizione}: il cliente ha completato un ordine ed il pagamento è andato a buon fine.
\item \textbf{Postcondizione}: viene visualizzato un riepilogo dell'ordine effettuato.
\end{itemize}

\subsubsection{UC9.1 - Visualizzazione dati del profilo}
\begin{itemize}
\item \textbf{Attori primari}: cliente;
\item \textbf{Descrizione}: il cliente visualizza i suoi dati personali presenti nel sistema;
\item \textbf{Scenario Principale}: il cliente si trova nella pagina del profilo e visualizza i seguenti dati personali:
\begin{itemize}
\item nome;
\item cognome;
\item indirizzo di fatturazione;
\item email.
\end{itemize}
\item \textbf{Precondizione}: il cliente si trova nella pagina del profilo;
\item \textbf{Postcondizione}: vengono visualizzati i dati personali del cliente.
\end{itemize}

\subsubsection{UC9.2 - Modifica dati del profilo}
\begin{itemize}
\item \textbf{Attori primari}: cliente;
\item \textbf{Attori secondari}: AWS Cognito\ped{G};
\item \textbf{Descrizione}: il cliente può modificare i dati personali presenti nella sua pagina del profilo;
\item \textbf{Scenario Principale}: il cliente si trova nella pagina del profilo e clicca un tasto dedicato per andare nella pagina di modifica. Da li può modificare le seguenti informazioni:
\begin{itemize}
\item nome [\textbf{UC9.2.1}];
\item cognome [\textbf{UC9.2.2}];
\item indirizzo di fatturazione [\textbf{UC9.2.3}];
\item email [\textbf{UC9.2.4}];
\item password [\textbf{UC9.2.5}].
\end{itemize}
Una volta che i campi hanno al suo interno i valori previsti dal cliente, il cliente può premere il pulsante per confermare la modifica dei dati [\textbf{UC9.2.6}];
\item \textbf{Precondizione}: il cliente si trova nella pagina del profilo;
\item \textbf{Postcondizione}: vengono modificati i dati personali del cliente presenti nel profilo in base alle sue intenzioni.
\end{itemize}

\begin{figure}[H]
\centering
\includegraphics[scale=0.6]{res/UseCase/Immagini/ModificaProfilo}
\caption{Diagramma UML\ped{G} per UC9.2 - Modifica dati del profilo}
\end{figure}

\subsubsection{UC9.2.1 - Modifica nome}
\begin{itemize}
\item \textbf{Attori primari}: cliente;
\item \textbf{Descrizione}: il cliente può modificare il nome;
\item \textbf{Scenario Principale}: il cliente visualizza nel campo relativo al nome il dato attualmente presente, che andrà a sovrascrivere con il nuovo nome aggiornato;
\item \textbf{Precondizione}: il cliente si trova nella pagina dedicata alla modifica del profilo;
\item \textbf{Postcondizione}: il cliente ha compilato il campo dedicato al nome, sovrascrivendo il precedente.
\end{itemize}

\subsubsection{UC9.2.2 - Modifica cognome}
\begin{itemize}
\item \textbf{Attori primari}: cliente;
\item \textbf{Descrizione}: il cliente può modificare il cognome;
\item \textbf{Scenario Principale}: il cliente visualizza nel campo relativo al cognome il dato attualmente presente, che andrà a sovrascrivere con il nuovo cognome aggiornato;
\item \textbf{Precondizione}: il cliente si trova nella pagina dedicata alla modifica del profilo;
\item \textbf{Postcondizione}: il cliente ha compilato il campo dedicato al cognome, sovrascrivendo il precedente.
\end{itemize}

\subsubsection{UC9.2.3 - Modifica indirizzo di fatturazione}
\begin{itemize}
\item \textbf{Attori primari}: cliente;
\item \textbf{Descrizione}: il cliente può modificare l'indirizzo di fatturazione;
\item \textbf{Scenario Principale}: il cliente visualizza nel campo relativo all'indirizzo di fatturazione il dato attualmente presente, che andrà a sovrascrivere con il nuovo indirizzo aggiornato;
\item \textbf{Precondizione}: il cliente si trova nella pagina dedicata alla modifica del profilo;
\item \textbf{Postcondizione}: il cliente ha compilato il campo dedicato all'indirizzo di fatturazione, sovrascrivendo il precedente.
\end{itemize}

\subsubsection{UC9.2.4 - Modifica email}
\begin{itemize}
\item \textbf{Attori primari}: cliente;
\item \textbf{Descrizione}: il cliente può modificare l'email;
\item \textbf{Scenario Principale}: il cliente visualizza nel campo relativo all'email il dato attualmente presente, che andrà a sovrascrivere con la nuova email aggiornata;
\item \textbf{Precondizione}: il cliente si trova nella pagina dedicata alla modifica del profilo;
\item \textbf{Postcondizione}: il cliente ha compilato il campo dedicato all'email, sovrascrivendo la precedente.
\end{itemize}

\subsubsection{UC9.2.5 - Modifica password}
\begin{itemize}
\item \textbf{Attori primari}: cliente;
\item \textbf{Descrizione}: il cliente può modificare la password;
\item \textbf{Scenario Principale}: il cliente visualizza un campo vuoto riguardante la password, che potrà compilare per modificarla;
\item \textbf{Precondizione}: il cliente si trova nella pagina dedicata alla modifica del profilo;
\item \textbf{Postcondizione}: il cliente ha compilato il campo dedicato alla password, sovrascrivendo la precedente.
\end{itemize}

\subsubsection{UC9.2.6 - Conferma modifica profilo}
\begin{itemize}
\item \textbf{Attori primari}: cliente;
\item \textbf{Descrizione}: il cliente conferma i dati presenti nel form e il profilo viene modificato;
\item \textbf{Scenario Principale}: il cliente preme il pulsante di conferma, la richiesta viene gestita dall'Identity Manager ed il profilo viene modificato con i valori precedentemente inseriti;
\item \textbf{Estensioni}: 
\begin{itemize}
\item se i campi risultano non validi oppure l'email è già stata utilizzata nel sistema, viene mostrato un messaggio di errore [\textbf{UC9.2.7}].
\end{itemize}
\item \textbf{Precondizione}: il cliente si trova nella pagina dedicata alla modifica del profilo;
\item \textbf{Postcondizione}: il cliente ha modificato il profilo con i dati inseriti.
\end{itemize}

\subsubsection{UC9.2.7 - Visualizzazione errori di modifica del profilo}
\begin{itemize}
\item \textbf{Attori primari}: cliente;
\item \textbf{Descrizione}: il cliente visualizza un errore riguardante i campi di modifica profilo che ha inserito. Questi errori possono essere:
\begin{itemize}
\item \textbf{Campo non valido}: un campo risulta vuoto o con caratteri non validi;
\item \textbf{Email già utilizzata}: l'email è già stata utilizzata da un altro cliente.
\end{itemize}
\item \textbf{Scenario Principale}: il sistema riconosce uno o più errori nei campi inseriti dal cliente e vengono visualizzati nella pagina di modifica profilo;
\item \textbf{Precondizione}: il cliente ha inserito i campi ed ha provato ad effettuare la modifica del profilo;
\item \textbf{Postcondizione}: viene visualizzato un messaggio di errore nella pagina di modifica del profilo e i dati non sono stati modificati.
\end{itemize}

\subsubsection{UC9.3 - Visualizzazione ordini effettuati dall'utente}
\begin{itemize}
\item \textbf{Attori primari}: cliente;
\item \textbf{Descrizione}: il cliente può visualizzare una lista di ordini effettuati;
\item \textbf{Scenario Principale}: il cliente si trova nella pagina del profilo e visualizza una lista di ordini effettuati. Per ogni ordine vengono visualizzati una serie di dati [\textbf{UC9.3.1}];
\item \textbf{Precondizione}: il cliente si trova nella pagina del profilo e ha effettuato almeno un ordine nel sito;
\item \textbf{Postcondizione}: viene visualizzata una lista di tutti gli ordini che il cliente ha effettuato.
\end{itemize}

\begin{figure}[H]
\centering
\includegraphics[scale=0.6]{res/UseCase/Immagini/VisualizzazioneOrdini}
\caption{Diagramma UML\ped{G} per UC9.3 - Visualizzazione ordini effettuati dall'utente}
\end{figure}

\subsubsection{UC9.3.1 - Visualizzazione dati singolo ordine}
\begin{itemize}
\item \textbf{Attori primari}: cliente;
\item \textbf{Descrizione}: per ogni ordine nella lista degli ordini effettuati, vengono visualizzati una serie di dati;
\item \textbf{Scenario Principale}: Il cliente si trova nella pagina del profilo e, cliccando un ordine effettuato, visualizza  una pagina di riepilogo con i seguenti dati:
\begin{itemize}
\item id ordine;
\item prodotti acquistati;
\item quantità prodotti acquistati;
\item costo per prodotto;
\item costo totale;
\item tasse applicate;
\item data d'acquisto.
\end{itemize}
\item \textbf{Precondizione}: il cliente, cliccando su un ordine nella sua pagina di profilo, si trova nella pagina di riepilogo di un ordine;
\item \textbf{Postcondizione}: vengono visualizzati i dati dell'ordine scelto.
\end{itemize}

\subsubsection{UC9.4 - Eliminazione account}
\begin{itemize}
\item \textbf{Attori primari}: cliente;
\item \textbf{Attori secondari}: AWS Cognito\ped{G};
\item \textbf{Descrizione}: il cliente può eliminare il suo account dal sistema;
\item \textbf{Scenario Principale}: il cliente si trova nella pagina del profilo e clicca il tasto dedicato per mandare la richiesta di eliminazione dell'account all'Identity Manager;
\item \textbf{Precondizione}: il cliente si trova nella pagina del profilo;
\item \textbf{Postcondizione}: l'account è stato eliminato dal sistema e il cliente diventa un utente non autenticato.
\end{itemize}

\subsubsection{UC16 - Ricerca nel catalogo digitale}
\begin{itemize}
\item \textbf{Attori primari}: utente generico;
\item \textbf{Descrizione}: l'utente può effettuare una ricerca tra tutti i prodotti in vendita nella piattaforma attraverso la digitazione di una parola o frase chiave nella barra di ricerca;
\item \textbf{Scenario Principale}: l'utente seleziona la barra di ricerca e inserisce la parola chiave corrispondente al prodotto desiderato. Infine se la ricerca va a buon fine viene visualizzata una lista di prodotti compatibili con la parola digitata, altrimenti viene visualizzato un messaggio che informa che la ricerca non ha prodotto nessun risultato;
\item \textbf{Precondizione}: l'utente si trova sulla piattaforma e necessita di fare una ricerca per trovare un prodotto specifico;
\item \textbf{Postcondizione}: l'utente visualizza il risultato della ricerca effettuata.
\end{itemize}
\subsubsection{UC17 - Accesso ad una pagina della categoria}
\begin{itemize}
\item \textbf{Attori primari}: utente generico;
\item \textbf{Descrizione}: l'utente può accedere ad una pagina contenente un insieme di prodotti listati facenti parte di una specifica categoria. In questa pagina, chiamata anche PLP\ped{G}, l'utente ha a disposizione ulteriori strumenti per interagire con i prodotti da lui desiderati;
\item \textbf{Scenario Principale}: l'utente clicca sulla categoria di prodotti da lui desiderata;
\item \textbf{Precondizione}: l'utente si trova sulla piattaforma e necessita di visualizzare tutti i prodotti contenuti in una specifica categoria;
\item \textbf{Postcondizione}: l'utente accede alla PLP\ped{G} corrispondente alla categoria desiderata.
\end{itemize}
\subsubsection{UC11.1 - Visualizzazione dei prodotti}
\begin{itemize}
\item \textbf{Attori primari}: utente generico;
\item \textbf{Descrizione}: l'utente può visualizzare un insieme di prodotti raggruppati per categoria accedendo alla relativa PLP\ped{G};
\item \textbf{Scenario Principale}: l'utente visualizza i prodotti appartenenti ad una determinata categoria. Ogni prodotto listato nella PLP\ped{G} deve esporre le seguenti informazioni:
\begin{enumerate}
\item[a.] nome;
\item[b.] immagine;
\item[c.] prezzo;
\end{enumerate}
\item \textbf{Precondizione}: l'utente ha cliccato sulla PLP\ped{G} dedicata alla categoria di prodotti scelta;
\item \textbf{Postcondizione}: l'utente visualizza le informazioni relative ai prodotti listati, con le eventuali operazioni disponibili su ognuno di essi.
\end{itemize}
\subsubsection{UC11.2 - Selezione dei prodotti}
\begin{itemize}
\item \textbf{Attori primari}: utente generico;
\item \textbf{Descrizione}: l'utente può selezionare prodotti multipli presenti sulla PLP\ped{G} in modo da poterli inserire successivamente nel carrello, con una quantità selezionata pari ad 1. Per selezionare un prodotto è sufficiente cliccare sulla relativa casella di selezione;
\item \textbf{Scenario Principale}:
\begin{enumerate}
\item[a.] l'utente visualizza i prodotti presenti nella lista corrispondente ad una categoria [\textbf{UC11.1}];
\item[b.] l'utente clicca sulle caselle relative ad uno o più prodotti per selezionarli.
\end{enumerate}
\item \textbf{Precondizione}: l'utente sta visualizzando la PLP\ped{G} contenente uno o più prodotti desiderati;
\item \textbf{Postcondizione}: l'utente ha selezionato i prodotti desiderati.
\end{itemize}
\subsubsection{UC11.3 - Inserimento nel carrello dei prodotti selezionati}
\begin{itemize}
\item \textbf{Attori primari}: utente generico;
\item \textbf{Descrizione}: l'utente può inserire un prodotto presente sulla PLP\ped{G} nel carrello, con una quantità selezionata pari ad 1. Per effettuare l'azione è sufficiente cliccare sull'apposito pulsante;
\item \textbf{Scenario Principale}:
\begin{enumerate}
\item[b.] l'utente seleziona i prodotti desiderati [\textbf{UC11.2}];
\item[c.] l'utente preme il pulsante per l'inserimento del prodotto desiderato nel carrello.
\end{enumerate}
\item \textbf{Precondizione}: l'utente sta visualizzando la PLP\ped{G} contenente prodotti;
\item \textbf{Postcondizione}: l'utente ha inserito nel proprio carrello uno o più prodotti selezionati.
\end{itemize}
\subsubsection{UC11.4 - Filtraggio dei prodotti}
\begin{itemize}
\item \textbf{Attori primari}: utente generico;
\item \textbf{Descrizione}: l'utente può raffinare la propria ricerca andando a filtrare i prodotti presenti nella PLP\ped{G}. Il parametro utilizzabile per eseguire questa operazione è il prezzo;
\item \textbf{Scenario Principale}:
\begin{enumerate}
\item[a.] l'utente abilita il filtro attraverso l'apposita casella;
\item[b.] l'utente digita il prezzo minimo e il prezzo massimo ai quali è interessato;
\item[c.] l'utente applica il filtro cliccando l'apposito bottone.
\end{enumerate}
\item \textbf{Estensioni}:
\begin{itemize}
\item se non viene trovato nessun risultato compatibile con il range di prezzo inserito dall'utente viene visualizzato un messaggio di errore [\textbf{UC11.5}];
\end{itemize}
\item \textbf{Precondizione}: l'utente sta visualizzando i prodotti in vendita listati nella specifica PLP\ped{G};
\item \textbf{Postcondizione}: l'utente ottiene una nuova pagina in cui può visualizzare solo i prodotti presenti nella PLP\ped{G} corrente e compresi nel range di prezzo specificato.
\end{itemize}
\subsubsection{UC11.5 - Visualizzazione errore nessun risultato filtraggio}
\begin{itemize}
\item \textbf{Attori primari}: utente generico;
\item \textbf{Descrizione}: l'utente visualizza un messaggio di errore che lo informa che il filtraggio dei prodotti effettuato utilizzando i parametri da lui inseriti non ha prodotto nessun risultato.
\item \textbf{Scenario Principale}: l'utente effettua un filtraggio sul catalogo digitale selezionando un range di prezzo nel quale non è presente nessun prodotto;
\item \textbf{Precondizione}: l'utente ha effettuato l'operazione di filtraggio nella corrente PLP\ped{G};
\item \textbf{Postcondizione}: viene visualizzato un messaggio che informa l'utente di non aver trovato prodotti compatibili con quanto stabilito attraverso il range di prezzo.
\end{itemize}
\begin{figure}[H]
\centering
\includegraphics[scale=0.6]{res/UseCase/Immagini/VisualizzazionePaginaProdotto}
\caption{Diagramma UML per UC11.4 - Visualizzazione pagina del prodotto}
\end{figure}
\subsubsection{UC11.6 - Visualizzazione pagina del prodotto}
\begin{itemize}
\item \textbf{Attori primari}: utente generico;
\item \textbf{Descrizione}: l'utente può accedere ad una pagina per visualizzare tutte le caratteristiche di un prodotto listato. In questa pagina, chiamata anche PDP\ped{G}, è possibile eventualmente aggiungere il prodotto al carrello.
\item \textbf{Scenario Principale}: l'utente clicca sul prodotto da lui desiderato;
\item \textbf{Precondizione}: l'utente si trova in una PLP\ped{G} contenente una lista di prodotti;
\item \textbf{Postcondizione}: l'utente accede alla PDP\ped{G} corrispondente al prodotto desiderato.
\end{itemize}
\subsubsection{UCX - Visualizzazione del prodotto in una PDP}
Figura \\
\begin{itemize}
\item \textbf{Attori primari}:
\item \textbf{Attori secondari}:
\item \textbf{Descrizione}:
\item \textbf{Scenario Principale}:
\item \textbf{Estensioni}:
\item \textbf{Specializzazioni}:
\item \textbf{Precondizione}:
\item \textbf{Postcondizione}:
\end{itemize}

\begin{figure}[H]
\centering
\includegraphics[scale=0.6]{res/UseCase/Immagini/MerchantDashboard}
\caption{Schema generale: modulo di gestione della merchant dashboard\ped{G}}
\end{figure}

\subsubsection{UC14 - Accesso alla merchant dashboard}
\begin{itemize}
\item \textbf{Attori primari}: venditore;
\item \textbf{Descrizione}: il venditore accede alla merchant dashboard\ped{G};
\item \textbf{Scenario Principale}: il venditore, tramite l'apposito pulsante, accede alla merchant dashboard\ped{G};
\item \textbf{Precondizione}: il venditore ha effettuato l'autenticazione ed è stato riconosciuto come venditore;
\item \textbf{Postcondizione}: il venditore si trova nella merchant dashboard\ped{G};
\end{itemize}

\subsubsection{UC15 - Gestione prodotti in vendita}
\begin{itemize}
\item \textbf{Attori primari}: venditore;
\item \textbf{Descrizione}: il venditore può gestire i prodotti dalla dashboard\ped{G};
\item \textbf{Scenario Principale}: il venditore può effettuare le seguenti operazioni sui prodotti:
\begin{itemize}
	\item inserimento \textbf{[UC15.1]};
	\item visualizzazione \textbf{[UC15.2]};
	\item modifica \textbf{[UC15.3]};
	\item rimozione \textbf{[UC15.4]}.
\end{itemize}
\item \textbf{Precondizione}: il venditore si trova nella merchant dashboard\ped{G};
\item \textbf{Postcondizione}: al venditore è permesso effettuare operazioni per gestire i propri prodotti.
\end{itemize}

\begin{figure}[H]
\centering
\includegraphics[scale=0.6]{res/UseCase/Immagini/GestioneProdotti}
\caption{Diagramma UML\ped{G} per UC15 - Gestione prodotti in vendita}
\end{figure}

\subsubsection{UC15.1 - Inserimento prodotto}
\begin{itemize}
\item \textbf{Attori primari}: venditore;
\item \textbf{Descrizione}: il venditore può inserire nel sistema nuovi prodotti da vendere;
\item \textbf{Scenario Principale}: il venditore accede alla pagina di inserimento prodotti tramite l'apposito pulsante ed inserisce le seguenti informazioni:
\begin{itemize}
	\item nome \textbf{[UC15.1.1]};
	\item descrizione \textbf{[UC15.1.2]};
	\item prezzo \textbf{[UC15.1.3]};
	\item immagine \textbf{[UC15.1.4]};
	\item categoria \textbf{[UC15.1.5]}.
\end{itemize}
In seguito verrà premuto il pulsante di conferma \textbf{[UC15.1.6]};
\item \textbf{Precondizione}: l'utente ha eseguito il login, è riconosciuto come venditore e si trova nella merchant dashboard\ped{G};
\item \textbf{Postcondizione}: il venditore ha inserito un nuovo prodotto nel sistema.
\end{itemize}

\begin{figure}[H]
\centering
\includegraphics[scale=0.6]{res/UseCase/Immagini/InserimentoProdotto}
\caption{Diagramma UML\ped{G} per UC15.1 - Inserimento prodotto}
\end{figure}

\subsubsection{UC15.1.1 - Inserimento nome}
\begin{itemize}
\item \textbf{Attori primari}: venditore;
\item \textbf{Descrizione}: al venditore è richiesto l'inserimento del nome del nuovo prodotto da inserire;
\item \textbf{Scenario Principale}: il venditore inserisce nel form dedicato il nome del nuovo prodotto;
\item \textbf{Precondizione}: il venditore si trova nella pagina dedicata all'inserimento di un nuovo prodotto;
\item \textbf{Postcondizione}: il venditore ha compilato il campo dedicato al nome del prodotto.
\end{itemize}

\subsubsection{UC15.1.2 - Inserimento descrizione}
\begin{itemize}
\item \textbf{Attori primari}: venditore;
\item \textbf{Descrizione}: al venditore è richiesto l'inserimento della descrizione del nuovo prodotto da inserire;
\item \textbf{Scenario Principale}: il venditore inserisce nel form dedicato la descrizione del nuovo prodotto;
\item \textbf{Precondizione}: il venditore si trova nella pagina dedicata all'inserimento di un nuovo prodotto;
\item \textbf{Postcondizione}: il venditore ha compilato il campo dedicato alla descrizione del prodotto.
\end{itemize}

\subsubsection{UC15.1.3 - Inserimento prezzo}
\begin{itemize}
\item \textbf{Attori primari}: venditore;
\item \textbf{Descrizione}: al venditore è richiesto l'inserimento del prezzo del nuovo prodotto da inserire;
\item \textbf{Scenario Principale}: il venditore inserisce nel form dedicato il prezzo del nuovo prodotto;
\item \textbf{Precondizione}: il venditore si trova nella pagina dedicata all'inserimento di un nuovo prodotto;
\item \textbf{Postcondizione}: il venditore ha compilato il campo dedicato al prezzo del prodotto.
\end{itemize}

\subsubsection{UC15.1.4 - Inserimento immagine}
\begin{itemize}
\item \textbf{Attori primari}: venditore;
\item \textbf{Descrizione}: al venditore è richiesto l'inserimento dell'immagine del nuovo prodotto da inserire;
\item \textbf{Scenario Principale}: il venditore carica tramite l'apposito bottone di upload l'immagine del nuovo prodotto;
\item \textbf{Precondizione}: il venditore si trova nella pagina dedicata all'inserimento di un nuovo prodotto;
\item \textbf{Postcondizione}: il venditore ha caricato l'immagine del nuovo prodotto.
\end{itemize}

\subsubsection{UC15.1.5 - Inserimento categoria}
\begin{itemize}
\item \textbf{Attori primari}: venditore;
\item \textbf{Descrizione}: al venditore è richiesto l'inserimento della categoria del nuovo prodotto da inserire;
\item \textbf{Scenario Principale}: il venditore inserisce nel form dedicato la categoria del nuovo prodotto;
\item \textbf{Precondizione}: il venditore si trova nella pagina dedicata all'inserimento di un nuovo prodotto;
\item \textbf{Postcondizione}: il venditore ha compilato il campo dedicato alla categoria del prodotto.
\end{itemize}

\subsubsection{UC15.1.6 - Conferma inserimento}
\begin{itemize}
\item \textbf{Attori primari}: venditore;
\item \textbf{Descrizione}: il venditore conferma i dati presenti nel form e inserisce il nuovo prodotto;
\item \textbf{Scenario Principale}: il venditore preme il pulsante di conferma e inserisce il nuovo prodotto con i dati precedentemente inseriti;
\item \textbf{Estensioni}: 
\begin{itemize}
\item nel caso qualche campo non sia compilato viene visualizzato un messaggio di errore \textbf{[UC15.1.7]}.
\end{itemize} 
\item \textbf{Precondizione}: il venditore si trova nella pagina dedicata all'inserimento di un nuovo prodotto e ha premuto il bottone di conferma;
\item \textbf{Postcondizione}: il venditore ha inserito un nuovo prodotto nel sistema.
\end{itemize}

\subsubsection{UC15.1.7 - Visualizzazione errore campi mancanti inserimento prodotto}
\begin{itemize}
\item \textbf{Attori primari}: venditore;
\item \textbf{Descrizione}: il venditore visualizza un messaggio di errore che lo informa che per procedere all'inserimento è necessario aver inserito tutti i dati richiesti;
\item \textbf{Scenario Principale}: il venditore prova ad inserire un nuovo prodotto senza aver compilato tutti i campi dati;
\item \textbf{Precondizione}: il venditore si trova nella pagina dedicata all'inserimento di un nuovo prodotto, ha premuto il bottone di conferma e non sono stati compilati tutti i campi;
\item \textbf{Postcondizione}: viene visualizzato un messaggio che informa il venditore della necessità di compilare tutti i campi dati per completare l'inserimento di un prodotto.
\end{itemize}

\subsubsection{UC15.2 - Visualizzazione lista prodotti}
\begin{itemize}
\item \textbf{Attori primari}: venditore;
\item \textbf{Descrizione}: il venditore visualizza nella merchant dashboard\ped{G} la lista dei prodotti presenti nel sistema;
\item \textbf{Scenario Principale}: il venditore accede alla merchant dashboard\ped{G} e, nell'apposita sezione, visualizza le seguenti informazioni sui prodotti nel sistema:
\begin{itemize}
\item nome;
\item descrizione;
\item categoria;
\item prezzo;
\item immagine.
\end{itemize}
\item \textbf{Precondizione}: l'utente ha eseguito il login, è riconosciuto come venditore e si trova nella merchant dashboard\ped{G};
\item \textbf{Postcondizione}: il venditore visualizza la lista dei prodotti presenti nel sistema, ognuno dei quali con le seguenti informazioni:
\begin{itemize}
\item nome;
\item descrizione;
\item categoria;
\item prezzo;
\item immagine.
\end{itemize}
\end{itemize}

\subsubsection{UC15.3 - Modifica prodotto}
\begin{itemize}
\item \textbf{Attori primari}: venditore;
\item \textbf{Descrizione}: il venditore può modificare i prodotti presenti nel sistema;
\item \textbf{Scenario Principale}: il venditore accede alla pagina di modifica tramite l'apposito pulsante collegato al prodotto da modificare, visualizzato nella lista dei prodotti \textbf{[UC15.1.2]}. Da li può modificare le seguenti informazioni:
\begin{itemize}
	\item nome \textbf{[UC15.3.1]};
	\item descrizione \textbf{[UC15.3.2]};
	\item prezzo \textbf{[UC15.3.3]};
	\item immagine \textbf{[UC15.3.4]};
	\item categoria \textbf{[UC15.3.5]}.
\end{itemize}
In seguito verrà premuto il pulsante di conferma \textbf{[UC15.3.6]};
\item \textbf{Precondizione}: l'utente ha eseguito il login, è riconosciuto come venditore e si trova nella merchant dashboard\ped{G};
\item \textbf{Postcondizione}: il venditore ha modificato il prodotto selezionato.
\end{itemize}

\begin{figure}[H]
\centering
\includegraphics[scale=0.6]{res/UseCase/Immagini/ModificaProdotto}
\caption{Diagramma UML\ped{G} per UC15.1.3 - Modifica prodotto}
\end{figure}

\subsubsection{UC15.3.1 - Modifica nome}
\begin{itemize}
\item \textbf{Attori primari}: venditore;
\item \textbf{Descrizione}: il venditore modifica il nome del prodotto selezionato;
\item \textbf{Scenario Principale}: il venditore visualizza nel campo relativo al nome del prodotto il dato attualmente presente, che andrà a sovrascrivere con il nuovo nome aggiornato;
\item \textbf{Precondizione}: il venditore si trova nella pagina dedicata alla modifica di un prodotto;
\item \textbf{Postcondizione}: il venditore ha compilato il campo dedicato al nome del prodotto, sovrascrivendo il precedente.
\end{itemize}

\subsubsection{UC15.3.2 - Modifica descrizione}
\begin{itemize}
\item \textbf{Attori primari}: venditore;
\item \textbf{Descrizione}: il venditore modifica la descrizione del prodotto selezionato;
\item \textbf{Scenario Principale}: il venditore visualizza nel campo relativo alla descrizione del prodotto il dato attualmente presente, che andrà a sovrascrivere con la nuova descrizione aggiornata;
\item \textbf{Precondizione}: il venditore si trova nella pagina dedicata alla modifica di un prodotto;
\item \textbf{Postcondizione}: il venditore ha compilato il campo dedicato alla descrizione del prodotto, sovrascrivendo la precedente.
\end{itemize}

\subsubsection{UC15.3.3 - Modifica prezzo}
\begin{itemize}
\item \textbf{Attori primari}: venditore;
\item \textbf{Descrizione}: il venditore modifica il prezzo del prodotto selezionato;
\item \textbf{Scenario Principale}: il venditore visualizza nel campo relativo al prezzo del prodotto il dato attualmente presente, che andrà a sovrascrivere con il nuovo prezzo aggiornato;
\item \textbf{Precondizione}: il venditore si trova nella pagina dedicata alla modifica di un prodotto;
\item \textbf{Postcondizione}: il venditore ha compilato il campo dedicato al prezzo del prodotto, sovrascrivendo il precedente.
\end{itemize}

\subsubsection{UC15.3.4 - Modifica immagine}
\begin{itemize}
\item \textbf{Attori primari}: venditore;
\item \textbf{Descrizione}: il venditore modifica l'immagine del prodotto selezionato;
\item \textbf{Scenario Principale}: il venditore visualizza nel campo relativo all'immagine del prodotto quella attualmente presente, che andrà a sovrascrivere caricando la nuova immagine aggiornata;
\item \textbf{Precondizione}: il venditore si trova nella pagina dedicata alla modifica di un prodotto;
\item \textbf{Postcondizione}: il venditore ha caricato l'immagine del prodotto, sovrascrivendo la precedente.
\end{itemize}

\subsubsection{UC15.3.5 - Modifica categoria}
\begin{itemize}
\item \textbf{Attori primari}: venditore;
\item \textbf{Descrizione}: il venditore modifica la categoria del prodotto selezionato;
\item \textbf{Scenario Principale}: il venditore visualizza nel campo relativo alla categoria del prodotto il dato attualmente presente, che andrà a sovrascrivere con la nuova categoria aggiornata;
\item \textbf{Precondizione}: il venditore si trova nella pagina dedicata alla modifica di un prodotto;
\item \textbf{Postcondizione}: il venditore ha compilato il campo dedicato alla categoria del prodotto, sovrascrivendo la precedente.
\end{itemize}

\subsubsection{UC15.3.6 - Conferma modifica}
\begin{itemize}
\item \textbf{Attori primari}: venditore;
\item \textbf{Descrizione}: il venditore conferma i dati presenti nel form e modifica il prodotto;
\item \textbf{Scenario Principale}: il venditore preme il pulsante di conferma e modifica il prodotto selezionato con i dati precedentemente inseriti;
\item \textbf{Estensioni}: 
\begin{itemize}
	\item se non viene sovrascritto nessun dato viene visualizzato un messaggio di errore \textbf{[UC15.3.7]}.
\end{itemize} 
\item \textbf{Precondizione}: il venditore si trova nella pagina dedicata alla modifica di un prodotto e ha premuto il bottone di conferma;
\item \textbf{Postcondizione}: il venditore ha modificato il prodotto selezionato.
\end{itemize}

\subsubsection{UC15.3.7 - Visualizzazione errore nessuna modifica}
\begin{itemize}
\item \textbf{Attori primari}: venditore;
\item \textbf{Descrizione}: il venditore visualizza un messaggio di errore che lo informa che per procedere alla modifica è necessario aver sovrascritto almeno un campo dati;
\item \textbf{Scenario Principale}: il venditore prova a modificare un nuovo prodotto senza aver sovrascritto almeno un campo dati;
\item \textbf{Precondizione}: il venditore si trova nella pagina dedicata alla modifica di un prodotto e ha premuto il bottone di conferma;
\item \textbf{Postcondizione}: viene visualizzato un messaggio che informa il venditore della necessità di compilare tutti i campi dati per completare la modifica di un prodotto.
\end{itemize}

\subsubsection{UC15.4 - Rimozione prodotto}
\begin{itemize}
\item \textbf{Attori primari}: venditore;
\item \textbf{Descrizione}: il venditore può rimuovere i prodotti presenti nel sistema;
\item \textbf{Scenario Principale}: il venditore elimina il prodotto scelto tramite l'apposito pulsante collegato al prodotto da eliminare, visualizzato nella lista dei prodotti \textbf{[UC15.1.2]};
\item \textbf{Precondizione}: l'utente ha eseguito il login, è riconosciuto come venditore e si trova nella merchant dashboard\ped{G};
\item \textbf{Postcondizione}: il venditore ha eliminato il prodotto selezionato.
\end{itemize}

\subsubsection{UC16 - Gestione categorie di prodotti}
\begin{itemize}
\item \textbf{Attori primari}: venditore;
\item \textbf{Descrizione}: il venditore può gestire le categorie di prodotti dalla dashboard\ped{G};
\item \textbf{Scenario Principale}: il venditore può effettuare le seguenti operazioni sulle categorie di prodotti:
\begin{itemize}
	\item visualizzazione \textbf{[UC16.1]};
	\item inserimento \textbf{[UC16.2]};
	\item rimozione \textbf{[UC16.3]}.
\end{itemize}
\item \textbf{Precondizione}: il venditore si trova nella merchant dashboard\ped{G};
\item \textbf{Postcondizione}: al venditore è permesso effettuare operazioni per gestire le categorie di prodotti.
\end{itemize}

\begin{figure}[H]
\centering
\includegraphics[scale=0.6]{res/UseCase/Immagini/GestioneCategorie}
\caption{Diagramma UML\ped{G} per UC16 - Gestione categorie di prodotti}
\end{figure}

\subsubsection{UC16.1 - Visualizzazione lista categorie}
\begin{itemize}
\item \textbf{Attori primari}: venditore;
\item \textbf{Descrizione}: il venditore visualizza nella merchant dashboard\ped{G} la lista delle categorie presenti nel sistema;
\item \textbf{Scenario Principale}: il venditore accede alla merchant dashboard\ped{G} e, nell'apposita sezione, visualizza una lista dei nomi delle categorie presenti;
\item \textbf{Precondizione}: l'utente ha eseguito il login, è riconosciuto come venditore e si trova nella merchant dashboard\ped{G};
\item \textbf{Postcondizione}: il venditore visualizza la lista dei nomi delle categorie presenti nel sistema.
\end{itemize}

\subsubsection{UC16.2 - Inserimento categoria}
\begin{itemize}
\item \textbf{Attori primari}: venditore;
\item \textbf{Descrizione}: il venditore può inserire nel sistema nuove categorie di prodotti;
\item \textbf{Scenario Principale}: il venditore accede alla pagina di inserimento categorie tramite l'apposito pulsante ed inserisce il nome della categoria \textbf{[UC16.2.1]} per poi premere il pulsante di conferma \textbf{[UC16.2.2]};
\item \textbf{Precondizione}: l'utente ha eseguito il login, è riconosciuto come venditore e si trova nella merchant dashboard\ped{G};
\item \textbf{Postcondizione}: il venditore ha inserito una nuova categoria nel sistema.
\end{itemize}

\begin{figure}[H]
\centering
\includegraphics[scale=0.6]{res/UseCase/Immagini/InserimentoCategoria}
\caption{Diagramma UML\ped{G} per UC16.2 - Inserimento categoria}
\end{figure}

\subsubsection{UC16.2.1 - Inserimento nome}
\begin{itemize}
\item \textbf{Attori primari}: venditore;
\item \textbf{Descrizione}: al venditore è richiesto l'inserimento del nome della nuova categoria da inserire;
\item \textbf{Scenario Principale}: il venditore inserisce nel form dedicato il nome della nuova categoria;
\item \textbf{Precondizione}: il venditore si trova nella pagina dedicata all'inserimento di una nuova categoria;
\item \textbf{Postcondizione}: il venditore ha compilato il campo dedicato al nome della categoria.
\end{itemize}

\subsubsection{UC16.2.2 - Conferma inserimento}
\begin{itemize}
\item \textbf{Attori primari}: venditore;
\item \textbf{Descrizione}: il venditore conferma i dati presenti nel form e inserisce la nuova categoria;
\item \textbf{Scenario Principale}: il venditore preme il pulsante di conferma e inserisce la nuova categoria con il nome precedentemente inserito;
\item \textbf{Estensioni}: 
\begin{itemize}
	\item se il nome della categoria non è stato inserito viene visualizzato un messaggio di errore \textbf{[UC16.2.3]}.
\end{itemize} 
\item \textbf{Precondizione}: il venditore si trova nella pagina dedicata all'inserimento di una nuova categoria e ha premuto il bottone di conferma;
\item \textbf{Postcondizione}: il venditore ha inserito una nuova categoria nel sistema.
\end{itemize}

\subsubsection{UC16.2.3 - Visualizzazione errore campi mancanti inserimento categoria}
\begin{itemize}
\item \textbf{Attori primari}: venditore;
\item \textbf{Descrizione}: il venditore visualizza un messaggio di errore che lo informa che per procedere all'inserimento è necessario aver inserito il nome della categoria;
\item \textbf{Scenario Principale}: il venditore prova ad inserire una nuova categoria senza aver compilato il campo dati del nome;
\item \textbf{Precondizione}: il venditore si trova nella pagina dedicata all'inserimento di una nuova categoria e ha premuto il bottone di conferma;
\item \textbf{Postcondizione}: viene visualizzato un messaggio che informa il venditore della necessità di compilare il campo del nome per inserire una nuova categoria.
\end{itemize}

\subsubsection{UC16.3 - Rimozione categoria}
\begin{itemize}
\item \textbf{Attori primari}: venditore;
\item \textbf{Descrizione}: il venditore può rimuovere le categorie;
\item \textbf{Scenario Principale}: il venditore elimina la categoria scelta tramite l'apposito pulsante collegato alla categoria da eliminare, visualizzato nella lista delle categorie \textbf{[UC16.1]};
\item \textbf{Precondizione}: l'utente ha eseguito il login, è riconosciuto come venditore e si trova nella merchant dashboard\ped{G};
\item \textbf{Postcondizione}: il venditore ha eliminato la categoria selezionata.
\end{itemize}


\subsubsection{UC17 - Visualizzazione lista ordini}
\begin{itemize}
\item \textbf{Attori primari}: venditore;
\item \textbf{Descrizione}: il venditore visualizza nella merchant dashboard\ped{G} la lista degli ordini ricevuti;
\item \textbf{Scenario Principale}: il venditore accede alla merchant dashboard\ped{G} e, nell'apposita sezione, visualizza una lista degli ordini ricevuti. Per ogni ordine viene visualizzato il numero dell'ordine, il costo totale e la data. Il venditore può accedere ad altre informazioni premendo il bottone 'Dettagli' collegato all'ordine interessato \textbf{[UC17.1]};
\item \textbf{Precondizione}: l'utente ha eseguito il login, è riconosciuto come venditore e si trova nella merchant dashboard\ped{G};
\item \textbf{Postcondizione}: il venditore visualizza la lista degli ordini presenti nel sistema.
\end{itemize}

\begin{figure}[H]
\centering
\includegraphics[scale=0.6]{res/UseCase/Immagini/VisualizzazioneOrdiniMerchant}
\caption{Diagramma UML\ped{G} per UC17 - Visualizzazione lista ricevuti}
\end{figure}

\subsubsection{UC17.1 - Visualizzazione informazioni ordine}
\begin{itemize}
\item \textbf{Attori primari}: venditore;
\item \textbf{Descrizione}: il venditore visualizza le informazioni dell'ordine selezionato;
\item \textbf{Scenario Principale}: il venditore accede alla pagina di visualizzazione informazioni ordine tramite il pulsante 'Dettagli' collegato all'ordine selezionato nella lista degli ordini ricevuti \textbf{[UC17]}. In particolare vengono visualizzati i seguenti dati:
\begin{itemize}
	\item numero ordine;
	\item prodotti acquistati;
	\item quantità prodotti acquistati;
	\item costo per singola voce;
	\item costo totale;
	\item tasse applicate;
	\item data acquisto.
\end{itemize}
\item \textbf{Precondizione}: l'utente ha eseguito il login, è riconosciuto come venditore e si trova nella merchant dashboard\ped{G};
\item \textbf{Postcondizione}: il venditore visualizza le informazioni relative all'ordine selezionato.
\end{itemize}

\subsubsection{UC18 - Visualizzazione collegamenti a sistemi di gestione}
\begin{itemize}
\item \textbf{Attori primari}: venditore;
\item \textbf{Descrizione}: il venditore visualizza nella merchant dashboard\ped{G} i collegamenti agli strumenti di gestione esterni;
\item \textbf{Scenario Principale}: il venditore accede alla merchant dashboard\ped{G} e, nell'apposita sezione, visualizza dei collegamenti alla piattaforma di monitoraggio dell'applicazione e agli strumenti di configurazione;
\item \textbf{Precondizione}: l'utente ha eseguito il login, è riconosciuto come venditore e si trova nella merchant dashboard\ped{G};
\item \textbf{Postcondizione}: il venditore visualizza la lista dei collegamenti ai sistemi di gestione del sito.
\end{itemize}

