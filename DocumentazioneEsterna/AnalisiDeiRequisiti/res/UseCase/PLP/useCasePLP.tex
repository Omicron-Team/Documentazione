\subsubsection{UC18 - Visualizzazione dei prodotti}
\begin{itemize}
\item \textbf{Attori primari}: utente generico;
\item \textbf{Descrizione}: l'utente può visualizzare un insieme di prodotti raggruppati per categoria accedendo alla relativa PLP\ped{G};
\item \textbf{Scenario Principale}: l'utente visualizza i prodotti appartenenti ad una determinata categoria. Ogni prodotto listato nella PLP\ped{G} deve esporre le seguenti informazioni:
\begin{itemize}
\item nome;
\item immagine;
\item prezzo.
\end{itemize}
\item \textbf{Precondizione}: l'utente ha cliccato sulla PLP\ped{G} dedicata alla categoria di prodotti scelta;
\item \textbf{Postcondizione}: l'utente visualizza le informazioni relative ai prodotti listati, con le eventuali operazioni disponibili su ognuno di essi.
\end{itemize}

\subsubsection{UC19 - Selezione dei prodotti}
\begin{itemize}
\item \textbf{Attori primari}: utente generico;
\item \textbf{Descrizione}: l'utente può selezionare prodotti multipli presenti sulla PLP\ped{G} in modo da poterli inserire successivamente nel carrello, con una quantità selezionata pari ad 1. Per selezionare un prodotto è sufficiente cliccare sulla relativa casella di selezione;
\item \textbf{Scenario Principale}:
\begin{enumerate}
\item l'utente visualizza i prodotti presenti nella lista corrispondente ad una categoria [\textbf{UC18}];
\item l'utente clicca sulle caselle relative ad uno o più prodotti per selezionarli.
\end{enumerate}
\item \textbf{Precondizione}: l'utente sta visualizzando la PLP\ped{G} contenente uno o più prodotti desiderati;
\item \textbf{Postcondizione}: l'utente ha selezionato i prodotti desiderati.
\end{itemize}

\subsubsection{UC20 - Inserimento nel carrello dei prodotti selezionati}
\begin{itemize}
\item \textbf{Attori primari}: utente generico;
\item \textbf{Descrizione}: l'utente può inserire un prodotto presente sulla PLP\ped{G} nel carrello, con una quantità selezionata pari ad 1. Per effettuare l'azione è sufficiente cliccare sull'apposito pulsante;
\item \textbf{Scenario Principale}:
\begin{enumerate}
\item l'utente seleziona i prodotti desiderati [\textbf{UC19}];
\item l'utente preme il pulsante per l'inserimento del prodotto desiderato nel carrello.
\end{enumerate}
\item \textbf{Precondizione}: l'utente sta visualizzando la PLP\ped{G} contenente prodotti;
\item \textbf{Postcondizione}: l'utente ha inserito nel proprio carrello uno o più prodotti selezionati.
\end{itemize}

\subsubsection{UC21 - Filtraggio dei prodotti}
\begin{itemize}
\item \textbf{Attori primari}: utente generico;
\item \textbf{Descrizione}: l'utente può raffinare la propria ricerca andando a filtrare i prodotti presenti nella PLP\ped{G}. Il parametro utilizzabile per eseguire questa operazione è il prezzo;
\item \textbf{Scenario Principale}:
\begin{enumerate}
\item l'utente abilita il filtro attraverso l'apposita casella;
\item l'utente digita il prezzo minimo e il prezzo massimo ai quali è interessato;
\item l'utente applica il filtro cliccando l'apposito bottone;
\item l'utente visualizza il risultato del filtraggio, che può essere di due tipologie:
\begin{itemize}
\item una nuova pagina in cui sono presenti solo i prodotti della PLP\ped{G} corrente e compresi nel range di prezzo specificato;
\item un messaggio che informa l'utente che il filtraggio dei prodotti effettuato utilizzando i parametri da lui inseriti non ha prodotto alcun risultato.
\end{itemize}
\end{enumerate}
\item \textbf{Precondizione}: l'utente sta visualizzando i prodotti in vendita listati nella specifica PLP\ped{G};
\item \textbf{Postcondizione}: l'utente visualizza il risultato del filtraggio effettuato sulla PLP\ped{G} corrente selezionando un determinato range di prezzo.
\end{itemize}

\subsubsection{UC22 - Accesso ad una pagina del prodotto}
\begin{itemize}
\item \textbf{Attori primari}: utente generico;
\item \textbf{Descrizione}: l'utente può accedere ad una pagina per visualizzare tutte le caratteristiche di un prodotto listato. E' inoltre possibile interagire con questa pagina, chiamata anche PDP\ped{G}, attraverso apposite funzionalità;
\item \textbf{Scenario Principale}: l'utente clicca sul prodotto da lui desiderato;
\item \textbf{Precondizione}: l'utente si trova in una PLP\ped{G} contenente una lista di prodotti;
\item \textbf{Postcondizione}: l'utente accede alla PDP\ped{G} corrispondente al prodotto desiderato.
\end{itemize}