\begin{figure}[H]
\centering
\includegraphics[scale=0.6]{res/UseCase/Immagini/Login}
\caption{Diagramma UML per modulo di login}
\end{figure}

\subsubsection{UC3 - Login}
\begin{itemize}
\item \textbf{Attori primari}: utente non autenticato;
\item \textbf{Attori secondari}: identity manager;
\item \textbf{Descrizione}: l'utente, inserendo le proprie credenziali, viene autenticato alla piattaforma;
\item \textbf{Scenario Principale}: l'utente non ancora autenticato seleziona il tipo di utente \textbf{[UC3.1]}, inserisce le proprie credenziali negli appositi campi dati \textbf{[UC3.2]} e richiede il login \textbf{[UC3.3]};
\item \textbf{Estensioni}:
\begin{itemize}
	\item \textbf{UC4}: se le credenziali inserite non vengono riconosciute dall'identity manager, viene visualizzato un messaggio che informa l'utente dell'errore;
\end{itemize}
\item \textbf{Precondizione}: l'utente prova ad autenticarsi alla piattaforma;
\item \textbf{Postcondizione}: l'utente viene autenticato come il tipo di utente selezionato in \textbf{[UC3.1]}.
\end{itemize}

\begin{figure}[H]
\centering
\includegraphics[scale=0.6]{res/UseCase/Immagini/LoginSottocasi}
\caption{Diagramma UML per UC3 - Login}
\end{figure}

\subsubsection{UC3.1 - Selezione tipo utente}
\begin{itemize}
\item \textbf{Attori primari}: utente non autenticato;
\item \textbf{Descrizione}: l'utente seleziona il tipo di utente per effettuare il login;
\item \textbf{Scenario Principale}: l'utente indica se vuole effettuare il login come cliente o come venditore;
\item \textbf{Precondizione}: il form di inserimento dati per selezione del tipo di utente è disponibile;
\item \textbf{Postcondizione}: è stato selezionato il tipo di utente per il login.
\end{itemize}

\subsubsection{UC3.2 - Inserimento dati login}
\begin{itemize}
\item \textbf{Attori primari}: utente non autenticato;
\item \textbf{Descrizione}: l'utente compila il form per il login;
\item \textbf{Scenario Principale}: l'utente inserisce nel form apposito i dati necessari al login;
\item \textbf{Specializzazioni}: A seconda del tipo di utente selezionato in \textbf{[UC3.1]}
\begin{itemize}
	\item Inserimento dati login cliente \textbf{[UC3.2.3]}
	\item Inserimento dati login venditore \textbf{[UC3.2.4]}
\end{itemize}
\item \textbf{Precondizione}: il form di inserimento dati per il login è disponibile;
\item \textbf{Postcondizione}: i dati necessari al login sono stati compilati.
\end{itemize}

\subsubsection{UC3.2.1 - Inserimento password}
\begin{itemize}
\item \textbf{Attori primari}: utente non autenticato;
\item \textbf{Descrizione}: l'utente deve compilare il campo "Password" per procedere al login;
\item \textbf{Scenario Principale}: l'utente inserisce la sua password nell'apposito campo;
\item \textbf{Precondizione}: il campo "Password" risulta vuoto;
\item \textbf{Postcondizione}: il campo "Password" è stato compilato.
\end{itemize}

\subsubsection{UC3.2.2 - Inserimento email}
\begin{itemize}
\item \textbf{Attori primari}: utente non autenticato;
\item \textbf{Descrizione}: l'utente deve compilare il campo "Email" per procedere al login;
\item \textbf{Scenario Principale}: l'utente inserisce il suo indirizzo email nell'apposito campo;
\item \textbf{Precondizione}: il campo "Email" risulta vuoto;
\item \textbf{Postcondizione}: il campo "Email" è stato compilato.
\end{itemize}

\begin{figure}[H]
\centering
\includegraphics[scale=0.6]{res/UseCase/Immagini/InserimentoDatiLoginCliente}
\caption{Diagramma UML per UC3.2.3 - Inserimento dati login cliente}
\end{figure}

\subsubsection{UC3.2.3 - Inserimento dati login cliente}
\begin{itemize}
\item \textbf{Attori primari}: utente non autenticato;
\item \textbf{Descrizione}: l'utente compila il form per il login;
\item \textbf{Scenario Principale}: l'utente che ha selezionato il tipo utente 'cliente' \textbf{[UC3.1]} inserisce nel form apposito la propria email \textbf{[UC3.2.2]} e la propria password \textbf{[UC3.2.1]};
\item \textbf{Precondizione}: il form di inserimento dati per il login è disponibile;
\item \textbf{Postcondizione}: i dati necessari al login sono stati compilati.
\end{itemize}

\begin{figure}[H]
\centering
\includegraphics[scale=0.6]{res/UseCase/Immagini/InserimentoDatiLoginVenditore}
\caption{Diagramma UML per UC3.2.4 - Inserimento dati login venditore}
\end{figure}

\subsubsection{UC3.2.4 - Inserimento dati login venditore}
\begin{itemize}
\item \textbf{Attori primari}: utente non autenticato;
\item \textbf{Descrizione}: l'utente compila il form per il login;
\item \textbf{Scenario Principale}: l'utente che ha selezionato il tipo utente 'venditore' \textbf{[UC3.1]} inserisce nel form apposito la propria password \textbf{[UC3.2.1]};
\item \textbf{Precondizione}: il form di inserimento dati per il login è disponibile;
\item \textbf{Postcondizione}: i dati necessari al login sono stati compilati.
\end{itemize}

\subsubsection{UC3.3 - Richiesta di login}
\begin{itemize}
\item \textbf{Attori primari}: utente non autenticato;
\item \textbf{Descrizione}: l'utente richiede l'autenticazione con i dati inseriti;
\item \textbf{Scenario Principale}: l'utente preme il tasto di login e manda la richiesta al sistema con i dati presenti nel form;
\item \textbf{Precondizione}: l'utente prova ad autenticarsi alla piattaforma;
\item \textbf{Postcondizione}: la richiesta di login è stata mandata al sistema;
\end{itemize} 

\subsubsection{UC4 - Visualizzazione errore dati login errati}
\begin{itemize}
\item \textbf{Attori primari}: utente non autenticato;
\item \textbf{Attori secondari}: identity manager;
\item \textbf{Descrizione}: l'utente visualizza un messaggio di errore che lo informa che i dati da lui inseriti durante il login non sono riconosciuti dall'identity manager;
\item \textbf{Scenario Principale}: l'utente tenta di effettuare il login usando credenziali non presenti nel sistema;
\item \textbf{Precondizione}: l'utente prova ad autenticarsi alla piattaforma;
\item \textbf{Postcondizione}: viene visualizzato un messaggio che informa l'utente dell'errore di riconoscimento delle credenziali.
\end{itemize}
