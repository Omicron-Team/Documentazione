\section{Resoconto}
\subsection{Presentazione dei membri del gruppo al proponente}
Il gruppo si è presentato al proponente, l'azienda Red Babel, ed ha esposto la volontà di conseguire il progetto \textit{EmporioLambda}.
\subsection{Richiesta di spiegazioni al proponente su punti non chiari del capitolato da parte dei membri del gruppo}
Il gruppo ha preparato per l'incontro le seguenti domande (precedentemente riferite attraverso Slack\ped{G}):
\begin{itemize}
\item Per l'utilizzo della piattaforma di computazione serverless\ped{G} AWS Lambda\ped{G} è necessario l'apprendimento di un linguaggio di programmazione funzionale?
\item I ``business requirements" (\href{https://www.math.unipd.it/~tullio/IS-1/2020/Progetto/C2.pdf}{Documento descrittivo capitolato C2} , §1.1) sono vincolanti per una corretta implementazione della soluzione al capitolato?
\item L'azienda proponente ha una preferenza riguardo all'utilizzo di AWS Cognito Identity\ped{G} oppure Auth0\ped{G} per l'implementazione dello strato di autenticazione della piattaforma?
\end{itemize}
Seguono le risposte offerte dal proponente:
\begin{itemize}
\item \textbf{Necessità di apprendimento di un linguaggio di programmazione funzionale:} \\
Non è necessario l'apprendimento del paradigma di programmazione funzionale in quanto questo pone dei vincoli nella sintassi e nelle tecnologie utilizzabili non richiesti. Il requisito di utilizzo della piattaforma di computazione AWS Lambda\ped{G} descrive invece il comportamento che ci si aspetta da parte di un modulo dell'architettura: essendo \textit{EmporioLambda} una piattaforma da sviluppare con tecnologie serverless\ped{G} le componenti necessarie al suo funzionamento devono poter completare il proprio obiettivo sotto forma di funzioni che, assumendo non abbiano uno stato interno, avendo in ingresso un input producono un output.
\item \textbf{Vincolatività dei ``business requirements":} \\
Questa tipologia di requisiti inseriti nella parte introduttiva del capitolato sono stati forniti per offrire una panoramica di base al contesto dell'E-commerce\ped{G} e alla sua elevata complessità, sono quindi da non considerarsi dei requisiti minimi al completamento della piattaforma.
\item \textbf{Preferenza riguardo all'utilizzo di AWS Cognito Identity o Auth0 per lo strato di autenticazione della piattaforma:} \\
L'azienda afferma che non ha preferenze per nessuno dei due servizi e che risulta importante una analisi e comparazione da parte del gruppo per effettuare la scelta migliore. \textit{Red Babel} aggiunge inoltre che mentre Cognito\ped{G} si integra meglio nella piattaforma e nei servizi forniti da Amazon\ped{G}, Auth0\ped{G} fornisce un'ampia documentazione sul sito proprietario, permettendo una migliore esperienza di auto-apprendimento della tecnologia.
\end{itemize}

