\section{Resoconto}
Il gruppo ha preparato per l'incontro le seguenti domande:
\begin{itemize}
\item Sono necessarie due pagine di login diverse? Una per il cliente e una per il venditore?
\item Si deve poter visualizzare e modificare l'indirizzo email durante la fase di checkout?
\item In che lingue devono essere redatti i manuali utente e sviluppatore?
\item In che lingue deve essere realizzato il prodotto?
\item Deve essere prevista la gestione delle categorie dei prodotti dalla dashboard del venditore?

\end{itemize}
Seguono le risposte offerte dal proponente:
\begin{itemize}
\item \textbf{Pagine di login:}\\
Si, viene consigliato di creare due form di login distinte, in modo da semplificare la logica di autenticazione;
\item \textbf{Email nella fase di checkout:}\\
Si, il cliente dovrebbe vedere la propria email, dove i prodotti verranno spediti. La funzionalità di modifica dell'email viene ritenuta desiderabile, non obbligatoria; 
\item \textbf{Lingua manuali:}\\
La lingua richiesta per entrambi i manuali è l'inglese. Facoltativa invece la lingua italiana;
\item \textbf{Lingua prodotto:}\\
Come per il manuale, la lingua richiesta è l'inglese. Facoltativa invece la lingua italiana;
\item \textbf{Gestione categorie prodotti:}\\
Si, viene richiesta una gestione delle categorie dei prodotti dalla dashboard del venditore. Viene inoltre specificato che questa gestione non deve provocare un re-deploy dell'applicazione.
\end{itemize}


